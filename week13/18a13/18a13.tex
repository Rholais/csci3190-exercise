% This is "sig-alternate.tex" V2.1 April 2013
% This file should be compiled with V2.5 of "sig-alternate.cls" May 2012
%
% This example file demonstrates the use of the 'sig-alternate.cls'
% V2.5 LaTeX2e document class file. It is for those submitting
% articles to ACM Conference Proceedings WHO DO NOT WISH TO
% STRICTLY ADHERE TO THE SIGS (PUBS-BOARD-ENDORSED) STYLE.
% The 'sig-alternate.cls' file will produce a similar-looking,
% albeit, 'tighter' paper resulting in, invariably, fewer pages.
%
% ----------------------------------------------------------------------------------------------------------------
% This .tex file (and associated .cls V2.5) produces:
%       1) The Permission Statement
%       2) The Conference (location) Info information
%       3) The Copyright Line with ACM data
%       4) NO page numbers
%
% as against the acm_proc_article-sp.cls file which
% DOES NOT produce 1) thru' 3) above.
%
% Using 'sig-alternate.cls' you have control, however, from within
% the source .tex file, over both the CopyrightYear
% (defaulted to 200X) and the ACM Copyright Data
% (defaulted to X-XXXXX-XX-X/XX/XX).
% e.g.
% \CopyrightYear{2007} will cause 2007 to appear in the copyright line.
% \crdata{0-12345-67-8/90/12} will cause 0-12345-67-8/90/12 to appear in the copyright line.
%
% ---------------------------------------------------------------------------------------------------------------
% This .tex source is an example which *does* use
% the .bib file (from which the .bbl file % is produced).
% REMEMBER HOWEVER: After having produced the .bbl file,
% and prior to final submission, you *NEED* to 'insert'
% your .bbl file into your source .tex file so as to provide
% ONE 'self-contained' source file.
%
% ================= IF YOU HAVE QUESTIONS =======================
% Questions regarding the SIGS styles, SIGS policies and
% procedures, Conferences etc. should be sent to
% Adrienne Griscti (griscti@acm.org)
%
% Technical questions _only_ to
% Gerald Murray (murray@hq.acm.org)
% ===============================================================
%
% For tracking purposes - this is V2.0 - May 2012

\documentclass{../../cls/sig-alternate-05-2015}

\usepackage{algorithm}
\usepackage{algpseudocode}

\usepackage{booktabs}
\usepackage{cleveref}
\usepackage{color}
\usepackage{enumitem}
\usepackage{mathtools}
\usepackage{soul}
\usepackage{textcomp}

\usepackage{tkz-graph}
% \GraphInit[vstyle = Shade]
\tikzset{
    LabelStyle/.style = { rectangle, rounded corners, draw,
        minimum width = 1em, fill = white,
        text = black, font = \bfseries },
    VertexStyle/.append style = {font = \bfseries},
    EdgeStyle/.append style = {->, bend left = 10} }
\thispagestyle{empty}

\begin{document}

% Copyright
%\setcopyright{acmcopyright}
%\setcopyright{acmlicensed}
%\setcopyright{rightsretained}
%\setcopyright{usgov}
%\setcopyright{usgovmixed}
%\setcopyright{cagov}
%\setcopyright{cagovmixed}


% DOI
%\doi{10.475/123_4}

% ISBN
%\isbn{123-4567-24-567/08/06}

%Conference
%\conferenceinfo{PLDI '13}{June 16--19, 2013, Seattle, WA, USA}

%\acmPrice{\$15.00}

%
% --- Author Metadata here ---
%\conferenceinfo{WOODSTOCK}{'97 El Paso, Texas USA}
%\CopyrightYear{2007} % Allows default copyright year (20XX) to be over-ridden - IF NEED BE.
%\crdata{0-12345-67-8/90/01}  % Allows default copyright data (0-89791-88-6/97/05) to be over-ridden - IF NEED BE.
% --- End of Author Metadata ---

\makeatletter
\def\old@comma{,}
\catcode`\,=13
\def,{%
    \ifmmode%
    \old@comma\discretionary{}{}{}%
    \else%
    \old@comma%
    \fi%
}
\makeatother

\title{CSCI 3190 \\ Introduction to Discrete Mathematics and Algorithms}
\subtitle{Sample Solution of Assignment 3}

\maketitle
\begin{abstract}

\end{abstract}

\keywords{}

\section{Ford-Fulkerson Algorithm}
\textbf{Solution.}
\begin{enumerate}[label={(\alph*)}]
    \item Shown in~\Cref{fig:1:a:1,fig:1:a:2,fig:1:a:3,fig:1:a:4,fig:1:a:5,fig:1:a:6,fig:1:a:7}, the maximum flow is 7.
    \begin{figure}[H]
        \resizebox{\linewidth}{!}{\begin{tikzpicture}
        \SetGraphUnit{3}
        \Vertex{s}
        \EA(s){a}
        \NOEA(a){b}
        \SOEA(a){c}
        \SOEA(b){t}
        \Edge[label = 6](s)(b)
        \Edge[label = 5](s)(c)
        \Edge[label = 2](a)(b)
        \Edge[label = 2](b)(c)
        \Edge[label = 3](b)(t)
        \tikzset{EdgeStyle/.append style = {very thick}}
        \Edge[label = 3](s)(a)
        \Edge[label = 3](a)(c)
        \Edge[label = 4](c)(t)
        % \tikzset{EdgeStyle/.append style = {bend left = 50}}
        \end{tikzpicture}}
        \caption{Step 1. Residual Graph. BottleNeck = 3.}
        \label{fig:1:a:1}
    \end{figure}
    \begin{figure}[H]
        \resizebox{\linewidth}{!}{\begin{tikzpicture}
            \SetGraphUnit{3}
            \Vertex{s}
            \EA(s){a}
            \NOEA(a){b}
            \SOEA(a){c}
            \SOEA(b){t}
            \Edge[label = 6](s)(b)
            \Edge[label = 5](s)(c)
            \Edge[label = 2](a)(b)
            \Edge[label = 2](b)(c)
            \Edge[label = 3](b)(t)
            % \tikzset{EdgeStyle/.append style = {very thick}}
            \Edge[label = 3/3](s)(a)
            \Edge[label = 3/3](a)(c)
            \Edge[label = 3/4](c)(t)
            % \tikzset{EdgeStyle/.append style = {bend left = 50}}
            \end{tikzpicture}}
        \caption{Step 1. Flow Graph. Flow = 3.}
        \label{fig:1:a:2}
    \end{figure}
    \begin{figure}[H]
        \resizebox{\linewidth}{!}{\begin{tikzpicture}
            \SetGraphUnit{3}
            \Vertex{s}
            \EA(s){a}
            \NOEA(a){b}
            \SOEA(a){c}
            \SOEA(b){t}
            \Edge[label = 5](s)(c)
            \Edge[label = 2](a)(b)
            \Edge[label = 2](b)(c)
            \Edge[label = 3](b)(t)
            \Edge[label = 1](c)(t)
            \tikzset{EdgeStyle/.append style = {very thin}}
            \Edge[label = 3](a)(s)
            \Edge[label = 3](c)(a)
            \Edge[label = 3](t)(c)
            \tikzset{EdgeStyle/.append style = {very thick}}
            \Edge[label = 6](s)(b)
            \Edge[label = 3](b)(t)
            % \tikzset{EdgeStyle/.append style = {bend left = 50}}
            \end{tikzpicture}}
        \caption{Step 2. Residual Graph. BottleNeck = 3.}
        \label{fig:1:a:3}
    \end{figure}
    \begin{figure}[H]
        \resizebox{\linewidth}{!}{\begin{tikzpicture}
            \SetGraphUnit{3}
            \Vertex{s}
            \EA(s){a}
            \NOEA(a){b}
            \SOEA(a){c}
            \SOEA(b){t}
            \Edge[label = 5](s)(c)
            \Edge[label = 2](a)(b)
            \Edge[label = 2](b)(c)
            % \tikzset{EdgeStyle/.append style = {very thick}}
            \Edge[label = 3/3](s)(a)
            \Edge[label = 3/3](a)(c)
            \Edge[label = 3/4](c)(t)
            \Edge[label = 3/6](s)(b)
            \Edge[label = 3/3](b)(t)
            % \tikzset{EdgeStyle/.append style = {bend left = 50}}
            \end{tikzpicture}}
        \caption{Step 2. Flow Graph. Flow = 6.}
        \label{fig:1:a:4}
    \end{figure}
    \begin{figure}[H]
        \resizebox{\linewidth}{!}{\begin{tikzpicture}
            \SetGraphUnit{3}
            \Vertex{s}
            \EA(s){a}
            \NOEA(a){b}
            \SOEA(a){c}
            \SOEA(b){t}
            \Edge[label = 5](s)(c)
            \Edge[label = 2](a)(b)
            \tikzset{EdgeStyle/.append style = {very thin}}
            \Edge[label = 3](a)(s)
            \Edge[label = 3](c)(a)
            \Edge[label = 3](t)(c)
            \Edge[label = 3](b)(s)
            \Edge[label = 3](t)(b)
            \tikzset{EdgeStyle/.append style = {very thick}}
            \Edge[label = 3](s)(b)
            \Edge[label = 2](b)(c)
            \Edge[label = 1](c)(t)
            % \tikzset{EdgeStyle/.append style = {bend left = 50}}
            \end{tikzpicture}}
        \caption{Step 3. Residual Graph. BottleNeck = 1.}
        \label{fig:1:a:5}
    \end{figure}
    \begin{figure}[H]
        \resizebox{\linewidth}{!}{\begin{tikzpicture}
            \SetGraphUnit{3}
            \Vertex{s}
            \EA(s){a}
            \NOEA(a){b}
            \SOEA(a){c}
            \SOEA(b){t}
            \Edge[label = 5](s)(c)
            \Edge[label = 2](a)(b)
            % \tikzset{EdgeStyle/.append style = {very thick}}
            \Edge[label = 3/3](s)(a)
            \Edge[label = 3/3](a)(c)
            \Edge[label = 4/4](c)(t)
            \Edge[label = 4/6](s)(b)
            \Edge[label = 3/3](b)(t)
            \Edge[label = 1/2](b)(c)
            % \tikzset{EdgeStyle/.append style = {bend left = 50}}
            \end{tikzpicture}}
        \caption{Step 3. Flow Graph. Flow = 7.}
        \label{fig:1:a:6}
    \end{figure}
    \begin{figure}[H]
        \resizebox{\linewidth}{!}{\begin{tikzpicture}
            \SetGraphUnit{3}
            \Vertex{s}
            \EA(s){a}
            \NOEA(a){b}
            \SOEA(a){c}
            \SOEA(b){t}
            \Edge[label = 2](s)(b)
            \Edge[label = 5](s)(c)
            \Edge[label = 2](a)(b)
            \Edge[label = 1](b)(c)
            \tikzset{EdgeStyle/.append style = {very thin}}
            \Edge[label = 3](a)(s)
            \Edge[label = 3](c)(a)
            \Edge[label = 3](t)(c)
            \Edge[label = 4](b)(s)
            \Edge[label = 3](t)(b)
            \Edge[label = 1](c)(b)
            \Edge[label = 4](t)(c)
            \tikzset{EdgeStyle/.append style = {very thick}}
            % \tikzset{EdgeStyle/.append style = {bend left = 50}}
            \end{tikzpicture}}
        \caption{Step 4. Residual Graph.}
        \label{fig:1:a:7}
    \end{figure}
    \item The minimum cut is $(\{s, a, b, c\}, \{t\})$.
\end{enumerate}

\section{National Football League}
\textbf{Solution.} It's possible.

For the games within each conference,
each team needs to play with every other teams inside the conference.

For the games between the two conferences, it easy to model it as a network flow problem.
Let the two conferences be $A = \{a_1, a_2, \ldots, a_{12}\}$
and $B = \{b_1, b_2, \ldots, b_{12}\}$,
we then have a source node for each $a_i$ and a target node for each $b_i$
with $1 \le i \le 12$.
Since each team plays three cross-conference games,
the supply of each source node is three and the demand of each target node is also three.
Every team is able to play with each of the teams in the other conference,
so there is an edge from each source to each target and the capacitance is one.

It's very obvious that the network flow problem has solution.
We can partition teams in each conference to four groups $A_1, A_2, A_3, A_4$
and $B_1, B_2, B_3, B_4$ where each group contains three team.
One solution of the network flow problem above is to let each team in $A_j$ play
against each team in $B_j$ with $1 \le j \le 4$.


\iffalse
It is impossible.
For each conference, the 13 teams produce in $13 \times 11 = 143$ appearances in the matches.
It is impossible because the total number of appearance should be even.
\fi

\section{Longest Common Subsequence}
\textbf{Solution.} See~\Cref{a:3}.
\begin{algorithm}[H]
    \caption{Print Out ALL the LCSes}
    \label{a:3}
    \begin{algorithmic}
        \Procedure{All}{$T$, $A$, $B$, $j$, $k$}
        \If{$j \cdot k = 0$} \Return $\{``"\}$
        \ElsIf{$A[j] = B[k]$}
        \State \Return $\{C + A[j] \colon Z \in \Call{All}{T, A, B, j - 1, k - 1}\}$
        \Else
        \State $R \coloneqq \{\}$
        \If{$T[j, k - 1] \ge T[j - 1, k]$}
        \State $R \coloneqq R \cup \Call{All}{T, A, B, j, k - 1}$
        \EndIf
        \If{$T[j - 1, k] \ge T[j, k - 1]$}
        \State $R \coloneqq R \cup \Call{All}{T, A, B, j - 1, k}$
        \EndIf
        \State \Return $R$
        \EndIf
        \EndProcedure
    \end{algorithmic}
\end{algorithm}

\section{Dijkstra's Algorithm}
\begin{enumerate}[label={(\alph*)}]
    \item Shown in~\Cref{fig:4:a:1,fig:4:a:2,fig:4:a:3,fig:4:a:4,fig:4:a:5,fig:4:a:6,fig:4:a:7,fig:4:a:8}.
    \begin{figure}[H]
        \resizebox{\linewidth}{!}{\begin{tikzpicture}
            \SetGraphUnit{3}
            \Vertex{a}
            \EA(a){b}
            \SO(a){c}
            \SO(b){d}
            \WE(c){e}
            \EA(d){f}
            \EA(b){g}
            \Edge[label = 6](a)(b)
            \Edge[label = 1](a)(c)
            \Edge[label = 2](a)(d)
            \Edge[label = 2](b)(g)
            \Edge[label = 6](c)(b)
            \Edge[label = 2](c)(d)
            \Edge[label = 3](d)(b)
            \Edge[label = 5](d)(f)
            \Edge[label = 7](e)(a)
            \Edge[label = 4](e)(c)
            \tikzset{EdgeStyle/.append style = {very thin}}
            \tikzset{EdgeStyle/.append style = {very thick}}
            \tikzset{EdgeStyle/.append style = {bend left = 10}}
            \WE(c){e 0}
            \end{tikzpicture}}
        \caption{Step 0.}
        \label{fig:4:a:1}
    \end{figure}
    \begin{figure}[H]
        \resizebox{\linewidth}{!}{\begin{tikzpicture}
            \SetGraphUnit{3}
            \Vertex{a}
            \EA(a){b}
            \SO(a){c}
            \SO(b){d}
            \WE(c){e}
            \EA(d){f}
            \EA(b){g}
            \Edge[label = 6](a)(b)
            \Edge[label = 1](a)(c)
            \Edge[label = 2](a)(d)
            \Edge[label = 2](b)(g)
            \Edge[label = 6](c)(b)
            \Edge[label = 2](c)(d)
            \Edge[label = 3](d)(b)
            \Edge[label = 5](d)(f)
            \tikzset{EdgeStyle/.append style = {very thin}}
            \tikzset{EdgeStyle/.append style = {very thick}}
            \Edge[label = 7](e)(a)
            \Edge[label = 4](e)(c)
            \tikzset{EdgeStyle/.append style = {bend left = 10}}
            \WE(b){a 7}
            \SO(a){c 4}
            \WE(c){e 0}
            \end{tikzpicture}}
        \caption{Step 1. Update with $e$.}
        \label{fig:4:a:2}
    \end{figure}
    \begin{figure}[H]
        \resizebox{\linewidth}{!}{\begin{tikzpicture}
            \SetGraphUnit{3}
            \Vertex{a}
            \EA(a){b}
            \SO(a){c}
            \SO(b){d}
            \WE(c){e}
            \EA(d){f}
            \EA(b){g}
            \Edge[label = 6](a)(b)
            \Edge[label = 1](a)(c)
            \Edge[label = 2](a)(d)
            \Edge[label = 2](b)(g)
            \Edge[label = 3](d)(b)
            \Edge[label = 5](d)(f)
            \Edge[label = 7](e)(a)
            \Edge[label = 4](e)(c)
            \tikzset{EdgeStyle/.append style = {very thin}}
            \tikzset{EdgeStyle/.append style = {very thick}}
            \Edge[label = 6](c)(b)
            \Edge[label = 2](c)(d)
            \tikzset{EdgeStyle/.append style = {bend left = 10}}
            \WE(b){a 7}
            \EA(a){b 10}
            \SO(a){c 4}
            \WE(c){e 0}
            \SO(b){d 6}
            \end{tikzpicture}}
        \caption{Step 2. Update with $c$.}
        \label{fig:4:a:3}
    \end{figure}
    \begin{figure}[H]
        \resizebox{\linewidth}{!}{\begin{tikzpicture}
            \SetGraphUnit{3}
            \Vertex{a}
            \EA(a){b}
            \SO(a){c}
            \SO(b){d}
            \WE(c){e}
            \EA(d){f}
            \EA(b){g}
            \Edge[label = 6](a)(b)
            \Edge[label = 1](a)(c)
            \Edge[label = 2](a)(d)
            \Edge[label = 2](b)(g)
            \Edge[label = 7](e)(a)
            \Edge[label = 4](e)(c)
            \Edge[label = 6](c)(b)
            \Edge[label = 2](c)(d)
            \tikzset{EdgeStyle/.append style = {very thin}}
            \tikzset{EdgeStyle/.append style = {very thick}}
            \Edge[label = 3](d)(b)
            \Edge[label = 5](d)(f)
            \tikzset{EdgeStyle/.append style = {bend left = 10}}
            \WE(b){a 7}
            \EA(a){b 9}
            \SO(a){c 4}
            \WE(c){e 0}
            \SO(b){d 6}
            \EA(d){f 11}
            \end{tikzpicture}}
        \caption{Step 3. Update with $d$.}
        \label{fig:4:a:4}
    \end{figure}
    \begin{figure}[H]
        \resizebox{\linewidth}{!}{\begin{tikzpicture}
            \SetGraphUnit{3}
            \Vertex{a}
            \EA(a){b}
            \SO(a){c}
            \SO(b){d}
            \WE(c){e}
            \EA(d){f}
            \EA(b){g}
            \Edge[label = 2](b)(g)
            \Edge[label = 7](e)(a)
            \Edge[label = 4](e)(c)
            \Edge[label = 6](c)(b)
            \Edge[label = 2](c)(d)
            \Edge[label = 3](d)(b)
            \Edge[label = 5](d)(f)
            \tikzset{EdgeStyle/.append style = {very thin}}
            \tikzset{EdgeStyle/.append style = {very thick}}
            \Edge[label = 6](a)(b)
            \Edge[label = 1](a)(c)
            \Edge[label = 2](a)(d)
            \tikzset{EdgeStyle/.append style = {bend left = 10}}
            \WE(b){a 7}
            \EA(a){b 9}
            \SO(a){c 4}
            \WE(c){e 0}
            \SO(b){d 6}
            \EA(d){f 11}
            \end{tikzpicture}}
        \caption{Step 4. Update with $a$.}
        \label{fig:4:a:5}
    \end{figure}
    \begin{figure}[H]
        \resizebox{\linewidth}{!}{\begin{tikzpicture}
            \SetGraphUnit{3}
            \Vertex{a}
            \EA(a){b}
            \SO(a){c}
            \SO(b){d}
            \WE(c){e}
            \EA(d){f}
            \EA(b){g}
            \Edge[label = 7](e)(a)
            \Edge[label = 4](e)(c)
            \Edge[label = 6](c)(b)
            \Edge[label = 2](c)(d)
            \Edge[label = 3](d)(b)
            \Edge[label = 5](d)(f)
            \Edge[label = 6](a)(b)
            \Edge[label = 1](a)(c)
            \Edge[label = 2](a)(d)
            \tikzset{EdgeStyle/.append style = {very thin}}
            \tikzset{EdgeStyle/.append style = {very thick}}
            \Edge[label = 2](b)(g)
            \tikzset{EdgeStyle/.append style = {bend left = 10}}
            \WE(b){a 7}
            \EA(a){b 9}
            \SO(a){c 4}
            \WE(c){e 0}
            \SO(b){d 6}
            \EA(d){f 11}
            \EA(b){g 11}
            \end{tikzpicture}}
        \caption{Step 5. Update with $b$.}
        \label{fig:4:a:6}
    \end{figure}
    \begin{figure}[H]
        \resizebox{\linewidth}{!}{\begin{tikzpicture}
            \SetGraphUnit{3}
            \Vertex{a}
            \EA(a){b}
            \SO(a){c}
            \SO(b){d}
            \WE(c){e}
            \EA(d){f}
            \EA(b){g}
            \Edge[label = 7](e)(a)
            \Edge[label = 4](e)(c)
            \Edge[label = 6](c)(b)
            \Edge[label = 2](c)(d)
            \Edge[label = 3](d)(b)
            \Edge[label = 5](d)(f)
            \Edge[label = 6](a)(b)
            \Edge[label = 1](a)(c)
            \Edge[label = 2](a)(d)
            \Edge[label = 2](b)(g)
            \tikzset{EdgeStyle/.append style = {very thin}}
            \tikzset{EdgeStyle/.append style = {very thick}}
            \tikzset{EdgeStyle/.append style = {bend left = 10}}
            \WE(b){a 7}
            \EA(a){b 9}
            \SO(a){c 4}
            \WE(c){e 0}
            \SO(b){d 6}
            \EA(d){f 11}
            \EA(b){g 11}
            \end{tikzpicture}}
        \caption{Step 6. Update with $f$.}
        \label{fig:4:a:7}
    \end{figure}
    \begin{figure}[H]
        \resizebox{\linewidth}{!}{\begin{tikzpicture}
            \SetGraphUnit{3}
            \Vertex{a}
            \EA(a){b}
            \SO(a){c}
            \SO(b){d}
            \WE(c){e}
            \EA(d){f}
            \EA(b){g}
            \Edge[label = 7](e)(a)
            \Edge[label = 4](e)(c)
            \Edge[label = 6](c)(b)
            \Edge[label = 2](c)(d)
            \Edge[label = 3](d)(b)
            \Edge[label = 5](d)(f)
            \Edge[label = 6](a)(b)
            \Edge[label = 1](a)(c)
            \Edge[label = 2](a)(d)
            \Edge[label = 2](b)(g)
            \tikzset{EdgeStyle/.append style = {very thin}}
            \tikzset{EdgeStyle/.append style = {very thick}}
            \tikzset{EdgeStyle/.append style = {bend left = 10}}
            \WE(b){a 7}
            \EA(a){b 9}
            \SO(a){c 4}
            \WE(c){e 0}
            \SO(b){d 6}
            \EA(d){f 11}
            \EA(b){g 11}
            \end{tikzpicture}}
        \caption{Step 7. Update with $g$.}
        \label{fig:4:a:8}
    \end{figure}
    \item To find the single source shortest paths for an un-directed graph,
    we can transform the un-directed graph to an equivalent directed graph
    by duplicating each un-directed edge as two directed edge in both direction,
    as shown in~\Cref{fig:4:b:1,fig:4:b:2,fig:4:b:3,fig:4:b:4,fig:4:b:5,fig:4:b:6,fig:4:b:7,fig:4:b:8}.
    \begin{figure}[H]
        \resizebox{\linewidth}{!}{\begin{tikzpicture}
            \SetGraphUnit{3}
            \Vertex{a}
            \EA(a){b}
            \SO(a){c}
            \SO(b){d}
            \WE(c){e}
            \EA(d){f}
            \EA(b){g}
            \tikzset{EdgeStyle/.append style = {-}}
            \Edge[label = 7](e)(a)
            \Edge[label = 4](e)(c)
            \Edge[label = 6](c)(b)
            \Edge[label = 2](c)(d)
            \Edge[label = 3](d)(b)
            \Edge[label = 5](d)(f)
            \Edge[label = 6](a)(b)
            \Edge[label = 1](a)(c)
            \Edge[label = 2](a)(d)
            \Edge[label = 2](b)(g)
            \tikzset{EdgeStyle/.append style = {very thin}}
            \tikzset{EdgeStyle/.append style = {very thick}}
            \tikzset{EdgeStyle/.append style = {bend left = 10}}
            % \WE(b){a 7}
            % \EA(a){b 9}
            % \SO(a){c 4}
            \WE(c){e 0}
            % \SO(b){d 6}
            % \EA(d){f 11}
            % \EA(b){g 11}
            \end{tikzpicture}}
        \caption{Step 0.}
        \label{fig:4:b:1}
    \end{figure}
    \begin{figure}[H]
        \resizebox{\linewidth}{!}{\begin{tikzpicture}
            \SetGraphUnit{3}
            \Vertex{a}
            \EA(a){b}
            \SO(a){c}
            \SO(b){d}
            \WE(c){e}
            \EA(d){f}
            \EA(b){g}
            \tikzset{EdgeStyle/.append style = {-}}
            \Edge[label = 6](c)(b)
            \Edge[label = 2](c)(d)
            \Edge[label = 3](d)(b)
            \Edge[label = 5](d)(f)
            \Edge[label = 6](a)(b)
            \Edge[label = 1](a)(c)
            \Edge[label = 2](a)(d)
            \Edge[label = 2](b)(g)
            \tikzset{EdgeStyle/.append style = {very thin}}
            \tikzset{EdgeStyle/.append style = {very thick}}
            \Edge[label = 7](e)(a)
            \Edge[label = 4](e)(c)
            \tikzset{EdgeStyle/.append style = {bend left = 10}}
            \WE(b){a 7}
            % \EA(a){b 9}
            \SO(a){c 4}
            \WE(c){e 0}
            % \SO(b){d 6}
            % \EA(d){f 11}
            % \EA(b){g 11}
            \end{tikzpicture}}
        \caption{Step 1. Update with $e$.}
        \label{fig:4:b:2}
    \end{figure}
    \begin{figure}[H]
        \resizebox{\linewidth}{!}{\begin{tikzpicture}
            \SetGraphUnit{3}
            \Vertex{a}
            \EA(a){b}
            \SO(a){c}
            \SO(b){d}
            \WE(c){e}
            \EA(d){f}
            \EA(b){g}
            \tikzset{EdgeStyle/.append style = {-}}
            \Edge[label = 3](d)(b)
            \Edge[label = 5](d)(f)
            \Edge[label = 6](a)(b)
            \Edge[label = 2](a)(d)
            \Edge[label = 2](b)(g)
            \Edge[label = 7](e)(a)
            \tikzset{EdgeStyle/.append style = {very thin}}
            \tikzset{EdgeStyle/.append style = {very thick}}
            \Edge[label = 6](c)(b)
            \Edge[label = 2](c)(d)
            \Edge[label = 1](a)(c)
            \Edge[label = 4](e)(c)
            \tikzset{EdgeStyle/.append style = {bend left = 10}}
            \WE(b){a 5}
            \EA(a){b 10}
            \SO(a){c 4}
            \WE(c){e 0}
            \SO(b){d 6}
            % \EA(d){f 11}
            % \EA(b){g 11}
            \end{tikzpicture}}
        \caption{Step 2. Update with $c$.}
        \label{fig:4:b:3}
    \end{figure}
    \begin{figure}[H]
        \resizebox{\linewidth}{!}{\begin{tikzpicture}
            \SetGraphUnit{3}
            \Vertex{a}
            \EA(a){b}
            \SO(a){c}
            \SO(b){d}
            \WE(c){e}
            \EA(d){f}
            \EA(b){g}
            \tikzset{EdgeStyle/.append style = {-}}
            \Edge[label = 3](d)(b)
            \Edge[label = 5](d)(f)
            \Edge[label = 2](b)(g)
            \Edge[label = 6](c)(b)
            \Edge[label = 2](c)(d)
            \Edge[label = 4](e)(c)
            \tikzset{EdgeStyle/.append style = {very thin}}
            \tikzset{EdgeStyle/.append style = {very thick}}
            \Edge[label = 6](a)(b)
            \Edge[label = 2](a)(d)
            \Edge[label = 7](e)(a)
            \Edge[label = 1](a)(c)
            \tikzset{EdgeStyle/.append style = {bend left = 10}}
            \WE(b){a 5}
            \EA(a){b 10}
            \SO(a){c 4}
            \WE(c){e 0}
            \SO(b){d 6}
            % \EA(d){f 11}
            % \EA(b){g 11}
            \end{tikzpicture}}
        \caption{Step 3. Update with $a$.}
        \label{fig:4:b:4}
    \end{figure}
    \begin{figure}[H]
        \resizebox{\linewidth}{!}{\begin{tikzpicture}
            \SetGraphUnit{3}
            \Vertex{a}
            \EA(a){b}
            \SO(a){c}
            \SO(b){d}
            \WE(c){e}
            \EA(d){f}
            \EA(b){g}
            \tikzset{EdgeStyle/.append style = {-}}
            \Edge[label = 2](b)(g)
            \Edge[label = 6](c)(b)
            \Edge[label = 4](e)(c)
            \Edge[label = 6](a)(b)
            \Edge[label = 7](e)(a)
            \Edge[label = 1](a)(c)
            \tikzset{EdgeStyle/.append style = {very thin}}
            \tikzset{EdgeStyle/.append style = {very thick}}
            \Edge[label = 3](d)(b)
            \Edge[label = 5](d)(f)
            \Edge[label = 2](c)(d)
            \Edge[label = 2](a)(d)
            \tikzset{EdgeStyle/.append style = {bend left = 10}}
            \WE(b){a 5}
            \EA(a){b 9}
            \SO(a){c 4}
            \WE(c){e 0}
            \SO(b){d 6}
            \EA(d){f 11}
            % \EA(b){g 11}
            \end{tikzpicture}}
        \caption{Step 4. Update with $d$.}
        \label{fig:4:b:5}
    \end{figure}
    \begin{figure}[H]
        \resizebox{\linewidth}{!}{\begin{tikzpicture}
            \SetGraphUnit{3}
            \Vertex{a}
            \EA(a){b}
            \SO(a){c}
            \SO(b){d}
            \WE(c){e}
            \EA(d){f}
            \EA(b){g}
            \tikzset{EdgeStyle/.append style = {-}}
            \Edge[label = 4](e)(c)
            \Edge[label = 7](e)(a)
            \Edge[label = 1](a)(c)
            \Edge[label = 5](d)(f)
            \Edge[label = 2](c)(d)
            \Edge[label = 2](a)(d)
            \tikzset{EdgeStyle/.append style = {very thin}}
            \tikzset{EdgeStyle/.append style = {very thick}}
            \Edge[label = 2](b)(g)
            \Edge[label = 6](c)(b)
            \Edge[label = 6](a)(b)
            \Edge[label = 3](d)(b)
            \tikzset{EdgeStyle/.append style = {bend left = 10}}
            \WE(b){a 5}
            \EA(a){b 9}
            \SO(a){c 4}
            \WE(c){e 0}
            \SO(b){d 6}
            \EA(d){f 11}
            \EA(b){g 11}
            \end{tikzpicture}}
        \caption{Step 5. Update with $b$.}
        \label{fig:4:b:6}
    \end{figure}
    \begin{figure}[H]
        \resizebox{\linewidth}{!}{\begin{tikzpicture}
            \SetGraphUnit{3}
            \Vertex{a}
            \EA(a){b}
            \SO(a){c}
            \SO(b){d}
            \WE(c){e}
            \EA(d){f}
            \EA(b){g}
            \tikzset{EdgeStyle/.append style = {-}}
            \Edge[label = 4](e)(c)
            \Edge[label = 7](e)(a)
            \Edge[label = 1](a)(c)
            \Edge[label = 2](c)(d)
            \Edge[label = 2](a)(d)
            \Edge[label = 2](b)(g)
            \Edge[label = 6](c)(b)
            \Edge[label = 6](a)(b)
            \Edge[label = 3](d)(b)
            \tikzset{EdgeStyle/.append style = {very thin}}
            \tikzset{EdgeStyle/.append style = {very thick}}
            \Edge[label = 5](d)(f)
            \tikzset{EdgeStyle/.append style = {bend left = 10}}
            \WE(b){a 5}
            \EA(a){b 9}
            \SO(a){c 4}
            \WE(c){e 0}
            \SO(b){d 6}
            \EA(d){f 11}
            \EA(b){g 11}
            \end{tikzpicture}}
        \caption{Step 6. Update with $f$.}
        \label{fig:4:b:7}
    \end{figure}
    \begin{figure}[H]
        \resizebox{\linewidth}{!}{\begin{tikzpicture}
            \SetGraphUnit{3}
            \Vertex{a}
            \EA(a){b}
            \SO(a){c}
            \SO(b){d}
            \WE(c){e}
            \EA(d){f}
            \EA(b){g}
            \tikzset{EdgeStyle/.append style = {-}}
            \Edge[label = 4](e)(c)
            \Edge[label = 7](e)(a)
            \Edge[label = 1](a)(c)
            \Edge[label = 2](c)(d)
            \Edge[label = 2](a)(d)
            \Edge[label = 6](c)(b)
            \Edge[label = 6](a)(b)
            \Edge[label = 3](d)(b)
            \Edge[label = 5](d)(f)
            \tikzset{EdgeStyle/.append style = {very thin}}
            \tikzset{EdgeStyle/.append style = {very thick}}
            \Edge[label = 2](b)(g)
            \tikzset{EdgeStyle/.append style = {bend left = 10}}
            \WE(b){a 5}
            \EA(a){b 9}
            \SO(a){c 4}
            \WE(c){e 0}
            \SO(b){d 6}
            \EA(d){f 11}
            \EA(b){g 11}
            \end{tikzpicture}}
        \caption{Step 7. Update with $g$.}
        \label{fig:4:b:8}
    \end{figure}
    \item Shown as~\Cref{fig:4:c:1,fig:4:c:2,fig:4:c:3,fig:4:c:4,fig:4:c:5,fig:4:c:6}.
    \begin{figure}[H]
        \resizebox{\linewidth}{!}{\begin{tikzpicture}
            \SetGraphUnit{3}
            \Vertex{a}
            \EA(a){b}
            \SO(a){c}
            \SO(b){d}
            \WE(c){e}
            \EA(d){f}
            \EA(b){g}
            \tikzset{EdgeStyle/.append style = {-}}
            % \Edge[label = 4](e)(c)
            % \Edge[label = 7](e)(a)
            \Edge[label = 1](a)(c)
            % \Edge[label = 2](c)(d)
            % \Edge[label = 2](a)(d)
            % \Edge[label = 6](c)(b)
            % \Edge[label = 6](a)(b)
            % \Edge[label = 3](d)(b)
            % \Edge[label = 5](d)(f)
            % \Edge[label = 2](b)(g)
            \end{tikzpicture}}
        \caption{Step 1. Add $ac$.}
        \label{fig:4:c:1}
    \end{figure}
    \begin{figure}[H]
        \resizebox{\linewidth}{!}{\begin{tikzpicture}
            \SetGraphUnit{3}
            \Vertex{a}
            \EA(a){b}
            \SO(a){c}
            \SO(b){d}
            \WE(c){e}
            \EA(d){f}
            \EA(b){g}
            \tikzset{EdgeStyle/.append style = {-}}
            % \Edge[label = 4](e)(c)
            % \Edge[label = 7](e)(a)
            \Edge[label = 1](a)(c)
            % \Edge[label = 2](c)(d)
            \Edge[label = 2](a)(d)
            % \Edge[label = 6](c)(b)
            % \Edge[label = 6](a)(b)
            % \Edge[label = 3](d)(b)
            % \Edge[label = 5](d)(f)
            % \Edge[label = 2](b)(g)
            \end{tikzpicture}}
        \caption{Step 2. Add $ad$.}
        \label{fig:4:c:2}
    \end{figure}
    \begin{figure}[H]
        \resizebox{\linewidth}{!}{\begin{tikzpicture}
            \SetGraphUnit{3}
            \Vertex{a}
            \EA(a){b}
            \SO(a){c}
            \SO(b){d}
            \WE(c){e}
            \EA(d){f}
            \EA(b){g}
            \tikzset{EdgeStyle/.append style = {-}}
            % \Edge[label = 4](e)(c)
            % \Edge[label = 7](e)(a)
            \Edge[label = 1](a)(c)
            % \Edge[label = 2](c)(d)
            \Edge[label = 2](a)(d)
            % \Edge[label = 6](c)(b)
            % \Edge[label = 6](a)(b)
            % \Edge[label = 3](d)(b)
            % \Edge[label = 5](d)(f)
            \Edge[label = 2](b)(g)
            \end{tikzpicture}}
        \caption{Step 3. Add $bg$.}
        \label{fig:4:c:3}
    \end{figure}
    \begin{figure}[H]
        \resizebox{\linewidth}{!}{\begin{tikzpicture}
            \SetGraphUnit{3}
            \Vertex{a}
            \EA(a){b}
            \SO(a){c}
            \SO(b){d}
            \WE(c){e}
            \EA(d){f}
            \EA(b){g}
            \tikzset{EdgeStyle/.append style = {-}}
            % \Edge[label = 4](e)(c)
            % \Edge[label = 7](e)(a)
            \Edge[label = 1](a)(c)
            % \Edge[label = 2](c)(d)
            \Edge[label = 2](a)(d)
            % \Edge[label = 6](c)(b)
            % \Edge[label = 6](a)(b)
            \Edge[label = 3](d)(b)
            % \Edge[label = 5](d)(f)
            \Edge[label = 2](b)(g)
            \end{tikzpicture}}
        \caption{Step 4. Add $bd$.}
        \label{fig:4:c:4}
    \end{figure}
    \begin{figure}[H]
        \resizebox{\linewidth}{!}{\begin{tikzpicture}
            \SetGraphUnit{3}
            \Vertex{a}
            \EA(a){b}
            \SO(a){c}
            \SO(b){d}
            \WE(c){e}
            \EA(d){f}
            \EA(b){g}
            \tikzset{EdgeStyle/.append style = {-}}
            \Edge[label = 4](e)(c)
            % \Edge[label = 7](e)(a)
            \Edge[label = 1](a)(c)
            % \Edge[label = 2](c)(d)
            \Edge[label = 2](a)(d)
            % \Edge[label = 6](c)(b)
            % \Edge[label = 6](a)(b)
            \Edge[label = 3](d)(b)
            % \Edge[label = 5](d)(f)
            \Edge[label = 2](b)(g)
            \end{tikzpicture}}
        \caption{Step 5. Add $ce$.}
        \label{fig:4:c:5}
    \end{figure}
    \begin{figure}[H]
        \resizebox{\linewidth}{!}{\begin{tikzpicture}
            \SetGraphUnit{3}
            \Vertex{a}
            \EA(a){b}
            \SO(a){c}
            \SO(b){d}
            \WE(c){e}
            \EA(d){f}
            \EA(b){g}
            \tikzset{EdgeStyle/.append style = {-}}
            \Edge[label = 4](e)(c)
            % \Edge[label = 7](e)(a)
            \Edge[label = 1](a)(c)
            % \Edge[label = 2](c)(d)
            \Edge[label = 2](a)(d)
            % \Edge[label = 6](c)(b)
            % \Edge[label = 6](a)(b)
            \Edge[label = 3](d)(b)
            \Edge[label = 5](d)(f)
            \Edge[label = 2](b)(g)
            \end{tikzpicture}}
        \caption{Step 6. Add $df$.}
        \label{fig:4:c:6}
    \end{figure}
    \item Shown as~\Cref{fig:4:d:1,fig:4:d:2,fig:4:d:3,fig:4:d:4,fig:4:d:5,fig:4:d:6}.
    \begin{figure}[H]
        \resizebox{\linewidth}{!}{\begin{tikzpicture}
            \SetGraphUnit{3}
            \Vertex{a}
            \EA(a){b}
            \SO(a){c}
            \SO(b){d}
            \WE(c){e}
            \EA(d){f}
            \EA(b){g}
            \tikzset{EdgeStyle/.append style = {-}}
            % \Edge[label = 4](e)(c)
            % \Edge[label = 7](e)(a)
            \Edge[label = 1](a)(c)
            % \Edge[label = 2](c)(d)
            % \Edge[label = 2](a)(d)
            % \Edge[label = 6](c)(b)
            % \Edge[label = 6](a)(b)
            % \Edge[label = 3](d)(b)
            % \Edge[label = 5](d)(f)
            % \Edge[label = 2](b)(g)
            \end{tikzpicture}}
        \caption{Step 1. Add $ac$.}
        \label{fig:4:d:1}
    \end{figure}
    \begin{figure}[H]
        \resizebox{\linewidth}{!}{\begin{tikzpicture}
            \SetGraphUnit{3}
            \Vertex{a}
            \EA(a){b}
            \SO(a){c}
            \SO(b){d}
            \WE(c){e}
            \EA(d){f}
            \EA(b){g}
            \tikzset{EdgeStyle/.append style = {-}}
            % \Edge[label = 4](e)(c)
            % \Edge[label = 7](e)(a)
            \Edge[label = 1](a)(c)
            % \Edge[label = 2](c)(d)
            \Edge[label = 2](a)(d)
            % \Edge[label = 6](c)(b)
            % \Edge[label = 6](a)(b)
            % \Edge[label = 3](d)(b)
            % \Edge[label = 5](d)(f)
            % \Edge[label = 2](b)(g)
            \end{tikzpicture}}
        \caption{Step 2. Add $ad$.}
        \label{fig:4:d:2}
    \end{figure}
    \begin{figure}[H]
        \resizebox{\linewidth}{!}{\begin{tikzpicture}
            \SetGraphUnit{3}
            \Vertex{a}
            \EA(a){b}
            \SO(a){c}
            \SO(b){d}
            \WE(c){e}
            \EA(d){f}
            \EA(b){g}
            \tikzset{EdgeStyle/.append style = {-}}
            % \Edge[label = 4](e)(c)
            % \Edge[label = 7](e)(a)
            \Edge[label = 1](a)(c)
            % \Edge[label = 2](c)(d)
            \Edge[label = 2](a)(d)
            % \Edge[label = 6](c)(b)
            % \Edge[label = 6](a)(b)
            \Edge[label = 3](d)(b)
            % \Edge[label = 5](d)(f)
            % \Edge[label = 2](b)(g)
            \end{tikzpicture}}
        \caption{Step 3. Add $bd$.}
        \label{fig:4:d:3}
    \end{figure}
    \begin{figure}[H]
        \resizebox{\linewidth}{!}{\begin{tikzpicture}
            \SetGraphUnit{3}
            \Vertex{a}
            \EA(a){b}
            \SO(a){c}
            \SO(b){d}
            \WE(c){e}
            \EA(d){f}
            \EA(b){g}
            \tikzset{EdgeStyle/.append style = {-}}
            % \Edge[label = 4](e)(c)
            % \Edge[label = 7](e)(a)
            \Edge[label = 1](a)(c)
            % \Edge[label = 2](c)(d)
            \Edge[label = 2](a)(d)
            % \Edge[label = 6](c)(b)
            % \Edge[label = 6](a)(b)
            \Edge[label = 3](d)(b)
            % \Edge[label = 5](d)(f)
            \Edge[label = 2](b)(g)
            \end{tikzpicture}}
        \caption{Step 4. Add $bg$.}
        \label{fig:4:d:4}
    \end{figure}
    \begin{figure}[H]
        \resizebox{\linewidth}{!}{\begin{tikzpicture}
            \SetGraphUnit{3}
            \Vertex{a}
            \EA(a){b}
            \SO(a){c}
            \SO(b){d}
            \WE(c){e}
            \EA(d){f}
            \EA(b){g}
            \tikzset{EdgeStyle/.append style = {-}}
            \Edge[label = 4](e)(c)
            % \Edge[label = 7](e)(a)
            \Edge[label = 1](a)(c)
            % \Edge[label = 2](c)(d)
            \Edge[label = 2](a)(d)
            % \Edge[label = 6](c)(b)
            % \Edge[label = 6](a)(b)
            \Edge[label = 3](d)(b)
            % \Edge[label = 5](d)(f)
            \Edge[label = 2](b)(g)
            \end{tikzpicture}}
        \caption{Step 5. Add $ce$.}
        \label{fig:4:d:5}
    \end{figure}
    \begin{figure}[H]
        \resizebox{\linewidth}{!}{\begin{tikzpicture}
            \SetGraphUnit{3}
            \Vertex{a}
            \EA(a){b}
            \SO(a){c}
            \SO(b){d}
            \WE(c){e}
            \EA(d){f}
            \EA(b){g}
            \tikzset{EdgeStyle/.append style = {-}}
            \Edge[label = 4](e)(c)
            % \Edge[label = 7](e)(a)
            \Edge[label = 1](a)(c)
            % \Edge[label = 2](c)(d)
            \Edge[label = 2](a)(d)
            % \Edge[label = 6](c)(b)
            % \Edge[label = 6](a)(b)
            \Edge[label = 3](d)(b)
            \Edge[label = 5](d)(f)
            \Edge[label = 2](b)(g)
            \end{tikzpicture}}
        \caption{Step 6. Add $df$.}
        \label{fig:4:d:6}
    \end{figure}
\end{enumerate}

\section{Network Flow}
Let the five vertices be a(1, 3), b(1, 3), c(2, 1), d(2, 1), e(3, 1).

\textbf{Solution I.}
If self loop is allowed,
we construct a network flow graph as shown in~\Cref{fig:5:1:1},
where the supply of vertex s is 9 and the demand of vertex t is also 9.
The directed graph $G$ can be constructed if the supply and demand can be satisfied.
\begin{figure}[H]
    \resizebox{\linewidth}{!}{\begin{tikzpicture}
        \SetGraphUnit{3}
        \Vertex{s}
        \EA(s){cs}
        \NO(cs){bs}
        \NO(bs){as}
        \SO(cs){ds}
        \SO(ds){es}
        \EA(cs){ct}
        \NO(ct){bt}
        \NO(bt){at}
        \SO(ct){dt}
        \SO(dt){et}
        \EA(ct){t}
        % \tikzset{EdgeStyle/.append style = {-}}
        \tikzset{EdgeStyle/.append style = {bend left = 20}}
        \Edge[label = 1](s)(as)
        \Edge[label = 1](s)(bs)
        \Edge[label = 2](s)(cs)
        \Edge[label = 2](s)(ds)
        \Edge[label = 3](s)(es)
        \Edge[label = 1](as)(bt)
        \Edge[label = 1](as)(ct)
        \Edge[label = 1](as)(dt)
        \Edge[label = 1](as)(et)
        \Edge[label = 1](bs)(at)
        \Edge[label = 1](bs)(ct)
        \Edge[label = 1](bs)(dt)
        \Edge[label = 1](bs)(et)
        \Edge[label = 1](cs)(at)
        \Edge[label = 1](cs)(bt)
        \Edge[label = 1](cs)(dt)
        \Edge[label = 1](cs)(et)
        \Edge[label = 1](ds)(at)
        \Edge[label = 1](ds)(bt)
        \Edge[label = 1](ds)(ct)
        \Edge[label = 1](ds)(et)
        \Edge[label = 1](es)(at)
        \Edge[label = 1](es)(bt)
        \Edge[label = 1](es)(ct)
        \Edge[label = 1](es)(dt)
        \Edge[label = 3](at)(t)
        \Edge[label = 3](bt)(t)
        \Edge[label = 1](ct)(t)
        \Edge[label = 1](dt)(t)
        \Edge[label = 1](et)(t)
        \end{tikzpicture}}
    \caption{Network Flow Graph.}
    \label{fig:5:1:1}
\end{figure}
One solution of the network flow problem is shown as~\Cref{fig:5:1:2}.
\begin{figure}[H]
    \resizebox{\linewidth}{!}{\begin{tikzpicture}
        \SetGraphUnit{3}
        \Vertex{s}
        \EA(s){cs}
        \NO(cs){bs}
        \NO(bs){as}
        \SO(cs){ds}
        \SO(ds){es}
        \EA(cs){ct}
        \NO(ct){bt}
        \NO(bt){at}
        \SO(ct){dt}
        \SO(dt){et}
        \EA(ct){t}
        % \tikzset{EdgeStyle/.append style = {-}}
        \tikzset{EdgeStyle/.append style = {bend left = 20}}
        \Edge[label = 1/1](s)(as)
        \Edge[label = 1/1](s)(bs)
        \Edge[label = 2/2](s)(cs)
        \Edge[label = 2/2](s)(ds)
        \Edge[label = 3/3](s)(es)
        \Edge[label = 1](as)(bt)
        \Edge[label = 1](as)(ct)
        \Edge[label = 1/1](as)(dt)
        \Edge[label = 1](as)(et)
        \Edge[label = 1](bs)(at)
        \Edge[label = 1](bs)(ct)
        \Edge[label = 1](bs)(dt)
        \Edge[label = 1/1](bs)(et)
        \Edge[label = 1/1](cs)(at)
        \Edge[label = 1/1](cs)(bt)
        \Edge[label = 1](cs)(dt)
        \Edge[label = 1](cs)(et)
        \Edge[label = 1/1](ds)(at)
        \Edge[label = 1/1](ds)(bt)
        \Edge[label = 1](ds)(ct)
        \Edge[label = 1](ds)(et)
        \Edge[label = 1/1](es)(at)
        \Edge[label = 1/1](es)(bt)
        \Edge[label = 1/1](es)(ct)
        \Edge[label = 1](es)(dt)
        \Edge[label = 3/3](at)(t)
        \Edge[label = 3/3](bt)(t)
        \Edge[label = 1/1](ct)(t)
        \Edge[label = 1/1](dt)(t)
        \Edge[label = 1/1](et)(t)
        \end{tikzpicture}}
    \caption{Network Flow Graph.}
    \label{fig:5:1:2}
\end{figure}
The corresponding constructed graph is shown as~\Cref{fig:5:1:3}.
\begin{figure}[H]
    \centering
    \resizebox{.4\linewidth}{!}{\begin{tikzpicture}
        \SetGraphUnit{3}
        \Vertex{a}
        \SO(a){b}
        \WE(a){c}
        \SO(c){d}
        \SO(d){e}
        \tikzset{EdgeStyle/.append style = {bend left = 20}}
        \Edge(a)(d)
        \Edge(b)(e)
        \Edge(c)(a)
        \Edge(c)(b)
        \Edge(d)(a)
        \Edge(d)(b)
        \Edge(e)(a)
        \Edge(e)(b)
        \Edge(e)(c)
        \end{tikzpicture}}
    \caption{Graph Solution.}
    \label{fig:5:1:3}
\end{figure}
\textbf{Solution II.}
If self loop is allowed,
observed that the incoming and outgoing degree of all the vertices are larger than
or equal to one, for each vertex,
we construct a edge whose source and target are the same vertex.
Therefore, the remaining incoming and outgoing degree is (0, 2), (0, 2), (1, 0), (1, 0),
(2, 0), respectively.
We construct a network flow graph as shown in~\Cref{fig:5:2:1},
where the supply of vertex s is 4 and the demand of vertex t is also 4.
The directed graph $G$ can be constructed if the supply and demand can be satisfied.
\begin{figure}[H]
    \resizebox{\linewidth}{!}{\begin{tikzpicture}
        \SetGraphUnit{3}
        \Vertex{s}
        \EA(s){d}
        % \NO(cs){bs}
        % \NO(bs){as}
        \NO(d){c}
        \SO(d){e}
        \EA(c){a}
        % \NO(ct){bt}
        \SO(a){b}
        % \SO(ct){dt}
        % \SO(dt){et}
        \EA(b){t}
        % \tikzset{EdgeStyle/.append style = {-}}
        \tikzset{EdgeStyle/.append style = {bend left = 20}}
        % \Edge[label = 1](s)(as)
        % \Edge[label = 1](s)(bs)
        \Edge[label = 1](s)(c)
        \Edge[label = 1](s)(d)
        \Edge[label = 2](s)(e)
        % \Edge[label = 1](as)(bt)
        % \Edge[label = 1](as)(ct)
        % \Edge[label = 1](as)(dt)
        % \Edge[label = 1](as)(et)
        % \Edge[label = 1](bs)(at)
        % \Edge[label = 1](bs)(ct)
        % \Edge[label = 1](bs)(dt)
        % \Edge[label = 1](bs)(et)
        \Edge[label = 1](c)(a)
        \Edge[label = 1](c)(b)
        % \Edge[label = 1](cs)(dt)
        % \Edge[label = 1](cs)(et)
        \Edge[label = 1](d)(a)
        \Edge[label = 1](d)(b)
        % \Edge[label = 1](ds)(ct)
        % \Edge[label = 1](ds)(et)
        \Edge[label = 1](e)(a)
        \Edge[label = 1](e)(b)
        \Edge[label = 2](a)(t)
        \Edge[label = 2](b)(t)
        \end{tikzpicture}}
    \caption{Network Flow Graph.}
    \label{fig:5:2:1}
\end{figure}
One solution of the network flow problem is shown as~\Cref{fig:5:2:2}.
\begin{figure}[H]
    \resizebox{\linewidth}{!}{\begin{tikzpicture}
        \SetGraphUnit{3}
        \Vertex{s}
        \EA(s){d}
        % \NO(cs){bs}
        % \NO(bs){as}
        \NO(d){c}
        \SO(d){e}
        \EA(c){a}
        % \NO(ct){bt}
        \SO(a){b}
        % \SO(ct){dt}
        % \SO(dt){et}
        \EA(b){t}
        % \tikzset{EdgeStyle/.append style = {-}}
        \tikzset{EdgeStyle/.append style = {bend left = 20}}
        % \Edge[label = 1](s)(as)
        % \Edge[label = 1](s)(bs)
        \Edge[label = 1/1](s)(c)
        \Edge[label = 1/1](s)(d)
        \Edge[label = 2/2](s)(e)
        % \Edge[label = 1](as)(bt)
        % \Edge[label = 1](as)(ct)
        % \Edge[label = 1](as)(dt)
        % \Edge[label = 1](as)(et)
        % \Edge[label = 1](bs)(at)
        % \Edge[label = 1](bs)(ct)
        % \Edge[label = 1](bs)(dt)
        % \Edge[label = 1](bs)(et)
        \Edge[label = 1/1](c)(a)
        \Edge[label = 1](c)(b)
        % \Edge[label = 1](cs)(dt)
        % \Edge[label = 1](cs)(et)
        \Edge[label = 1](d)(a)
        \Edge[label = 1/1](d)(b)
        % \Edge[label = 1](ds)(ct)
        % \Edge[label = 1](ds)(et)
        \Edge[label = 1/1](e)(a)
        \Edge[label = 1/1](e)(b)
        \Edge[label = 2/2](a)(t)
        \Edge[label = 2/2](b)(t)
        \end{tikzpicture}}
    \caption{Network Flow Solution.}
    \label{fig:5:2:2}
\end{figure}
The corresponding constructed graph is shown as~\Cref{fig:5:2:3}.
\begin{figure}[H]
    \centering
    \resizebox{.7\linewidth}{!}{\begin{tikzpicture}
        \SetGraphUnit{3}
        \Vertex{a}
        \SO(a){b}
        \WE(a){c}
        \SO(c){d}
        \SO(d){e}
        \tikzset{EdgeStyle/.append style = {bend left = 20}}
        \Edge(c)(a)
        \Edge(d)(b)
        \Edge(e)(a)
        \Edge(e)(b)
        \Loop[dir=SO,dist=2cm,labelstyle={right,color=white}](a.east)
        \Loop[dir=SO,dist=2cm,labelstyle={right,color=white}](b.east)
        \Loop[dir=NO,dist=2cm,labelstyle={left,color=white}](c.west)
        \Loop[dir=NO,dist=2cm,labelstyle={left,color=white}](d.west)
        \Loop[dir=NO,dist=2cm,labelstyle={left,color=white}](e.west)
        \end{tikzpicture}}
    \caption{Graph Solution.}
    \label{fig:5:2:3}
\end{figure}

\section{Maximum Sub-array}
\textbf{Solution.} The complexity of~\Cref{a:5} is $O(n)$.
\begin{algorithm}[H]
    \caption{Maximum Sub-array}
    \label{a:5}
    \begin{algorithmic}
        \Procedure{Kadane}{$A$, $n$}
        \State $max\_ending\_here \coloneqq 0$
        \State $max \coloneqq 0$
        \State $s \coloneqq 1$
        \State $t \coloneqq 0$
        \For{$i \coloneqq 1, \ldots n$}
        \If{$max\_ending\_here + A[i] \le 0$}
        \State $max\_ending\_here \coloneqq 0$
        \State $s \coloneqq i + 1$
        \State \textbf{continue}
        \EndIf
        \State $max\_ending\_here \coloneqq max\_ending\_here + A[i]$
        \If{$max\_ending\_here > max$}
        \State $max \coloneqq max\_ending\_here$
        \State $t \coloneqq i$
        \EndIf
        \EndFor
        \State \Return $s, t$
        \EndProcedure
    \end{algorithmic}
\end{algorithm}

\end{document}
