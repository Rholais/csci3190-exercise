%!TEX program = xelatex
\documentclass[10pt, compress, handout]{beamer}
\usepackage[titleprogressbar]{../../cls/beamerthemem}

\setbeamertemplate{caption}[numbered]
\setbeamertemplate{theorems}[numbered]
\newcounter{example}
\resetcounteronoverlays{example}
\newtheorem{crl}{Corollary}[theorem]
\newtheorem{eg}[example]{Example}
\newtheorem*{solution*}{Solution}

\usepackage{booktabs}
\usepackage[scale=2]{ccicons}
\usepackage{minted}

\usepackage{cleveref}
\crefname{example}{Example}{Examples}

\usepgfplotslibrary{dateplot}

\usemintedstyle{trac}

\usepackage{algorithm}
\usepackage[noend]{algpseudocode}
\resetcounteronoverlays{algorithm}

\usepackage{version}
% \excludeversion{proof}
% \excludeversion{solution*}

\usepackage{mathtools}
\usepackage{multicol}
\usepackage{qtree}

\usepackage{tikz}

\makeatletter
\def\old@comma{,}
\catcode`\,=13
\def,{%
    \ifmmode%
    \old@comma\discretionary{}{}{}%
    \else%
    \old@comma%
    \fi%
}
\makeatother

\title{CSCI 3190 Tutorial of Week 13}
\subtitle{Tree}
\author{LI Haocheng}
\institute{Department of Computer Science and Engineering}

\begin{document}

\maketitle

\begin{frame}[fragile]
\frametitle{Distributive Property}
\begin{eg}
    Prove that \begin{enumerate}
        \item $(A - B) - (A - C) \equiv A \cap (C - B)$;
        \item $(A - C) - (B - C) \equiv (A - B) - C$.
    \end{enumerate}
\end{eg}
\begin{proof}
    \begin{columns}
        \onslide<2->\begin{column}{.4\linewidth}
            \begin{align}
                \begin{aligned}
                    & (A - B) - (A - C)\\
                    \equiv & (A \cap \overline{B}) \cap \overline{A \cap \overline{C}}\\
                    \equiv & A \cap \overline{B} \cap (\overline{A} \cup C)\\
                    \equiv & (A \cap \overline{A} \cap \overline{B}) \cup (A \cap \overline{B} \cap C)\\
                    \equiv & A \cap (C - B).
                \end{aligned}
            \end{align}
        \end{column}
        \onslide<3>\begin{column}{.4\linewidth}
            \begin{align}
                \begin{aligned}
                    & (A - C) - (B - C)\\
                    \equiv & A \cap \overline{C} \cap \overline{B \cap \overline{C}}\\
                    \equiv & A \cap \overline{C} \cap (\overline{B} \cup C)\\
                    \equiv & A \cap \overline{B} \cap \overline{C}\\
                    \equiv & (A - B) - C.
                \end{aligned}
            \end{align}
        \end{column}
    \end{columns}
\end{proof}
\end{frame}

\begin{frame}[fragile]
\frametitle{Propositional Satisfiability}
\begin{eg}
    Determine whether each of the compound propositions
    $(p \lor \neg q) \land (q \lor \neg r) \land (r \lor \neg p) \land (p \lor q \lor r)$
    $\land (\neg p \lor \neg q \lor \neg r)$ is satisfiable.
\end{eg}
\onslide<2>\begin{solution*}
    Note that for
    $(p \lor \neg q) \land (q \lor \neg r) \land (r \lor \neg p) \land (p \lor q \lor r)$
    $\land (\neg p \lor \neg q \lor \neg r)$ to be true,
    $(p \lor \neg q) \land (q \lor \neg r) \land (r \lor \neg p)$ and
    $(p \lor q \lor r) \land (\neg p \lor \neg q \lor \neg r)$ must both be true.
    For the first to be true, the three variables must have the same truth values,
    and for the second to be true,
    at least one of three variables must be true and at least one must be false.
    However, these conditions are contradictory.
    From these observations we conclude that no assignment of truth values to $p$, $q$,
    and $r$ makes
    $(p \lor \neg q) \land (q \lor \neg r) \land (r \lor \neg p) \land (p \lor q \lor r)$
    $\land (\neg p \lor \neg q \lor \neg r)$ true. Hence, it is unsatisfiable.
\end{solution*}
\end{frame}

\begin{frame}[fragile]
\frametitle{Quantifiers}
\begin{columns}
    \begin{column}{.7\linewidth}
        \begin{eg}
            Determine the truth value of the followings, where $x, y \in \mathbb{I}^+$.
            \begin{enumerate}
                \item $\forall x \forall y (2x^2 < y + 1)$
                \item $\forall x \exists y (2x^2 < y + 1)$
                \item $\exists x \forall y (2x^2 < y + 1)$
                \item $\exists x \exists y (2x^2 < y + 1)$
            \end{enumerate}
        \end{eg}
        \begin{solution*}
            \begin{enumerate}
                \item<2-> False. Let $x = 1, y = 1$, then $2x^2 = y + 1$.
                \item<3-> True. $\forall x \exists y = 2x^2$ s.t. $2x^2 < y + 1$.
                \item<4-> False. Let $y = 1, 2x^2 < y + 1$, then $x < 1$.
                \item<5-> True. Let $x = 1, y = 2$, then $2x^2 < y + 1$.
            \end{enumerate}
        \end{solution*}
    \end{column}
    \begin{column}{.4\linewidth}
        \begin{eg}
            Can you conclude that $A = B$ if $A, B$ and $C$ are sets such that: \begin{enumerate}
                \item $A \cup C = B \cup C$;
                \item $A \cap C = B \cap C$.
            \end{enumerate}
        \end{eg}
        \begin{solution*}
            \begin{enumerate}
                \item<6-> No. Let $A = \emptyset, B = C \supset A$, then $A \cup C = B \cup C$.
                \item<7-> No. Let $A \ne B, C = \emptyset$, then $A \cap C = B \cap C$.
            \end{enumerate}
        \end{solution*}
    \end{column}
\end{columns}
\end{frame}

\begin{frame}[fragile]
\frametitle{Equivalence Class}
\begin{eg}
    Find smallest equivalence relation on set $\{a, b, c, d, e\}$ containing relation $\{(a, b), (b, c), (d, c)\}$.
\end{eg}
\begin{solution*}
    \begin{enumerate}
        \item<2-> Find Equivalence Classes: $[a] = \{a, b, c, d\}, [e] = \{e\}$.
        \item<3> Find Equivalence Relation: $\{(a, a), (a, b), (a, c), (a, d), (b, a), (b, b), (b, c), (b, d), (c, a),$
        $(c, b), (c, c), (c, d), (d, a), (d, b), (d, c), (d, d), (e, e)\}$
    \end{enumerate}
\end{solution*}
\end{frame}

\begin{frame}
\frametitle{Equivalence Relation}
\begin{eg}
    Consider the relation $R$ over the set of all positive integers $\mathbb{I}^+$:    $R = \{(a, b) \mid |a - b| \equiv 0 \pmod{2}\}$
    \begin{enumerate}
        \item Is $R$ an equivalence relation? Explain your answer.
        \item How many equivalence classes of $R$ in $\mathbb{I}^+$ are there? What are they?
    \end{enumerate}
\end{eg}
\begin{solution*}
    \begin{enumerate}
        \item<2-> Yes. \begin{itemize}
            \item $\forall a \in \mathbb{I}^+, |a - a| = 0$.
            \item $|b - a| = |a - b|$.
            \item $|a - c| \equiv a - c \equiv (a - b) + (b - c) \equiv |a - b| + |b - c| \pmod{2}$.
        \end{itemize}
        
        \item<3-> \begin{itemize}
            \item 2.
            \item $[1] = \{a \mid a \equiv 1 \pmod{2}\}$, $[2] = \{a \mid a \equiv 0 \pmod{2}\}$.
        \end{itemize}
    \end{enumerate}
\end{solution*}
\end{frame}

\begin{frame}[fragile]
\frametitle{Relation}
\begin{columns}
    \begin{column}{.6\linewidth}
        \begin{eg}
            \begin{enumerate}
                \item Let $A = \{1, 2, 3, 4, 5\}$, $R$ is a relation on $A \times A$ such that $((a, b), (c, d)) \in R$ if and only if $a - d = c - b$. Show that $R$ is an equivalence relation.
                \item How many equivalence classes are there in $R$?
            \end{enumerate}
        \end{eg}
        \begin{proof}
            \begin{enumerate}
                \item<2-> \begin{itemize}
                    \item $a - b = a - b$.
                    \item $a - d = c - b \Leftrightarrow c - b = a - d$.
                    \item $a - d = c - b, c - f = e - d \Rightarrow a - f = e - b$.
                \end{itemize}
                \item<3-> 9.
            \end{enumerate}
        \end{proof}
    \end{column}
    \begin{column}{.5\linewidth}
        \onslide<1->\begin{eg}
            Let $n \in \mathbb{Z}^+$, let $U = \{1, 2, \cdots, n\}$. Define the relation $R$ on the power set of $U$ by $(A, B) \in R$ if and only if $A \not\subset B$ and $B \not\subset A$. Is $R$ an equivalence relation? What is $|R|$?
        \end{eg}
        \begin{solution*}
            \begin{enumerate}
                \item<4->. $R$ is not an equivalence relation. Let $n = 3, A = \{1\}, B = \{2\}$, $C = \{1, 3\}$ so that $(A, B) \in R, (B, C) \in R, (A, C) \notin R$.
                \item<5> $|R| = 2^{2n} - 2(3^n - 2^n)$.
            \end{enumerate}
        \end{solution*}
    \end{column}
\end{columns}
\end{frame}

\begin{frame}[fragile]
\frametitle{Pigeonhole Principle}
\begin{columns}
    \begin{column}{.6\linewidth}
        \begin{eg}
            Let $S$ be a set of 7 positive integers the maximum of which is at most 24. Prove that the sums of the elements in all the nonempty subsets of $S$ cannot be distinct.
        \end{eg}
        \onslide<2->\begin{proof}
            Consider nonempty subsets of size at most 5. Their maximum sum is at most $20 + 21 + \cdots + 24 = 110$. There are $126 - 7 = 119$ nonempty subsets with at most 5 elements so at least 2 of them have equal sum.
        \end{proof}
    \end{column}
    \begin{column}{.5\linewidth}
        \onslide<1->\begin{eg}
            Given a closed form expression for the generating function of the following sequence:\begin{equation}
            1, -2, 3, -4, 5, -6, \cdots
            \end{equation}
        \end{eg}
        \onslide<3>\begin{solution*}
            \begin{align}
            a_r & = -1^{r + 1} & \leftrightarrow & -\frac{1}{1 + x},\\
            b_r & = -1^r (r + 1) & \leftrightarrow & \frac{1}{(1 + x)^2}.
            \end{align}
        \end{solution*}
    \end{column}
\end{columns}
\end{frame}

\begin{frame}[fragile]
\frametitle{Complexity}
\onslide<1->\begin{eg}
    Show the following:\begin{enumerate}[(a)]
        \item $3n^2 + 5n + 10 = O(n^2)$
    \end{enumerate}
\end{eg}
\onslide<2>\begin{proof}
    \begin{enumerate}[(a)]
        \item \begin{description}
            \item[Method 1] $\exists c = 4, \exists N = 6$, such that $\forall n > N, 3n^2 + 5n + 10 < 4n^2$.
            \item[Method 2] Since $3 + \frac{5}{n} + \frac{10}{n^2} = O(1)$ and $n^2 = O(n^2)$, $3n^2 + 5n + 10 = O(n^2)$.
        \end{description}
    \end{enumerate}
\end{proof}
\end{frame}

\begin{frame}[fragile]
\frametitle{Terminology}
\begin{columns}
    \begin{column}{.6\linewidth}
        \onslide<1->\begin{eg}
            In the rooted tree $T$ (with root $a$) shown in Figure \ref{f-11-1-5}, find the parent of $c$, the children of $g$, the
            siblings of $h$, all ancestors of $e$, all descendants of $b$, all internal vertices, and all leaves. What
            is the subtree rooted at $g$?
        \end{eg}
        \onslide<2>\begin{solution*}
            The parent of $c$ is $b$. The children of $g$ are $h$, $i$, and $j$. The siblings of $h$ are $i$ and $j$. The ancestors of $e$ are $c$, $b$, and $a$. The descendants of $b$ are $c$, $d$, and $e$. The internal vertices are $a$, $b$, $c$, $g$, $h$, and $j$. The leaves are $d$, $e$, $f$, $i$, $k$, $l$, and $m$.
        \end{solution*}
    \end{column}
    \onslide<1->\begin{column}{.4\linewidth}
        \begin{figure}
            \centering
            $\Tree [.a [.b [.c d e ] ] f [.g [.h k ] i [.j l m ]]]$
            \caption{A Rooted Tree $T$}
            \label{f-11-1-5}
        \end{figure}
    \end{column}
\end{columns}
\end{frame}

\plain{Questions?}

\end{document}
