%!TEX program = xelatex
\documentclass[10pt, compress, handout]{beamer}
\usepackage[titleprogressbar]{../../cls/beamerthemem}

\usepackage{booktabs}
\usepackage[scale=2]{ccicons}
\usepackage{minted}

\usepgfplotslibrary{dateplot}

\usemintedstyle{trac}

\setbeamertemplate{caption}[numbered]
\setbeamertemplate{theorems}[numbered]
\newtheorem{crl}{Corollary}[theorem]
\newtheorem*{solution*}{Solution}

\usepackage{algorithm}
\usepackage[noend]{algpseudocode}

\usepackage{version}
%\excludeversion{proof}
%\excludeversion{solution*}

\usepackage{mathtools}
\usepackage{multicol}
\usepackage{qtree}

\usepackage{tikz}

\makeatletter
\def\old@comma{,}
\catcode`\,=13
\def,{%
    \ifmmode%
    \old@comma\discretionary{}{}{}%
    \else%
    \old@comma%
    \fi%
}
\makeatother

\title{CSCI 3190 Tutorial of Week 12}
\subtitle{Trees}
\author{LI Haocheng}
\institute{Department of Computer Science and Engineering}

\begin{document}

\maketitle

\begin{frame}[fragile]
\frametitle{Left Child}
\begin{columns}
    \begin{column}{.6\linewidth}
        \onslide<1->\begin{example}
            What are the left and right children of $d$ in the binary tree $T$ shown in Figure \ref{f-11-1-8} (where the order is that implied by the drawing)? What are the left and right subtrees of $c$?
        \end{example}
        \onslide<2>\begin{solution*}
            The left child of $d$ is $f$ and the right child is $g$.
        \end{solution*}
    \end{column}
    \onslide<1->\begin{column}{.4\linewidth}
        \begin{figure}
            \centering
            $\Tree [.a [.b [.d f g ] e ] [.c [.h j ] [.i k [.l m ]]]]$
            \caption{A Binary Tree $T$}
            \label{f-11-1-8}
        \end{figure}
    \end{column}
\end{columns}
\end{frame}

\begin{frame}[fragile]
    \frametitle{$m$-ary Tree}
    \onslide<1->\begin{theorem}\label{t-11-1-3}
        A full $m$-ary tree with $i$ internal vertices contains $n = m i + 1$ vertices.
    \end{theorem}
    \onslide<2>\begin{proof}
        Every vertex, except the root, is the child of an internal vertex. Because each of the $i$ internal vertices has $m$ children, there are $m i$ vertices in the tree other than the root. Therefore, the tree contains $n = m i + 1$ vertices
    \end{proof}
\end{frame}

\begin{frame}[fragile]
\frametitle{Full $m$-ary Tree}
\onslide<1->\begin{theorem}\label{t-11-1-4}
    A full $m$-ary tree with \begin{enumerate}
        \item $n$ vertices has $i = \frac{n - 1}{m}$ internal vertices and $l =\frac{(m - 1)n + 1}{m}$ leaves,
        \item $i$ internal vertices has $n = mi + 1$ vertices and $l = (m - 1) i + 1$ leaves,
        \item $l$ leaves has $n = \frac{m l - 1}{m - 1}$ vertices and $i = \frac{l - 1}{m - 1}$ internal vertices.
    \end{enumerate}
\end{theorem}
\onslide<2>\begin{proof}
    The three parts of the theorem can all be proved using the equality given in Theorem \ref{t-11-1-3}, that is, $n = m i + 1$, together with the equality $n = l + i$, which is true because each vertex is either a leaf or an internal vertex. 
    
    For example, solving for $i$ in $n = m i + 1$ gives $i = \frac{n - 1}{m}$. Then inserting this expression for $i$ into the equation $n = l + i$ shows that $l = n - i = n - \frac{n - 1}{m} = \frac{(m - 1)n + 1}{m}$.
\end{proof}
\end{frame}

\begin{frame}[fragile]
\frametitle{Preorder}
\begin{columns}
    \begin{column}{.6\linewidth}
        \onslide<1->\begin{example}
            Determine the orders in which a preorder, a inorder and a postorder traversal visits the vertices of the given ordered rooted tree in Figure \ref{f-11-3-e7}.
        \end{example}
        \begin{solution*}
            \begin{enumerate}
                \item<2-> $a, b, d, e, f, g, c$.
                \item<3-> $d, b, f, e, g, a, c$.
                \item<4-> $d, f, g, e, b, c, a$.
            \end{enumerate}
        \end{solution*}
    \end{column}
    \onslide<1->\begin{column}{.4\linewidth}
        \begin{figure}
            \centering
            $\Tree [.a [.b d [.e f g ]] c ]$
            \caption{A Rooted Tree $T$}
            \label{f-11-3-e7}
        \end{figure}
    \end{column}
\end{columns}
\end{frame}

\begin{frame}[fragile]
\frametitle{Expression}
\begin{columns}
    \begin{column}{.6\linewidth}
        \onslide<1->\begin{example}
            \begin{enumerate}
                \item Represent the expression $((x + 2) \uparrow 3) \cdot (y - (3 + x)) - 5$ using a binary tree.
                \item Write this expression in prefix notation.
                \item Write this expression in postfix notation.
                \item Write this expression in infix notation.
            \end{enumerate}
        \end{example}
        \begin{solution*}
            \begin{enumerate}
                \item<3-> $-\ \cdot\ \uparrow\ +\ x\ 2\ 3\ -\ y\ +\ 3\ x\ 5$
                \item<4-> $x\ 2\ +\ 3\ \uparrow\ y\ 3\ x\ +\ -\ \cdot\ 5\ -$
                \item<5-> $x\ +\ 2\ \uparrow\ 3\ \cdot\ y\ -\ 3\ +\ x\ -\ 5$
            \end{enumerate}
        \end{solution*}
    \end{column}
    \onslide<2->\begin{column}{.4\linewidth}
        \begin{solution*}
            \begin{figure}
                \centering
                $\Tree [.- [.\(\cdot\) [.\(\uparrow\) [.+ x 2 ] 3 ] [.- y [.+ 3 x ] ] ] 5 ]$
                \caption{A Rooted Tree $T$}
                \label{f-11-3-e16}
            \end{figure}
        \end{solution*}
    \end{column}
\end{columns}
\end{frame}

\begin{frame}[fragile]
    \frametitle{Construction}
    \begin{columns}
        \begin{column}{.6\linewidth}
            \onslide<1->\begin{example}
                Construct the ordered rooted tree whose preorder traversal is $a, b, f, c, g, h, i, d, e, j, k, l$, where $a$ has four children, $c$ has three children, $j$ has two children, $b$ and $e$ have one child each, and all other vertices are leaves.
            \end{example}
        \end{column}
        \onslide<2>\begin{column}{.4\linewidth}
            \begin{solution*}
                \begin{figure}
                    \centering
                    $\Tree [.a [.b f ] [.c g h i ] d [.e [.j k l ] ] ]$
                    \caption{A Rooted Tree $T$}
                    \label{f-11-3-e25}
                \end{figure}
            \end{solution*}
        \end{column}
    \end{columns}
\end{frame}

\begin{frame}[allowframebreaks]
\frametitle{Counting}
\begin{example}
    \begin{enumerate}[(a)]
        \item Write a recursive algorithm to count the number of leaves (nodes) in a binary tree pointed to by a tree node pointer called \texttt{root}. You can assume that each tree node \texttt{x} has a pointer \texttt{x.left} pointing to its left sub-tree and a pointer \texttt{x.right} pointing to its right sub-tree.
        \item Express the time complexity of your algorithm with a recurrence equation.
        \item solve the recurrence.
    \end{enumerate}
\end{example}
\begin{solution*}
    \begin{columns}
        \begin{column}{.4\linewidth}
            \begin{enumerate}[(a)]
                \item See Algorithm \ref{a-1}.
                \item $T(root) = T(root.left) + T(root.right) + 1$
                \item $T(root) = T(root.left.left) + T(root.left.right) + T(root.right.left) + T(root.right.right) + 3 = \cdots = n$
            \end{enumerate}
        \end{column}
        \begin{column}{.6\linewidth}
            \begin{algorithm}[H]
                \caption{Counting}
                \label{a-1}
                \begin{algorithmic}
                    \Procedure{Count}{$root$}
                    \If{$root.left = 0 \wedge root.right = 0$}
                    \State \Return 1
                    \EndIf
                    \State $c \coloneqq 0$\Comment{1 if counting nodes.}
                    \If{$root.left \ne 0$}
                    \State $c \coloneqq c + \Call{Count}{root.left}$
                    \EndIf
                    \If{$root.right \ne 0$}
                    \State $c \coloneqq c + \Call{Count}{root.right}$
                    \EndIf
                    \State \Return $c$
                    \EndProcedure
                \end{algorithmic}
            \end{algorithm}
        \end{column}
    \end{columns}
\end{solution*}
\end{frame}

\plain{Questions?}

\end{document}
