%!TEX program = xelatex
\documentclass[10pt, compress, handout]{beamer}
\usepackage[titleprogressbar]{../../cls/beamerthemem}

\usepackage{booktabs}
\usepackage[scale=2]{ccicons}
\usepackage{minted}

\usepgfplotslibrary{dateplot}

\usemintedstyle{trac}

\setbeamertemplate{caption}[numbered]
\setbeamertemplate{theorems}[numbered]
\newtheorem{crl}{Corollary}[theorem]
\newtheorem*{solution*}{Solution}

\usepackage{algorithm}
\usepackage[noend]{algpseudocode}

\usepackage{version}
%\excludeversion{proof}
%\excludeversion{solution*}

\usepackage{mathtools}
\usepackage{multicol}
\usepackage{qtree}

\usepackage{tikz}

\makeatletter
\def\old@comma{,}
\catcode`\,=13
\def,{%
    \ifmmode%
    \old@comma\discretionary{}{}{}%
    \else%
    \old@comma%
    \fi%
}
\makeatother

\title{CSCI 3190 Tutorial of Week 12}
\subtitle{Trees}
\author{LI Haocheng}
\institute{Department of Computer Science and Engineering}

\begin{document}

\maketitle

\begin{frame}[allowframebreaks]
\frametitle{Counting}
\begin{example}
    \begin{enumerate}[(a)]
        \item Write a recursive algorithm to count the number of leaves (nodes) in a binary tree pointed to by a tree node pointer called \texttt{root}. You can assume that each tree node \texttt{x} has a pointer \texttt{x.left} pointing to its left sub-tree and a pointer \texttt{x.right} pointing to its right sub-tree.
        \item Express the time complexity of your algorithm with a recurrence equation.
        \item solve the recurrence.
    \end{enumerate}
\end{example}
\begin{solution*}
    \begin{columns}
        \begin{column}{.4\linewidth}
            \begin{enumerate}[(a)]
                \item See Algorithm \ref{a-1}.
                \item $T(root) = T(root.left) + T(root.right) + 1$
                \item $T(root) = T(root.left.left) + T(root.left.right) + T(root.right.left) + T(root.right.right) + 3 = \cdots = n$
            \end{enumerate}
        \end{column}
        \begin{column}{.6\linewidth}
            \begin{algorithm}[H]
                \caption{Counting}
                \label{a-1}
                \begin{algorithmic}
                    \Procedure{Count}{$root$}
                    \If{$root.left = 0 \wedge root.right = 0$}
                    \State \Return 1
                    \EndIf
                    \State $c \coloneqq 0$\Comment{1 if counting nodes.}
                    \If{$root.left \ne 0$}
                    \State $c \coloneqq c + \Call{Count}{root.left}$
                    \EndIf
                    \If{$root.right \ne 0$}
                    \State $c \coloneqq c + \Call{Count}{root.right}$
                    \EndIf
                    \State \Return $c$
                    \EndProcedure
                \end{algorithmic}
            \end{algorithm}
        \end{column}
    \end{columns}
\end{solution*}
\end{frame}

\plain{Questions?}

\end{document}
