%!TEX program = xelatex
\documentclass[10pt, compress]{beamer}
\usepackage[titleprogressbar]{../../cls/beamerthemem}

\usepackage{booktabs}
\usepackage[scale=2]{ccicons}
\usepackage{minted}

\usepgfplotslibrary{dateplot}

\usemintedstyle{trac}

\setbeamertemplate{caption}[numbered]
\setbeamertemplate{theorems}[numbered]
\newtheorem{crl}{Corollary}[theorem]
\newtheorem*{solution*}{Solution}

\usepackage{algorithm}
\usepackage[noend]{algpseudocode}

\usepackage{version}
%\excludeversion{proof}
%\excludeversion{solution*}

\usepackage{mathtools}
\usepackage{multicol}
\usepackage{qtree}

\usepackage{tikz}

\makeatletter
\def\old@comma{,}
\catcode`\,=13
\def,{%
	\ifmmode%
	\old@comma\discretionary{}{}{}%
	\else%
	\old@comma%
	\fi%
}
\makeatother

\title{CSCI 3190 Tutorial of Week 12}
\subtitle{Quiz 2}
\author{LI Haocheng}
\institute{Department of Computer Science and Engineering}

\begin{document}

\maketitle

\begin{frame}[fragile]
	\frametitle{Complexity}
	\onslide<1->\begin{example}
		\begin{enumerate}
			\item Show that $f(n) = 3^n = O(n!)$.
			\item Show that $f(n) = 4^n = O(n!)$.
		\end{enumerate}
	\end{example}
	\onslide<2>\begin{solution*}
		\begin{enumerate}
			\item $\exists N = 0, \exists C = 5, \forall n \ge N, 3^n < C(n!)$.
			\item $\exists N = 0, \exists C = 11, \forall n \ge N, 4^n < C(n!)$.
		\end{enumerate}
	\end{solution*}
\end{frame}

\begin{frame}[allowframebreaks]
	\frametitle{Longest Common Subsequence}
	\begin{example}
		\begin{enumerate}[(a)]
			\item Find the longest common subsequence between $\alpha = \mathtt{abcbdaccdb}$ and $\beta = \mathtt{bbcad}$ by constructing a table $len$ where $len[i, j]$ is the length of the longest common subsequence between $\alpha[1]\alpha[2]\cdots\alpha[i]$ and $\beta[1]\beta[2]\cdots\beta[j]$ where $i = 1\cdots 10$ and $j = 1\cdots 5$.
			\item From the table $len$, find all the longest common subsequences between $\alpha$ and $\beta$. For each of the longest common subsequences, show how it can be found from the table.
		\end{enumerate}
	\end{example}
	\newpage
	\begin{solution*}
		\begin{table}
			\centering
			\caption{The Table $len$}
			\label{t-6}
			\begin{tabular}{c|ccccccccccc}
				\toprule 
				& 0 & a & b & c & b & d & a & c & c & d & b \\ 
				\midrule 
				0 & 0 & 0 & 0 & 0 & 0 & 0 & 0 & 0 & 0 & 0 & 0 \\ 
				b & 0 & 0 & \alert{1} & 1 & 1 & 1 & 1 & 1 & 1 & 1 & 1 \\ 
				b & 0 & 0 & 1 & 1 & \alert{2} & 2 & 2 & 2 & 2 & 2 & 2 \\ 
				c & 0 & 0 & 1 & \alert{2} & 2 & 2 & 2 & \alert{3} & 3 & 3 & 3 \\ 
				a & 0 & 1 & 1 & 2 & 2 & 2 & \alert{3} & 3 & 3 & 3 & 3 \\ 
				d & 0 & 1 & 1 & 2 & 2 & 3 & 3 & 3 & 3 & \alert{4} & 4 \\ 
				\bottomrule 
			\end{tabular}
		\end{table}
	\end{solution*}
\end{frame}

\begin{frame}[fragile]
	\frametitle{Euler Path}
	\onslide<1->\begin{columns}
		\begin{column}{.5\linewidth}
			\begin{example}
				Determine whether the given graph has an
				Euler circuit. Construct such a circuit when one exists. If
				no Euler circuit exists, determine whether the graph has an
				Euler path and construct such a path if one exists.
			\end{example}
		\end{column}
		\begin{column}{.5\linewidth}
			\begin{figure}
				\centering
				\includegraphics[width=.6\linewidth]{f-10-5-e-1}
			\end{figure}
		\end{column}
	\end{columns}
	\onslide<2>\begin{solution*}
		Neither.
	\end{solution*}
\end{frame}

\begin{frame}[fragile]
	\frametitle{Euler Circuit}
	\onslide<1->\begin{columns}
		\begin{column}{.5\linewidth}
			\begin{example}
				Determine whether the given graph has an
				Euler circuit. Construct such a circuit when one exists. If
				no Euler circuit exists, determine whether the graph has an
				Euler path and construct such a path if one exists.
			\end{example}
		\end{column}
		\begin{column}{.5\linewidth}
			\begin{figure}
				\centering
				\includegraphics[width=.8\linewidth]{f-10-5-e-2}
			\end{figure}
		\end{column}
	\end{columns}
	\onslide<2>\begin{solution*}
		$a, b, c, f, e, b, d, e, h, f, i, h, g, d, a.$
	\end{solution*}
\end{frame}

\begin{frame}[fragile]
	\frametitle{Hamilton Path}
	\onslide<1->\begin{columns}
		\begin{column}{.5\linewidth}
			\begin{example}
				Which of the simple graphs in Figure have a Hamilton circuit or, if not, a Hamilton path?
			\end{example}
		\end{column}
		\begin{column}{.5\linewidth}
			\begin{figure}
				\centering
				\includegraphics[width=\linewidth]{f-10-5-10}
			\end{figure}
		\end{column}
	\end{columns}
	\onslide<2>\begin{solution*}
		$G_1$ has a Hamilton circuit: $a, b, c, d, e, a$. There is no Hamilton circuit in $G_2$ (this can
		be seen by noting that any circuit containing every vertex must contain the edge $\{a, b\}$ twice),
		but $G_2$ does have a Hamilton path, namely, $a, b, c, d$. $G_3$ has neither a Hamilton circuit nor a
		Hamilton path, because any path containing all vertices must contain one of the edges $\{a, b\}, \{e, f\}$, and $\{c, d\}$ more than once.
	\end{solution*}
\end{frame}

\begin{frame}[fragile]
	\frametitle{Hamilton Circuit}
	\onslide<1->\begin{columns}
		\begin{column}{.5\linewidth}
			\begin{example}
				Show that neither graph displayed in Figure has a Hamilton circuit.
			\end{example}
		\end{column}
		\begin{column}{.5\linewidth}
			\begin{figure}
				\centering
				\includegraphics[width=\linewidth]{f-10-5-11}
			\end{figure}
		\end{column}
	\end{columns}
	\onslide<2>\begin{solution*}
		There is no Hamilton circuit in $G$ because $G$ has a vertex of degree one, namely, $e$. Now consider $H$. Because the degrees of the vertices $a, b, d$, and $e$ are all two, every edge incident with these vertices must be part of any Hamilton circuit. It is now easy to see that no Hamilton circuit can exist in $H$, for any Hamilton circuit would have to contain four edges incident with c, which is impossible.
	\end{solution*}
\end{frame}

\begin{frame}[fragile]
	\frametitle{Shortest Path}
	\onslide<1->\begin{columns}
		\begin{column}{.5\linewidth}
			\begin{example}
				Find the length of a shortest path between $a$
				and $z$ in the given weighted graph.
			\end{example}
		\end{column}
		\begin{column}{.5\linewidth}
			\begin{figure}
				\centering
				\includegraphics[width=\linewidth]{f-10-6-e-2}
			\end{figure}
		\end{column}
	\end{columns}
	\onslide<2>\begin{solution*}
		\begin{equation}
			2 + 2 + 1 + 2 = 7.
		\end{equation}
	\end{solution*}
\end{frame}

\begin{frame}[fragile]
	\frametitle{Newark}
	\onslide<1->\begin{columns}
		\begin{column}{.5\linewidth}
			\begin{example}
				Find a shortest route in distance between Newark and
				Camden, and between Newark and Cape May, using
				these roads.
			\end{example}
		\end{column}
		\begin{column}{.5\linewidth}
			\begin{figure}
				\centering
				\includegraphics[width=\linewidth]{f-10-6-e-17-a}
			\end{figure}
		\end{column}
	\end{columns}
	\onslide<2>\begin{solution*}
		\begin{description}
			\item[Camden] 80
			\item[Cape May] 165
		\end{description}
	\end{solution*}
\end{frame}

\plain{Questions?}

\end{document}
