% This is "sig-alternate.tex" V2.1 April 2013
% This file should be compiled with V2.5 of "sig-alternate.cls" May 2012
%
% This example file demonstrates the use of the 'sig-alternate.cls'
% V2.5 LaTeX2e document class file. It is for those submitting
% articles to ACM Conference Proceedings WHO DO NOT WISH TO
% STRICTLY ADHERE TO THE SIGS (PUBS-BOARD-ENDORSED) STYLE.
% The 'sig-alternate.cls' file will produce a similar-looking,
% albeit, 'tighter' paper resulting in, invariably, fewer pages.
%
% ----------------------------------------------------------------------------------------------------------------
% This .tex file (and associated .cls V2.5) produces:
%       1) The Permission Statement
%       2) The Conference (location) Info information
%       3) The Copyright Line with ACM data
%       4) NO page numbers
%
% as against the acm_proc_article-sp.cls file which
% DOES NOT produce 1) thru' 3) above.
%
% Using 'sig-alternate.cls' you have control, however, from within
% the source .tex file, over both the CopyrightYear
% (defaulted to 200X) and the ACM Copyright Data
% (defaulted to X-XXXXX-XX-X/XX/XX).
% e.g.
% \CopyrightYear{2007} will cause 2007 to appear in the copyright line.
% \crdata{0-12345-67-8/90/12} will cause 0-12345-67-8/90/12 to appear in the copyright line.
%
% ---------------------------------------------------------------------------------------------------------------
% This .tex source is an example which *does* use
% the .bib file (from which the .bbl file % is produced).
% REMEMBER HOWEVER: After having produced the .bbl file,
% and prior to final submission, you *NEED* to 'insert'
% your .bbl file into your source .tex file so as to provide
% ONE 'self-contained' source file.
%
% ================= IF YOU HAVE QUESTIONS =======================
% Questions regarding the SIGS styles, SIGS policies and
% procedures, Conferences etc. should be sent to
% Adrienne Griscti (griscti@acm.org)
%
% Technical questions _only_ to
% Gerald Murray (murray@hq.acm.org)
% ===============================================================
%
% For tracking purposes - this is V2.0 - May 2012

\documentclass{../../cls/sig-alternate-05-2015}

\usepackage{algorithm}
\usepackage{algpseudocode}

\usepackage{booktabs}
\usepackage{color}
\usepackage{enumitem}
\usepackage{mathtools}
\usepackage{soul}
\usepackage{textcomp}


\begin{document}

% Copyright
%\setcopyright{acmcopyright}
%\setcopyright{acmlicensed}
%\setcopyright{rightsretained}
%\setcopyright{usgov}
%\setcopyright{usgovmixed}
%\setcopyright{cagov}
%\setcopyright{cagovmixed}


% DOI
%\doi{10.475/123_4}

% ISBN
%\isbn{123-4567-24-567/08/06}

%Conference
%\conferenceinfo{PLDI '13}{June 16--19, 2013, Seattle, WA, USA}

%\acmPrice{\$15.00}

%
% --- Author Metadata here ---
%\conferenceinfo{WOODSTOCK}{'97 El Paso, Texas USA}
%\CopyrightYear{2007} % Allows default copyright year (20XX) to be over-ridden - IF NEED BE.
%\crdata{0-12345-67-8/90/01}  % Allows default copyright data (0-89791-88-6/97/05) to be over-ridden - IF NEED BE.
% --- End of Author Metadata ---

\makeatletter
\def\old@comma{,}
\catcode`\,=13
\def,{%
    \ifmmode%
    \old@comma\discretionary{}{}{}%
    \else%
    \old@comma%
    \fi%
}
\makeatother

\title{CSCI 3190 \\ Introduction to Discrete Mathematics and Algorithms}
\subtitle{Sample Solution of Assignment 1}

\maketitle
\begin{abstract}

\end{abstract}

\keywords{}

\section{Logical Equivalence}
\textbf{Proof}\begin{enumerate}[label=(\alph*)]
    \item \begin{align}
    \begin{aligned}
    & \lnot p \land \lnot q \rightarrow \lnot r\\
    \equiv & \lnot r \lor \lnot (\lnot p \land \lnot q)\\
    \equiv & \lnot r \lor (p \lor q)\\
    \equiv & r \rightarrow q \lor p.
    \end{aligned}
    \end{align}
    \item \begin{align}
        \begin{aligned}
        & p \rightarrow (q \land r)\\
        \equiv & (q \land r) \lor \lnot p\\
        \equiv & (\lnot p \lor q) \land (\lnot p \lor r)\\
        \equiv & (p \rightarrow q) \land (p \rightarrow r).
        \end{aligned}
    \end{align}
    \item \begin{align}
        \begin{aligned}
        & (p \lor q) \rightarrow r\\
        \equiv & \lnot (p \lor q) \lor r\\
        \equiv & (\lnot p \land \lnot q) \lor r\\
        \equiv & (\lnot p \lor r) \land (\lnot q \lor r)\\
        \equiv & (p \rightarrow r) \land (q \rightarrow r).
        \end{aligned}
    \end{align}
    \item \begin{align}
        \begin{aligned}
        & ((q \rightarrow p) \land (\lnot p \rightarrow q) \land (q \rightarrow q))\\
        \equiv & ((\lnot q \lor p) \land (p \lor q) \land (\lnot q \lor q))\\
        \equiv & ((p \lor (q \land \lnot q)) \land TRUE)\\
        \equiv & (p \lor FALSE)\\
        \equiv & p.
        \end{aligned}
    \end{align}
    \item \begin{align}
        \begin{aligned}
        & ((p \rightarrow \lnot p) \land (\lnot p \rightarrow p))\\
        \equiv & ((\lnot p \lor \lnot p) \land (p \lor p))\\ \equiv & (\lnot p \land p)\\
        \equiv & FALSE.
        \end{aligned}
    \end{align}
\end{enumerate}

\section{Tautology}
\textbf{Proof}\begin{enumerate}[label=(\alph*)]
    \item \hl{Claim: $\forall x \forall y [(x^2 = y^2) \rightarrow (x = y)]$. $x = 1 \land y = -1$ is a counterexample.}
    \item Claim: $\forall x \exists y [y^2 = x]$. $x = 2$ is a counterexample.
    \item Claim: $\forall x \forall y [xy \ge x]$. $x = 1 \land y = -1$ is a counterexample.
    \item Claim: $\exists x \exists y [x - y = 5 \land 2x + 4y = 4]$. Let $x = 4, y = -1$, we have \begin{equation}
        x - y = 5 \land 2x + 4y = 4.
    \end{equation}
\end{enumerate}

\section{Set}
\textbf{Proof}\begin{enumerate}[label=(\alph*)]
    \item \begin{align}
        \begin{aligned}
        & ((a, b) \in A \times (B \cap C))\\
        \equiv & ((a \in A) \land (b \in B \cap C))\\
        \equiv & ((a \in A) \land (b \in B) \land (b \in C))\\
        \equiv & (((a, b) \in A \times B) \land ((a, b) \in A \times C))\\
        \equiv & ((a, b) \in (A \times B) \cap (A \times C)).
        \end{aligned}
    \end{align}
    \item \begin{align}
        \begin{aligned}
        & (A \cup B) - C\\
        = & (A \cup B) \cap \overline{C}\\
        = & (A \cap \overline{C}) \cup (B \cap \overline{C})\\
        = & (A - C) \cup (B - C).
        \end{aligned}
    \end{align}
    \item \begin{align}
        \begin{aligned}
        & (A - B) \cup (A - C)\\
        = & (A \cap \overline{B}) \cup (A \cap \overline{C})\\
        = & A \cap (\overline{B} \cup \overline{C})\\
        = & A \cap \overline{B \cap C}\\
        = & A - (B \cap C).
        \end{aligned}
    \end{align}
\end{enumerate}

\section{Function Composition}
\textbf{Solution}\begin{enumerate}[label=(\alph*)]
\item $(f \circ f)(n) = 2 (2n + 1) + 1 = 4n + 3$.
\item $(f \circ g)(n) = 2(n + 5) + 1 = 2n + 11$.
\item $(g \circ f)(n) = 2n + 1 + 5 = 2n + 6$.
\item \begin{equation}
(g \circ h)(n) = \begin{cases}
5 & \text{when $n$ is even},\\
6 & \text{when $n$ is odd}.
\end{cases}
\end{equation}
\item \begin{equation}
(h \circ g)(n) = \begin{cases}
1 & \text{when $n$ is even},\\
0 & \text{when $n$ is odd}.
\end{cases}
\end{equation}
\item \begin{equation}
((f \circ g) \circ h)(n) = \begin{cases}
(f \circ g)(0) = 11 & \text{when $n$ is even},\\
(f \circ g)(1) = 13 & \text{when $n$ is odd}.
\end{cases}
\end{equation}
\end{enumerate}

\section{Tautological Implication}
\textbf{Proof}\begin{enumerate}[label=(\alph*)]
\item $(p \land q) \rightarrow p \equiv \lnot (p \land q) \lor p \equiv \lnot p \lor \lnot q \lor p \equiv TRUE$.
\item \begin{align}
    \begin{aligned}
    & (p \land (p \rightarrow q) \land (s \lor r) \land (r \rightarrow \lnot q)) \rightarrow s\\
    \equiv & \lnot (p \land (\lnot p \lor q) \land (s \lor r) \land (\lnot r \lor \lnot q)) \lor s\\
    \equiv & \lnot (p \land q \land (\lnot r \lor \lnot q) \land (s \lor r)) \lor s\\
    \equiv & \lnot (p \land q \land \lnot r \land (r \lor s)) \lor s\\
    \equiv & \lnot (p \land q \land \lnot r \land s) \lor s\\ \equiv & \lnot p \lor \lnot q \lor r \lor \lnot s \lor s\\
    \equiv & TRUE.
    \end{aligned}
\end{align}
\item \begin{align}
    \begin{aligned}
    & ((p \rightarrow r) \land (q \rightarrow r)) \rightarrow ((p \lor q) \rightarrow r)\\
    \equiv & \lnot ((\lnot p \lor r) \land (\lnot q \lor r)) \lor (\lnot (p \lor q) \lor r)\\
    \equiv & \lnot ((\lnot p \land \lnot q) \lor r) \lor ((\lnot p \land \lnot q) \lor r)\\
    \equiv & TRUE.
    \end{aligned}
\end{align}
\end{enumerate}

\section{Infinite Set}
\textbf{Proof}\begin{enumerate}[label=(\alph*)]
    \item False. Let \begin{equation}
        \begin{cases}
        A = \{x \mid x > 0 \land x \in \mathbb{Z}\},\\
        B = \{x \mid x < 0 \land x \in \mathbb{Z}\},
        \end{cases}
    \end{equation} then $A \cap B = \emptyset$.
    \item False. Let $B = \mathbb{Z}, A = \emptyset$, then $A \subseteq B, \left|A\right| = 0$.
    \item True. Otherwise, $\left|A\right| > \left|B\right|$ which is contradictory to $A \subseteq B$.
    \item False. Let $A = \emptyset, B = \mathbb{Z}$, then $A \subseteq B, \left|B\right| = \infty$.
\end{enumerate}

\section{Invertible Function}
\textbf{Solution}\begin{enumerate}[label=(\alph*)]
    \item Let $A = \{a_0, a_1, \ldots, a_4\}$, to form function $f$ we need to decide the value of $f(a_0), f(a_1), \ldots, f(a_4)$, each of which has five possible values. Therefore, the total number of function candidates is $5^5$.
    \item $5! = 120$.
\end{enumerate}

\section{Closure}
\textbf{Solution}\begin{enumerate}[label=(\alph*)]
    \item \begin{description}
        \item[Symmetric closure] $\{(1, 3), (1, 5), (2, 4), (2, 5), (3, 1), (4, 2), (5, 1), (5, 2)\}$.
        \item[Reflexive closure] $\{(1, 1), (1, 3), (1, 5), (2, 2), (2, 4), (2, 5), (3, 1), (3, 3), (4, 2), (4, 4), (5, 1), (5, 2), (5, 5)\}$.
    \end{description}
    \item \begin{description}
        \item[Reflexive closure] $\{(1, 1), (1, 5), (2, 2), (2, 4), (3, 1), (3, 3), (4, 4), (5, 2), (5, 5)\}$.
        \item[Transitive closure] $\{(1, 1), (1, 2), (1, 4), (1, 5), (2, 2), (2, 4), (3, 1), (3, 2), (3, 3), (3, 4), (3, 5), (4, 4), (5, 2), (5, 4), (5, 5)\}$.
    \end{description}
\end{enumerate}

\section{Equivalence Relation}
\textbf{Proof}\begin{description}
    \item[Reflexive] $a + b = b + a$ so that $((a, b), (a, b)) \in R$.
    \item[Symmetric] $a + d = b + c \Leftrightarrow c + b = d + a$ so that \begin{equation}
        ((a, b), (c, d)) \in R \Leftrightarrow ((c, d), (a, b)) \in R.
    \end{equation}
    \item[Transitive] $a + d = b + c \land c + f = d + e \Rightarrow a + f = b + e$ so that \begin{equation}
        ((a, b), (c, d)) \in R \land ((c, d), (e, f)) \in R \Rightarrow ((a, b), (e, f)) \in R.
    \end{equation}
\end{description}

\section{Equivalence Class}
\textbf{Proof}\begin{enumerate}[label=(\alph*)]
    \item \begin{description}
        \item[$\Rightarrow$] Let $c \in [a]$, then $(c, a) \in R$. Since $(a, b) \in R$, $(c, b) \in R$. Therefore, $c \in [b]$, so that $[a] \subseteq [b]$. Similarly, we can prove $[b] \subseteq [a]$. Thus $[a] = [b]$.
        \item[$\Leftarrow$] $[a] = [b] \Rightarrow b \in [a] \Rightarrow (a, b) \in R$.
    \end{description}
    \item \begin{description}
        \item[$\Rightarrow$] \begin{align}
            \begin{aligned}
            & [a] \cap [b] \ne \emptyset\\
            \Rightarrow & \exists c \in [a] \cap [b]\\
            \Rightarrow & (a, c) \in R \land (c, b) \in R\\
            \Rightarrow & (a, b) \in R.
            \end{aligned}
        \end{align}
        \item[$\Leftarrow$] $(a, b) \in R \Rightarrow b \in [a] \Rightarrow b \in [a] \cap [b] \ne \emptyset$.
    \end{description}
\end{enumerate}
\end{document}
