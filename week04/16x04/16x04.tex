% This is "sig-alternate.tex" V2.1 April 2013
% This file should be compiled with V2.5 of "sig-alternate.cls" May 2012
%
% This example file demonstrates the use of the 'sig-alternate.cls'
% V2.5 LaTeX2e document class file. It is for those submitting
% articles to ACM Conference Proceedings WHO DO NOT WISH TO
% STRICTLY ADHERE TO THE SIGS (PUBS-BOARD-ENDORSED) STYLE.
% The 'sig-alternate.cls' file will produce a similar-looking,
% albeit, 'tighter' paper resulting in, invariably, fewer pages.
%
% ----------------------------------------------------------------------------------------------------------------
% This .tex file (and associated .cls V2.5) produces:
%       1) The Permission Statement
%       2) The Conference (location) Info information
%       3) The Copyright Line with ACM data
%       4) NO page numbers
%
% as against the acm_proc_article-sp.cls file which
% DOES NOT produce 1) thru' 3) above.
%
% Using 'sig-alternate.cls' you have control, however, from within
% the source .tex file, over both the CopyrightYear
% (defaulted to 200X) and the ACM Copyright Data
% (defaulted to X-XXXXX-XX-X/XX/XX).
% e.g.
% \CopyrightYear{2007} will cause 2007 to appear in the copyright line.
% \crdata{0-12345-67-8/90/12} will cause 0-12345-67-8/90/12 to appear in the copyright line.
%
% ---------------------------------------------------------------------------------------------------------------
% This .tex source is an example which *does* use
% the .bib file (from which the .bbl file % is produced).
% REMEMBER HOWEVER: After having produced the .bbl file,
% and prior to final submission, you *NEED* to 'insert'
% your .bbl file into your source .tex file so as to provide
% ONE 'self-contained' source file.
%
% ================= IF YOU HAVE QUESTIONS =======================
% Questions regarding the SIGS styles, SIGS policies and
% procedures, Conferences etc. should be sent to
% Adrienne Griscti (griscti@acm.org)
%
% Technical questions _only_ to
% Gerald Murray (murray@hq.acm.org)
% ===============================================================
%
% For tracking purposes - this is V2.0 - May 2012

\documentclass{sig-alternate-05-2015}
\usepackage{booktabs}


\begin{document}

% Copyright
%\setcopyright{acmcopyright}
%\setcopyright{acmlicensed}
%\setcopyright{rightsretained}
%\setcopyright{usgov}
%\setcopyright{usgovmixed}
%\setcopyright{cagov}
%\setcopyright{cagovmixed}


% DOI
%\doi{10.475/123_4}

% ISBN
%\isbn{123-4567-24-567/08/06}

%Conference
%\conferenceinfo{PLDI '13}{June 16--19, 2013, Seattle, WA, USA}

%\acmPrice{\$15.00}

%
% --- Author Metadata here ---
%\conferenceinfo{WOODSTOCK}{'97 El Paso, Texas USA}
%\CopyrightYear{2007} % Allows default copyright year (20XX) to be over-ridden - IF NEED BE.
%\crdata{0-12345-67-8/90/01}  % Allows default copyright data (0-89791-88-6/97/05) to be over-ridden - IF NEED BE.
% --- End of Author Metadata ---

%\\TODO:1.tautology prove section 2. set section operation proof 3. function section basic proving onto bijection...
\title{CSCI 3190 \\ Introduction to Discrete Mathematics and Algorithms}
\subtitle{Extended Exercise 4}

\maketitle
\begin{abstract}

\end{abstract}

\keywords{}

\section{Basic Structures}
\subsection{Sequences and Summations}
\begin{enumerate}
\item Let $a_n$ be the $n$-th term of the sequence 1, 2, 2, 3, 3, 3, 4, 4, 4, 4, 5, 5, 5, 5, 5, 6, 6, 6, 6, 6, 6,..., constructed by including the integer $k$ exactly $k$ times. Showthat $a_n = \lfloor \sqrt{2n}+ \frac{1}{2n} \rfloor$.

\item Show that $\Sigma^n_{j = 1}(a_j - a_{j - 1}) = a_n - a_0$, where
$a_0$, $a_1$, ..., $a_n$ is a sequence of real numbers. This type of sum is called \textbf{telescoping}.

\item Sum both sides of the identity $k^2 - (k - 1)^2 = 2k - 1$ from $k = 1$ to $k = n$ and use last exercise find:
\begin{enumerate}
	\item a formula for $\Sigma^n_{k = 1}(2k - 1)$ (the sum of the first $n$ odd natural numbers).
	\item a formula for $\Sigma^n_{k = 1} k$.
\end{enumerate}

\item Use the technique given in last Exercise, to derive the formula for $\Sigma^n_{k = 1} k^2$

\end{enumerate}
\subsection{Cardinality of Sets}
\begin{enumerate}
\item Show that a finite group of guests arriving at Hilbert \textquoteright s
fully occupied Grand Hotel can be given rooms without
evicting any current guest.
\item Suppose that Hilbert \textquoteright s Grand Hotel is fully occupied, but
the hotel closes all the even numbered rooms for maintenance.
Show that all guests can remain in the hotel.
\item Suppose that Hilbert \textquoteright s Grand Hotel is fully occupied on
the day the hotel expands to a second building which also
contains a countably infinite number of rooms. Show that
the current guests can be spread out to fill every room of
the two buildings of the hotel.
\item Show that a countably infinite number of guests arriving
at Hilbert \textquoteright s fully occupied Grand Hotel can be given
rooms without evicting any current guest.
\item Suppose that a countably infinite number of buses, each
containing a countably infinite number of guests, arrive
at Hilbert \textquoteright s fully occupied Grand Hotel. Show that all the
arriving guests can be accommodated without evicting
any current guest.
\end{enumerate}

\section{Advanced Counting Techniques}
\subsection{Generating Functions}
\begin{enumerate}
	\item For each of these generating functions, provide a closed
	formula for the sequence it determines.
	\begin{enumerate}
		\item $(x^2 + 1)^3$
		\item $(3x - 1)^3$
		\item $\frac{1}{1 - 2x^2} $
		\item $\frac{x^2}{(1 - x)^3}$
		\item $x - 1 + \frac{1}{1 - 3x}$
		\item $\frac{1 + x^3}{(1 + x)^3}$
		\item $\frac{x}{1 + x + x^2}$
		\item $e^{3x^2} - 1$
	\end{enumerate}
\end{enumerate}

\nocite{*}
\bibliographystyle{abbrv}
\bibliography{ref}  % sigproc.bib is the name of the Bibliography in this case
 
\clearpage
%APPENDICES are optional
%\balancecolumns
\appendix
%Appendix A
\section{Answer}
\subsection{Basic Structures}
\subsubsection{Sequences and Summations}
\begin{enumerate}
\item Assume $a_n = k$, so that:
\begin{align}
	\frac{(k - 1)k}{2} + 1 \le n & \le \frac{k(k + 1)}{2}\\
	k^2 - k + 2 \le 2n & \le k^2 + k\\
	(k - \frac{1}{2})^2 < (k - \frac{1}{2})^2 + \frac{7}{4} \le 2n & \le (k + \frac{1}{2})^2 - \frac{1}{4} < (k + \frac{1}{2})^2\\
	k - \frac{1}{2} < \sqrt{2n} & < k + \frac{1}{2}\\
	\sqrt{2n} - \frac{1}{2} < k & < \sqrt{2n} + \frac{1}{2}\\
	k & = \lfloor \sqrt{2n} + \frac{1}{2} \rfloor
\end{align}

\item 
\begin{align}
	& \Sigma^n_{j = 1}(a_j - a_{j - 1})\\
	= & \Sigma^n_{j = 1}a_j - \Sigma^n_{j = 1}a_{j - 1}\\
	= & a_n + \Sigma^{n - 1}_{j = 1}a_j - \Sigma^{n - 1}_{j = 1}a_j + a_0\\
	= & a_n - a_0
\end{align}

\item 
\begin{enumerate}
	\item 
	\begin{equation}
		\Sigma^n_{k = 1}(2k - 1) = n^2
	\end{equation}
	
	\item 
	\begin{align}
		& \Sigma^n_{k = 1} k\\
		= & \frac{\Sigma^n_{k = 1}(2k - 1) + \Sigma^n_{k = 1} 1}{2}\\
		= & \frac{n^2 + n}{2}
	\end{align}
\end{enumerate}

\item Since $k^3 - (k - 1)^3 = 3k^2 - 3k + 1$:
\begin{align}
	& \Sigma_{k = 1}^n k^2\\
	= & \frac{n^3 + 3\Sigma_{k = 1}^n k - \Sigma_{k = 1}^n 1}{3}\\
	= & \frac{2n^3 + 3n^2 + 3n - 2n}{6}\\
	= & \frac{2n^3 + 3n^2 + n}{6}
\end{align}
\end{enumerate}
\subsubsection{Cardinality of Sets}
\begin{enumerate}
\item Suppose $m$
new guests arrive at the fully occupied hotel. Move the guest
in Room $n$ to Room $m + n$ for $n = 1, 2, 3, ...$; then the new
guests can occupy rooms 1 to $m$.
\item Move the guest
in Room $n$ to Room $2n$ for $n = 1, 2, 3, ...$.
\item For $n = 1, 2, 3, ...$, put the guest currently in Room $2n$ into Room $n$, and the guest
currently in Room $2n - 1$ into Room $n$ of the new building.
\item Move the guest
in Room $n$ to Room $2n - 1$ for $n = 1, 2, 3, ...$; then the $m$-th new guests can occupy room $2m$.
\item Move the guess currently Room $i$ to Room $2i + 1$ for $i = 1, 2, 3, ...$. Put the $j$ th guest from the $k$-th bus into
Room $2^k(2j + 1)$.
\end{enumerate}

\subsection{Advanced Counting Techniques}
\subsubsection{Generating Functions}
\begin{enumerate}
	\item \begin{enumerate}
		\item \begin{equation}
			a_n = \begin{cases}
				1,&n \in \{0, 6\},\\
				3,&n \in \{2, 4\},\\
				0,&\text{otherwise.}
			\end{cases}
		\end{equation}
		
		\item \begin{equation}
			a_n = \begin{cases}
				-1,&n = 0,\\
				9,&n = 1,\\
				27,&n = 2,\\
				-27,&n = 3.
			\end{cases}
		\end{equation}
		
		\item \begin{equation}
			a_n = \begin{cases}
				2^{\frac{n}{2}}, & n \equiv 0 \pmod{2},\\
				0, & \text{otherwise}.
			\end{cases}
		\end{equation}
		
		\item \begin{equation}
			a_n = \frac{(n - 1)n}{2}.
		\end{equation}
		\item \begin{equation}
			a_n = \begin{cases}
				0, & n = 0,\\
				4, & n = 1,\\
				3^n, & \text{otherwise}.
			\end{cases}
		\end{equation}
		\item \begin{align}
			& \frac{1 + x^3}{(1 + x)^3}\\
			= & (1 + x^3) \Sigma_{n = 0}^\infty (-1)^n \frac{(n + 1)(n + 2)}{2} x^n\\
			= & \Sigma_{n = 0}^\infty (-1)^n \frac{(n + 1)(n + 2)}{2} x^n\\
			& - \Sigma_{n = 3}^\infty (-1)^n \frac{(n - 2)(n - 1)}{2} x^n\\
			= & 1 + \Sigma_{n = 1}^\infty (-1)^n 3n x^n
		\end{align}
		\begin{equation}
			a_n = \begin{cases}
			1, & n = 0,\\
			(-1)^n 3n, & \text{otherwise}.
			\end{cases}
		\end{equation}
		\item \begin{equation}
			a_n = \begin{cases}
				0, & n \equiv 0 \pmod{3},\\
				1, & n \equiv 1 \pmod{3},\\
				-1, & \text{otherwise}.
			\end{cases}
		\end{equation}
		\item \begin{equation}
			a_n = \begin{cases}
				\frac{3^{\frac{n}{2}}}{(\frac{n}{2})!}, & n \ge 2, n \equiv 0 \pmod{2},\\
				0, & \text{otherwise}.
			\end{cases}
		\end{equation}
	\end{enumerate}
\end{enumerate}

\end{document}
