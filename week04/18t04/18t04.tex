%!TEX program = xelatex
\documentclass[10pt, compress, handout]{beamer}
\usepackage[titleprogressbar]{../../cls/beamerthemem}

\usepackage{booktabs}
\usepackage[scale=2]{ccicons}
\usepackage{minted}

\usepgfplotslibrary{dateplot}

\usemintedstyle{trac}

\setbeamertemplate{caption}[numbered]
\setbeamertemplate{theorems}[numbered]
\newtheorem{crl}{Corollary}[theorem]
\newtheorem*{solution*}{Solution}

\usepackage{algorithm}
\usepackage[noend]{algpseudocode}

\usepackage{version}
%\excludeversion{proof}
%\excludeversion{solution*}

\usepackage{mathtools}
\usepackage{multicol}
\usepackage{qtree}

\usepackage{tikz}

\makeatletter
\def\old@comma{,}
\catcode`\,=13
\def,{%
	\ifmmode%
	\old@comma\discretionary{}{}{}%
	\else%
	\old@comma%
	\fi%
}
\makeatother

\title{CSCI 3190 Tutorial of Week 04}
\subtitle{Assignment 1}
\author{LI Haocheng}
\institute{Department of Computer Science and Engineering}

\begin{document}

\maketitle

\begin{frame}[fragile]
\frametitle{Logical Equivalence}
\onslide<1->\begin{example}
	Show that the following are logically equivalent:\begin{enumerate}[(a)]
		\item $\lnot p \land \lnot q \rightarrow \lnot r \equiv r \rightarrow q \lor p$.
		\item $p \rightarrow (q \land r) \equiv (p \rightarrow q) \land (p \rightarrow r)$.
		\item $(p \lor q) \rightarrow r \equiv (p \rightarrow r) \land (q \rightarrow r)$.
		\item $((q \rightarrow p) \land (\lnot p \rightarrow q) \land (q \rightarrow q)) \equiv p$.
	\end{enumerate}
\end{example}
\onslide<2>\begin{proof}
	\begin{enumerate}[(a)]
		\item $\lnot p \land \lnot q \rightarrow \lnot r \equiv \lnot r \lor \lnot (\lnot p \land \lnot q) \equiv \lnot r \lor (p \lor q) \equiv r \rightarrow q \lor p$.
		\item $p \rightarrow (q \land r) \equiv (q \land r) \lor \lnot p \equiv (\lnot p \lor q) \land (\lnot p \lor r) \equiv (p \rightarrow q) \land (p \rightarrow r)$.
		\item $(p \lor q) \rightarrow r \equiv \lnot (p \lor q) \lor r \equiv (\lnot p \land \lnot q) \lor r	\equiv (\lnot p \lor r) \land (\lnot q \lor r) \equiv (p \rightarrow r) \land (q \rightarrow r)$.
		\item $((q \rightarrow p) \land (\lnot p \rightarrow q) \land (q \rightarrow q)) \equiv ((\lnot q \lor p) \land (p \lor q) \land (\lnot q \lor q)) \equiv ((p \lor (q \land \lnot q)) \land TRUE) \equiv (p \lor FALSE) \equiv p$.
	\end{enumerate}
\end{proof}
\end{frame}

\begin{frame}[fragile]
\frametitle{Tautology}
\onslide<1->\begin{example}
	Prove the following statement or find a counterexample to disprove it. All the variables have the
	same domain of the set of all integers:\begin{enumerate}[(a)]
		\item $\forall x \forall y [(x^2 = y^2) \rightarrow (x = y)]$.
		\item $\forall x \exists y [y^2 = x]$.
		\item $\forall x \forall y [xy \ge x]$.
		\item $\exists x \exists y [(x - y = 5) \land (2x + 4y = 4)]$.
	\end{enumerate}
\end{example}
\onslide<2>\begin{proof}
	\begin{enumerate}[(a)]
		\item False. $x = 1 \land y = -1$ is a counterexample.
		\item False. Let $x = 2$, no integer $y$ satisfying $y^2 = x$.
		\item False. Let $x = 1, y = -1$, we have $xy < x$.
		\item True. Let $x = 4, y = -1$, we have $x - y = 5 \land 2x + 4y = 4$.
	\end{enumerate}
\end{proof}
\end{frame}

\begin{frame}[fragile]
\frametitle{Set}
\onslide<1->\begin{example}
	Prove the followings:\begin{enumerate}[(a)]
		\item $A \times (B \cap C) = (A \times B) \cap (A \times C)$.
		\item $(A \cup B) - C = (A - C) \cup (B - C)$.
	\end{enumerate}
\end{example}
\onslide<2>\begin{proof}
	\begin{enumerate}[(a)]
		\item \begin{align}
		\begin{aligned}
		& ((a, b) \in A \times (B \cap C))\\
		\equiv & ((a \in A) \land (b \in B \cap C))\\
		\equiv & ((a \in A) \land (b \in B) \land (b \in C))\\
		\equiv & (((a, b) \in A \times B) \land ((a, b) \in A \times C))\\
		\equiv & ((a, b) \in (A \times B) \cap (A \times C)).
		\end{aligned}
		\end{align}
		\item $(A \cup B) - C = (A \cup B) \cap \overline{C} = (A \cap \overline{C}) \cup (B \cap \overline{C}) = (A - C) \cup (B - C)$.
	\end{enumerate}
\end{proof}
\end{frame}

\begin{frame}[fragile]
\frametitle{Function Composition}
\onslide<1->\begin{example}
	Let $f$, $g$ and $h$ be functions from $\mathbb{N}$ to $\mathbb{N}$, where $\mathbb{N}$ is the set of all natural numbers, i.e., $1, 2, 3, \ldots$
	and are defined as follows: \begin{align}
		f(n) = & 2n + 1,\\
		g(n) = & n + 5,\\
		h(n) = & \begin{cases}
		0 & \text{when $n$ is even},\\
		1 & \text{when $n$ is odd}.
		\end{cases}
	\end{align} Determine $f \circ f, f \circ g, g \circ f$.
\end{example}
\onslide<2>\begin{solution*}
	\begin{enumerate}[(a)]
		\item $(f \circ f)(n) = 2 (2n + 1) + 1 = 4n + 3$.
		\item $(f \circ g)(n) = 2(n + 5) + 1 = 2n + 11$.
		\item $(g \circ f)(n) = 2n + 1 + 5 = 2n + 6$.
	\end{enumerate}
\end{solution*}
\end{frame}

\begin{frame}[fragile]
\frametitle{Tautological Implication}
\onslide<1->\begin{example}
	Prove the following tautological implication: \begin{enumerate}[(b)]
		\item $(p \land (p \rightarrow q) \land (s \lor r) \land (r \rightarrow \lnot q)) \Rightarrow s$.
	\end{enumerate}
\end{example}
\onslide<2>\begin{proof}
	\begin{enumerate}[(b)]
		\item \begin{align}
		\begin{aligned}
		& (p \land (p \rightarrow q) \land (s \lor r) \land (r \rightarrow \lnot q)) \rightarrow s\\
		\equiv & \lnot (p \land (\lnot p \lor q) \land (s \lor r) \land (\lnot r \lor \lnot q)) \lor s\\
		\equiv & \lnot (p \land q \land (\lnot r \lor \lnot q) \land (s \lor r)) \lor s\\
		\equiv & \lnot (p \land q \land \lnot r \land (r \lor s)) \lor s\\
		\equiv & \lnot (p \land q \land \lnot r \land s) \lor s\\ \equiv & \lnot p \lor \lnot q \lor r \lor \lnot s \lor s\\
		\equiv & TRUE.
		\end{aligned}
		\end{align}
	\end{enumerate}
\end{proof}
\end{frame}

\begin{frame}[fragile]
\frametitle{Infinite Set}
\onslide<1->\begin{example}
	Determine whether the following statements are true or false. \begin{enumerate}[(a)]
		\item If $A$ and $B$ are infinite sets then $A \cap B$ is infinite.
		\item If $B$ is infinite and $A \subseteq B$, then $A$ is infinite.
		\item If $A \subseteq B$ with $B$ finite, then $A$ is finite.
		\item If $A \subseteq B$ with $A$ finite, then $B$ is finite.
	\end{enumerate}
\end{example}
\onslide<2>\begin{proof}
	\begin{enumerate}[(a)]
		\item False. Let $A = \{x \mid x > 0 \land x \in \mathbb{Z}\}, B = \{x \mid x < 0 \land x \in \mathbb{Z}\}$, then $A \cap B = \emptyset$.
		\item False. Let $B = \mathbb{Z}, A = \emptyset$, then $A \subseteq B, \left|A\right| = 0$.
		\item True. Otherwise, $\left|A\right| > \left|B\right|$ which is contradictory to $A \subseteq B$.
		\item False. Let $A = \emptyset, B = \mathbb{Z}$, then $A \subseteq B, \left|B\right| = \infty$.
	\end{enumerate}
\end{proof}
\end{frame}

\begin{frame}[fragile]
\frametitle{Invertible Function}
\onslide<1->\begin{example}
	If $\left|A\right| = \left|B\right| = 5$, \begin{enumerate}[(a)]
		\item how many function $f: A \rightarrow B$ can be formed?
		\item How many of them are invertible?
	\end{enumerate}
\end{example}
\onslide<2>\begin{solution*}
	\begin{enumerate}[(a)]
		\item $5^5$.
		\item $5! = 120$.
	\end{enumerate}
\end{solution*}
\end{frame}

\begin{frame}[fragile]
\frametitle{Inverse Relation}
\begin{columns}
	\begin{column}{.6\linewidth}
		Let $R$ be a relation from a set $A$ to a set $B$.
		\begin{definition}
			The \textbf{inverse relation} from $B$ to $A$, denoted by $R^{-1}$, is the set of ordered pairs $\{(b, a) \mid (a, b) \in R \}$.
		\end{definition}
		\begin{definition}
			The \textbf{complementary relation} $\bar{R}$ is the set of ordered pairs $\{(a, b) \mid (a, b) \notin R \}$.
		\end{definition}
	\end{column}
	\begin{column}{.4\linewidth}
		\begin{example}
			Let $R$ be the relation $R = \{(a, b) \mid a < b\}$ on the set of integers. Find \begin{enumerate}
				\onslide<1->\item $R^{-1}$ \onslide<2>$= \{(a, b) \mid a > b\}$
				\onslide<1->\item $\bar{R}$ \onslide<2>$= \{(a, b) \mid a \ge b\}$
			\end{enumerate}
		\end{example}
	\end{column}
\end{columns}
\end{frame}

\begin{frame}[fragile]
\frametitle{Closure}
\onslide<1->\begin{example}
	Consider the relation $R = \{(3, 1), (1, 5), (2, 4), (5, 2)\}$ over the set $A = \{1, 2, 3, 4, 5\}$: \begin{enumerate}[(a)]
		\item What is the reflexive closure of the symmetric closure of $R$?
		\item What is the transitive closure of the reflexive closure of $R$?
	\end{enumerate}
\end{example}
\onslide<2>\begin{solution*}
	\begin{enumerate}[(a)]
		\item \begin{description}
			\item[Symmetric closure] $\{(1, 3), (1, 5), (2, 4), (2, 5), (3, 1), (4, 2), (5, 1), (5, 2)\}$.
			\item[Reflexive closure] $\{(1, 1), (1, 3), (1, 5), (2, 2), (2, 4), (2, 5), (3, 1), (3, 3), (4, 2), (4, 4), (5, 1), (5, 2), (5, 5)\}$.
		\end{description}
		\item \begin{description}
			\item[Reflexive closure] $\{(1, 1), (1, 5), (2, 2), (2, 4), (3, 1), (3, 3), (4, 4), (5, 2), (5, 5)\}$.
			\item[Transitive closure] $\{(1, 1), (1, 2), (1, 4), (1, 5), (2, 2), (2, 4), (3, 1), (3, 2), (3, 3), (3, 4), (3, 5), (4, 4), (5, 2), (5, 4), (5, 5)\}$.
		\end{description}
	\end{enumerate}
\end{solution*}
\end{frame}

\plain{Questions?}

\end{document}
