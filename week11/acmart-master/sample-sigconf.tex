\documentclass[sigconf]{acmart}

\usepackage{booktabs} % For formal tables


% Copyright

\setcopyright{none}
%\setcopyright{acmcopyright}
%\setcopyright{acmlicensed}
%\setcopyright{rightsretained}
%\setcopyright{usgov}
%\setcopyright{usgovmixed}
%\setcopyright{cagov}
%\setcopyright{cagovmixed}


% DOI
%\acmDOI{10.475/123_4}

% ISBN
%\acmISBN{123-4567-24-567/08/06}

%Conference
\acmConference[CSCI'3190]{Introduction to Discrete Mathematics and Algorithms}{2017}{The Chinese University of Hong Kong}
\acmYear{2017}
\copyrightyear{2017}


%\acmArticle{4}
%\acmPrice{15.00}

% These commands are optional
%\acmBooktitle{Transactions of the ACM Woodstock conference}
%\editor{Jennifer B. Sartor}
%\editor{Theo D'Hondt}
%\editor{Wolfgang De Meuter}

\newtheorem*{solution*}{Solution}

\usepackage{algorithm}
\usepackage[noend]{algpseudocode}

\makeatletter
\def\old@comma{,}
\catcode`\,=13
\def,{%
	\ifmmode%
	\old@comma\discretionary{}{}{}%
	\else%
	\old@comma%
	\fi%
}
\makeatother

\begin{document}
\title{Assignment 2}
%\titlenote{Produces the permission block, and copyright information}
%\subtitle{}
%\subtitlenote{The full version of the author's guide is available as \texttt{acmart.pdf} document}

% The default list of authors is too long for headers.
%\renewcommand{\shortauthors}{B. Trovato et al.}

%
% The code below should be generated by the tool at
% http://dl.acm.org/ccs.cfm
% Please copy and paste the code instead of the example below. 
%

%\keywords{ACM proceedings, \LaTeX, text tagging}


\maketitle

%\input{samplebody-conf}
\section{Mathematical Induction}
\begin{proof}
	By mathematical induction:\begin{description}
		\item[Base] \begin{equation}
			0 \le 2 \times 0.
		\end{equation}
		\item[Induction] $a + 2 \le 2b + 2 = 2(b + 1)$.
	\end{description}
\end{proof}

\section{Recurrence Relation}
\begin{solution*}
	\begin{align}
		\begin{aligned}
			& \begin{cases}
				w_0 & = 1,\\
				w_1 & = 1,\\
				w_{n + 1} & = \begin{bmatrix}
					1 & 2
				\end{bmatrix}\begin{bmatrix}
				w_n\\
				w_{n - 1}
			\end{bmatrix}
			\end{cases}\\
			\Leftrightarrow & \begin{bmatrix}
			w_{n + 1}\\
			w_n
			\end{bmatrix} = \begin{bmatrix}
			1 & 2\\
			1 & 0
			\end{bmatrix}^n\begin{bmatrix}
			1\\
			1
			\end{bmatrix}\\
			\Leftrightarrow & \begin{bmatrix}
			w_{n + 1}\\
			w_n
			\end{bmatrix} = \frac{1}{3}\begin{bmatrix}
			2 & 1\\
			1 & -1
			\end{bmatrix}\begin{bmatrix}
			2 & 0\\
			0 & -1
			\end{bmatrix}^n\begin{bmatrix}
			1 & 1\\
			1 & -2
			\end{bmatrix}\begin{bmatrix}
			1\\
			1
			\end{bmatrix}\\
			\Leftrightarrow & \begin{bmatrix}
			w_{n + 1}\\
			w_n
			\end{bmatrix} = \frac{1}{3}\begin{bmatrix}
			2 & 1\\
			1 & -1
			\end{bmatrix}\begin{bmatrix}
			2^n & 0\\
			0 & (-1)^n
			\end{bmatrix}\begin{bmatrix}
			2\\
			-1
			\end{bmatrix}\\
			\Leftrightarrow & \begin{bmatrix}
			w_{n + 1}\\
			w_n
			\end{bmatrix} = \frac{1}{3}\begin{bmatrix}
			2 & 1\\
			1 & -1
			\end{bmatrix}\begin{bmatrix}
			2^{n + 1}\\
			(-1)^{n + 1}
			\end{bmatrix}\\
			\Leftrightarrow & w_n = \frac{2^{n + 1} + (-1)^n}{3}.
		\end{aligned}
	\end{align}
\end{solution*}

\section{Hanoi}
\subsection{Indistinguishable}
\begin{enumerate}
	\item See Algorithm \ref{a-3}.
	\begin{algorithm}
		\caption{Indistinguishable Double Tower of Hanoi}
		\label{a-3}
		\begin{algorithmic}
			\Procedure{Hanoi}{$height$, $begin$, $end$, $mid$}
			\If{$height$ = 0}
			\State \Return
			\Else
			\State \Call{Hanoi}{$height - 2$, $begin$, $mid$, $end$}
			\State \Call{Move}{$begin$, $end$}
			\State \Call{Move}{$begin$, $end$}
			\State \Call{Hanoi}{$height - 2$, $mid$, $end$, $begin$}
			\EndIf
			\EndProcedure
		\end{algorithmic}
	\end{algorithm}
	
	\item \begin{equation}
		\begin{cases}
			h_0 & = 0,\\
			h_n & = 2h_{n - 1} + 2.
		\end{cases}
	\end{equation}
	\item \begin{align}
		\begin{aligned}
			\begin{bmatrix}
			w_n\\
			1
			\end{bmatrix} & = \begin{bmatrix}
			2 & 2\\
			0 & 1
			\end{bmatrix}^n\begin{bmatrix}
			0\\
			1
			\end{bmatrix}\\
			& = \begin{bmatrix}
			1 & 2\\
			0 & -1
			\end{bmatrix}\begin{bmatrix}
			2 & 0\\
			0 & 1
			\end{bmatrix}^n\begin{bmatrix}
			1 & 2\\
			0 & -1
			\end{bmatrix}\begin{bmatrix}
			0\\
			1
			\end{bmatrix}\\
			& = \begin{bmatrix}
			1 & 2\\
			0 & -1
			\end{bmatrix}\begin{bmatrix}
			2^n & 0\\
			0 & 1
			\end{bmatrix}\begin{bmatrix}
			2\\
			-1
			\end{bmatrix}\\
			& = \begin{bmatrix}
			1 & 2\\
			0 & -1
			\end{bmatrix}\begin{bmatrix}
			2^{n + 1}\\
			-1
			\end{bmatrix}\\
			w_n & = 2^{n + 1} - 2\\
			& = O(2^n).
		\end{aligned}
	\end{align}
\end{enumerate}

\subsection{Original}
\begin{enumerate}
	\item See Algorithm \ref{a-4}.
	\begin{algorithm}
		\caption{Original Double Tower of Hanoi}
		\label{a-4}
		\begin{algorithmic}
			\Procedure{Hanoi}{$height$, $begin$, $end$, $mid$}
			\If{$height$ = 0}
			\State \Return
			\Else
			\State \Call{Hanoi}{$height - 2$, $begin$, $end$, $mid$}
			\State \Call{Move}{$begin$, $mid$}
			\State \Call{Hanoi}{$height - 2$, $end$, $mid$, $begin$}
			\State \Call{Move}{$begin$, $end$}
			\State \Call{Hanoi}{$height - 2$, $mid$, $begin$, $end$}
			\State \Call{Move}{$mid$, $end$}
			\State \Call{Hanoi}{$height - 2$, $begin$, $end$, $mid$}
			\EndIf
			\EndProcedure
		\end{algorithmic}
	\end{algorithm}
	
	\item \begin{equation}
	\begin{cases}
	h_0 & = 0,\\
	h_n & = 4h_{n - 1} + 3.
	\end{cases}
	\end{equation}
	\item \begin{align}
	\begin{aligned}
	\begin{bmatrix}
	w_n\\
	1
	\end{bmatrix} & = \begin{bmatrix}
	4 & 3\\
	0 & 1
	\end{bmatrix}^n\begin{bmatrix}
	0\\
	1
	\end{bmatrix}\\
	& = \begin{bmatrix}
	1 & 1\\
	0 & -1
	\end{bmatrix}\begin{bmatrix}
	4 & 0\\
	0 & 1
	\end{bmatrix}^n\begin{bmatrix}
	1 & 1\\
	0 & -1
	\end{bmatrix}\begin{bmatrix}
	0\\
	1
	\end{bmatrix}\\
	& = \begin{bmatrix}
	1 & 1\\
	0 & -1
	\end{bmatrix}\begin{bmatrix}
	4^n & 0\\
	0 & 1
	\end{bmatrix}\begin{bmatrix}
	1\\
	-1
	\end{bmatrix}\\
	& = \begin{bmatrix}
	1 & 1\\
	0 & -1
	\end{bmatrix}\begin{bmatrix}
	4^n\\
	-1
	\end{bmatrix}\\
	w_n & = 4^n - 1\\
	& = O(4^n).
	\end{aligned}
	\end{align}
\end{enumerate}

\section{Exponentiation}
\subsection{Recursive Algorithm}
\begin{enumerate}
	\item See Algorithm \ref{alg:a2n}.
	\begin{algorithm}
		\caption{Recursive Algorithm to Find $a^{2^n}$}
		\label{alg:a2n}
		\begin{algorithmic}[1]
			\Require $a \in \mathbb{R}, n \in \mathbb{N}$.
			\Ensure $a^{2^n}$.
			\Procedure{pow2}{$a$, $n$}
			\If {$n = 0$}
			\State\Return $a$
			\EndIf
			\State $p \gets$ \Call{pow2}{$a$, $n - 1$}
			\State\Return $p \times p$
			\EndProcedure
		\end{algorithmic}
	\end{algorithm}
	\item $O(n)$.
\end{enumerate}
\subsection{Iterative Algorithm}
\begin{enumerate}
	\item See Algorithm \ref{alg:a2ni}.
	\begin{algorithm}
		\caption{Iterative Algorithm to Find $a^{2^n}$}
		\label{alg:a2ni}
		\begin{algorithmic}[1]
			\Require $a \in \mathbb{R}, n \in \mathbb{N}$.
			\Ensure $a^{2^n}$.
			\Procedure{pow2i}{$a$, $n$}
			\State $p \gets a$
			\For {$i \gets 1, 2, \cdots, n$}
			\State $p \gets p \times p$
			\EndFor
			\State\Return $p$
			\EndProcedure
		\end{algorithmic}
	\end{algorithm}
	\item $O(n)$.
\end{enumerate}

\section{Longest Common Subsequence}
\begin{enumerate}
	\item See Algorithm \ref{a-7}.
	\begin{algorithm}
		\caption{Compute the LCS with Unique Symbols}
		\label{a-7}
		\begin{algorithmic}
			\Procedure{Unique}{$A$, $B$}
			\State $C \gets \Call{Zeros}{n}$
			\State $D \gets \Call{Zeros}{{}}$
			\For{$j \gets 1 \cdots m$}
			\State $D[A[j]] \gets j$
			\EndFor
			\For{$k \gets 1 \cdots n$}
			\State $C[k] \gets D[B[k]]$
			\EndFor
			\State $P \gets \Call{Zeros}{n}$
			\State $M \gets \Call{Zeros}{n + 1}$
			\State $l \gets 0$
			\For{$k \gets 1 \cdots n$}
			\State $lo \gets 1$
			\State $hi \gets l$
			\While{$lo \le hi$}
			\State $mid \gets \lceil\frac{lo + hi}{2}\rceil$
			\If{$C[M[mid + 1]] < C[k]$}
			\State $lo \gets mid + 1$
			\Else
			\State $hi \gets mid - 1$
			\EndIf
			\EndWhile
			\State $P[k] \gets M[lo]$
			\State $M[lo + 1] \gets k$
			\If{$lo > l$}
			\State $l \gets lo$
			\EndIf
			\EndFor
			\State $S \gets \Call{Zeros}{l}$
			\State $k \gets M[l + 1]$
			\For{$i \gets l \cdots 1$}
			\State $S[i] \gets A[C[k]]$
			\State $k \gets P[k]$
			\EndFor
			\State \Return $S$
			\EndProcedure
		\end{algorithmic}
	\end{algorithm}
	\item $O(n \log m)$
\end{enumerate}

\section{Maximum Subarray}
\begin{enumerate}
	\item See Algorithm \ref{a-8}.
	\begin{algorithm}
		\caption{Maximum Subarray}
		\label{a-8}
		\begin{algorithmic}
			\Procedure{MaximumSubarray}{$A$, $s$, $l$}
			\State $maxEndHere \gets 0$
			\State $maxEndHereIdx \gets 1$
			\State $maxSoFar \gets 0$
			\State $s \gets 0$
			\State $l \gets 0$
			\For{$i \gets 1, 2, \cdots, n$}
			\If{$maxEndHere + A[i] \le 0$}
			\State $maxEndHere \gets 0$
			\State $maxEndHereIdx \gets i + 1$
			\State \textbf{continue}
			\EndIf
			\State $maxEndHere \gets maxEndHere + A[i]$
			\If{$maxEndHere > maxSoFar$}
			\State $maxSoFar \gets maxEndHere$
			\State $s \gets maxEndHereIdx$
			\State $l \gets i$
			\EndIf
			\EndFor
			\State\Return $maxSoFar$
			\EndProcedure
		\end{algorithmic}
	\end{algorithm}
	\item $O(n)$
\end{enumerate}

\bibliographystyle{../../cls/ACM-Reference-Format}
%\bibliography{sample-bibliography} 

\end{document}
