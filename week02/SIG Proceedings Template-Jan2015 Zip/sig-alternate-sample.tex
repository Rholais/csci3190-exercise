% This is "sig-alternate.tex" V2.1 April 2013
% This file should be compiled with V2.5 of "sig-alternate.cls" May 2012
%
% This example file demonstrates the use of the 'sig-alternate.cls'
% V2.5 LaTeX2e document class file. It is for those submitting
% articles to ACM Conference Proceedings WHO DO NOT WISH TO
% STRICTLY ADHERE TO THE SIGS (PUBS-BOARD-ENDORSED) STYLE.
% The 'sig-alternate.cls' file will produce a similar-looking,
% albeit, 'tighter' paper resulting in, invariably, fewer pages.
%
% ----------------------------------------------------------------------------------------------------------------
% This .tex file (and associated .cls V2.5) produces:
%       1) The Permission Statement
%       2) The Conference (location) Info information
%       3) The Copyright Line with ACM data
%       4) NO page numbers
%
% as against the acm_proc_article-sp.cls file which
% DOES NOT produce 1) thru' 3) above.
%
% Using 'sig-alternate.cls' you have control, however, from within
% the source .tex file, over both the CopyrightYear
% (defaulted to 200X) and the ACM Copyright Data
% (defaulted to X-XXXXX-XX-X/XX/XX).
% e.g.
% \CopyrightYear{2007} will cause 2007 to appear in the copyright line.
% \crdata{0-12345-67-8/90/12} will cause 0-12345-67-8/90/12 to appear in the copyright line.
%
% ---------------------------------------------------------------------------------------------------------------
% This .tex source is an example which *does* use
% the .bib file (from which the .bbl file % is produced).
% REMEMBER HOWEVER: After having produced the .bbl file,
% and prior to final submission, you *NEED* to 'insert'
% your .bbl file into your source .tex file so as to provide
% ONE 'self-contained' source file.
%
% ================= IF YOU HAVE QUESTIONS =======================
% Questions regarding the SIGS styles, SIGS policies and
% procedures, Conferences etc. should be sent to
% Adrienne Griscti (griscti@acm.org)
%
% Technical questions _only_ to
% Gerald Murray (murray@hq.acm.org)
% ===============================================================
%
% For tracking purposes - this is V2.0 - May 2012

\documentclass{sig-alternate-05-2015}
\usepackage{booktabs}


\begin{document}

% Copyright
%\setcopyright{acmcopyright}
%\setcopyright{acmlicensed}
%\setcopyright{rightsretained}
%\setcopyright{usgov}
%\setcopyright{usgovmixed}
%\setcopyright{cagov}
%\setcopyright{cagovmixed}


% DOI
%\doi{10.475/123_4}

% ISBN
%\isbn{123-4567-24-567/08/06}

%Conference
%\conferenceinfo{PLDI '13}{June 16--19, 2013, Seattle, WA, USA}

%\acmPrice{\$15.00}

%
% --- Author Metadata here ---
%\conferenceinfo{WOODSTOCK}{'97 El Paso, Texas USA}
%\CopyrightYear{2007} % Allows default copyright year (20XX) to be over-ridden - IF NEED BE.
%\crdata{0-12345-67-8/90/01}  % Allows default copyright data (0-89791-88-6/97/05) to be over-ridden - IF NEED BE.
% --- End of Author Metadata ---

%\\TODO:1.tautology prove section 2. set section operation proof 3. function section basic proving onto bijection...
\title{CSCI 3190 \\ Introduction to Discrete Mathematics and Algorithms}
\subtitle{Extended Exercise 2}

\maketitle
\begin{abstract}

\end{abstract}

\keywords{}
\section{Tautology Proof}
Try to prove the followings are tautologies.
\begin{enumerate}
\item $p\Rightarrow (p\vee q )$
\item $\neg p \Rightarrow (p\rightarrow q)$
\item $(p\wedge q)\Rightarrow (p\rightarrow q)$
\item $\neg (p\rightarrow q)\Rightarrow p$
\item $\neg (p \rightarrow q) \Rightarrow \neg q$
\item $(p\wedge q)\Rightarrow p$
\item $(p\wedge (p\rightarrow q))\Rightarrow q$
\item $(\neg q \wedge (p \rightarrow q))\Rightarrow \neg p$
\item $ (p\vee q)\wedge (\neg p) \Rightarrow q$
\item $ ((p \vee q) \wedge (\neg p \vee r)) \Rightarrow (q\vee r)$
\end{enumerate}


\section{Sets}
\begin{enumerate}
\item List the members of these sets.
\begin{enumerate}
	\item $\{x \mid x \text{ is a real number such that } x^2 = 1\}$
	\item $\{x \mid x \text{ is a positive integer less than 12}\}$
	\item $\{x \mid x \text{ is the square of an integer and } x < 100\}$
	\item $\{x \mid x \text{ is an integer such that } x^2 = 2\}$
\end{enumerate}

\item For each of these pairs of sets, determine whether the first
is a subset of the second, the second is a subset of the first,
or neither is a subset of the other.
\begin{enumerate}
	\item the set of airline flights from HK to Auckland,
	the set of nonstop airline flights from HK to
	Auckland
	\item the set of people who speak English, the set of people
	who speak Chinese
	\item the set of flying squirrels, the set of living creatures
	that can fly
\end{enumerate}

\item Determine whether each of these pairs of sets are equal.
\begin{enumerate}
	\item $\{1, 3, 3, 3, 5, 5, 5, 5, 5\}$, $\{5, 3, 1\}$
	\item $\{\{1\}\}$, $\{1, \{1\}\}$
	\item $\emptyset$, $\{\emptyset\}$
\end{enumerate}

\item Determine whether each of these statements is true or
false.
\begin{enumerate}
	\item $0 \in \emptyset$
	\item $\emptyset \in \{0\}$
	\item $\{0\} \subset \emptyset$
	\item $\emptyset \subset \{0\}$
	\item $\{0\} \in \{0\}$
	\item $\{0\} \subset \{0\}$
	\item $\{\emptyset\} \subseteq \{\emptyset\}$
\end{enumerate}

\item Suppose that $A$, $B$, and $C$ are sets such that $A \subseteq B$ and $B \subseteq C$. Show that $A \subseteq C$.

\item What is the cardinality of each of these sets?
\begin{enumerate}
	\item $\{a\}$
	\item $\{\{a\}\}$
	\item $\{a, \{a\}\}$ 
	\item $\{a, \{a\}, \{a, \{a\}\}\}$
\end{enumerate}

\item Find the power set of each of these sets, where $a$ and $b$
are distinct elements.
\begin{enumerate}
	\item $\{a\}$
	\item $\{a, b\}$
	\item $\{\emptyset, \{\emptyset\}\}$
\end{enumerate}

\item How many elements does each of these sets have where
$a$ and $b$ are distinct elements?
\begin{enumerate}
	\item $\mathcal{P}(\{a, b, \{a, b\}\})$
	\item $\mathcal{P}(\{\emptyset, a, \{a\}, \{\{a\}\}\})$
	\item $\mathcal{P}(\mathcal{P}(\emptyset))$
\end{enumerate}

\item This exercise presents \textbf{Russell\textquoteright s paradox}. Let $S$ be the
set that contains a set $x$ if the set $x$ does not belong to
itself, so that $S = \{x | x \notin x\}$.
\begin{enumerate}
	\item Show the assumption that $S$ is a member of $S$ leads to
	a contradiction.
	\item Show the assumption that $S$ is not a member of $S$ leads
	to a contradiction.
\end{enumerate}
\item Describe a procedure for listing all the subsets of a finite
set.
\end{enumerate}

\section{Set Operations}
\begin{enumerate}
\item Let $A$ be the set of students who live within one mile
of school and let $B$ be the set of students who walk to
classes. Describe the students in each of these sets.
\begin{enumerate}
	\item $A \cap B$
	\item $A \cup B$
	\item $A - B$
	\item $B - A$
\end{enumerate}

\item Let $A$ and $B$ be sets. Prove the \textbf{Commutative Laws} by showing that
\begin{enumerate}
	\item $A \cup B = B \cup A$.
	\item $A \cap B = B \cap A$.
\end{enumerate}

\item Let $A$ and $B$ be sets. Prove the \textbf{Absorption Laws} by showing that
\begin{enumerate}
	\item $A \cup (A \cap B) = A$.
	\item $A \cap (A \cup B) = A$.
\end{enumerate}

\item Let $A$ and $B$ be sets. Prove the \textbf{De Morgan's Laws} by showing that
\begin{enumerate}
	\item $\overline{A \cap B} =\overline{A} \cup \overline{B} $.
	\item $\overline{A \cup B} =\overline{A} \cap \overline{B} $.
\end{enumerate}

\item Let A, B, and C be sets. Show that
\begin{enumerate}
	\item $(A\cup B) \subseteq (A\cup B\cup C)$
	\item $(A\cap B\cap C) \subseteq (A\cap B)$
	\item $(A-B)-C \subseteq A-C$
	\item $(A-C)\cap (C-B)=\emptyset$
	\item $(B-A)\cup (C-A) =(B\cup C) - A$
\end{enumerate}
\item Determine whether the symmetric difference is associative;
that is, if $A$, $B$, and $C$ are sets, does it follow that
$A \oplus (B \oplus C) = (A \oplus B) \oplus C$?
\item Suppose that $A$, $B$, and $C$ are sets such that $A \oplus C =
B \oplus C$. Must it be the case that $A = B$?

\item Show that if $A$, $B$, and $C$ are sets, then
\begin{equation}
	|A \cup B \cup C| = |A| + |B| + |C| - |A \cap B| - |A \cap C| - |B \cap C| + |A \cap B \cap C|.
\end{equation}

\end{enumerate}
\section{Functions}

\begin{enumerate}
\item Suppose that $f$ is a function from $A$ to $B$, where $A$ and $B$
are finite sets with $|A| = |B|$. Show that $f$ is one-to-one
if and only if it is onto.

\item Show that if a set $S$ has cardinality $m$, where $m$ is a
positive integer, then there is a one-to-one correspondence
between $S$ and the set $\{1, 2, . . . , m\}$.

\item Show that if $S$ and $T$ are two sets each with $m$ elements,
where $m$ is a positive integer, then there is a
one-to-one correspondence between $S$ and $T$.

\item Show that a set $S$ is infinite if and only if there is a proper
subset $A$ of $S$ such that there is a one-to-one correspondence
between $A$ and $S$.
\end{enumerate}


%\nocite{*}
\bibliographystyle{abbrv}
\bibliography{ref}  % sigproc.bib is the name of the Bibliography in this case
 
\newpage
%APPENDICES are optional
%\balancecolumns
\appendix
%Appendix A
\section{Answer}

\subsection{TAUTOLOGY PROOF}
\begin{enumerate}
\item $p\rightarrow (p\vee q )$

$\Leftrightarrow \neg p \vee (p \vee q)$

$\Leftrightarrow (\neg p \vee p) \vee q$

$\Leftrightarrow T \vee q$ 

$\Leftrightarrow T$
\item 	$\neg p \rightarrow (p\rightarrow q)$

$\Leftrightarrow p \vee (\neg p\vee q)$

$\Leftrightarrow (p \vee \neg p) \vee q$

$\Leftrightarrow T \vee q$
	
$\Leftrightarrow T$
\item $(p\wedge q)\rightarrow (p\rightarrow q)$

$\Leftrightarrow \neg (p\wedge q)\vee (\neg p\vee q)$

$\Leftrightarrow \neg p\vee \neg q\vee \neg p\vee q$

$\Leftrightarrow \neg p\vee (\neg q \vee q)$

$\Leftrightarrow \neg p\vee T$

$\Leftrightarrow  T$
\item $\neg (p\rightarrow q)\rightarrow p$

$\Leftrightarrow  (p\rightarrow q)\vee p$

$\Leftrightarrow  (\neg p\vee q)\vee p$

$\Leftrightarrow  (\neg p\vee p)\vee q$

$\Leftrightarrow  T\vee p$

$\Leftrightarrow  T$
\item $\neg (p \rightarrow q) \rightarrow \neg q$

$\Leftrightarrow  (p \rightarrow q) \vee \neg q$

$\Leftrightarrow  (\neg p \vee q) \vee \neg q$

$\Leftrightarrow  \neg p \vee (q \vee \neg q)$

$\Leftrightarrow  \neg p \vee T$

$\Leftrightarrow   T$
	
\item $(p\wedge q)\rightarrow p$

$\Leftrightarrow \neg (p\wedge q)\vee p$

$\Leftrightarrow \neg p\vee \neg q \vee p$

$\Leftrightarrow (\neg p\vee p )\vee \neg q $

$\Leftrightarrow T\vee \neg q $

$\Leftrightarrow T $
\item $(p\wedge (p\rightarrow q))\rightarrow q$

$\Leftrightarrow \neg (p\wedge (\neg p\vee q))\vee q$

$\Leftrightarrow \neg p\vee \neg(\neg p\vee q)\vee q$

$\Leftrightarrow \neg p\vee (p\wedge \neg q)\vee q$

$\Leftrightarrow ((\neg p\vee p)\wedge (\neg p \vee \neg q))\vee q$

$\Leftrightarrow (T\wedge (\neg p \vee \neg q))\vee q$

$\Leftrightarrow \neg p \vee \neg q \vee q$

$\Leftrightarrow \neg p \vee (\neg q \vee q)$

$\Leftrightarrow \neg p \vee T$

$\Leftrightarrow T $
\item $(\neg q \wedge (p \rightarrow q))\rightarrow \neg p$

$\Leftrightarrow  \neg (\neg q \wedge (\neg p \vee q))\vee \neg p$

$\Leftrightarrow   ( q \vee \neg(\neg p \vee q))\vee \neg p$

$\Leftrightarrow   ( q \vee  (p \wedge \neg q))\vee \neg p$

$\Leftrightarrow   (( q \vee p) \wedge (q\vee \neg q))\vee \neg p$

$\Leftrightarrow   (( q \vee p) \wedge T)\vee \neg p$

$\Leftrightarrow    q \vee (p\vee \neg p)$

$\Leftrightarrow    q \vee T$

$\Leftrightarrow     T$
\item $ (p\vee q)\wedge (\neg p) \rightarrow q$

$\Leftrightarrow \neg ((p\vee q)\wedge (\neg p)) \vee q$

$\Leftrightarrow \neg(p\vee q)\vee \neg (\neg p)) \vee q$

$\Leftrightarrow (\neg p\wedge \neg q )\vee p \vee q$

$\Leftrightarrow ((\neg p \vee p )\wedge (\neg q \vee p)) \vee q$

$\Leftrightarrow (T\wedge (\neg q \vee p)) \vee q$

$\Leftrightarrow (\neg q \vee q) \vee p$

$\Leftrightarrow T \vee p$

$\Leftrightarrow T$
\item $ ((p \vee q) \wedge (\neg p \vee r)) \rightarrow (q\vee r)$

$\Leftrightarrow \neg  ((p \vee q) \wedge (\neg p \vee r)) \vee q\vee r$

$\Leftrightarrow   (\neg(p \vee q) \vee \neg (\neg p \vee r)) \vee q \vee r$

$\Leftrightarrow   (\neg p \wedge \neg q) \vee (p \wedge \neg r) \vee q\vee r$

$\Leftrightarrow   ((\neg p \wedge \neg q)\vee q) \vee (p \wedge \neg r) \vee r$

$\Leftrightarrow   ((\neg p\vee q) \wedge (\neg q\vee q) )\vee ( p \wedge \neg r) \vee r$

$\Leftrightarrow   ((\neg p\vee q) \wedge T )\vee ( p \wedge \neg r) \vee r$

$\Leftrightarrow   \neg p\vee q\vee (p \wedge \neg r) \vee r$

$\Leftrightarrow   \neg p\vee q\vee ((p \vee r)\wedge (\neg r\vee r)) $

$\Leftrightarrow   \neg p\vee q\vee (( p \vee r)\wedge T) $

$\Leftrightarrow   \neg p\vee q\vee p \vee r $

$\Leftrightarrow   (\neg p\vee p)\vee q \vee r $

$\Leftrightarrow   T\vee q \vee r $

$\Leftrightarrow   T $
\end{enumerate}

\subsection{Sets}
\begin{enumerate}
\item 
\begin{enumerate}
	\item $\{-1,1\}$ 
	\item $\{1,2,3,4,5,6,7,8,9,10,11\}$ 
	\item $\{0,1,4, 9, 16, 25, 36, 49, 64, 81\}$ 
	\item $\emptyset$
\end{enumerate}

\item 
\begin{enumerate}
	\item The first is a subset of the
	second, but the second is not a subset of the first. 
	\item Neither
	is a subset of the other. 
	\item The first is a subset of the second, but the second is not a subset of the first.
\end{enumerate}

\item 
\begin{enumerate}
	\item Yes
	\item No 
	\item No
\end{enumerate}

\item 
\begin{enumerate}
	\item False 
	\item False 
	\item False 
	\item True 
	\item False 
	\item False
	\item True
\end{enumerate}

\item Suppose that $x \in A$. Because $A \subseteq B$, this implies that
$x \in B$. Because $B \subseteq C$, we see that $x \in C$. Because
$x \in A$ implies that $x \in C$, it follows that $A \subseteq C$.

\item 
\begin{enumerate}
	\item 1
	\item 1 
	\item 2 
	\item 3
\end{enumerate}

\item 
\begin{enumerate}
	\item $\{\emptyset, \{a\}\}$ 
	\item $\{\emptyset, \{a\}, \{b\}, \{a, b\}\}$
	\item $\{\emptyset, \{\emptyset\}, \{\{\emptyset\}\}, \{\emptyset, \{\emptyset\}\}\}$
\end{enumerate}

\item 
\begin{enumerate}
	\item 8 
	\item 16 
	\item 2
\end{enumerate}

\item 
\begin{enumerate}
	\item If $S$ is a member of itself then according to the definition, $S$ doesn\textquoteright t belong to $S$, contradictrion.
	\item If $S$ isn\textquoteright t a member of itself then according to the definition, $S$ belongs to $S$, contradictrion.
\end{enumerate}

\item Let $S = \{a_1 , a_2 , ..., a_n \}$.
Represent each subset of $S$ with a bit string of length $n$, where
the $i$th bit is 1 if and only if $a_i \in S$. To generate all subsets of
$S$, list all $2^n$ bit strings of length $n$ (for instance, in increasing
order), and write down the corresponding subsets.
\end{enumerate}
\subsection{Set Operations}
\begin{enumerate}
\item 
\begin{enumerate}
	\item The set of students who live within one mile of school
	and walk to classes 
	\item The set of students who live within
	one mile of school or walk to classes (or do both) 
	\item The
	set of students who live within one mile of school but
	do not walk to classes 
	\item The set of students who walk
	to classes but live more than one mile away from school
\end{enumerate}

	\begin{align}
	A \cup B & = \{x \mid x \in A \vee x \in B\}\\
	& = \{x \mid x \in B \vee x \in A\}\\
	& = B \cup A
	\end{align}


\item 
\begin{enumerate}
	\item 
	\begin{align}
		A \cup B & = \{x \mid x \in A \vee x \in B\}\\
		& = \{x \mid x \in B \vee x \in A\}\\
		& = B \cup A
	\end{align}
	\item 
	\begin{align}
		A \cap B & = \{x \mid x \in A \wedge x \in B\}\\
		& = \{x \mid x \in B \wedge x \in A\}\\
		& = B \cap A
	\end{align}
\end{enumerate}

\item 
\begin{enumerate}
	\item Suppose $x \in A \cup (A \cap B)$. Then $x \in A$ or $x \in A \cap B$ by the definition of union. Then by the definition of intersection, $x \in A \cap B \rightarrow x \in A$, we have proved that the left-hand side is a subset of the right-hand side. Conversely, let $x \in A$. Then by the definition of union, $x \in A \cup (A \cap B)$, so the right-hand side is
	a subset of the left-hand side.
	\item Suppose $x \in A \cap (A \cup B)$. Then $x \in A$ and
	$x \in A \cup B$ by the definition of intersection. Because $x \in A$,	we have proved that the left-hand side is a subset of the right-hand side. Conversely, let $x \in A$. Then by the definition of
	union, $x \in A \cup B$ as well. Therefore $x \in A \cap (A \cup B)$
	by the definition of intersection, so the right-hand side is
	a subset of the left-hand side.
\end{enumerate}
\item
\begin{enumerate}
	\item \begin{equation}
	\begin{aligned}
	\overline{A\cap B} =& \{x| x\notin A \cap B\}\\
	=&\{x|\neg (x\in (A\cap B))\}\\
	=&\{x|\neg (x\in A \wedge x\in B)\}\\
	=&\{x|\neg(x\in A) \vee \neg (x\in B)\}\\
	=&\{x|x\notin A \vee x\notin B\}\\
	=&\{x|x\in \overline{A} \vee x \in \overline{B}\}\\
	=&\{x|x\in \overline{A}\cup \overline{B}\}\\
	=&\overline{A} \cup \overline{B} 
	\end{aligned}
	\end{equation}
	\item \begin{equation}
	\begin{aligned}
	\overline{A\cup B} =& \{x| x\notin A \cup B\}\\
	=&\{x|\neg (x\in (A\cup B))\}\\
	=&\{x|\neg (x\in A \vee x\in B)\}\\
	=&\{x|\neg(x\in A) \wedge \neg (x\in B)\}\\
	=&\{x|x\notin A \wedge x\notin B\}\\
	=&\{x|x\in \overline{A} \wedge x \in \overline{B}\}\\
	=&\{x|x\in \overline{A}\cap \overline{B}\}\\
	=&\overline{A} \cap \overline{B} 
	\end{aligned}
	\end{equation}
	
\end{enumerate}


\item 
\begin{align}
	& A \oplus (B \oplus C) \\
	= & (A \cap \overline{B \oplus C}) \cup (\overline{A} \cap (B \oplus C))\\
	= & (A \cap (B \otimes C)) \cup (\overline{A} \cap ((B \cap \overline{C}) \cup (\overline{B} \cap C)))\\
	= & (A \cap ((B \cap C) \cup (\overline{B} \cap \overline{C})) \cup (\overline{A} \cap B \cap \overline{C}) \cup (\overline{A} \cap \overline{B} \cap C)\\
	= & (A \cap B \cap C) \cup (A \cap \overline{B} \cap \overline{C}) \cup (\overline{A} \cap B \cap \overline{C}) \cup (\overline{A} \cap \overline{B} \cap C)\\
	= & (((A \cap B) \cup (\overline{A} \cap \overline{B})) \cap C) \cup (((A \cap \overline{B}) \cup (\overline{A} \cap B)) \cap \overline{C})\\
	= & ((A \otimes B)\cap C) \cup ((A \oplus B) \cap \overline{C})\\
	= & (\overline{A \oplus B} \cap C) \cup ((A \oplus B) \cap \overline{C})\\
	= & (A \oplus B) \oplus C
\end{align}

\item Yes.

\item 
\begin{align}
	&|A \cup B \cup C|\\
	=&|A \cup B| + |C| - |(A \cup B) \cap C|\\
	=&|A| + |B| + |C| - |A \cap B| - |(A \cap C) \cup (B \cap C)|\\
	=&|A| + |B| + |C| - |A \cap B| - |A \cap C| - |B \cap C| + |A \cap B \cap C|
\end{align}
\end{enumerate}
\subsection{Functions}

\begin{enumerate}
\item 
\begin{enumerate}
	\item $\Rightarrow$
		\begin{align}
			&\text{If $f$ is not onto}\\
			&\because f\ \text{is not onto}\\
			&\therefore \exists b \in B, \neg \exists a \in A, f(a) = b\\
			&\because |A| = |B|\\
			&\therefore |f(A)| < |A| = |B|\\
			&\therefore \exists a_1 \ne a_2 \in A, f(a_1) \ne f(a_2)\\
			&\because f\ \text{is not one-to-one, contradiction.}
		\end{align}
	\item $\Leftarrow$
		\begin{align}
			&\text{If $f$ is not one-to-one}\\
			&\because f\ \text{is not one-to-one}\\
			&\therefore \exists a_1 \ne a_2 \in A, f(a_1) \ne f(a_2)\\
			&\because |A| = |B|\\
			&\therefore |f(A)| < |A| = |B|\\
			&\therefore \exists b \in B, \neg \exists a \in A, f(a) = b\\
			&\therefore f\ \text{is not onto, contradiction.}
		\end{align}
\end{enumerate}

\item By definition, to say that $S$ has cardinality
$m$ is to say that $S$ has exactly $m$ distinct elements. Therefore
we can assign the first object to 1, the second to 2, and so on.
This provides the one-to-one correspondence.

\item By last exercise, there is a bijection $f$ from $S$ to ${1, 2, . . . , m}$ and a bijection $g$ from $T$ to $\{1, 2, . . . , m\}$. Then the composition $g^{-1} \circ f$ is
the desired bijection from $S$ to $T$ .

\item 
\begin{enumerate}
	\item If $S$ is finite, then $S$ is not equivalent to a proper subset since they don\textquoteright t have a same cardinality.
	\item If $S$ is infinite, then $S$ has countably infinite subset $\{s_0, s_1, s_2, ...\}$ by the previous exercise. Show that $S$ is
	equivalent to $A = S \backslash \{s_0\}$. Construct the bijection as follows:
	\item Map $\{s_0, s_1, s_2, ...\}$ one-to-one onto $\{s_1, s_2, s_3, ...\}$. 
	\item For every other element $x \in S \backslash \{s_0, s_1, s_2, ...\}$, map $x$ to $x$.
\end{enumerate}
\end{enumerate}


\end{document}
