% This is "sig-alternate.tex" V2.1 April 2013
% This file should be compiled with V2.5 of "sig-alternate.cls" May 2012
%
% This example file demonstrates the use of the 'sig-alternate.cls'
% V2.5 LaTeX2e document class file. It is for those submitting
% articles to ACM Conference Proceedings WHO DO NOT WISH TO
% STRICTLY ADHERE TO THE SIGS (PUBS-BOARD-ENDORSED) STYLE.
% The 'sig-alternate.cls' file will produce a similar-looking,
% albeit, 'tighter' paper resulting in, invariably, fewer pages.
%
% ----------------------------------------------------------------------------------------------------------------
% This .tex file (and associated .cls V2.5) produces:
%       1) The Permission Statement
%       2) The Conference (location) Info information
%       3) The Copyright Line with ACM data
%       4) NO page numbers
%
% as against the acm_proc_article-sp.cls file which
% DOES NOT produce 1) thru' 3) above.
%
% Using 'sig-alternate.cls' you have control, however, from within
% the source .tex file, over both the CopyrightYear
% (defaulted to 200X) and the ACM Copyright Data
% (defaulted to X-XXXXX-XX-X/XX/XX).
% e.g.
% \CopyrightYear{2007} will cause 2007 to appear in the copyright line.
% \crdata{0-12345-67-8/90/12} will cause 0-12345-67-8/90/12 to appear in the copyright line.
%
% ---------------------------------------------------------------------------------------------------------------
% This .tex source is an example which *does* use
% the .bib file (from which the .bbl file % is produced).
% REMEMBER HOWEVER: After having produced the .bbl file,
% and prior to final submission, you *NEED* to 'insert'
% your .bbl file into your source .tex file so as to provide
% ONE 'self-contained' source file.
%
% ================= IF YOU HAVE QUESTIONS =======================
% Questions regarding the SIGS styles, SIGS policies and
% procedures, Conferences etc. should be sent to
% Adrienne Griscti (griscti@acm.org)
%
% Technical questions _only_ to
% Gerald Murray (murray@hq.acm.org)
% ===============================================================
%
% For tracking purposes - this is V2.0 - May 2012

\documentclass{../../cls/sig-alternate-05-2015}

\usepackage{algorithm}
\usepackage{algpseudocode}

\usepackage{booktabs}
\usepackage{color}
\usepackage{enumitem}
\usepackage{mathtools}

\usepackage{ntheorem}
\newtheorem{example}{Example}
\theoremindent 1em
\theoremheaderfont{\scshape}
\theoremseparator{.}
\theorembodyfont{\upshape}
\newtheorem*{solution*}{Solution}

\usepackage{soul}
\usepackage{textcomp}

\begin{document}

% Copyright
%\setcopyright{acmcopyright}
%\setcopyright{acmlicensed}
%\setcopyright{rightsretained}
%\setcopyright{usgov}
%\setcopyright{usgovmixed}
%\setcopyright{cagov}
%\setcopyright{cagovmixed}


% DOI
%\doi{10.475/123_4}

% ISBN
%\isbn{123-4567-24-567/08/06}

%Conference
%\conferenceinfo{PLDI '13}{June 16--19, 2013, Seattle, WA, USA}

%\acmPrice{\$15.00}

%
% --- Author Metadata here ---
%\conferenceinfo{WOODSTOCK}{'97 El Paso, Texas USA}
%\CopyrightYear{2007} % Allows default copyright year (20XX) to be over-ridden - IF NEED BE.
%\crdata{0-12345-67-8/90/01}  % Allows default copyright data (0-89791-88-6/97/05) to be over-ridden - IF NEED BE.
% --- End of Author Metadata ---

\makeatletter
\def\old@comma{,}
\catcode`\,=13
\def,{%
    \ifmmode%
    \old@comma\discretionary{}{}{}%
    \else%
    \old@comma%
    \fi%
}
\makeatother

\title{CSCI 3190 \\ Introduction to Discrete Mathematics and Algorithms}
\subtitle{Sample Solution of Quiz 1}

\maketitle
\begin{abstract}

\end{abstract}

\keywords{}

\section{Equivalence Class}
\begin{example}
    Let \begin{equation}
        A = \{1, 2, 3, 4, 5, 6, 7, 8\} \times \{1, 2, 3, 4, 5, 6, 7, 8\}.
    \end{equation} Define a relation $R$ on $A$ by: $((a, b), (c, d)) \in R$ if and only if $a \times b = c \times d$ \begin{enumerate}[label=(\alph*)]
        \item Show that $R$ is an equivalence relation on $A$.
        \item Determin the equivalence class of \begin{enumerate}[label=(\roman*)]
            \item $[(1, 1)]$,
            \item $[(2, 2)]$,
            \item $[(2, 4)]$,
            \item $[(4, 3)]$.
        \end{enumerate}
    \end{enumerate}
\end{example}
\begin{solution*}
	\begin{enumerate}[label=(\alph*)]
		\item \begin{description}
			\item[Reflexive] \begin{align}
				\begin{aligned}
                    & (a, b) \in A \land a \times b = a \times b\\
                    \Leftrightarrow & ((a, b), (a, b)) \in R.
                \end{aligned}
			\end{align}
			\item[Symmetric] \begin{equation}
				((a, b), (c, d)) \in R \Leftrightarrow a \times b = c \times d \Leftrightarrow ((c, d), (a, b)) \in R.
			\end{equation}
			\item[Transitive] \begin{align}
				\begin{aligned}
				& ((a, b), (c, d)) \in R \land ((c, d), (e, f)) \in R\\
				\Leftrightarrow & a \times b = c \times d = e \times f\\
				\Rightarrow & ((a, b), (e, f)) \in R.
				\end{aligned}
			\end{align}
		\end{description}
		\item \begin{enumerate}[label=(\roman*)]
			\item $\{(1, 1)\}$,
			\item $\{(1, 4), (2, 2), (4, 1)\}$,
			\item $\{(1, 8), (2, 4), (4, 2), (8, 1)\}$,
			\item $\{(2, 6), (3, 4), (4, 3), (6, 2)\}$.
		\end{enumerate}
	\end{enumerate}
\end{solution*}

\section{Reflexive Relation}
\begin{example}
	If $R$ is a relation on a set $A$, prove or disprove that $R^2$ is reflexive $\Rightarrow$ $R$ is reflexive.
\end{example}
\begin{proof}
	Let $A = \{0, 1\}, R = \{(0, 1), (1, 0)\}$ which is not reflexive, then $R^2 = \{(0, 0), (1, 1)\}$ is reflexive.
\end{proof}

\section{Set}
\begin{example}
	Determine  whether each of the following statements is true for arbitrary sets $A, B$, and $C$. Give a counter-example if your answer is ``No''.\begin{enumerate}
		\item If $A \in B$ and $B \subseteq C$, then $A \in C$.
		\item If $A \subseteq B$ and $B \in C$, then $A \subseteq C$.
		\item If $A \subseteq B$ and $B \in C$, then $A \in C$.
	\end{enumerate}
\end{example}
\begin{solution*}
	\begin{enumerate}
		\item Yes.
		\item No. Let $A = \{0\}, B = \{0\}, C = \{\{0\}\}$.
		\item No. Let $A = \emptyset, B = \{0\}, C = \{\{0\}\}$.
	\end{enumerate}
\end{solution*}

\section{Inclusion Exclusion}
\begin{example}
	Out of 30 students, 15 of them have taken the course Discrete Mathematics, 8 of them have taken Algorithms and 6 of them have taken Economics. Moreover, we know that 3 of them have taken all three courses. Let $x$ be the number of students not taking any of these three courses. What is the smallest possible $x$? Explain your answer.
\end{example}
\begin{solution*}
	Let $U$ be the set of all students, $A$ be the set of student taken Algorithms, $D$ be the set of students taken Discrete Mathematics and $E$ be the set of students taken Economics, then \begin{align}
		\begin{aligned}
		x = & \left|\overline{A \cup D \cup E}\right|\\
		= & \left|U\right| - \left|A\right| - \left|D\right| - \left|E\right| + \left|A \cap D\right| + \left|A \cap E\right| + \left|D \cap E\right| - \left|A \cap D \cap E\right|\\
		= & 1 + 2 \left|A \cap D \cap E\right| + \left|A \cap D \cap \overline{E}\right| + \left|A \cap \overline{D} \cap E\right| + \left|\overline{A} \cap D \cap E\right|\\
		\ge & 7.
		\end{aligned}
	\end{align} $x = 7$ when \begin{equation}
		\left|A \cap D \cap \overline{E}\right| = \left|A \cap \overline{D} \cap E\right| = \left|\overline{A} \cap D \cap E\right| = 0.
	\end{equation}
\end{solution*}
\end{document}
