%!TEX program = xelatex
\documentclass[10pt, compress]{beamer}
\usepackage[titleprogressbar]{../../cls/beamerthemem}

\usepackage{booktabs}
\usepackage[scale=2]{ccicons}
\usepackage{minted}

\usepgfplotslibrary{dateplot}

\usemintedstyle{trac}

\setbeamertemplate{caption}[numbered]
\setbeamertemplate{theorems}[numbered]
\newtheorem{crl}{Corollary}[theorem]
\newtheorem*{solution*}{Solution}

\usepackage{algorithm}
\usepackage[noend]{algpseudocode}

\usepackage{version}
%\excludeversion{proof}
%\excludeversion{solution*}

\usepackage{mathtools}
\usepackage{multicol}
\usepackage{qtree}

\usepackage{tikz}

\makeatletter
\def\old@comma{,}
\catcode`\,=13
\def,{%
	\ifmmode%
	\old@comma\discretionary{}{}{}%
	\else%
	\old@comma%
	\fi%
}
\makeatother

\title{CSCI 3190 Tutorial of Week 05}
\subtitle{Quiz 1}
\author{LI Haocheng}
\institute{Department of Computer Science and Engineering}

\begin{document}

\maketitle

\begin{frame}[fragile]
\frametitle{Closures}
\begin{columns}
	\begin{column}{.4\linewidth}
		\begin{definition}
			A \textbf{reflexive closure} of $R$\begin{enumerate}
				\item contains $R$,
				\item is reflexive,
				\item is contained within every reflexive relation that contains $R$.
			\end{enumerate}
		\end{definition}
		\begin{definition}
			A \textbf{diagonal relation} $\Delta$ on $A$ is $\{(a, a) \mid a \in A \}$.
		\end{definition}
	\end{column}
	\begin{column}{.6\linewidth}
		\begin{theorem}
			Given a relation $R$ on a set $A$, the reflexive closure of $R$ equals $R \cup \Delta$.
		\end{theorem}
		\onslide<2>\begin{proof}
			\begin{enumerate}
				\item $R \cup \Delta \supseteq R$.
				\item $R \cup \Delta \supseteq \Delta$.
				\item If there exists a reflexive set $S \supset \Delta$, $r \in R \cup \Delta$, $r \notin S$. Therefore $r \notin \Delta$, $r \in R$ so that $S \nsupseteq R$ which forms a contradictory.
			\end{enumerate}
		\end{proof}
	\end{column}
\end{columns}
\end{frame}

\begin{frame}[fragile]
\frametitle{Symmetric Relation}
\begin{columns}
	\begin{column}{.6\linewidth}
		\onslide<1->\begin{theorem}
			Given a relation $R$ on a set $A$, the reflexive closure of $R$ equals $R \cup R^{-1}$.
		\end{theorem}
		\onslide<2->\begin{proof}
			\begin{enumerate}
				\item $R \cup R^{-1} \supseteq R$.
				\item $(R \cup R^{-1})^{-1} = R^{-1} \cup {R^{-1}}^{-1} = R \cup R^{-1}$.
				\item If there exists a set $S \supseteq R$, $r \in R \cup R^{-1}$, $r \notin S$. Therefore $r \notin R$, $r \in R^{-1}$ so that $S \ne S^{-1}$ which forms a contradictory.
			\end{enumerate}
		\end{proof}
	\end{column}
	\begin{column}{.5\linewidth}
		\onslide<1->\begin{example}
			What is the symmetric closure of the relation $R =\{(a, b) \mid a > b\}$ on the set of positive integers?
		\end{example}
		\onslide<3>\begin{solution*}
			\begin{align}
			\begin{aligned}
			& R \cup R^{-1} \\
			= & \{(a, b) \mid a > b \} \cup \{(a, a) \mid a < b\} \\
			= & \{(a, b) \mid a \ne b\}.
			\end{aligned}
			\end{align}
		\end{solution*}
	\end{column}
\end{columns}
\end{frame}

\begin{frame}[fragile]
\frametitle{Equivalence Relation}
\begin{definition}
	A relation on a set $A$ is called an \textbf{equivalence relation} if it is reflexive, symmetric, and transitive.
\end{definition}
\begin{example}
	Let $R$ be the relation on the set of real numbers such that $(a, b) \in R$ if and only if $a - b$ is an integer. Is $R$ an equivalence relation?
\end{example}
\onslide<2>\begin{solution*}
	Because $a - a = 0$ is an integer for all real numbers $a$, $\forall a \in \mathbb{R}, (a, a) \in R$. Hence, $R$ is reflexive. Now suppose that $(a, b) \in R$. Then $a - b$ is an integer, so $b - a$ is also an integer. Hence, $(b, a) \in R$. It follows that $R$ is symmetric. If $(a, b), (b, c) \in R$, then $a - b$ and $b - c$ are integers. Therefore, $a - c = (a - b) + (b - c)$ is also an integer. Hence, $(a, c) \in R$. Thus, $R$ is transitive. Consequently, $R$ is an equivalence relation.
\end{solution*}
\end{frame}

\begin{frame}[fragile]
\frametitle{Example of Equivalence Relation}
\begin{example}
	Show the smallest equivalence relation contains $\{(a, b), (c, b), (d, c)\}$ on set $\{a, b, c, d, e\}$.
\end{example}
\onslide<2>\begin{solution*}
	\begin{enumerate}
		\item Find Reflexsive Closure: $\{(a, a), (a, b), (b, b), (c, b), (c, c), (d, c), (d, d), (e, e)\}$.
		\item Find Symmetric Closure: $\{(a, a), (a, b), (b, a), (b, b), (b, c), (c, b), (c, c), (c, d), (d, c), (d, d), (e, e)\}$.
		\item Find Transitive Closure: $\{(a, a), (a, b), (a, c), (a, d), (b, a), (b, b), (b, c), (b, d), (c, a), (c, b), (c, c), (c, d), (d, a), (d, b), (d, c), (d, d), (e, e)\}$
	\end{enumerate}
\end{solution*}
\end{frame}

\begin{frame}[fragile]
\frametitle{Equivalence Class}
\onslide<1->\begin{example}
	Let $A = \{1, 2, 3, 4, 5, 6, 7, 8\} \times \{1, 2, 3, 4, 5, 6, 7, 8\}$. Define a relation $R$ on $A$ by: $((a, b), (c, d)) \in R$ if and only if $a \times b = c \times d$ \begin{enumerate}[(a)]
		\item Show that $R$ is an equivalence relation on $A$.
		\item Determin the equivalence class of $[(1, 1)], [(2, 2)], [(2, 4)]$ and $[(4, 3)]$.
	\end{enumerate}
\end{example}
\onslide<2>\begin{solution*}
	\begin{enumerate}[(a)]
		\item \begin{description}
			\item[Reflexive] $(a, b) \in A \land a \times b = a \times b \Leftrightarrow ((a, b), (a, b)) \in R$.
			\item[Symmetric] $((a, b), (c, d)) \in R \Leftrightarrow a \times b = c \times d \Leftrightarrow ((c, d), (a, b)) \in R$.
			\item[Transitive] $((a, b), (c, d)) \in R \land ((c, d), (e, f)) \in R \Leftrightarrow a \times b = c \times d = e \times f \Rightarrow ((a, b), (e, f)) \in R$.
		\end{description}
		\item $\{(1, 1)\}$, $\{(1, 4), (2, 2), (4, 1)\}$, $\{(1, 8), (2, 4), (4, 2), (8, 1)\}$, $\{(2, 6), (3, 4), (4, 3), (6, 2)\}$.
	\end{enumerate}
\end{solution*}
\end{frame}

\begin{frame}[fragile]
\frametitle{Reflexive Relation}
\onslide<1->\begin{example}
	If $R$ is a relation on a set $A$, prove or disprove that $R^2$ is reflexive $\Rightarrow$ $R$ is reflexive.
\end{example}
\onslide<2->\begin{proof}
	Let $A = \{0, 1\}, R = \{(0, 1), (1, 0)\}$ which is not reflexive, then $R^2 = \{(0, 0), (1, 1)\}$ is reflexive.
\end{proof}
\onslide<1->\begin{example}
	What is the reflexive closure of the relation $R = \{(a, b) \mid a < b\}$ on the set of integers?
\onslide<3>\begin{solution*}
	$R \cup \Delta = \{(a, b) \mid a < b \} \cup \{(a, a) \mid a \in \mathbb{Z}\} = \{(a, b) \mid a \le b\}$.
\end{solution*}
\end{example}
\end{frame}

\begin{frame}[fragile]
\frametitle{Set}
\begin{columns}
	\begin{column}{.9\linewidth}
		\onslide<1->\begin{example}
			Determine  whether each of the following statements is true for arbitrary sets $A, B$, and $C$. Give a counter-example if your answer is ``No''.\begin{enumerate}
				\item If $A \in B$ and $B \subseteq C$, then $A \in C$.
				\item If $A \subseteq B$ and $B \in C$, then $A \subseteq C$.
				\item If $A \subseteq B$ and $B \in C$, then $A \in C$.
			\end{enumerate}
		\end{example}
		\onslide<2>\begin{solution*}
			\begin{enumerate}
				\item Yes.
				\item No. Let $A = \{0\}, B = \{0\}, C = \{\{0\}\}$.
				\item No. Let $A = \emptyset, B = \{0\}, C = \{\{0\}\}$.
			\end{enumerate}
		\end{solution*}
	\end{column}
\end{columns}
\end{frame}

\begin{frame}[fragile]
\frametitle{Inclusion Exclusion}
\begin{columns}
	\begin{column}{\linewidth}
		\onslide<1->\begin{example}
			Out of 30 students, 15 of them have taken the course Discrete Mathematics, 8 of them have taken Algorithms and 6 of them have taken Economics. Moreover, we know that 3 of them have taken all three courses. Let $x$ be the number of students not taking any of these three courses. What is the smallest possible $x$? Explain your answer.
		\end{example}
		\onslide<2>\begin{solution*}
			Let $U$ be the set of all students, $A$ be the set of student taken Algorithms, $D$ be the set of students taken Discrete Mathematics and $E$ be the set of students taken Economics, then $x = \left|\overline{A \cup D \cup E}\right| = \left|U\right| - \left|A\right| - \left|D\right| - \left|E\right| + \left|A \cap D\right| + \left|A \cap E\right| + \left|D \cap E\right| - \left|A \cap D \cap E\right| = 1 + 2 \left|A \cap D \cap E\right| + \left|A \cap D \cap \overline{E}\right| + \left|A \cap \overline{D} \cap E\right| + \left|\overline{A} \cap D \cap E\right| \ge 7$. $x = 7$ when $\left|A \cap D \cap \overline{E}\right| = \left|A \cap \overline{D} \cap E\right| = \left|\overline{A} \cap D \cap E\right| = 0$.
		\end{solution*}
	\end{column}
\end{columns}
\end{frame}

\begin{frame}[fragile]
\frametitle{The Pigeonhole Principle}
\begin{example}
	Assume that in a group of six people, each pair of individuals consists of two friends or two enemies.
	Show that there are either three mutual friends or three mutual enemies in the group.
\end{example}
\onslide<2>\begin{solution*}
	Let $A$ be one of the six individuals.
	Following from the generalized pigeonhole principle, of the five other people in the group,
	there are either three or more who are friends or enemies of $A$.
	
	Without loss of generality, suppose that $B$, $C$, and $D$ are friends of $A$.
	If any two of these three people are friends,
	then these two and $A$ form a group of three mutual friends.
	Otherwise, $B$, $C$, and $D$ form a set of three mutual enemies.
\end{solution*}
\end{frame}

\plain{Questions?}

\end{document}
