%!TEX program = xelatex
\documentclass[10pt, compress]{beamer}
\usepackage[titleprogressbar]{../../cls/beamerthemem}

\usepackage{booktabs}
\usepackage[scale=2]{ccicons}
\usepackage{minted}

\usepgfplotslibrary{dateplot}

\usemintedstyle{trac}

\setbeamertemplate{caption}[numbered]
\setbeamertemplate{theorems}[numbered]
\newtheorem{crl}{Corollary}[theorem]
\newtheorem*{solution*}{Solution}

\usepackage{algorithm}
\usepackage[noend]{algpseudocode}

\usepackage{version}
%\excludeversion{proof}
%\excludeversion{solution*}

\usepackage{mathtools}
\usepackage{multicol}
\usepackage{qtree}

\usepackage{tikz}

\makeatletter
\def\old@comma{,}
\catcode`\,=13
\def,{%
	\ifmmode%
	\old@comma\discretionary{}{}{}%
	\else%
	\old@comma%
	\fi%
}
\makeatother

\title{CSCI 3190 Tutorial of Week 05}
\subtitle{Counting}
\author{LI Haocheng}
\institute{Department of Computer Science and Engineering}

\begin{document}

\maketitle

\begin{frame}[fragile]
\frametitle{The Pigeonhole Principle}
\begin{example}
	Assume that in a group of six people, each pair of individuals consists of two friends or two enemies.
	Show that there are either three mutual friends or three mutual enemies in the group.
\end{example}
\onslide<2>\begin{solution*}
	Let A be one of the six people.
	Following from the generalized pigeonhole principle, of the five other people in the group,
	there are either three or more who are friends or enemies of A.
	
	Without loss of generality, suppose that B, C, and D are friends of A.
	If any two of these three individuals are friends, then these two and A form a group of three mutual friends.
	Otherwise, B, C, and D form a set of three mutual enemies.
\end{solution*}
\end{frame}

\begin{frame}[fragile]

\frametitle{Ramsey Number}
\begin{definition}
	The \textbf{Ramsey number} $R(m, n)$, where $m$ and $n$ are positive integers greater than or equal to 2,
	denotes the minimum number of people at a party such that there are either $m$ mutual friends or $n$ mutual enemies,
	assuming that every pair of people at the party are friends or enemies.
\end{definition}
\begin{example}
	Show that in a group of five people (where any two people are either friends or enemies), there are not necessarily three mutual friends or three mutual enemies.
\end{example}
\end{frame}

\begin{frame}[fragile]
\frametitle{$R(2, n)$}
\begin{example}
	Show that if $n$ is an integer with $n \ge 2$, then the Ramsey number $R(2,n)$ equals $n$.
\end{example}
\onslide<2>\begin{solution*}
	We need to show two
	things: that if we have a group of $n$ people, then among them
	we must find either a pair of friends or a subset of $n$ of them
	all of whom are mutual enemies; and that there exists a group
	of $n - 1$ people for which this is not possible.
	
	For the first
	statement, if there is any pair of friends, then the condition is
	satisfied, and if not, then every pair of people are enemies, so
	the second condition is satisfied. For the second statement, if
	we have a group of $n - 1$ people all of whom are enemies of
	each other, then there is neither a pair of friends nor a subset
	of $n$ of them all of whom are mutual enemies.
\end{solution*}
\end{frame}

\begin{frame}[fragile]
\frametitle{Permutation}
\begin{columns}
	\begin{column}{.6\linewidth}
		\begin{definition}
			A \textbf{permutation} of a set of distinct objects is an ordered arrangement of these objects.
		\end{definition}
		\begin{definition}
			An ordered arrangement of $r$ elements of a set is called an \textbf{$r$-permutation}.
		\end{definition}
		\begin{theorem}
			\label{thm:p}
			If $n$ is a positive integer and $r$ is an integer with $1 \le r \le n$, then there are $P(n, r) = n(n - 1)(n - 2) \cdots (n - r + 1)$
			$r$-permutations of a set with $n$ distinct elements.
		\end{theorem}
	\end{column}
	\begin{column}{.5\linewidth}
		\onslide<2>\begin{proof}
			The first element of the
		permutation can be chosen in $n$ ways. There are $n - 1$ ways to choose the second element of the permutation. Similarly, there are $n - 2$ ways to choose the third element, and so on, until there are exactly $n - (r - 1) = n - r + 1$ ways to choose the $r$th element. Consequently, by the product rule, there are
		$n(n - 1)(n - 2) \cdots (n - r + 1)$
		$r$-permutations of the set.
		\end{proof}
	\end{column}
\end{columns}
\end{frame}

\begin{frame}[fragile]
\frametitle{Factorial Expression}
\begin{columns}
	\begin{column}{.4\linewidth}
		\begin{crl}
			If $n$ and $r$ are integers with $0 \le r \le n$, then $P(n, r) = \frac{n!}{(n - r)!}$.
		\end{crl}
		\onslide<2->\begin{proof}
			When $n$ and $r$ are integers with $1 \le r \le n$, we have it by Theorem~\ref{thm:p}. Because $\frac{n!}{(n - 0)!} = 1$ whenever $n \ge 0$, we have it when $r - 0$.
		\end{proof}
	\end{column}
	\begin{column}{.6\linewidth}
		\onslide<1->\begin{example}
			How many ways are there to select a first-prize winner, a second-prize winner, and a third-prize winner from 100 different people who have entered a contest?
		\end{example}
		\onslide<3>\begin{solution*}
			The number is the number of 3-permutations of a set of 100 elements. Consequently, the answer is $P(100, 3) = 100 \times 99 \times 98 = 970200$.
		\end{solution*}
	\end{column}
\end{columns}
\end{frame}

\begin{frame}[fragile]
\frametitle{Example of Permutation}
\begin{example}
	How many times does \texttt{plan}, \texttt{than} and \texttt{both} NOT occur in permutations of the 26 letters?
\end{example}
\onslide<2>\begin{solution*}
	\texttt{plan} occurs $22! \times 23 = 23!$ times.
	
	\texttt{than} occurs $22! \times 23 = 23!$ times.
	
	\texttt{both} occurs $22! \times 23 = 23!$ times.
	
	\texttt{plan} and \texttt{than} occurs $0$ times.
	
	\texttt{plan} and \texttt{both} occurs $18! \times 19 \times 20 = 20!$ times.
	
	\texttt{than} and \texttt{both} occurs $20! \times 21 = 21!$ times.
	
	\texttt{plan}, \texttt{than} and \texttt{both} occurs $0$ times.
	
	So the answer is $26! - 23! \times 3 + 21! + 20!$.
\end{solution*}
\end{frame}

\begin{frame}[fragile]
\frametitle{Combination}
\begin{columns}
	\begin{column}{.4\linewidth}
		\begin{example}
			How many different committees of 3 students can be formed from a group of 4 students?
		\end{example}
		\onslide<2->\begin{solution*}
			4. 
		\end{solution*}
		\onslide<1->\begin{definition}
			An \textbf{$r$-combination} of elements of a set is an unordered selection of $r$ elements from the set.
		\end{definition}
	\end{column}
	\begin{column}{.6\linewidth}
		\begin{theorem}
			The number of $r$-combinations of a set with $n$ elements, where $0 \le r \le n$, equals $C(n, r) = \frac{n!}{r!(n - r)!}$.
		\end{theorem}
		\onslide<3>\begin{proof}
			By the product rule, $P(n, r) = C(n, r) \cdot P(r, r)$, which implies that\begin{equation}
				C(n, r) = \frac{P(n, r)}{P(r, r)} = \frac{\frac{n!}{(n - r)!}}{\frac{r!}{(r - r)!}} = \frac{n!}{r!(n - r)!}.
			\end{equation}
		\end{proof}
	\end{column}
\end{columns}
\end{frame}

\begin{frame}[fragile]
\frametitle{Combinatorial Proof}
\begin{columns}
	\begin{column}{.6\linewidth}
		\begin{crl}
			Let $0 \le r \le n$. Then $C(n, r) = C(n, n - r)$.
		\end{crl}
		\begin{definition}
			A \textbf{combinatorial proof} of an identity is a proof that uses counting arguments to prove that both sides of the identity count the same objects but in different ways or a proof that is based on showing that there is a bijection between the sets of objects counted by the two sides of the identity. These two types of proofs are called \textbf{double counting proofs} and \textbf{bijective proofs}, respectively.
		\end{definition}
	\end{column}
	\begin{column}{.4\linewidth}
		\begin{proof}
			Suppose that $|S| = n$. The function that maps a subset $A$ of $S$ to $\bar{A}$ is a bijection between subsets of $S$ with $r$ elements and subsets with $n - r$ elements. The identity $C(n, r) = C(n, n - r)$ follows because when there is a bijection between two finite sets, the two sets must have the same number of elements.
		\end{proof}
	\end{column}
\end{columns}
\end{frame}

\begin{frame}[fragile]
	\frametitle{Binomial Coefficients}
	\begin{example}
		Show that if $n$ and $k$ are integers with $1 \le k \le n$, then $\binom{n}{k} \le \frac{n^k}{2^{k - 1}}$.
	\end{example}
	\onslide<2>\begin{solution*}
		\begin{align}
		\begin{aligned}
		&\binom{n}{k}\\
		= & \frac{n(n - 1)(n - 2) \cdots (n - k + 1)}{k(k - 1)(k - 2) \cdots 2}\\
		\le & \frac{n \cdot n \cdot \cdots \cdot n}{2 \cdot 2 \cdot \cdots \cdot 2}\\
		= & \frac{n^k}{2^{k - 1}}.
		\end{aligned}
		\end{align}
	\end{solution*}
\end{frame}

\plain{Questions?}

\end{document}
