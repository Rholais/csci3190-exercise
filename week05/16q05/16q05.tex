% This is "sig-alternate.tex" V2.1 April 2013
% This file should be compiled with V2.5 of "sig-alternate.cls" May 2012
%
% This example file demonstrates the use of the 'sig-alternate.cls'
% V2.5 LaTeX2e document class file. It is for those submitting
% articles to ACM Conference Proceedings WHO DO NOT WISH TO
% STRICTLY ADHERE TO THE SIGS (PUBS-BOARD-ENDORSED) STYLE.
% The 'sig-alternate.cls' file will produce a similar-looking,
% albeit, 'tighter' paper resulting in, invariably, fewer pages.
%
% ----------------------------------------------------------------------------------------------------------------
% This .tex file (and associated .cls V2.5) produces:
%       1) The Permission Statement
%       2) The Conference (location) Info information
%       3) The Copyright Line with ACM data
%       4) NO page numbers
%
% as against the acm_proc_article-sp.cls file which
% DOES NOT produce 1) thru' 3) above.
%
% Using 'sig-alternate.cls' you have control, however, from within
% the source .tex file, over both the CopyrightYear
% (defaulted to 200X) and the ACM Copyright Data
% (defaulted to X-XXXXX-XX-X/XX/XX).
% e.g.
% \CopyrightYear{2007} will cause 2007 to appear in the copyright line.
% \crdata{0-12345-67-8/90/12} will cause 0-12345-67-8/90/12 to appear in the copyright line.
%
% ---------------------------------------------------------------------------------------------------------------
% This .tex source is an example which *does* use
% the .bib file (from which the .bbl file % is produced).
% REMEMBER HOWEVER: After having produced the .bbl file,
% and prior to final submission, you *NEED* to 'insert'
% your .bbl file into your source .tex file so as to provide
% ONE 'self-contained' source file.
%
% ================= IF YOU HAVE QUESTIONS =======================
% Questions regarding the SIGS styles, SIGS policies and
% procedures, Conferences etc. should be sent to
% Adrienne Griscti (griscti@acm.org)
%
% Technical questions _only_ to
% Gerald Murray (murray@hq.acm.org)
% ===============================================================
%
% For tracking purposes - this is V2.0 - May 2012

\documentclass{../../cls/sig-alternate-05-2015}
\usepackage{booktabs}
\usepackage{textcomp}


\begin{document}

% Copyright
%\setcopyright{acmcopyright}
%\setcopyright{acmlicensed}
%\setcopyright{rightsretained}
%\setcopyright{usgov}
%\setcopyright{usgovmixed}
%\setcopyright{cagov}
%\setcopyright{cagovmixed}


% DOI
%\doi{10.475/123_4}

% ISBN
%\isbn{123-4567-24-567/08/06}

%Conference
%\conferenceinfo{PLDI '13}{June 16--19, 2013, Seattle, WA, USA}

%\acmPrice{\$15.00}

%
% --- Author Metadata here ---
%\conferenceinfo{WOODSTOCK}{'97 El Paso, Texas USA}
%\CopyrightYear{2007} % Allows default copyright year (20XX) to be over-ridden - IF NEED BE.
%\crdata{0-12345-67-8/90/01}  % Allows default copyright data (0-89791-88-6/97/05) to be over-ridden - IF NEED BE.
% --- End of Author Metadata ---

%\\TODO:1.tautology prove section 2. set section operation proof 3. function section basic proving onto bijection...
\title{CSCI 3190 \\ Introduction to Discrete Mathematics and Algorithms}
\subtitle{Answer of Quiz 1}

\maketitle

\section{Test1}
\subsection{Tautology}
\begin{align}
	& (\neg (q \rightarrow r) \rightarrow (q \vee r))\\
	= & (q \rightarrow r) \vee q \vee r\\
	= & \neg q \vee r \vee q\\
	= & T
\end{align}

\subsection{Equivalence Relation}
\begin{enumerate}
	\item Find Reflexsive Closure: $\{(a, a), (a, b), (b, b), (b, c), \\(c, c), (d, c), (d, d), (e, e)\}$.
	\item Find Symmetric Closure: $\{(a, a), (a, b), (b, a), (b, b), \\(b, c), (c, b), (c, c), (c, d), (d, c), (d, d), (e, e)\}$.
	\item Find Transitive Closure: $\{(a, a), (a, b), (a, c), (a, d), \\(b, a), (b, b), (b, c), (b, d), (c, a), (c, b), (c, c), (c, d), (d, a), \\(d, b), (d, c), (d, d), (e, e)\}$
\end{enumerate}

\subsection{Permutation}
\texttt{plan} occurs $22! \times 23 = 23!$ times.

\texttt{than} occurs $22! \times 23 = 23!$ times.

\texttt{both} occurs $22! \times 23 = 23!$ times.

\texttt{plan} and \texttt{than} occurs $0$ times.

\texttt{plan} and \texttt{both} occurs $18! \times 19 \times 20$ times.

\texttt{than} and \texttt{both} occurs $20! \times 21$ times.

\texttt{plan}, \texttt{than} and \texttt{both} occurs $0$ times.

So the answer is $26! - 23! \times 3 + 21! + 20!$.

\newpage

\subsection{Function}
\begin{enumerate}
	\item $5^7$.
	\item $0$, because $|A| > |B|$.
	\item There are 2 kinds of situations, 1 of them is that 1 of $B$ matches 3 of $A$ and others match 1 of $A$ for each, the other 1 is that 2 of $B$ match 2 of $A$ for each and others match 1 of $A$ for each. We will discuss these 2 situations separately.
	
	For the first situation, we select 1 of 5 in $B$ and then for each of other 4 items in $B$ select 1 different item in $A$ so that there are ${}_5 C_1 \times {}_7 P_4$ functions in the first situation.
	
	For the second situation, we select 2 of 5 in $B$, for each of other 3 items in $B$ select 1 different item in $A$ and then from the rest 4 items in $A$ select 2 of them to match 1 of 2 rest item in $B$ so that there are ${}_5 C_2 \times {}_7 P_3 \times {}_4 C_2$ functions in the section situation.
	
	Totally, there are ${}_5 C_1 \times {}_7 P_4 + {}_5 C_2 \times {}_7 P_3 \times {}_4 C_2 = 16800$ functions.
	
	One of the functions is: 
	
	$\{(1, a), (2, b), (3, c), (4, d), (5, e), (6, a), (7, b)\}$.
\end{enumerate}

\subsection{Generating Function}
Number of newborns is \begin{align}
	n_r = & \begin{cases}
	4 \times 3^r, & r \ge 0\\
	0, & \text{otherwise}.
	\end{cases}\\
	\leftrightarrow & \frac{4}{1 - 3x}
\end{align} so that number of racoons is \begin{align}
	b_r = & \Sigma_{i = r - 5}^{r} n_i\\
	\leftrightarrow & \frac{4(1 - x^6)}{(1 - 3x)(1 - x)}
\end{align}

\clearpage

\section{Test2}
\subsection{Tautology}
\begin{align}
	& (\neg (q \rightarrow \neg r) \rightarrow (q \vee r))\\
	= & (q \rightarrow \neg r) \vee q \vee r\\
	= & \neg q \vee \neg r \vee q \vee r\\
	= & T
\end{align}

\subsection{Equivalence Relation}
\begin{enumerate}
	\item Find Reflexsive Closure: $\{(a, a), (a, b), (b, b), (b, d), \\(c, c), (d, c), (d, d), (e, e)\}$.
	\item Find Symmetric Closure: $\{(a, a), (a, b), (b, a), (b, b), \\(b, d), (c, c), (c, d), (d, b), (d, c), (d, d), (e, e)\}$.
	\item Find Transitive Closure: $\{(a, a), (a, b), (a, c), (a, d), \\(b, a), (b, b), (b, c), (b, d), (c, a), (c, b), (c, c), (c, d), (d, a), \\(d, b), (d, c), (d, d), (e, e)\}$
\end{enumerate}

\subsection{Permutation}
\texttt{game} occurs $22! \times 23 = 23!$ times.

\texttt{meal} occurs $22! \times 23 = 23!$ times.

\texttt{also} occurs $22! \times 23 = 23!$ times.

\texttt{game} and \texttt{meal} occurs $0$ times.

\texttt{game} and \texttt{also} occurs $0$ times.

\texttt{meal} and \texttt{also} occurs $20! \times 21$ times.

\texttt{game}, \texttt{meal} and \texttt{also} occurs $0$ times.

So the answer is $26! - 23! \times 3 + 21!$.

\newpage

\subsection{Function}
\begin{enumerate}
	\item $4^6$.
	\item $0$, because $|A| > |B|$.
	\item There are 2 kinds of situations, 1 of them is that 1 of $B$ matches 3 of $A$ and others match 1 of $A$ for each, the other 1 is that 2 of $B$ match 2 of $A$ for each and others match 1 of $A$ for each. We will discuss these 2 situations separately.
	
	For the first situation, we select 1 of 4 in $B$ and then for each of other 3 items in $B$ select 1 different item in $A$ so that there are ${}_4 C_1 \times {}_6 P_3$ functions in the first situation.
	
	For the second situation, we select 2 of 4 in $B$, for each of other 2 items in $B$ select 1 different item in $A$ and then from the rest 4 items in $A$ select 2 of them to match 1 of 2 rest item in $B$ so that there are ${}_4 C_2 \times {}_6 P_2 \times {}_4 C_2$ functions in the section situation.
	
	Totally, there are ${}_4 C_1 \times {}_6 P_3 + {}_4 C_2 \times {}_6 P_2 \times {}_4 C_2 = 1560$ functions.
	
	One of the functions is: 
	
	$\{(a, 1), (b, 2), (c, 3), (d, 4), (e, 1), (f, 2)\}$.
\end{enumerate}

\subsection{Generating Function}
Number of newborns is \begin{align}
	n_r = & \begin{cases}
		4 \times 3^r, & r \ge 0\\
		0, & \text{otherwise}.
	\end{cases}\\
	\leftrightarrow & \frac{4}{1 - 3x}
\end{align} so that number of racoons is \begin{align}
	b_r = & \Sigma_{i = r - 5}^{r} n_i\\
	\leftrightarrow & \frac{4(1 - x^6)}{(1 - 3x)(1 - x)}
\end{align}

\nocite{*}
\bibliographystyle{abbrv}
\bibliography{ref}  % sigproc.bib is the name of the Bibliography in this case
 
\clearpage
%APPENDICES are optional
%\balancecolumns
%\appendix
%Appendix A

\end{document}
