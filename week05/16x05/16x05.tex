% This is "sig-alternate.tex" V2.1 April 2013
% This file should be compiled with V2.5 of "sig-alternate.cls" May 2012
%
% This example file demonstrates the use of the 'sig-alternate.cls'
% V2.5 LaTeX2e document class file. It is for those submitting
% articles to ACM Conference Proceedings WHO DO NOT WISH TO
% STRICTLY ADHERE TO THE SIGS (PUBS-BOARD-ENDORSED) STYLE.
% The 'sig-alternate.cls' file will produce a similar-looking,
% albeit, 'tighter' paper resulting in, invariably, fewer pages.
%
% ----------------------------------------------------------------------------------------------------------------
% This .tex file (and associated .cls V2.5) produces:
%       1) The Permission Statement
%       2) The Conference (location) Info information
%       3) The Copyright Line with ACM data
%       4) NO page numbers
%
% as against the acm_proc_article-sp.cls file which
% DOES NOT produce 1) thru' 3) above.
%
% Using 'sig-alternate.cls' you have control, however, from within
% the source .tex file, over both the CopyrightYear
% (defaulted to 200X) and the ACM Copyright Data
% (defaulted to X-XXXXX-XX-X/XX/XX).
% e.g.
% \CopyrightYear{2007} will cause 2007 to appear in the copyright line.
% \crdata{0-12345-67-8/90/12} will cause 0-12345-67-8/90/12 to appear in the copyright line.
%
% ---------------------------------------------------------------------------------------------------------------
% This .tex source is an example which *does* use
% the .bib file (from which the .bbl file % is produced).
% REMEMBER HOWEVER: After having produced the .bbl file,
% and prior to final submission, you *NEED* to 'insert'
% your .bbl file into your source .tex file so as to provide
% ONE 'self-contained' source file.
%
% ================= IF YOU HAVE QUESTIONS =======================
% Questions regarding the SIGS styles, SIGS policies and
% procedures, Conferences etc. should be sent to
% Adrienne Griscti (griscti@acm.org)
%
% Technical questions _only_ to
% Gerald Murray (murray@hq.acm.org)
% ===============================================================
%
% For tracking purposes - this is V2.0 - May 2012

\documentclass{../../cls/sig-alternate-05-2015}
\usepackage{booktabs}
\usepackage{textcomp}


\begin{document}

% Copyright
%\setcopyright{acmcopyright}
%\setcopyright{acmlicensed}
%\setcopyright{rightsretained}
%\setcopyright{usgov}
%\setcopyright{usgovmixed}
%\setcopyright{cagov}
%\setcopyright{cagovmixed}


% DOI
%\doi{10.475/123_4}

% ISBN
%\isbn{123-4567-24-567/08/06}

%Conference
%\conferenceinfo{PLDI '13}{June 16--19, 2013, Seattle, WA, USA}

%\acmPrice{\$15.00}

%
% --- Author Metadata here ---
%\conferenceinfo{WOODSTOCK}{'97 El Paso, Texas USA}
%\CopyrightYear{2007} % Allows default copyright year (20XX) to be over-ridden - IF NEED BE.
%\crdata{0-12345-67-8/90/01}  % Allows default copyright data (0-89791-88-6/97/05) to be over-ridden - IF NEED BE.
% --- End of Author Metadata ---

%\\TODO:1.tautology prove section 2. set section operation proof 3. function section basic proving onto bijection...
\title{CSCI 3190 \\ Introduction to Discrete Mathematics and Algorithms}
\subtitle{Extended Exercise 5}

\maketitle
\begin{abstract}

\end{abstract}

\keywords{}

\section{Induction and Recursion}
\subsection{Mathematical Induction}
\begin{enumerate}
\item Prove \textbf{Bernoulli’s inequality}: if $h > -1$, then $1+nh \le (1+h)n$ for all non-negative integers $n$.

\item Suppose that $a$ and $b$ are real numbers with $0 < b < a$.
Prove that if $n$ is a positive integer, then $a^n - b^n \le na^{n - 1}(a - b)$.
\end{enumerate}

\nocite{*}
\bibliographystyle{abbrv}
\bibliography{ref}  % sigproc.bib is the name of the Bibliography in this case
 
\clearpage
%APPENDICES are optional
%\balancecolumns
\appendix
%Appendix A
\section{Answer}
\subsection{Induction and Recursion}
\subsubsection{Mathematical Induction}
\begin{enumerate}
\item Let $P(n)$ be \textquotedblleft $1 + nh \le (1 + h)^n, h > -1$\textquotedblright.
Basis step: $P(0)$ is true because $1 + 0 \cdot h = 1 \le 1 = (1 + h)^0$.
Inductive step: Assume $1 + kh \le (1 + h)^k$. Then because
$(1+h) > 0$, $(1 + h)^{k + 1} = (1 + h)(1 + h)^k \ge (1 + h)(1 + k^h) = 1 + (k + 1)^h + kh^2 \ge 1 + (k + 1)h$.
	
\item Let $P(n)$ be \textquotedblleft $a^n - b^n \le na^{n - 1}(a - b)$\textquotedblright. Basis step: $P(1)$ is true because $a^1 - b^1 = a - b \le 1 \cdot a^0 (a - b)$. Inductive step: Assume $a^k - b^k \le k a^{k - 1}(a - b)$. Then because $0 < b < a$: \begin{align}
	a^{k + 1} - b^{k + 1} = & (a - b)\Sigma_{i = 0}^k a^i b^{k - i}\\
	= & a^k(a - b) + b(a - b) \Sigma_{i = 0}^{k - 1} a^i b^{k - 1 - i}\\
	= & a^k(a - b) + b(a^k - b^k)\\
	\le & a^k(a - b) + k a^{k - 1} b(a - b)\\
	\le & (k + 1) a^k (a - b).
\end{align}
\end{enumerate}

\end{document}
