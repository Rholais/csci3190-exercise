%!TEX program = xelatex
\documentclass[10pt, compress]{beamer}
\usetheme[titleprogressbar]{m}

\usepackage{booktabs}
\usepackage[scale=2]{ccicons}
\usepackage{minted}

\usepgfplotslibrary{dateplot}

\usemintedstyle{trac}

\setbeamertemplate{theorems}[numbered]
\newtheorem{crl}{Corollary}[theorem]

\usepackage{multicol}

\makeatletter
\def\old@comma{,}
\catcode`\,=13
\def,{%
	\ifmmode%
	\old@comma\discretionary{}{}{}%
	\else%
	\old@comma%
	\fi%
}
\makeatother

\title{CSCI 3190 Tutorial of Week 8}
\subtitle{Trees}
\author{LI Haocheng}
\institute{Department of Computer Science and Engineering}

\begin{document}

\maketitle

\begin{frame}[fragile]
	\frametitle{Generating Function}
	\onslide<1->\begin{example}
		Let the lifespan of a raccoon be exactly 6 years. Suppose there are 4 new-born raccoons at the 0th year and the number of new-born raccoons in each year is 3 times that in the previous year. Let $b_r$ be the number of raccoons on year $r$ where $r \ge 0$. Give a closed form generating function for $b_r$.
	\end{example}
	
	\onslide<2>\textbf{Solution} Number of newborns is \begin{equation}
	n_r = \begin{cases}
	4 \times 3^r, & r \ge 0\\
	0, & \text{otherwise}.
	\end{cases} \leftrightarrow \frac{4}{1 - 3x},
	\end{equation} so that number of racoons is \begin{equation}
	b_r = \Sigma_{i = r - 5}^{r} n_i \leftrightarrow \frac{4(1 - x^6)}{(1 - 3x)(1 - x)}.
	\end{equation}
\end{frame}

\begin{frame}[fragile]
	\frametitle{$n \log n$}
	\onslide<1->\begin{example}
		Show that $n \log n$ is $O(\log n!)$.
	\end{example}
	
	\onslide<2>\begin{proof}
		We can easily show that $(n - i)(i + 1) = n + (n - i - 1) i \ge n$ for $0 \le i \le n - 1$. Hence, $(n!)^2 = (n \cdot 1)((n - 1) \cdot 2) \cdot ((n - 2) \cdot 3) \cdots (2 \cdot (n -
		1)) \cdot (1 \cdot n) \ge n^n$. Therefore, $2 log n! \ge n log n$.
	\end{proof}
\end{frame}

\begin{frame}[fragile]
	\frametitle{Subset}
	\onslide<1->\begin{example}
		Let $S$ be a set of seven positive integers the maximum of which is at most 24. Prove that the sums of the elements in all the nonempty subsets of $S$ cannot be distinct.
	\end{example}
	
	\onslide<2>\begin{proof}
		Consider subset of size at most 5. Their total sum is at most $20 + 21 + \cdots + 24 = 110$. There are $128 - 1 - 1 - 7 = 119$ non-empty subsets with at most 5 elements so at least 2 of them have equal sum.
	\end{proof}
\end{frame}

\begin{frame}[fragile]
	\frametitle{Function}
	\onslide<1->\begin{example}
		Let $A_1$, $A$ and $B$ be sets such that $\{1, 2, 3, 4, 5\} = A_1 \subset A$, $B = \{s, t, u, v, w, x\}$ and $f$ is a function from $A_1$ to $B$. If $f$ can be extended to $A$ (by defining the mapping for those in $A - A_1$) in 216 ways, what is $|A|$?
	\end{example}
	\onslide<2>\textbf{Solution} $|A - A_1| = 6$ so that $|A| = 11$.
\end{frame}

\begin{frame}[fragile]
	\frametitle{Distributive Property}
	\begin{example}
		For any $A, B, C \subseteq U$, prove that $(A - B) - C = (A - C) - (B - C)$ where the notation $X - Y$ where $X$ and $Y$ are sets denotes the set of all elements that are in $X$ but not in $Y$.
	\end{example}
	\begin{proof}
		\begin{align}
			(A - C) - (B - C) & = A \cap \overline{C} \cap \overline{B \cap \overline{C}}\\
			& = A \cap \overline{C} \cap (\overline{B} \cup C)\\
			& = A \cap \overline{B} \cap \overline{C}\\
			& = (A - B) - C
		\end{align}
	\end{proof}
\end{frame}

\plain{Questions?}

\end{document}