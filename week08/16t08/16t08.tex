%!TEX program = xelatex
\documentclass[10pt, compress, handout]{beamer}
\usepackage[titleprogressbar]{../../cls/beamerthemem}

\usepackage{booktabs}
\usepackage[scale=2]{ccicons}
\usepackage{minted}

\usepgfplotslibrary{dateplot}

\usemintedstyle{trac}

\setbeamertemplate{caption}[numbered]
\setbeamertemplate{theorems}[numbered]
\newtheorem{crl}{Corollary}[theorem]

\usepackage{multicol}
\usepackage{qtree}

\makeatletter
\def\old@comma{,}
\catcode`\,=13
\def,{%
	\ifmmode%
	\old@comma\discretionary{}{}{}%
	\else%
	\old@comma%
	\fi%
}
\makeatother

\title{CSCI 3190 Tutorial of Week 8}
\subtitle{Trees}
\author{LI Haocheng}
\institute{Department of Computer Science and Engineering}

\begin{document}

\maketitle

\begin{frame}[fragile]
	\frametitle{Circuit}
	\onslide<1->\begin{example}
		Let $P_1$ and $P_2$ be two paths without cycle between the vertices $u$ and $v$ in the simple graph $G$ that do not contain the same set of edges. Show that there is a circuit in $G$.
	\end{example}
	\onslide<2>\textbf{Proof.}	Let
	the paths $P_1$ and $P_2$ be $u = x_0, x_1, \cdots, x_n = v$ and $u = y_0, y_1$, $\cdots$, $y_m = v$, respectively. Since the paths do not contain cycles, they must diverge eventually. We can suppose that $x_0 = y_0, x_1 = y_1$, $\cdots$, $x_i = y_i$, but $x_{i + 1} \ne y_{i + 1}$. To form our circuit, we follow the path $y_i, y_{i + 1}, y_{i + 2}$, and so on,
	until it once again first encounters a vertex on $P_1$. Once we are back on $P_1$, we
	follow it along-forwards or backwards, as necessary-to return
	to $x_i$. Since $x_i = y_i$, this certainly forms a circuit.
\end{frame}

\begin{frame}[allowframebreaks]
	\frametitle{Tree}
	\begin{theorem}
		An undirected graph is a tree if and only if there is a unique path without cycle between any two of its vertices.
	\end{theorem}
	\textbf{Proof.} First assume that $T$ is a tree. Then $T$ is a connected graph without circuit. Let $x$ and $y$ be two vertices of $T$ . Because $T$ is connected, there is a path without cycle between $x$ and $y$. Moreover, this path must be unique, for if there were a second such path, the path formed by combining the first path from $x$ to $y$ followed by the path from $y$ to $x$ obtained by reversing the order of the second path from $x$ to $y$ would form a circuit. 
	
	\newpage
	
	This implies that there is a circuit in $T$. Hence, there is a unique path without cycle between any two vertices of a tree. 
	
	Now assume that there is a unique path without cycle between any two vertices of a graph $T$. Then $T$ is connected, because there is a path between any two of its vertices. Furthermore, $T$ can have no circuits. To see that this is true, suppose $T$ had a circuit that contained the vertices $x$ and $y$. Then there would be two simple paths between $x$ and $y$, because the circuit is made up of a path from $x$ to $y$ and a second path from $y$ to $x$. Hence, a graph with a unique simple path between any two vertices is a tree.
\end{frame}

\begin{frame}[fragile]
	\frametitle{Terminology}
	\begin{columns}
		\begin{column}{.6\linewidth}
			\onslide<1->\begin{example}
				In the rooted tree $T$ (with root $a$) shown in Figure \ref{f-11-1-5}, find the parent of $c$, the children of $g$, the
				siblings of $h$, all ancestors of $e$, all descendants of $b$, all internal vertices, and all leaves. What
				is the subtree rooted at $g$?
			\end{example}
			\onslide<2>\textbf{Solution} The parent of $c$ is $b$. The children of $g$ are $h$, $i$, and $j$. The siblings of $h$ are $i$ and $j$. The ancestors of $e$ are $c$, $b$, and $a$. The descendants of $b$ are $c$, $d$, and $e$. The internal vertices are $a$, $b$, $c$, $g$, $h$, and $j$. The leaves are $d$, $e$, $f$, $i$, $k$, $l$, and $m$.
		\end{column}
		\onslide<1->\begin{column}{.4\linewidth}
			\begin{figure}
				\centering
				$\Tree [.a [.b [.c d e ] ] f [.g [.h k ] i [.j l m ]]]$
				\caption{A Rooted Tree $T$}
				\label{f-11-1-5}
			\end{figure}
		\end{column}
	\end{columns}
\end{frame}

\begin{frame}[fragile]
	\frametitle{Left Child}
	\begin{columns}
		\begin{column}{.6\linewidth}
			\onslide<1->\begin{example}
				What are the left and right children of $d$ in the binary tree $T$ shown in Figure \ref{f-11-1-8} (where the order is that implied by the drawing)? What are the left and right subtrees of $c$?
			\end{example}
			\onslide<2>\textbf{Solution} The left child of $d$ is $f$ and the right child is $g$.
		\end{column}
		\onslide<1->\begin{column}{.4\linewidth}
			\begin{figure}
				\centering
				$\Tree [.a [.b [.d f g ] e ] [.c [.h j ] [.i k [.l m ]]]]$
				\caption{A Binary Tree $T$}
				\label{f-11-1-8}
			\end{figure}
		\end{column}
	\end{columns}
\end{frame}

\begin{frame}[fragile]
	\frametitle{$m$-ary Tree}
	\onslide<1->\begin{theorem}\label{t-11-1-3}
		A full $m$-ary tree with $i$ internal vertices contains $n = m i + 1$ vertices.
	\end{theorem}
	\onslide<2>\begin{proof}
		Every vertex, except the root, is the child of an internal vertex. Because each of the $i$ internal vertices has $m$ children, there are $m i$ vertices in the tree other than the root. Therefore, the tree contains $n = m i + 1$ vertices
	\end{proof}
\end{frame}

\begin{frame}[allowframebreaks]
	\frametitle{Full $m$-ary Tree}
	\onslide<1->\begin{theorem}\label{t-11-1-4}
		A full $m$-ary tree with \begin{enumerate}
			\item $n$ vertices has $i = \frac{n - 1}{m}$ internal vertices and $l =\frac{(m - 1)n + 1}{m}$ leaves,
			\item $i$ internal vertices has $n = mi + 1$ vertices and $l = (m - 1) i + 1$ leaves,
			\item $l$ leaves has $n = \frac{m l - 1}{m - 1}$ vertices and $i = \frac{l - 1}{m - 1}$ internal vertices.
		\end{enumerate}
	\end{theorem}
	\onslide<2>\textbf{Proof.} The three parts of the theorem can all be proved using the equality given in Theorem \ref{t-11-1-3}, that is, $n = m i + 1$, together with the equality $n = l + i$, which is true because each vertex is either a leaf or an internal vertex. 
		
	For example, solving for $i$ in $n = m i + 1$ gives $i = \frac{n - 1}{m}$. Then inserting this expression for $i$ into the equation $n = l + i$ shows that $l = n - i = n - \frac{n - 1}{m} = \frac{(m - 1)n + 1}{m}$.
\end{frame}

\begin{frame}[fragile]
	\frametitle{Chain Letter}
	\begin{example}
		Suppose that someone starts a chain letter. Each person who receives the letter is asked to send it on to four other people. Some people do this, but others do not send any letters. How many people have seen the letter, including the first person, if no one receives more than one letter and if the chain letter ends after there have been 100 people who read it but did not send it out? How many people sent out the letter?
	\end{example}
	\textbf{Solution} The chain letter can be represented using a 4-ary tree. The internal vertices correspond to people who sent out the letter, and the leaves correspond to people who did not send it out. Because 100 people did not send out the letter, the number of leaves in this rooted tree is
	$l = 100$. Hence, Theorem \ref{t-11-1-4} shows that the number of people who have seen the letter is $n = \frac{4 \times 100 - 1}{4 - 1} = 133$. Also, the number of internal vertices is $133 - 100 = 33$, so 33 people sent out the letter.
\end{frame}

\plain{Questions?}

\end{document}
