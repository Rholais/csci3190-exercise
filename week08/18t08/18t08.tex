%!TEX program = xelatex
\documentclass[10pt, compress, handout]{beamer}
\usepackage[titleprogressbar]{../../cls/beamerthemem}

\setbeamertemplate{caption}[numbered]
\setbeamertemplate{theorems}[numbered]
\newcounter{example}
\resetcounteronoverlays{example}
\newtheorem{crl}{Corollary}[theorem]
\newtheorem{eg}[example]{Example}
\newtheorem*{solution*}{Solution}

\usepackage{booktabs}
\usepackage[scale=2]{ccicons}
\usepackage{minted}

\usepackage{cleveref}
\crefname{example}{Example}{Examples}

\usepgfplotslibrary{dateplot}

\usemintedstyle{trac}

\usepackage{algorithm}
\usepackage[noend]{algpseudocode}
\resetcounteronoverlays{algorithm}

\usepackage{version}
%\excludeversion{proof}
%\excludeversion{solution*}

\usepackage{mathtools}
\usepackage{multicol}
\usepackage{qtree}

\usepackage{tikz}

\makeatletter
\def\old@comma{,}
\catcode`\,=13
\def,{%
    \ifmmode%
    \old@comma\discretionary{}{}{}%
    \else%
    \old@comma%
    \fi%
}
\makeatother

\title{CSCI 3190 Tutorial of Week 08}
\subtitle{Assignment 2}
\author{LI Haocheng}
\institute{Department of Computer Science and Engineering}

\begin{document}

\maketitle

\begin{frame}
\frametitle{Pigeonhole Principle}
\begin{columns}
    \begin{column}{0.6\linewidth}
        \onslide<1->\begin{eg}
            How many numbers must be selected from the set $\{1, 3, 5, 7, 9, 11, 13, 15\}$ to guarantee that at
            least one pair of these numbers add up to 16? Explain your answer using the pigeonhole principle.
        \end{eg}
        \onslide<2->\begin{solution*}
            There are four pairs can add up to 16. Form the set into four pigeonholes accordingly, let the selected numbers be pigeons. At least five pigeons must be selected to guarantee that at least one hole has two pigeons.
        \end{solution*}
    \end{column}
    \begin{column}{0.5\linewidth}
        \onslide<1->\begin{eg}
            Give a recursive algorithm for finding the sum of the first $n$ positive integers.
        \end{eg}
        \onslide<3->\begin{solution*}
            \begin{algorithm}[H]
                \caption{A Recursive Algorithm for Summation}
                \label{a-2}
                \begin{algorithmic}
                    \Procedure{sum}{$n \in \mathbb{N}$}
                    \If{$n = 0$}
                    \State\Return $0$
                    \EndIf
                    \State\Return $n + $\Call{sum}{$n - 1$}
                    \EndProcedure
                \end{algorithmic}
            \end{algorithm}
        \end{solution*}
    \end{column}
\end{columns}
\end{frame}

\begin{frame}
\frametitle{Hanoi Tower}
\begin{columns}
    \begin{column}{0.6\linewidth}
        \onslide<1->\begin{eg}
            Consider the Tower of Hanoi problem with peg A, B, C. We want to transfer a tower of $n$ disks
            from A to B, if direct moves between A and B are disallowed. \begin{enumerate}
                \item Write a recursive pseudocode.
                \item Set up a recurrence equation to count the number of steps to move $n$ disks.
            \end{enumerate}
        \end{eg}
        \begin{solution*}
            \begin{enumerate}
                \item<2-> See~\Cref{a:5-1}.
                \item<3> Let the number of steps to move $n$ disks be $\{s_n\}$, then $s_0 = 0, s_n = 3s_{n - 1} + 2$.
            \end{enumerate}
        \end{solution*}
    \end{column}
    \begin{column}{0.5\linewidth}
        \onslide<2->\begin{algorithm}[H]
            \caption{Tower of Hanoi}
            \label{a:5-1}
            \begin{algorithmic}
                \Procedure{Hanoi}{$n$, $A$, $B$, $C$}
                \If{$n$ = 0}
                \State \Return
                \Else
                \State \Call{Hanoi}{$n - 1$, $A$, $B$, $C$}
                \State \Call{Move}{$A$, $C$}
                \State \Call{Hanoi}{$n - 1$, $B$, $A$, $C$}
                \State \Call{Move}{$C$, $B$}
                \State \Call{Hanoi}{$n - 1$, $A$, $B$, $C$}
                \EndIf
                \EndProcedure
            \end{algorithmic}
        \end{algorithm}
    \end{column}
\end{columns}
\end{frame}

\plain{Questions?}

\end{document}
