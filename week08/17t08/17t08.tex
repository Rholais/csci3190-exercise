%!TEX program = xelatex
\documentclass[10pt, compress, handout]{beamer}
\usepackage[titleprogressbar]{../../cls/beamerthemem}

\usepackage{booktabs}
\usepackage[scale=2]{ccicons}
\usepackage{minted}

\usepgfplotslibrary{dateplot}

\usemintedstyle{trac}

\setbeamertemplate{caption}[numbered]
\setbeamertemplate{theorems}[numbered]
\newtheorem{crl}{Corollary}[theorem]
\newtheorem*{solution*}{Solution}

\usepackage{algorithm}
\usepackage[noend]{algpseudocode}

\usepackage{version}
%\excludeversion{proof}
%\excludeversion{solution*}

\usepackage{mathtools}
\usepackage{multicol}
\usepackage{qtree}

\usepackage{tikz}

\makeatletter
\def\old@comma{,}
\catcode`\,=13
\def,{%
    \ifmmode%
    \old@comma\discretionary{}{}{}%
    \else%
    \old@comma%
    \fi%
}
\makeatother

\title{CSCI 3190 Tutorial of Week 08}
\subtitle{Quiz 1}
\author{LI Haocheng}
\institute{Department of Computer Science and Engineering}

\begin{document}

\maketitle

\begin{frame}[fragile]
\frametitle{Pigeonhole Principle}
\begin{columns}
    \begin{column}{.6\linewidth}
        \begin{example}
            Let $S$ be a set of 7 positive integers the maximum of which is at most 24. Prove that the sums of the elements in all the nonempty subsets of $S$ cannot be distinct.
        \end{example}
        \onslide<2->\begin{proof}
            Consider nonempty subsets of size at most 5. Their maximum sum is at most $20 + 21 + \cdots + 24 = 110$. There are $126 - 7 = 119$ nonempty subsets with at most 5 elements so at least 2 of them have equal sum.
        \end{proof}
    \end{column}
    \begin{column}{.5\linewidth}
        \onslide<1->\begin{example}
            Given a closed form expression for the generating function of the following sequence:\begin{equation}
            1, -2, 3, -4, 5, -6, \cdots
            \end{equation}
        \end{example}
        \onslide<3>\begin{solution*}
            \begin{align}
                a_r & = -1^{r + 1} & \leftrightarrow & -\frac{1}{1 + x},\\
                b_r & = -1^r (r + 1) & \leftrightarrow & \frac{1}{(1 + x)^2}.
            \end{align}
        \end{solution*}
    \end{column}
\end{columns}
\end{frame}

\plain{Questions?}

\end{document}
