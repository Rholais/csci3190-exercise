%!TEX program = xelatex
\documentclass[10pt, compress, handout]{beamer}
\usepackage[titleprogressbar]{../../cls/beamerthemem}

\usepackage{booktabs}
\usepackage[scale=2]{ccicons}
\usepackage{minted}

\usepgfplotslibrary{dateplot}

\usemintedstyle{trac}

\setbeamertemplate{caption}[numbered]
\setbeamertemplate{theorems}[numbered]
\newtheorem{crl}{Corollary}[theorem]
\newtheorem*{solution*}{Solution}

\usepackage{algorithm}
\usepackage[noend]{algpseudocode}

\usepackage{version}
%\excludeversion{proof}
%\excludeversion{solution*}

\usepackage{mathtools}
\usepackage{multicol}
\usepackage{qtree}

\usepackage{tikz}

\makeatletter
\def\old@comma{,}
\catcode`\,=13
\def,{%
	\ifmmode%
	\old@comma\discretionary{}{}{}%
	\else%
	\old@comma%
	\fi%
}
\makeatother

\title{CSCI 3190 Tutorial of Week 08}
\subtitle{Quiz 1}
\author{LI Haocheng}
\institute{Department of Computer Science and Engineering}

\begin{document}

\maketitle

\begin{frame}[fragile]
\frametitle{Tautologies}
\begin{example}
	\begin{enumerate}
		\item Determine if the following is a tautology. Show your proof. \begin{equation}
			(p \to q) \to ((p \lor q) \to q).
		\end{equation}
		\item Determine if the followings are logically equivalent. Show your proof. \begin{equation}
		(p \to q) \land (p \to r) \text{~and~} q \to r.
		\end{equation}
	\end{enumerate}
\end{example}
\begin{proof}
	\begin{enumerate}
		\item<2-> $((p \to q) \to ((p \lor q) \to q)) \equiv \neg (\neg p \lor q) \lor (\neg(p \lor q) \lor q)$
		$\equiv ((p \land \neg q) \lor (\neg p \land \neg q) \lor q)$
		$\equiv (((p \lor \neg q) \land (\neg p \lor \neg q) \land \neg q) \lor q ) \equiv ((\neg q \land \neg q) \lor q) \equiv T$.
		\item<3-> Let $p = 0, q = 1, r = 0$, then $lhs = 1, rhs = 0$.
	\end{enumerate}
\end{proof}
\end{frame}

\begin{frame}[fragile]
\frametitle{Distributive Property}
\begin{example}
	Prove that \begin{enumerate}
		\item $(A - B) - (A - C) \equiv A \cap (C - B)$;
		\item $(A - C) - (B - C) \equiv (A - B) - C$.
	\end{enumerate}
\end{example}
\begin{proof}
	\begin{columns}
		\onslide<2->\begin{column}{.4\linewidth}
			\begin{align}
				\begin{aligned}
					& (A - B) - (A - C)\\
					\equiv & (A \cap \overline{B}) \cap \overline{A \cap \overline{C}}\\
					\equiv & A \cap \overline{B} \cap (\overline{A} \cup C)\\
					\equiv & (A \cap \overline{A} \cap \overline{B}) \cup (A \cap \overline{B} \cap C)\\
					\equiv & A \cap (C - B).
				\end{aligned}
			\end{align}
		\end{column}
		\onslide<3>\begin{column}{.4\linewidth}
			\begin{align}
				\begin{aligned}
					& (A - C) - (B - C)\\
					\equiv & A \cap \overline{C} \cap \overline{B \cap \overline{C}}\\
					\equiv & A \cap \overline{C} \cap (\overline{B} \cup C)\\
					\equiv & A \cap \overline{B} \cap \overline{C}\\
					\equiv & (A - B) - C.
				\end{aligned}
			\end{align}
		\end{column}
	\end{columns}
\end{proof}
\end{frame}

\begin{frame}[fragile]
\frametitle{Relation}
\begin{columns}
	\begin{column}{.6\linewidth}
		\begin{example}
			\begin{enumerate}
				\item Let $A = \{1, 2, 3, 4, 5\}$, $R$ is a relation on $A \times A$ such that $((a, b), (c, d)) \in R$ if and only if $a - d = c - b$. Show that $R$ is an equivalence relation.
				\item How many equivalence classes are there in $R$?
			\end{enumerate}
		\end{example}
		\begin{proof}
			\begin{enumerate}
				\item<2-> \begin{itemize}
					\item $a - b = a - b$.
					\item $a - d = c - b \Leftrightarrow c - b = a - d$.
					\item $a - d = c - b, c - f = e - d \Rightarrow a - f = e - b$.
				\end{itemize}
				\item<3-> 9.
			\end{enumerate}
		\end{proof}
	\end{column}
	\begin{column}{.5\linewidth}
		\onslide<1->\begin{example}
			Let $n \in \mathbb{Z}^+$, let $U = \{1, 2, \cdots, n\}$. Define the relation $R$ on the power set of $U$ by $(A, B) \in R$ if and only if $A \not\subset B$ and $B \not\subset A$. Is $R$ an equivalence relation? What is $|R|$?
		\end{example}
		\begin{solution*}
			\begin{enumerate}
				\item<4->. $R$ is not an equivalence relation. Let $n = 3, A = \{1\}, B = \{2\}$, $C = \{1, 3\}$ so that $(A, B) \in R, (B, C) \in R, (A, C) \notin R$.
				\item<5> $|R| = 2^{2n} - 2(3^n - 2^n)$.
			\end{enumerate}
		\end{solution*}
	\end{column}
\end{columns}
\end{frame}

\begin{frame}[fragile]
\frametitle{Onto}
\begin{example}
	Consider the set $X$ of all functions $f \colon A \rightarrow B$, where $A = \{1, 2, 3, 4, 5, 6, 7\}$, $B = \{a, b, c, d, e\}$. How many functions in $X$ are onto?
\end{example}
\onslide<2>\begin{solution*}
	For the first situation that 1 of $B$ matches 3 of $A$ and others match 1 of $A$ for each, we select 1 of 5 in $B$ and then for each of other 4 items in $B$ select 1 different item in $A$ so that there are ${}_5 C_1 \times {}_7 P_4$ functions in the first situation.
	
	For the second situation that 2 of $B$ match 2 of $A$ for each and others match 1 of $A$ for each, we select 2 of 5 in $B$, for each of other 3 items in $B$ select 1 different item in $A$ and then from the rest 4 items in $A$ select 2 of them to match 1 of 2 rest item in $B$ so that there are ${}_5 C_2 \times {}_7 P_3 \times {}_4 C_2$ functions in the section situation.
	
	Totally, there are ${}_5 C_1 \times {}_7 P_4 + {}_5 C_2 \times {}_7 P_3 \times {}_4 C_2 = 16800$ functions.
\end{solution*}
\end{frame}

\begin{frame}[fragile]
\frametitle{Pigeonhole Principle}
\begin{columns}
	\begin{column}{.6\linewidth}
		\begin{example}
			Let $S$ be a set of 7 positive integers the maximum of which is at most 24. Prove that the sums of the elements in all the nonempty subsets of $S$ cannot be distinct.
		\end{example}
		\onslide<2->\begin{proof}
			Consider nonempty subsets of size at most 5. Their maximum sum is at most $20 + 21 + \cdots + 24 = 110$. There are $126 - 7 = 119$ nonempty subsets with at most 5 elements so at least 2 of them have equal sum.
		\end{proof}
	\end{column}
	\begin{column}{.5\linewidth}
		\onslide<1->\begin{example}
			Given a closed form expression for the generating function of the following sequence:\begin{equation}
			1, -2, 3, -4, 5, -6, \cdots
			\end{equation}
		\end{example}
		\onslide<3>\begin{solution*}
			\begin{align}
				a_r & = -1^{r + 1} & \leftrightarrow & -\frac{1}{1 + x},\\
				b_r & = -1^r (r + 1) & \leftrightarrow & \frac{1}{(1 + x)^2}.
			\end{align}
		\end{solution*}
	\end{column}
\end{columns}
\end{frame}

\plain{Questions?}

\end{document}
