%!TEX program = xelatex
\documentclass[10pt, compress]{beamer}
\usetheme[titleprogressbar]{m}

\usepackage{booktabs}
\usepackage[scale=2]{ccicons}
\usepackage{minted}

\usepgfplotslibrary{dateplot}

\usemintedstyle{trac}

\setbeamertemplate{theorems}[numbered]
\newtheorem{crl}{Corollary}[theorem]

\usepackage{multicol}

\title{CSCI 3190 Tutorial of Week 7}
\subtitle{Quiz 1}
\author{LI Haocheng}
\institute{Department of Computer Science and Engineering}

\begin{document}

\maketitle

\begin{frame}[fragile]
\frametitle{Tautology}
\onslide<1->Prove following expressions are tautology.
\begin{enumerate}
	\item $\neg (q \rightarrow r) \rightarrow (q \vee r)$
	\item $\neg (q \rightarrow \neg r) \rightarrow (q \vee r)$
\end{enumerate}
\onslide<2>\textbf{Solution}
\begin{multicols}{2}
\begin{enumerate}
	\item \begin{align}
	& (\neg (q \rightarrow r) \rightarrow (q \vee r))\\
	= & (q \rightarrow r) \vee q \vee r\\
	= & \neg q \vee r \vee q\\
	= & T
	\end{align}
	
	\columnbreak
	
	\item \begin{align}
	& (\neg (q \rightarrow \neg r) \rightarrow (q \vee r))\\
	= & (q \rightarrow \neg r) \vee q \vee r\\
	= & \neg q \vee \neg r \vee q \vee r\\
	= & T
	\end{align}
\end{enumerate}
\end{multicols}
\end{frame}

\begin{frame}[fragile]
\frametitle{Equivalence Relation}
\onslide<1->Find minimum equivalence relation of following relations.
\begin{enumerate}
	\item $\{(a, b), (b, c), (d, c)\} \subseteq \{a, b, c, d, e\}^2$.
	\item $\{(a, b), (b, d), (d, c)\} \subseteq \{a, b, c, d, e\}^2$.
\end{enumerate}
\onslide<2>\textbf{Solution}
\begin{enumerate}
	\item Find Reflexsive Closure: $\{(a, a), (a, b), (b, b), (b, c), (c, c), (d, c), (d, d), (e, e)\}$.
	\item Find Symmetric Closure: $\{(a, a), (a, b), (b, a), (b, b), (b, c), (c, b), (c, c), (c, d), (d, c), (d, d), (e, e)\}$.
	\item Find Transitive Closure: $\{(a, a), (a, b), (a, c), (a, d), (b, a), (b, b), (b, c), (b, d), (c, a), (c, b), (c, c)$, $(c, d), (d, a), (d, b), (d, c), (d, d), (e, e)\}$
\end{enumerate}
\end{frame}

\begin{frame}[fragile]
\frametitle{Permutation}
\onslide<1->\begin{enumerate}
	\item Permutations of 26 letters don\textquoteright t exist \texttt{plan}, \texttt{than} or \texttt{both}?
	\item Permutations of 26 letters don\textquoteright t exist \texttt{game}, \texttt{meal} or \texttt{also}?
\end{enumerate}
\onslide<2>\textbf{Solution}

\texttt{plan} occurs $22! \times 23 = 23!$ times.

\texttt{than} occurs $22! \times 23 = 23!$ times.

\texttt{both} occurs $22! \times 23 = 23!$ times.

\texttt{plan} and \texttt{than} occurs $0$ times.

\texttt{plan} and \texttt{both} occurs $18! \times 19 \times 20$ times.

\texttt{than} and \texttt{both} occurs $20! \times 21$ times.

\texttt{plan}, \texttt{than} and \texttt{both} occurs $0$ times.

So the answer is $26! - 23! \times 3 + 21! + 20!$.

\end{frame}

\begin{frame}[allowframebreaks]
\frametitle{Function}
Consider functions $f \colon A \rightarrow B$, where $A = \{1, 2, 3, 4, 5, 6, 7\}$, $B = \{a, b, c, d, e\}$.\begin{enumerate}
	\item How many different functions?
	\item How many of them are one-to-one?
	\item How many of them are onto?
	\item Name one of onto functions.
\end{enumerate}
\textbf{Solution} \begin{enumerate}
	\item $5^7$.
	\item $0$, because $|A| > |B|$.
	\item There are 2 kinds of situations, 1 of them is that 1 of $B$ matches 3 of $A$ and others match 1 of $A$ for each, the other 1 is that 2 of $B$ match 2 of $A$ for each and others match 1 of $A$ for each. We will discuss these 2 situations separately.
	
	For the first situation, we select 1 of 5 in $B$ and then for each of other 4 items in $B$ select 1 different item in $A$ so that there are ${}_5 C_1 \times {}_7 P_4$ functions in the first situation.
	
	For the second situation, we select 2 of 5 in $B$, for each of other 3 items in $B$ select 1 different item in $A$ and then from the rest 4 items in $A$ select 2 of them to match 1 of 2 rest item in $B$ so that there are ${}_5 C_2 \times {}_7 P_3 \times {}_4 C_2$ functions in the section situation.
	
	Totally, there are ${}_5 C_1 \times {}_7 P_4 + {}_5 C_2 \times {}_7 P_3 \times {}_4 C_2 = 16800$ functions.
	
	\item One of the onto functions is: $\{(1, a), (2, b), (3, c), (4, d), (5, e), (6, a), (7, b)\}$.
\end{enumerate}
\end{frame}

\begin{frame}[fragile]
\frametitle{Generating Function}
\onslide<1->Consider a bare island. Suppose there are 4 new-born raccoons in the 0th year. Number of new-born raccoons each year are 3 times of last year. Assume all raccoons can live for 6 years. Determine the generating function of number of raccoons on the island.

\onslide<2>\textbf{Solution} Number of newborns is \begin{align}
n_r = & \begin{cases}
4 \times 3^r, & r \ge 0\\
0, & \text{otherwise}.
\end{cases}\\
\leftrightarrow & \frac{4}{1 - 3x}
\end{align} so that number of racoons is \begin{align}
b_r = & \Sigma_{i = r - 5}^{r} n_i\\
\leftrightarrow & \frac{4(1 - x^6)}{(1 - 3x)(1 - x)}
\end{align}
\end{frame}

\begin{frame}[fragile]
\frametitle{$n \log n$}
\onslide<1->Show that $n \log n$ is $O(\log n!)$.

\onslide<2>\textbf{Solution} We
can easily show that $(n - i)(i + 1) \ge n$ for $i = 0, 1, \cdots, n - 1$.
Hence, $(n!)^2 = (n \cdot 1)((n - 1) \cdot 2) \cdot ((n - 2) \cdot 3) \cdots (2 \cdot (n -
1)) \cdot (1 \cdot n) \ge n^n$. Therefore, $2 log n! \ge n log n$.
\end{frame}

\begin{frame}[fragile]
\frametitle{Big-Theta}
\onslide<1->Determine whether $\log n!$ is $\Theta(n \log n)$.

\onslide<2->\textbf{Solution} We have known that $n \log n = O(\log n!)$.

\onslide<3->\textbf{Claim} $\log n! = O(n \log n)$ so that $n \log n = \Omega(\log n!)$. Hence, $\log n! = \Theta(n \log n)$.

\onslide<4>\textbf{Proof} $\exists c = 1, \exists N = 1$, such that $\forall n \ge N, n! \le n^n$, so that $\log n! \le c n\log n$.

\end{frame}

\begin{frame}
\frametitle{Sorting Algorithm}
\onslide<1->\begin{theorem} \label{t-11-2-1}
	A sorting algorithm based on binary comparisons requires at least $\lceil \log n! \rceil$ comparisons.
\end{theorem}
\onslide<2>\begin{proof}
	The complexity of a sort based on binary comparisons is measured in terms of the number	of such comparisons used. The largest number of binary comparisons ever needed to sort a list with $n$ elements gives the worst-case performance of the algorithm. The most comparisons used equals the longest path length in the decision tree representing the sorting procedure. In other words, the largest number of comparisons ever needed is equal to the height of the decision
	tree. Because the height of a binary tree with $n!$ leaves is at least $\lceil \log n! \rceil$, at least $\lceil \log n! \rceil$ comparisons are needed.
\end{proof}
\end{frame}

\begin{frame}[fragile]
\frametitle{Number of Comparisons}
\begin{crl}
	The number of comparisons used by a sorting algorithm to sort $n$ elements based on binary comparisons is $\Omega(n \log n)$.
\end{crl}
\begin{proof}
	We can use Theorem \ref{t-11-2-1} to provide a big-Omega estimate for the number of comparisons used by a sorting algorithm based on binary comparison. We need only note that $\lceil \log n! \rceil$ is $\Theta(n \log n)$.
\end{proof}
\end{frame}

\plain{Questions?}

\end{document}