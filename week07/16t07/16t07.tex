%!TEX program = xelatex
\documentclass[10pt, compress]{beamer}
\usetheme[titleprogressbar]{m}

\usepackage{booktabs}
\usepackage[scale=2]{ccicons}
\usepackage{minted}

\usepgfplotslibrary{dateplot}

\usemintedstyle{trac}

\usepackage{multicol}

\title{CSCI 3190 Tutorial of Week 7}
\subtitle{Quiz 1}
\author{LI Haocheng}
\institute{Department of Computer Science and Engineering}

\begin{document}

\maketitle

\begin{frame}[fragile]
\frametitle{Tautology}
\onslide<1->Prove following expressions are tautology.
\begin{enumerate}
	\item $\neg (q \rightarrow r) \rightarrow (q \vee r)$
	\item $\neg (q \rightarrow \neg r) \rightarrow (q \vee r)$
\end{enumerate}
\onslide<2>\textbf{Solution}
\begin{multicols}{2}
\begin{enumerate}
	\item \begin{align}
	& (\neg (q \rightarrow r) \rightarrow (q \vee r))\\
	= & (q \rightarrow r) \vee q \vee r\\
	= & \neg q \vee r \vee q\\
	= & T
	\end{align}
	
	\columnbreak
	
	\item \begin{align}
	& (\neg (q \rightarrow \neg r) \rightarrow (q \vee r))\\
	= & (q \rightarrow \neg r) \vee q \vee r\\
	= & \neg q \vee \neg r \vee q \vee r\\
	= & T
	\end{align}
\end{enumerate}
\end{multicols}
\end{frame}

\begin{frame}[fragile]
\frametitle{Equivalence Relation}
\onslide<1->Find minimum equivalence relation of following relations.
\begin{enumerate}
	\item $\{(a, b), (b, c), (d, c)\} \subseteq \{a, b, c, d, e\}^2$.
	\item $\{(a, b), (b, d), (d, c)\} \subseteq \{a, b, c, d, e\}^2$.
\end{enumerate}
\onslide<2>\textbf{Solution}
\begin{enumerate}
	\item Find Reflexsive Closure: $\{(a, a), (a, b), (b, b), (b, c), (c, c), (d, c), (d, d), (e, e)\}$.
	\item Find Symmetric Closure: $\{(a, a), (a, b), (b, a), (b, b), (b, c), (c, b), (c, c), (c, d), (d, c), (d, d), (e, e)\}$.
	\item Find Transitive Closure: $\{(a, a), (a, b), (a, c), (a, d), (b, a), (b, b), (b, c), (b, d), (c, a), (c, b), (c, c)$, $(c, d), (d, a), (d, b), (d, c), (d, d), (e, e)\}$
\end{enumerate}
\end{frame}

\begin{frame}[fragile]
\frametitle{Permutation}
\onslide<1->\begin{enumerate}
	\item Permutations of 26 letters don\textquoteright t exist \texttt{plan}, \texttt{than} or \texttt{both}?
	\item Permutations of 26 letters don\textquoteright t exist \texttt{game}, \texttt{meal} or \texttt{also}?
\end{enumerate}
\onslide<2>\textbf{Solution}

\texttt{plan} occurs $22! \times 23 = 23!$ times.

\texttt{than} occurs $22! \times 23 = 23!$ times.

\texttt{both} occurs $22! \times 23 = 23!$ times.

\texttt{plan} and \texttt{than} occurs $0$ times.

\texttt{plan} and \texttt{both} occurs $18! \times 19 \times 20$ times.

\texttt{than} and \texttt{both} occurs $20! \times 21$ times.

\texttt{plan}, \texttt{than} and \texttt{both} occurs $0$ times.

So the answer is $26! - 23! \times 3 + 21! + 20!$.

\end{frame}

\begin{frame}[allowframebreaks]
\frametitle{Function}
Consider functions $f \colon A \rightarrow B$, where $A = \{1, 2, 3, 4, 5, 6, 7\}$, $B = \{a, b, c, d, e\}$.\begin{enumerate}
	\item How many different functions?
	\item How many of them are one-to-one?
	\item How many of them are onto?
	\item Name one of onto functions.
\end{enumerate}
\textbf{Solution} \begin{enumerate}
	\item $5^7$.
	\item $0$, because $|A| > |B|$.
	\item There are 2 kinds of situations, 1 of them is that 1 of $B$ matches 3 of $A$ and others match 1 of $A$ for each, the other 1 is that 2 of $B$ match 2 of $A$ for each and others match 1 of $A$ for each. We will discuss these 2 situations separately.
	
	For the first situation, we select 1 of 5 in $B$ and then for each of other 4 items in $B$ select 1 different item in $A$ so that there are ${}_5 C_1 \times {}_7 P_4$ functions in the first situation.
	
	For the second situation, we select 2 of 5 in $B$, for each of other 3 items in $B$ select 1 different item in $A$ and then from the rest 4 items in $A$ select 2 of them to match 1 of 2 rest item in $B$ so that there are ${}_5 C_2 \times {}_7 P_3 \times {}_4 C_2$ functions in the section situation.
	
	Totally, there are ${}_5 C_1 \times {}_7 P_4 + {}_5 C_2 \times {}_7 P_3 \times {}_4 C_2 = 16800$ functions.
	
	\item One of the onto functions is: $\{(1, a), (2, b), (3, c), (4, d), (5, e), (6, a), (7, b)\}$.
\end{enumerate}
\end{frame}

\begin{frame}[fragile]
\frametitle{Generating Function}

\end{frame}

\plain{Questions?}

\end{document}