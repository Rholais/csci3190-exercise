%!TEX program = xelatex
\documentclass[10pt, compress, handout]{beamer}
\usetheme[titleprogressbar]{m}

\usepackage{booktabs}
\usepackage[scale=2]{ccicons}
\usepackage{minted}

\usepgfplotslibrary{dateplot}

\usemintedstyle{trac}

\setbeamertemplate{theorems}[numbered]
\newtheorem{crl}{Corollary}[theorem]

\usepackage{multicol}

\makeatletter
\def\old@comma{,}
\catcode`\,=13
\def,{%
	\ifmmode%
	\old@comma\discretionary{}{}{}%
	\else%
	\old@comma%
	\fi%
}
\makeatother

\title{CSCI 3190 Tutorial of Week 7}
\subtitle{Quiz 1}
\author{LI Haocheng}
\institute{Department of Computer Science and Engineering}

\begin{document}

\maketitle

\begin{frame}[fragile]
\frametitle{Propositional Logic}
\onslide<1->Explain, without using a truth table, why $(p \vee \neg q) \wedge (q \vee \neg r) \wedge (r \vee \neg p)$ is true when $p$, $q$, and $r$ have the same truth value and it is false otherwise.

\onslide<2>\textbf{Solution} It is easy to evaluate that the expression is true when all variables have the same truth value. Suppose at least one is true and at least one is false. There are 2 combinations of 2 variables which have different truth value. These 2 combinations can form 2 disjunction expression with false value and one of them is in the conjunction expression so that the whole expression is false.
\end{frame}

\begin{frame}[fragile]
\frametitle{Propositional Logic}
\onslide<1->Explain, without using a truth table, why $(p \vee q \vee r) \wedge
(\neg p \vee \neg q \vee \neg r)$ is true when at least one of $p$, $q$, and $r$
is true and at least one is false, but is false when all three
variables have the same truth value.

\onslide<2>\textbf{Proof} The first clause is true if and only if at least one of $p$, $q$, and
$r$ is true. The second clause is true if and only if at least one of
the three variables is false. Therefore the entire statement is
true if and only if there is at least one T and one F among the
truth values of the variables, in other words, that they don\textquoteright t all
have the same truth value.
\end{frame}

\begin{frame}[allowframebreaks]
\frametitle{Propositional Satisfiability}
\begin{definition}
	A compound proposition is \textbf{satisfiable} if there is an assignment of truth values to its variables that makes it true. When no such assignments exists, that is, when the compound proposition is false for all assignments of truth values to its variables, the compound proposition is \textbf{unsatisfiable}.
\end{definition}
\begin{example}
	Determine whether each of the compound propositions $(p \vee \neg q) \wedge (q \vee \neg r) \wedge (r \vee \neg p)$, $(p \vee q \vee r) \wedge (\neg p \vee \neg q \vee \neg r)$ and $(p \vee \neg q) \wedge (q \vee \neg r) \wedge (r \vee \neg p) \wedge (p \vee q \vee r) \wedge	(\neg p \vee \neg q \vee \neg r)$ is satisfiable.
\end{example}
\textbf{Solution} Instead of using truth table to solve this problem, we will reason about truth values. Note that $(p \vee \neg q) \wedge (q \vee \neg r) \wedge (r \vee \neg p)$ is true when the three variable $p$, $q$, and $r$ have the same truth value. Hence, it is satisfiable as there is at least one assignment of truth values for $p$, $q$, and $r$ that makes it true. 

Similarly, note that
$(p \vee q \vee r) \wedge (\neg p \vee \neg q \vee \neg r)$ is true when at least one of $p$, $q$, and $r$ is true and at least one is false. Hence, $(p \vee q \vee r) \wedge (\neg p \vee \neg q \vee \neg r)$ is satisfiable, as there is at least one assignment of truth values for $p$, $q$, and $r$ that makes it true.
\newpage
Finally, note that for $(p \vee \neg q) \wedge (q \vee \neg r) \wedge (r \vee \neg p) \wedge (p \vee q \vee r) \wedge	(\neg p \vee \neg q \vee \neg r)$	to be true, $(p \vee \neg q) \wedge (q \vee \neg r) \wedge (r \vee \neg p)$ and $(p \vee q \vee r) \wedge (\neg p \vee \neg q \vee \neg r)$ must both be true. For the first to be true, the three variables must have the same truth values, and
for the second to be true, at least one of three variables must be true and at least one must be false. However, these conditions are contradictory. From these observations we conclude
that no assignment of truth values to $p$, $q$, and $r$ makes $(p \vee \neg q) \wedge (q \vee \neg r) \wedge (r \vee \neg p) \wedge (p \vee q \vee r) \wedge	(\neg p \vee \neg q \vee \neg r)$ true. Hence, it is unsatisfiable.
\end{frame}

\begin{frame}[fragile]
	\frametitle{Tautology}
	\onslide<1->Prove that following expressions are tautology.
	\begin{enumerate}
		\item $\neg (q \rightarrow r) \rightarrow (q \vee r)$
		\item $\neg (q \rightarrow \neg r) \rightarrow (q \vee r)$
	\end{enumerate}
	\onslide<2->\begin{example}
		Show that $p \rightarrow q$ and $\neg p \vee q$ are logically equivalent.
	\end{example}
	\onslide<3>\textbf{Solution}
	\begin{multicols}{2}
		\begin{enumerate}
			\item \begin{align}
			& (\neg (q \rightarrow r) \rightarrow (q \vee r))\\
			= & (q \rightarrow r) \vee q \vee r\\
			= & \neg q \vee r \vee q\\
			= & T
			\end{align}
			
			\columnbreak
			
			\item \begin{align}
			& (\neg (q \rightarrow \neg r) \rightarrow (q \vee r))\\
			= & (q \rightarrow \neg r) \vee q \vee r\\
			= & \neg q \vee \neg r \vee q \vee r\\
			= & T
			\end{align}
		\end{enumerate}
	\end{multicols}
\end{frame}

\begin{frame}[fragile]
	\frametitle{$n \log n$}
	\onslide<1->Show that $n \log n$ is $O(\log n!)$.
	
	\onslide<2>\textbf{Solution} We
	can easily show that $(n - i)(i + 1) \ge n$ for $i = 0, 1, \cdots, n - 1$.
	Hence, $(n!)^2 = (n \cdot 1)((n - 1) \cdot 2) \cdot ((n - 2) \cdot 3) \cdots (2 \cdot (n -
	1)) \cdot (1 \cdot n) \ge n^n$. Therefore, $2 log n! \ge n log n$.
\end{frame}

\begin{frame}[fragile]
	\frametitle{Big-Theta}
	\onslide<1->Determine whether $\log n!$ is $\Theta(n \log n)$.
	
	\onslide<2->\textbf{Solution} We have known that $n \log n = O(\log n!)$.
	
	\onslide<3->\textbf{Claim} $\log n! = O(n \log n)$ so that $n \log n = \Omega(\log n!)$. Hence, $\log n! = \Theta(n \log n)$.
	
	\onslide<4>\textbf{Proof} $\exists c = 1, \exists N = 1$, such that $\forall n \ge N, n! \le n^n$, so that $\log n! \le c n\log n$.
	
\end{frame}

\begin{frame}
	\frametitle{Sorting Algorithm}
	\onslide<1->\begin{theorem} \label{t-11-2-1}
		A sorting algorithm based on binary comparisons requires at least $\lceil \log n! \rceil$ comparisons.
	\end{theorem}
	\onslide<2>\begin{proof}
		The complexity of a sort based on binary comparisons is measured in terms of the number	of such comparisons used. The largest number of binary comparisons ever needed to sort a list with $n$ elements gives the worst-case performance of the algorithm. The most comparisons used equals the longest path length in the decision tree representing the sorting procedure. In other words, the largest number of comparisons ever needed is equal to the height of the decision
		tree. Because the height of a binary tree with $n!$ leaves is at least $\lceil \log n! \rceil$, at least $\lceil \log n! \rceil$ comparisons are needed.
	\end{proof}
\end{frame}

\begin{frame}[fragile]
	\frametitle{Number of Comparisons}
	\onslide<1->\begin{crl}\label{c-11-2-1}
		The number of comparisons used by a sorting algorithm to sort $n$ elements based on binary comparisons is $\Omega(n \log n)$.
	\end{crl}
	\onslide<2->\begin{proof}
		We can use Theorem \ref{t-11-2-1} to provide a big-Omega estimate for the number of comparisons used by a sorting algorithm based on binary comparison. We need only note that $\lceil \log n! \rceil$ is $\Theta(n \log n)$.
	\end{proof}
	
	\onslide<3>A consequence of Corollary \ref{c-11-2-1} is that a sorting algorithm based on binary comparisons that
	uses $\Theta(n \log n)$ comparisons, in the worst case, to sort $n$ elements is optimal, in the sense that no other such algorithm has better worst-case complexity. Note that we see that the merge sort algorithm is optimal in this sense.
	
\end{frame}

\begin{frame}[fragile]
	\frametitle{Average Number of Comparisons}
	\onslide<1->\begin{theorem}
		The average number of comparisons used by a sorting algorithm to sort $n$ elements based on binary comparisons is $\Omega(n \log n)$.
	\end{theorem}
	\onslide<2>\begin{proof}
		The average number of comparisons used by a sorting algorithm based on binary comparisons is the average depth of a leaf in the decision tree representing the sorting algorithm. We know that the average depth of a leaf in a binary tree with $N$ vertices
		is $\Omega(\log N)$. We obtain the following estimate when we let $N = n!$ and note that a function that is $\Omega(\log n!)$ is also $\Omega(n \log n)$ because $\log n!$ is $\Omega (n \log n)$.
	\end{proof}
\end{frame}

\begin{frame}[fragile]
\frametitle{Equivalence Relation}
\onslide<1->Find smallest equivalence relation on set $\{a, b, c, d, e\}$ containing following relations.
\begin{enumerate}
	\item $\{(a, b), (b, c), (d, c)\}$.
	\item $\{(a, b), (b, d), (d, c)\}$.
\end{enumerate}
\onslide<2>\textbf{Solution}
\begin{enumerate}[(i)]
	\item Find Reflexsive Closure: $\{(a, a), (a, b), (b, b), (b, c), (c, c), (d, c), (d, d), (e, e)\}$.
	\item Find Symmetric Closure: $\{(a, a), (a, b), (b, a), (b, b), (b, c), (c, b), (c, c), (c, d), (d, c), (d, d), (e, e)\}$.
	\item Find Transitive Closure: $\{(a, a), (a, b), (a, c), (a, d), (b, a), (b, b), (b, c), (b, d), (c, a), (c, b), (c, c), (c, d), (d, a), (d, b), (d, c), (d, d), (e, e)\}$
\end{enumerate}
\end{frame}

\begin{frame}[fragile]
\frametitle{Permutation}
\onslide<1->\begin{enumerate}
	\item Permutations of 26 letters don\textquoteright t exist \texttt{plan}, \texttt{than} or \texttt{both}?
	\item Permutations of 26 letters don\textquoteright t exist \texttt{game}, \texttt{meal} or \texttt{also}?
\end{enumerate}
\onslide<2>\textbf{Solution}

\texttt{plan} occurs $22! \times 23 = 23!$ times.

\texttt{than} occurs $22! \times 23 = 23!$ times.

\texttt{both} occurs $22! \times 23 = 23!$ times.

\texttt{plan} and \texttt{than} occurs $0$ times.

\texttt{plan} and \texttt{both} occurs $18! \times 19 \times 20$ times.

\texttt{than} and \texttt{both} occurs $20! \times 21$ times.

\texttt{plan}, \texttt{than} and \texttt{both} occurs $0$ times.

So the answer is $26! - 23! \times 3 + 21! + 20!$.

\end{frame}

\begin{frame}[allowframebreaks]
\frametitle{Function}
Consider the set $X$ of all functions $f \colon A \rightarrow B$, where $A = \{1, 2, 3, 4, 5, 6, 7\}$, $B = \{a, b, c, d, e\}$.\begin{enumerate}
	\item How many functions are there in $X$?
	\item How many functions in $X$ are one-to-one? Give one example if there is at least one.
	\item How many functions in $X$ are onto? Give one example if there is at least one.
\end{enumerate}
\textbf{Solution} \begin{enumerate}
	\item $5^7$.
	\item $0$, because $|A| > |B|$.
	\item There are 2 kinds of situations, 1 of them is that 1 of $B$ matches 3 of $A$ and others match 1 of $A$ for each, the other 1 is that 2 of $B$ match 2 of $A$ for each and others match 1 of $A$ for each. We will discuss these 2 situations separately.
	
	For the first situation, we select 1 of 5 in $B$ and then for each of other 4 items in $B$ select 1 different item in $A$ so that there are ${}_5 C_1 \times {}_7 P_4$ functions in the first situation.
	
	For the second situation, we select 2 of 5 in $B$, for each of other 3 items in $B$ select 1 different item in $A$ and then from the rest 4 items in $A$ select 2 of them to match 1 of 2 rest item in $B$ so that there are ${}_5 C_2 \times {}_7 P_3 \times {}_4 C_2$ functions in the section situation.
	
	Totally, there are ${}_5 C_1 \times {}_7 P_4 + {}_5 C_2 \times {}_7 P_3 \times {}_4 C_2 = 16800$ functions.
	
	\item One of the onto functions is: $\{(1, a), (2, b), (3, c), (4, d), (5, e), (6, a), (7, b)\}$.
\end{enumerate}
\end{frame}

\begin{frame}[fragile]
\frametitle{Generating Function}
\onslide<1->Let the lifespan of a raccoon be exactly 6 years. Suppose there are 4 new-born raccoons at the 0th year and the number of new-born raccoons in each year is 3 times that in the previous year. Let $b_r$ be the number of raccoons on year $r$ where $r \ge 0$. Give a closed form generating function for $b_r$.

\onslide<2>\textbf{Solution} Number of newborns is \begin{equation}
n_r = \begin{cases}
4 \times 3^r, & r \ge 0\\
0, & \text{otherwise}.
\end{cases} \leftrightarrow \frac{4}{1 - 3x},
\end{equation} so that number of racoons is \begin{equation}
b_r = \Sigma_{i = r - 5}^{r} n_i \leftrightarrow \frac{4(1 - x^6)}{(1 - 3x)(1 - x)}.
\end{equation}
\end{frame}

\plain{Questions?}

\end{document}