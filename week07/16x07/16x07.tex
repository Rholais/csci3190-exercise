% This is "sig-alternate.tex" V2.1 April 2013
% This file should be compiled with V2.5 of "sig-alternate.cls" May 2012
%
% This example file demonstrates the use of the 'sig-alternate.cls'
% V2.5 LaTeX2e document class file. It is for those submitting
% articles to ACM Conference Proceedings WHO DO NOT WISH TO
% STRICTLY ADHERE TO THE SIGS (PUBS-BOARD-ENDORSED) STYLE.
% The 'sig-alternate.cls' file will produce a similar-looking,
% albeit, 'tighter' paper resulting in, invariably, fewer pages.
%
% ----------------------------------------------------------------------------------------------------------------
% This .tex file (and associated .cls V2.5) produces:
%       1) The Permission Statement
%       2) The Conference (location) Info information
%       3) The Copyright Line with ACM data
%       4) NO page numbers
%
% as against the acm_proc_article-sp.cls file which
% DOES NOT produce 1) thru' 3) above.
%
% Using 'sig-alternate.cls' you have control, however, from within
% the source .tex file, over both the CopyrightYear
% (defaulted to 200X) and the ACM Copyright Data
% (defaulted to X-XXXXX-XX-X/XX/XX).
% e.g.
% \CopyrightYear{2007} will cause 2007 to appear in the copyright line.
% \crdata{0-12345-67-8/90/12} will cause 0-12345-67-8/90/12 to appear in the copyright line.
%
% ---------------------------------------------------------------------------------------------------------------
% This .tex source is an example which *does* use
% the .bib file (from which the .bbl file % is produced).
% REMEMBER HOWEVER: After having produced the .bbl file,
% and prior to final submission, you *NEED* to 'insert'
% your .bbl file into your source .tex file so as to provide
% ONE 'self-contained' source file.
%
% ================= IF YOU HAVE QUESTIONS =======================
% Questions regarding the SIGS styles, SIGS policies and
% procedures, Conferences etc. should be sent to
% Adrienne Griscti (griscti@acm.org)
%
% Technical questions _only_ to
% Gerald Murray (murray@hq.acm.org)
% ===============================================================
%
% For tracking purposes - this is V2.0 - May 2012

\documentclass{../../cls/sig-alternate-05-2015}

\usepackage{algorithm}
\usepackage{algorithmic}

\usepackage{booktabs}
\usepackage{mathtools}
\usepackage{textcomp}


\begin{document}

% Copyright
%\setcopyright{acmcopyright}
%\setcopyright{acmlicensed}
%\setcopyright{rightsretained}
%\setcopyright{usgov}
%\setcopyright{usgovmixed}
%\setcopyright{cagov}
%\setcopyright{cagovmixed}


% DOI
%\doi{10.475/123_4}

% ISBN
%\isbn{123-4567-24-567/08/06}

%Conference
%\conferenceinfo{PLDI '13}{June 16--19, 2013, Seattle, WA, USA}

%\acmPrice{\$15.00}

%
% --- Author Metadata here ---
%\conferenceinfo{WOODSTOCK}{'97 El Paso, Texas USA}
%\CopyrightYear{2007} % Allows default copyright year (20XX) to be over-ridden - IF NEED BE.
%\crdata{0-12345-67-8/90/01}  % Allows default copyright data (0-89791-88-6/97/05) to be over-ridden - IF NEED BE.
% --- End of Author Metadata ---

%\\TODO:1.tautology prove section 2. set section operation proof 3. function section basic proving onto bijection...
\title{CSCI 3190 \\ Introduction to Discrete Mathematics and Algorithms}
\subtitle{Extended Exercise 7}

\maketitle
\begin{abstract}

\end{abstract}

\keywords{}

\section{Algorithms}
\subsection{The Growth of Functions}
\begin{enumerate}
\item Show that $n \log n$ is $O(\log n!)$.
\end{enumerate}

\subsection{Complexity of Algorithms}
\begin{enumerate}
\item Give a big-$O$ estimate for the number of operations (where an operation is an addition or a multiplication) used in Algorithm \ref{a-3-3-1}. \begin{algorithm}
	\caption{}
	\label{a-3-3-1}
	\begin{algorithmic}
		\STATE $t \coloneqq 0$
		\FOR{$i \coloneqq 1, 2, 3$}
		\FOR{$j \coloneqq 1, 2, 3, 4$}
		\STATE $t \coloneqq t + ij$
		\ENDFOR
		\ENDFOR
	\end{algorithmic}
\end{algorithm}

\item Give a big-$O$ estimate for the number of operations (where an operation is an addition or a multiplication) used in Algorithm \ref{a-3-3-2}. \begin{algorithm}
	\caption{}
	\label{a-3-3-2}
	\begin{algorithmic}
		\STATE $t \coloneqq 0$
		\FOR{$i \coloneqq 1, 2, \cdots, n$}
		\FOR{$j \coloneqq 1, 2, \cdots, n$}
		\STATE $t \coloneqq t + i + j$
		\ENDFOR
		\ENDFOR
	\end{algorithmic}
\end{algorithm}

\item Give a big-$O$ estimate for the number of operations, where an operation is a comparison or a multiplication, used in Algorithm \ref{a-3-3-3} (ignoring compar- isons used to test the conditions in the \textbf{for} loops, where $a_1, a_2, \cdots, a_n$ are positive real numbers). \begin{algorithm}
	\caption{}
	\label{a-3-3-3}
	\begin{algorithmic}
		\STATE $t \coloneqq 0$
		\FOR{$i \coloneqq 1, 2, \cdots, n$}
		\FOR{$j \coloneqq i + 1, i + 2, \cdots, n$}
		\STATE $t \coloneqq \max(t, a_i a_j)$
		\ENDFOR
		\ENDFOR
	\end{algorithmic}
\end{algorithm}

\end{enumerate}

\section{Induction and Recursion}
\subsection{Mathematical Induction}
\begin{enumerate}
\item Prove \textbf{Bernoulli’s inequality}: if $h > -1$, then $1+nh \le (1+h)^n$ for all non-negative integers $n$.

\item Suppose that $a$ and $b$ are real numbers with $0 < b < a$.
Prove that if $n$ is a positive integer, then $a^n - b^n \le na^{n - 1}(a - b)$.

\item Prove that for every positive integer n, \begin{equation}
	\Sigma_{i = 1}^n \frac{1}{\sqrt{i}} > 2(\sqrt{n + 1} - 1).
\end{equation}

\item Prove that $n^2 - 1$ is divisible by 8 whenever $n$ is an odd positive integer.

\item Prove that 21 divides $4^{n + 1} + 5^{2n - 1}$ whenever $n$ is a positive integer.

\item Show that \begin{equation}
	[(p_1 \rightarrow p_2) \wedge (p_2 \rightarrow p_3) \wedge \cdots \wedge (p_{n - 1} \rightarrow p_n)] \rightarrow [(p_1 \wedge p_2 \wedge \cdots \wedge p_{n - 1}) \rightarrow p_n]
\end{equation}
is a tautology whenever $p_1$, $p_2$,... ,$p_n$ are propositions, where $n \ge 2$.
\end{enumerate}

\nocite{*}
\bibliographystyle{abbrv}
\bibliography{ref}  % sigproc.bib is the name of the Bibliography in this case
 
\clearpage
%APPENDICES are optional
%\balancecolumns
\appendix
%Appendix A
\section{Answer}
\subsection{Algorithms}
\subsubsection{The Growth of Functions}
\begin{enumerate}
\item We
can easily show that $(n - i)(i + 1) \ge n$ for $i = 0, 1, \cdots, n - 1$.
Hence, $(n!)^2 = (n \cdot 1)((n - 1) \cdot 2) \cdot ((n - 2) \cdot 3) \cdots (2 \cdot (n -
1)) \cdot (1 \cdot n) \ge n^n$. Therefore, $2 log n! \ge n log n$.
\end{enumerate}

\subsubsection{Complexity of Algorithms}
\begin{enumerate}
\item $O(1)$.
\item $O(n^2)$.
\item $O(n^2)$.
\end{enumerate}

\subsection{Induction and Recursion}
\subsubsection{Mathematical Induction}
\begin{enumerate}
\item Let $P(n)$ be \textquotedblleft $1 + nh \le (1 + h)^n, h > -1$\textquotedblright.
\textit{Basis step:} $P(0)$ is true because $1 + 0 \cdot h = 1 \le 1 = (1 + h)^0$.
\textit{Inductive step:} Assume $1 + kh \le (1 + h)^k$. Then because
$(1+h) > 0$, $(1 + h)^{k + 1} = (1 + h)(1 + h)^k \ge (1 + h)(1 + k^h) = 1 + (k + 1)^h + kh^2 \ge 1 + (k + 1)h$.
	
\item Let $P(n)$ be \textquotedblleft $a^n - b^n \le na^{n - 1}(a - b)$\textquotedblright. \textit{Basis step:} $P(1)$ is true because $a^1 - b^1 = a - b \le 1 \cdot a^0 (a - b)$. \textit{Inductive step:} Assume $a^k - b^k \le k a^{k - 1}(a - b)$. Then because $0 < b < a$: \begin{align}
	a^{k + 1} - b^{k + 1} = & (a - b)\Sigma_{i = 0}^k a^i b^{k - i}\\
	= & a^k(a - b) + b(a - b) \Sigma_{i = 0}^{k - 1} a^i b^{k - 1 - i}\\
	= & a^k(a - b) + b(a^k - b^k)\\
	\le & a^k(a - b) + k a^{k - 1} b(a - b)\\
	\le & (k + 1) a^k (a - b).
\end{align}

\item Let $P(n)$ be \textquotedblleft $\Sigma_{i = 1}^n \frac{1}{\sqrt{i}} > 2(\sqrt{n + 1} - 1)$ \textquotedblright. \textit{Basis step:} $P(1)$ is true because $1 > 2(\sqrt{2} - 1)$. \textit{Inductive step:} Assume that $P(k)$ is true. Then $\Sigma_{i = 1}^{k + 1} \frac{1}{\sqrt{i}} > 2(\sqrt{k + 1} - 1) + \frac{1}{\sqrt{k + 1}}$. We claim that $2(\sqrt{k + 1} - 1) + \frac{1}{\sqrt{k + 1}} > 2(\sqrt{k + 2} - 1)$ which is equivalent to $2(\sqrt{k + 2} - \sqrt{k + 1}) (\sqrt{k + 2} + \sqrt{k + 1}) < \frac{\sqrt{k + 1}}{\sqrt{k + 1}} + \frac{\sqrt{k + 2}}{\sqrt{k + 1}}$. This is clearly true since $2 < 1 + \frac{\sqrt{k + 2}}{\sqrt{k + 1}}$.

\item Let $P(n)$ be the proposition that $(2n - 1)^2 - 1$ is divisible by
8. The basis case $P(1)$ is true because $8 \mid 0$. Now assume
that $P(k)$ is true. Because $[(2(k + 1) - 1]^2 − 1 =
[(2k - 1)^2 - 1] + 8k$, $P(k + 1)$ is true because both terms on
the right-hand side are divisible by 8. This shows that $P(n)$ is true for all positive integers $n$, so $m^2 − 1$ is divisible by
8 whenever $m$ is an odd positive integer.

\item Let $P(n)$ be the proposition that $4^{n + 1} + 5^{2n - 1}$ is divisible by
21. The basis case $P(1)$ is true because $21 \mid 21$. Now assume
that $P(k)$ is true. Because $4^{k + 2} + 5^{2n + 1} = 4 \times [4^{k + 1} + 5^{2k - 1}] + 21 \times 5^{2k - 1}$, $P(k + 1)$ is true because both terms on
the right-hand side are divisible by 21. This shows that $P(n)$ is true for all positive integers $n$, so $4^{n + 1} + 5^{2n - 1}$ is divisible by
8 whenever $m$ is an positive integer.

\item Let $P(n)$ be \textquotedblleft $[(p_1 \rightarrow p_2) \wedge (p_2 \rightarrow p_3) \wedge \cdots \wedge (p_{n - 1} \rightarrow p_n)] \rightarrow [(p_1 \wedge p_2 \wedge \cdots \wedge p_{n - 1}) \rightarrow p_n]$\textquotedblright . \textit{Basis step:} $P(2)$ is true because $(p_1 \rightarrow p_2) \rightarrow (p_1 \rightarrow p_2)$ is a tautology. \textit{Inductive step:} Assume $P(k)$ is true. To show $[(p_1 \rightarrow p_2) \wedge (p_2 \rightarrow p_3) \wedge \cdots \wedge (p_{k - 1} \rightarrow p_k) \wedge (p_k \rightarrow p_{k + 1})] \rightarrow [(p_1 \wedge p_2 \wedge \cdots \wedge p_{k - 1} \wedge p_k) \rightarrow p_{k + 1}]$ is a tautology, assume that the hypothesis of this conditional
statement is true. Because both the hypothesis and $P(k)$ are
true, it follows that $(p_1 \wedge p_2 \wedge \cdots \wedge p_{k - 1}) \rightarrow p_k$ is true. Because
this is true, and because $p_k \rightarrow p{k + 1}$ is true (it is part
of the assumption) it follows by hypothetical syllogism that
$(p_1 \wedge p_2 \wedge \cdots \wedge p_{k - 1}) \rightarrow p_{k + 1}$ is true. The weaker statement $(p_1 \wedge p_2 \wedge \cdots \wedge p_{k - 1} \wedge p_k) \rightarrow p_{k + 1}$ follows from this.
\end{enumerate}

\end{document}
