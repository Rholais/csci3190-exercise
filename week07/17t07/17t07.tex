%!TEX program = xelatex
\documentclass[10pt, compress]{beamer}
\usepackage[titleprogressbar]{../../cls/beamerthemem}

\usepackage{booktabs}
\usepackage[scale=2]{ccicons}
\usepackage{minted}

\usepgfplotslibrary{dateplot}

\usemintedstyle{trac}

\setbeamertemplate{caption}[numbered]
\setbeamertemplate{theorems}[numbered]
\newtheorem{crl}{Corollary}[theorem]
\newtheorem*{solution*}{Solution}

\usepackage{algorithm}
\usepackage[noend]{algpseudocode}

\usepackage{version}
%\excludeversion{proof}
%\excludeversion{solution*}

\usepackage{mathtools}
\usepackage{multicol}
\usepackage{qtree}

\usepackage{tikz}

\makeatletter
\def\old@comma{,}
\catcode`\,=13
\def,{%
	\ifmmode%
	\old@comma\discretionary{}{}{}%
	\else%
	\old@comma%
	\fi%
}
\makeatother

\title{CSCI 3190 Tutorial of Week 07}
\subtitle{Assignment 1}
\author{LI Haocheng}
\institute{Department of Computer Science and Engineering}

\begin{document}

\maketitle

\begin{frame}[fragile]
\frametitle{Tautologies}
\begin{example}
	\begin{enumerate}
		\item $(p \land q) \equiv \neg (p \to \neg q)$
		\item $((p \land q) \to r) \equiv ((p \to r) \lor (q \to r))$
		\item $\neg (p \leftrightarrow q) \equiv (p \leftrightarrow \neg q)$
		\item $((p \land q) \to r) \not\equiv ((p \to r) \land (q \to r))$
	\end{enumerate}
\end{example}
\begin{proof}
	\begin{enumerate}
		\item<2-> $\neg (p \to \neg q) \equiv \neg (\neg p \lor \neg q) \equiv (p \land q)$.
		\item<3-> $((p \land q) \to r) \equiv (\neg (p \land q) \lor r) \equiv ((\neg p \lor \neg q) \lor (r \lor r)) \equiv ((\neg p \lor r) \lor (\neg q \lor r)) \equiv ((p \to r) \lor (q \to r))$.
		\item<4-> $\neg (p \leftrightarrow q) \equiv \neg ((p \lor \neg q) \land (\neg p \lor q)) \equiv ((p \land \neg q) \lor (\neg p \land q)) \equiv ((p \lor q) \land (\neg p \lor \neg q)) \equiv (p \leftrightarrow \neg q)$.
		\item<5-> Let $p = 1, q = 0, r = 0$, then $lhs = 1, rhs = 0$.
	\end{enumerate}
\end{proof}
\end{frame}

\begin{frame}[fragile]
\frametitle{Propositional Logic}
\begin{example}
	Explain, without using a truth table, why $(p \vee \neg q) \wedge (q \vee \neg r) \wedge (r \vee \neg p)$ is true when $p$, $q$, and $r$ have the same truth value and it is false otherwise.
\end{example}

\onslide<2>\begin{proof}
It is easy to evaluate that the expression is true when all variables have the same truth value. Suppose at least one is true and at least one is false. There are 2 combinations of 2 variables which have different truth value. These 2 combinations can form 2 disjunction expression with false value and one of them is in the conjunction expression so that the whole expression is false.
\end{proof}
\end{frame}

\begin{frame}[fragile]
\frametitle{Tautological Implications}
\begin{columns}
	\begin{column}{.7\linewidth}
		\begin{example}
			\begin{enumerate}
				\item $p \Rightarrow (p \lor q)$
				\item $(p \land q) \Rightarrow (p \to q)$
				\item $\neg (p \to q) \Rightarrow p$
				\item $((p \lor q) \land \neg p) \Rightarrow q$
			\end{enumerate}
		\end{example}
		\begin{proof}
			\begin{enumerate}
				\item<2-> $(\neg p \lor (p \lor q)) \equiv (1 \lor q) \equiv 1$.
				\item<3-> $(\neg(p \land q) \lor (\neg p \lor q)) \equiv (\neg p \lor \neg q \lor q) \equiv (\neg p \lor 1) \equiv 1$.
				\item<4-> $\neg (p \to q) \equiv \neg (\neg p \lor q) \equiv (p \land \neg q) \Rightarrow p$.
				\item<5-> $((p \lor q) \land \neg p) \equiv (0 \lor (\neg p \land q)) \equiv (\neg p \land q) \Rightarrow q$.
			\end{enumerate}
		\end{proof}
	\end{column}
	\begin{column}{.4\linewidth}
		\begin{definition}
			A compound proposition is \textbf{satisfiable} if there is an assignment of truth values to its variables that makes it true. When no such assignments exists, that is, when the compound proposition is false for all assignments of truth values to its variables, the compound proposition is \textbf{unsatisfiable}.
		\end{definition}
	\end{column}
\end{columns}
\end{frame}

\begin{frame}[fragile]
\frametitle{Propositional Satisfiability}
\begin{example}
	Determine whether each of the compound propositions $(p \vee \neg q) \wedge (q \vee \neg r) \wedge (r \vee \neg p) \wedge (p \vee q \vee r) \wedge	(\neg p \vee \neg q \vee \neg r)$ is satisfiable.
\end{example}
\onslide<2>\begin{solution*}
	Note that for $(p \vee \neg q) \wedge (q \vee \neg r) \wedge (r \vee \neg p) \wedge (p \vee q \vee r) \wedge	(\neg p \vee \neg q \vee \neg r)$	to be true, $(p \vee \neg q) \wedge (q \vee \neg r) \wedge (r \vee \neg p)$ and $(p \vee q \vee r) \wedge (\neg p \vee \neg q \vee \neg r)$ must both be true. For the first to be true, the three variables must have the same truth values, and
for the second to be true, at least one of three variables must be true and at least one must be false. However, these conditions are contradictory. From these observations we conclude
that no assignment of truth values to $p$, $q$, and $r$ makes $(p \vee \neg q) \wedge (q \vee \neg r) \wedge (r \vee \neg p) \wedge (p \vee q \vee r) \wedge	(\neg p \vee \neg q \vee \neg r)$ true. Hence, it is unsatisfiable.
\end{solution*}
\end{frame}

\begin{frame}[fragile]
\frametitle{Factorial Expression}

\end{frame}

\begin{frame}[fragile]
\frametitle{Example of Permutation}
\begin{example}
	How many times does \texttt{plan}, \texttt{than} and \texttt{both} NOT occur in permutations of the 26 letters?
\end{example}
\onslide<2>\begin{solution*}
	\texttt{plan} occurs $22! \times 23 = 23!$ times.
	
	\texttt{than} occurs $22! \times 23 = 23!$ times.
	
	\texttt{both} occurs $22! \times 23 = 23!$ times.
	
	\texttt{plan} and \texttt{than} occurs $0$ times.
	
	\texttt{plan} and \texttt{both} occurs $18! \times 19 \times 20 = 20!$ times.
	
	\texttt{than} and \texttt{both} occurs $20! \times 21 = 21!$ times.
	
	\texttt{plan}, \texttt{than} and \texttt{both} occurs $0$ times.
	
	So the answer is $26! - 23! \times 3 + 21! + 20!$.
\end{solution*}
\end{frame}

\begin{frame}[fragile]
\frametitle{Combination}
\begin{columns}
	\begin{column}{.4\linewidth}
		\begin{example}
			How many different committees of 3 students can be formed from a group of 4 students?
		\end{example}
		\onslide<2->\begin{solution*}
			4. 
		\end{solution*}
		\onslide<1->\begin{definition}
			An \textbf{$r$-combination} of elements of a set is an unordered selection of $r$ elements from the set.
		\end{definition}
	\end{column}
	\begin{column}{.6\linewidth}
		\begin{theorem}
			The number of $r$-combinations of a set with $n$ elements, where $0 \le r \le n$, equals $C(n, r) = \frac{n!}{r!(n - r)!}$.
		\end{theorem}
		\onslide<3>\begin{proof}
			By the product rule, $P(n, r) = C(n, r) \cdot P(r, r)$, which implies that\begin{equation}
				C(n, r) = \frac{P(n, r)}{P(r, r)} = \frac{\frac{n!}{(n - r)!}}{\frac{r!}{(r - r)!}} = \frac{n!}{r!(n - r)!}.
			\end{equation}
		\end{proof}
	\end{column}
\end{columns}
\end{frame}

\begin{frame}[fragile]
\frametitle{Combinatorial Proof}
\begin{columns}
	\begin{column}{.6\linewidth}
		\begin{crl}
			Let $0 \le r \le n$. Then $C(n, r) = C(n, n - r)$.
		\end{crl}
		\begin{definition}
			A \textbf{combinatorial proof} of an identity is a proof that uses counting arguments to prove that both sides of the identity count the same objects but in different ways or a proof that is based on showing that there is a bijection between the sets of objects counted by the two sides of the identity. These two types of proofs are called \textbf{double counting proofs} and \textbf{bijective proofs}, respectively.
		\end{definition}
	\end{column}
	\begin{column}{.4\linewidth}
		\begin{proof}
			Suppose that $|S| = n$. The function that maps a subset $A$ of $S$ to $\bar{A}$ is a bijection between subsets of $S$ with $r$ elements and subsets with $n - r$ elements. The identity $C(n, r) = C(n, n - r)$ follows because when there is a bijection between two finite sets, the two sets must have the same number of elements.
		\end{proof}
	\end{column}
\end{columns}
\end{frame}

\begin{frame}[fragile]
	\frametitle{Binomial Coefficients}
	\begin{example}
		Show that if $n$ and $k$ are integers with $1 \le k \le n$, then $\binom{n}{k} \le \frac{n^k}{2^{k - 1}}$.
	\end{example}
	\onslide<2>\begin{solution*}
		\begin{align}
		\begin{aligned}
		&\binom{n}{k}\\
		= & \frac{n(n - 1)(n - 2) \cdots (n - k + 1)}{k(k - 1)(k - 2) \cdots 2}\\
		\le & \frac{n \cdot n \cdot \cdots \cdot n}{2 \cdot 2 \cdot \cdots \cdot 2}\\
		= & \frac{n^k}{2^{k - 1}}.
		\end{aligned}
		\end{align}
	\end{solution*}
\end{frame}

\plain{Questions?}

\end{document}
