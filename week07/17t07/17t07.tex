%!TEX program = xelatex
\documentclass[10pt, compress, handout]{beamer}
\usepackage[titleprogressbar]{../../cls/beamerthemem}

\usepackage{booktabs}
\usepackage[scale=2]{ccicons}
\usepackage{minted}

\usepgfplotslibrary{dateplot}

\usemintedstyle{trac}

\setbeamertemplate{caption}[numbered]
\setbeamertemplate{theorems}[numbered]
\newtheorem{crl}{Corollary}[theorem]
\newtheorem*{solution*}{Solution}

\usepackage{algorithm}
\usepackage[noend]{algpseudocode}

\usepackage{version}
%\excludeversion{proof}
%\excludeversion{solution*}

\usepackage{mathtools}
\usepackage{multicol}
\usepackage{qtree}

\usepackage{tikz}

\makeatletter
\def\old@comma{,}
\catcode`\,=13
\def,{%
    \ifmmode%
    \old@comma\discretionary{}{}{}%
    \else%
    \old@comma%
    \fi%
}
\makeatother

\title{CSCI 3190 Tutorial of Week 07}
\subtitle{Assignment 1}
\author{LI Haocheng}
\institute{Department of Computer Science and Engineering}

\begin{document}

\maketitle

\begin{frame}[fragile]
\frametitle{Quantifiers}
\begin{columns}
    \begin{column}{.7\linewidth}
        \begin{example}
            Determine the truth value of the followings, where $x, y \in \mathbb{I}^+$.
            \begin{enumerate}
                \item $\forall x \forall y (2x^2 < y + 1)$
                \item $\forall x \exists y (2x^2 < y + 1)$
                \item $\exists x \forall y (2x^2 < y + 1)$
                \item $\exists x \exists y (2x^2 < y + 1)$
            \end{enumerate}
        \end{example}
        \begin{solution*}
            \begin{enumerate}
                \item<2-> False. Let $x = 1, y = 1$, then $2x^2 = y + 1$.
                \item<3-> True. $\forall x \exists y = 2x^2$ s.t. $2x^2 < y + 1$.
                \item<4-> False. Let $y = 1, 2x^2 < y + 1$, then $x < 1$.
                \item<5-> True. Let $x = 1, y = 2$, then $2x^2 < y + 1$.
            \end{enumerate}
        \end{solution*}
    \end{column}
    \begin{column}{.4\linewidth}
        \begin{example}
            Can you conclude that $A = B$ if $A, B$ and $C$ are sets such that: \begin{enumerate}
                \item $A \cup C = B \cup C$;
                \item $A \cap C = B \cap C$.
            \end{enumerate}
        \end{example}
        \begin{solution*}
            \begin{enumerate}
                \item<6-> No. Let $A = \emptyset, B = C \supset A$, then $A \cup C = B \cup C$.
                \item<7-> No. Let $A \ne B, C = \emptyset$, then $A \cap C = B \cap C$.
            \end{enumerate}
        \end{solution*}
    \end{column}
\end{columns}
\end{frame}

\begin{frame}
\frametitle{Relation}
\begin{example}
    Consider the relation $R$ over the set of all positive integers $\mathbb{I}^+$:    $R = \{(a, b) \mid |a - b| \equiv 0 \pmod{2}\}$
    \begin{enumerate}
        \item Is $R$ an equivalence relation? Explain your answer.
        \item How many equivalence classes of $R$ in $\mathbb{I}^+$ are there? What are they?
    \end{enumerate}
\end{example}
\begin{solution*}
    \begin{enumerate}
        \item<2-> Yes. \begin{itemize}
            \item $\forall a \in \mathbb{I}^+, |a - a| = 0$.
            \item $|b - a| = |a - b|$.
            \item $|a - c| \equiv a - c \equiv (a - b) + (b - c) \equiv |a - b| + |b - c| \pmod{2}$.
        \end{itemize}
        
        \item<3-> \begin{itemize}
            \item 2.
            \item $[1] = \{a \mid a \equiv 1 \pmod{2}\}$, $[2] = \{a \mid a \equiv 0 \pmod{2}\}$.
        \end{itemize}
    \end{enumerate}
\end{solution*}
\end{frame}

\plain{Questions?}

\end{document}
