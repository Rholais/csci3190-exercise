%!TEX program = xelatex
\documentclass[10pt, compress]{beamer}
\usepackage[titleprogressbar]{../../cls/beamerthemem}

\usepackage{booktabs}
\usepackage[scale=2]{ccicons}
\usepackage{minted}

\usepgfplotslibrary{dateplot}

\usemintedstyle{trac}

\setbeamertemplate{caption}[numbered]
\setbeamertemplate{theorems}[numbered]
\newtheorem{crl}{Corollary}[theorem]
\newtheorem*{solution*}{Solution}

\usepackage{algorithm}
\usepackage[noend]{algpseudocode}
\resetcounteronoverlays{algorithm}

\usepackage{version}
%\excludeversion{proof}
%\excludeversion{solution*}

\usepackage{mathtools}
\usepackage{multicol}
\usepackage{qtree}

\usepackage{tikz}

\makeatletter
\def\old@comma{,}
\catcode`\,=13
\def,{%
	\ifmmode%
	\old@comma\discretionary{}{}{}%
	\else%
	\old@comma%
	\fi%
}
\makeatother

\title{CSCI 3190 Tutorial of Week 07}
\subtitle{Recursive Algorithm}
\author{LI Haocheng}
\institute{Department of Computer Science and Engineering}

\begin{document}

\maketitle

\begin{frame}
\frametitle{Minimum}
\begin{example}
	Give a recursive algorithm for finding the minimum of a finite set of integers,
	making use of the fact that the minimum of $n$ integers is the smaller of the last integer
	in the list and the minimum of the first $n - 1$ integers in the list.
\end{example}
\onslide<2>\begin{solution*}
	\begin{algorithm}[H]
		\caption{A Recursive Algorithm for Minimum}
		\label{a-3}
		\begin{algorithmic}
			\Procedure{smallest}{$a_1, \cdots, a_n \in \mathbb{Z}$}
			\If{$n = 1$}
			\State\Return $a_1$
			\EndIf
			\State\Return $\min(a_n, $ \Call{smallest}{$a_1, \cdots, a_{n - 1}$})
			\EndProcedure
		\end{algorithmic}
	\end{algorithm}
\end{solution*}
\end{frame}

\begin{frame}[allowframebreaks]
\frametitle{Fibonacci}
\begin{example}
	Consider the computation of the $n$-th Fibonacci number:\begin{enumerate}
		\item Give the pseudo code of a recursive algorithm to compute the nth Fibonacci number. What is the
		complexity of your algorithm? Explain your answer.
		\item Give the pseudo code of an iterative algorithm to compute the nth Fibonacci number. What is the
		complexity of your algorithm? Explain your answer. 
	\end{enumerate}
\end{example}

\newpage

\begin{solution*}
	\begin{columns}
		\begin{column}{.6\linewidth}
			\begin{algorithm}[H]
				\caption{A Recursive Algorithm for Fibonacci}
				\label{a-4-1}
				\begin{algorithmic}
					\Procedure{Fib}{$n$}
					\If{$n = 0$}
					\State\Return $0$
					\EndIf
					\If{$n = 1$}
					\State\Return $1$
					\EndIf
					\State\Return \Call{Fib}{$n - 1$} + \Call{Fib}{$n - 2$}
					\EndProcedure
				\end{algorithmic}
			\end{algorithm}
			
			$O(a_n) = 2^n$.
		\end{column}
		
		\begin{column}{.5\linewidth}
			\begin{algorithm}[H]
				\caption{A Iterative Algorithm for Fibonacci}
				\label{a-4-2}
				\begin{algorithmic}
					\Procedure{IterFib}{$n$}
					\If{$n = 0$}\ \Return $0$
					\EndIf
					\State $x \coloneqq 0, y \coloneqq 1$
					\For{$i \coloneqq 1, 2, \cdots, n - 1$}
					\State $z \coloneqq x + y, x \coloneqq y, y \coloneqq z$
					\EndFor
					\State \Return $y$
					\EndProcedure
				\end{algorithmic}
			\end{algorithm}
			
			The number of additions follows the expression $b_n = n - 1$ so that $O(b_n) = n$.
		\end{column}
	\end{columns}
\end{solution*}
\end{frame}

\begin{frame}
\frametitle{Mode}
\begin{example}
	Give a recursive algorithm for finding a mode of a list of integers.
\end{example}
\onslide<2>\begin{solution*}
	\begin{algorithm}[H]
		\caption{A Recursive Algorithm for Minimum}
		\label{a-5}
		\begin{algorithmic}
			\Procedure{mode}{$a_1, \cdots, a_n \in \mathbb{Z}$}
			\If{$n = 1$}
			\Return $a_1$
			\EndIf
			\State $m \eqqcolon $ \Call{mode}{$a_1, \cdots, a_{n - 1}$}
			\If{$m = a_n$}
			\Return $a_n$
			\EndIf
			\State $numM \eqqcolon$ number of $m$ in $a_1, \cdots, a_n$
			\State $numN \eqqcolon$ number of $a_n$ in $a_1, \cdots, a_n$
			\If{$numM \le numN$}
			\Return $a_n$
			\EndIf
			\State\Return $m$
			\EndProcedure
		\end{algorithmic}
	\end{algorithm}
\end{solution*}
\end{frame}

\plain{Questions?}

\end{document}
