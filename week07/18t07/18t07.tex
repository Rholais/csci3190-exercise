%!TEX program = xelatex
\documentclass[10pt, compress, handout]{beamer}
\usepackage[titleprogressbar]{../../cls/beamerthemem}

\setbeamertemplate{caption}[numbered]
\setbeamertemplate{theorems}[numbered]
\newcounter{example}
\resetcounteronoverlays{example}
\newtheorem{crl}{Corollary}[theorem]
\newtheorem{eg}[example]{Example}
\newtheorem*{solution*}{Solution}

\usepackage{booktabs}
\usepackage[scale=2]{ccicons}
\usepackage{minted}

\usepackage{cleveref}
\crefname{example}{Example}{Examples}

\usepgfplotslibrary{dateplot}

\usemintedstyle{trac}

\usepackage{algorithm}
\usepackage[noend]{algpseudocode}
\resetcounteronoverlays{algorithm}

\usepackage{version}
%\excludeversion{proof}
%\excludeversion{solution*}

\usepackage{mathtools}
\usepackage{multicol}
\usepackage{qtree}

\usepackage{tikz}

\makeatletter
\def\old@comma{,}
\catcode`\,=13
\def,{%
    \ifmmode%
    \old@comma\discretionary{}{}{}%
    \else%
    \old@comma%
    \fi%
}
\makeatother

\title{CSCI 3190 Tutorial of Week 07}
\subtitle{Recurrence Relation}
\author{LI Haocheng}
\institute{Department of Computer Science and Engineering}

\begin{document}

\maketitle

\begin{frame}[fragile]
\frametitle{Factorial Expression}
\begin{columns}
    \begin{column}{.5\linewidth}
        \onslide<1->\begin{eg}
            How many ways are there to select a first-prize winner, a second-prize winner, and a third-prize winner from 100 different people who have entered a contest?
        \end{eg}
        \onslide<2->\begin{solution*}
            The number is the number of 3-permutations of a set of 100 elements. Consequently, the answer is ${}_{100}P_3 = 100 \times 99 \times 98 = 970200$.
        \end{solution*}
    \end{column}
    \begin{column}{.6\linewidth}
        \onslide<1->\begin{eg}
            How many times does \texttt{plan}, \texttt{than} and \texttt{both} NOT occur in permutations of the 26 letters?
        \end{eg}
        \onslide<3>\begin{solution*}
            \texttt{plan} occurs $23!$ times.

            \texttt{than} occurs $23!$ times.

            \texttt{both} occurs $23!$ times.

            \texttt{plan} and \texttt{than} occurs $0$ times.

            \texttt{plan} and \texttt{both} occurs $20!$ times.

            \texttt{than} and \texttt{both} occurs $21!$ times.

            \texttt{plan}, \texttt{than} and \texttt{both} occurs $0$ times.

            So the answer is $26! - 23! \times 3 + 21! + 20!$.
        \end{solution*}
    \end{column}
\end{columns}
\end{frame}

\begin{frame}[fragile]
\frametitle{Codeword Enumeration}
\begin{eg}
	\label{eg:8}
	Computer system considers a string of decimal digits a valid
	codeword if it contains an even number of 0 digits. For instance, 1230407869 is valid,
	whereas 120987045608 is not valid. Let $a_n$ be the number of valid $n$-digit codewords. Find
	a recurrence relation for $a_n$.
\end{eg}

\onslide<2>\begin{solution*}
	There are nine 1-digit codewords so that $a_1 = 9$.
	There are two ways to form
	a valid string with $n$ digits from a string with one fewer digit.
	First, it can be obtained by appending a valid string of $n - 1$ digits with a digit other than 0, which has $9a_{n - 1}$ ways.
	Second, it can be obtained by appending a 0 to a string of length $n - 1$ that is not valid, which has $10_{n - 1} - a_{n - 1}$ ways.
	Because all valid strings of length n are produced in one of these two ways, it follows that there are $a_n = 9 a_{n - 1} + (10^{n - 1} - a_{n - 1}) = 8 a_{n - 1} + 10^{n - 1}$ valid strings of length $n$.
\end{solution*}
\end{frame}

\begin{frame}[fragile]
\frametitle{Solve Codeword Enumeration}
\begin{eg}
Suppose that a valid codeword is an $n$-digit number in decimal notation containing an even number of 0s.
Let an denote the number of valid codewords of length $n$.
In~\Cref{eg:8} we showed that the sequence $\{a_n\}$ satisfies the recurrence relation $a_n = 8 a_{n - 1} + 10^{n - 1}$ and the initial condition $a_1 = 9$. Use generating functions to find an explicit formula for $a_n$.
\end{eg}

\onslide<2>\begin{solution*}
we extend this sequence	by setting $a_0 = 1$.
Let $G(x) = \sum_{n = 0}^{\infty} a_n x_n$ be the generating function of the sequence $\{a_n\}$,
then $G(x) - 8x G(x) = 1 + \frac{x}{1 - 10x}$.
Solving for $G(x)$ shows that $G(x) = \frac{1 - 9x}{(1 - 8x)(1 - 10x)} = \frac{1}{2} \left(\frac{1}{1 - 8x} + \frac{1}{1 - 10x}\right) = \sum_{n = 0}^\infty \frac{1}{2} (8^n + 10^n) x^n$.
Consequently, we have shown that $a_n = \frac{1}{2} (8^n + 10^n)$.
\end{solution*}
\end{frame}

\begin{frame}[fragile]
\frametitle{Mathematical Induction}
\begin{eg}
For $n \in \mathbb{N}^+$, prove each of the following by mathematical induction:\begin{enumerate}
\item $3 | 2^{2n + 1} + 1$;
\item $9 | n^3 + (n + 1)^3 + (n + 2)^3$.
\end{enumerate}
\end{eg}
\begin{proof}
\begin{enumerate}
\item<2-> \begin{description}
	\item[Base] Let $n = 0$, $3 \mid 2^1 + 1 = 3$.
	\item[Induction] Suppose $3 \mid 2^{2n - 1} + 1$, $2^{2n + 1} = 2^{2n - 1} \times 3 + 2^{2n - 1} + 1$
	$\equiv 0 \pmod{3}$.
\end{description}
\item<3> \begin{description}
	\item[Base] Let $n = 0$, $9 \mid 1^3 + 2^3$.
	\item[Induction] Suppose $9 \mid (n - 1)^3 + n^3 + (n + 1)^3$, $n^3 + (n + 1)^3 + (n + 2)^3$
	$= (n - 1)^3 + n^3 + (n + 1)^3 + 3((n - 1)^2 + (n - 1)(n + 2) + (n + 2)^2)$
	$= (n - 1)^3 + n^3 + (n + 1)^3 + 9(n^2 + n + 1) \equiv 0 \pmod{9}$.
\end{description}
\end{enumerate}
\end{proof}
\end{frame}

\plain{Questions?}

\end{document}
