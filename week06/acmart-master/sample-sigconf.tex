\documentclass[sigconf]{acmart}

\usepackage{booktabs} % For formal tables


% Copyright

\setcopyright{none}
%\setcopyright{acmcopyright}
%\setcopyright{acmlicensed}
%\setcopyright{rightsretained}
%\setcopyright{usgov}
%\setcopyright{usgovmixed}
%\setcopyright{cagov}
%\setcopyright{cagovmixed}


% DOI
%\acmDOI{10.475/123_4}

% ISBN
%\acmISBN{123-4567-24-567/08/06}

%Conference
\acmConference[CSCI'3190]{Introduction to Discrete Mathematics and Algorithms}{2017}{The Chinese University of Hong Kong}
\acmYear{2017}
\copyrightyear{2017}


%\acmArticle{4}
%\acmPrice{15.00}

% These commands are optional
%\acmBooktitle{Transactions of the ACM Woodstock conference}
%\editor{Jennifer B. Sartor}
%\editor{Theo D'Hondt}
%\editor{Wolfgang De Meuter}

\makeatletter
\def\old@comma{,}
\catcode`\,=13
\def,{%
	\ifmmode%
	\old@comma\discretionary{}{}{}%
	\else%
	\old@comma%
	\fi%
}
\makeatother

\begin{document}
\title{Assignment 1}
%\titlenote{Produces the permission block, and copyright information}
%\subtitle{}
%\subtitlenote{The full version of the author's guide is available as \texttt{acmart.pdf} document}

% The default list of authors is too long for headers.
%\renewcommand{\shortauthors}{B. Trovato et al.}

%
% The code below should be generated by the tool at
% http://dl.acm.org/ccs.cfm
% Please copy and paste the code instead of the example below. 
%

%\keywords{ACM proceedings, \LaTeX, text tagging}


\maketitle

%\input{samplebody-conf}
\section{Tautologies}
\begin{enumerate}
	\item $\neg (p \to \neg q) \equiv \neg (\neg p \lor \neg q) \equiv (p \land q)$.
	\item $((p \land q) \to r) \equiv (\neg (p \land q) \lor r) \equiv ((\neg p \lor \neg q) \lor (r \lor r)) \equiv ((\neg p \lor r) \lor (\neg q \lor r)) \equiv ((p \to r) \lor (q \to r))$.
	\item $\neg (p \leftrightarrow q) \equiv \neg ((p \lor \neg q) \land (\neg p \lor q)) \equiv ((p \land \neg q) \lor (\neg p \land q)) \equiv ((p \lor q) \land (\neg p \lor \neg q)) \equiv (p \leftrightarrow \neg q)$.
	\item Let $p = 1, q = 0, r = 0$, then $lhs = 1, rhs = 0$.
\end{enumerate}

\section{Tautological Implications}
\begin{enumerate}
	\item $(\neg p \lor (p \lor q)) \equiv (1 \lor q) \equiv 1$.
	\item $(\neg(p \land q) \lor (\neg p \lor q)) \equiv (\neg p \lor \neg q \lor q) \equiv (\neg p \lor 1) \equiv 1$.
	\item $\neg (p \to q) \equiv \neg (\neg p \lor q) \equiv (p \land \neg q) \Rightarrow p$.
	\item $((p \lor q) \land \neg p) \equiv (0 \lor (\neg p \land q)) \equiv (\neg p \land q) \Rightarrow q$.
\end{enumerate}

\section{Quantifiers}
\begin{enumerate}
	\item False. Let $x = 1, y = 1$, then $2x^2 = y + 1$.
	\item True. $\forall x \exists y = 2x^2$ s.t. $2x^2 < y + 1$.
	\item False. Let $y = 1, 2x^2 < y + 1$, then $x < 1$.
	\item True. Let $x = 1, y = 2$, then $2x^2 < y + 1$.
\end{enumerate}

\section{Sets}
\begin{enumerate}
	\item No. Let $A = \emptyset, B = C \supset A$, then $A \cup C = B \cup C$.
	\item No. Let $A \ne B, C = \emptyset$, then $A \cap C = B \cap C$.
\end{enumerate}

\section{Functions}
\begin{enumerate}
	\item $5^6 = 15625$.
	\item $C(6, 2) \times P(5, 5) = 1800$.
\end{enumerate}

\section{Closures}
\subsection{Reflexive Closure of the Symmetric Closure}
\begin{description}
	\item[Symmetric Closure] $\{(1, 3), (1, 4), (2, 4), (3, 1), (4, 1), (4, 2)\}$.
	\item[Reflexive Closure] $\{(1, 1), (1, 3), (1, 4), (2, 2), (2, 4), (3, 1), (3, 3), (4, 1), (4, 2), (4, 4)\}$.
\end{description}

\subsection{Transitive Closure of the Reflexive closure}
\begin{description}
	\item[Reflexive Closure] $\{(1, 1), (1, 3), (1, 4), (2, 2), (3, 3), (4, 2), (4, 4)\}$.
	\item[Transitive Closure] $\{(1, 1), (1, 3), (1, 4), (2, 2), (3, 3), (4, 2), (4, 4)\}$.
\end{description}

\section{Relation}
\subsection{Equivalence Relation}
Yes. \begin{enumerate}
	\item $\forall a \in \mathbb{I}^+, |a - a| = 0$.
	\item $|b - a| = |a - b|$.
	\item $|a - c| \equiv a - c \equiv (a - b) + (b - c) \equiv |a - b| + |b - c| \pmod{2}$.
\end{enumerate}

\subsection{Equivalence Classes}
\begin{enumerate}
	\item 2.
	\item $[1] = \{a \mid a \equiv 1 \pmod{2}\}$, $[2] = \{a \mid a \equiv 0 \pmod{2}\}$.
\end{enumerate}

\section{Equivalence Relation}
\begin{enumerate}
	\item \begin{description}
		\item[$\Rightarrow$] Let $c \in [a]$, then $(c, a) \in R$. Since $(a, b) \in R$, $(c, b) \in R$. Therefore, $c \in [b]$, so that $[a] \subseteq [b]$. Similarly, we can prove $[b] \subseteq [a]$. Thus $[a] = [b]$.
		\item[$\Leftarrow$] $[a] = [b] \Rightarrow b \in [a] \Rightarrow (a, b) \in R$.
	\end{description}
	\item \begin{description}
		\item[$\Rightarrow$] $[a] \cap [b] \ne \emptyset \Rightarrow \exists c \in [a] \cap [b] \Rightarrow (a, c) \in R \land (c, b) \in R \Rightarrow (a, b) \in R$.
		\item[$\Leftarrow$] $(a, b) \in R \Rightarrow b \in [a] \Rightarrow b \in [a] \cap [b] \ne \emptyset$.
	\end{description}
\end{enumerate}

\section{Oranges}

Since there are enough oranges from Taiwan and U.S.A., only oranges from Mainland are under constraint. We will first calculated the number of ways without constraints and then exclude the ways require additional oranges from Mainland.

It's easy to see that the number of ways require 11 or 12 oranges from Mainland is 2 or 1 respectively. Therefore, the number of ways need to be excluded is 3.

To calculate the number of ways without constraints, suppose there are 14 positions in a line and we need to place 2 separators in 2 positions which separate the remaining to 3 parts for oranges from Mainland, Taiwan and U.S.A. respectively. The number of ways to place separators is $C(14, 2)$ so that the number of ways $C(14, 2) - 3 = 88$.

\section{Strings}
\begin{enumerate}
	\item $26^6 - 25^6 = 64775151$.
	\item $26^6 - 25^6 \times 2 + 24^6 = 11737502$.
	\item $P(24, 4) \times 5 = 1275120$.
	\item $C(6, 2) \times P(24, 4) = 3825360$.
\end{enumerate}

\section{Solutions}
\begin{enumerate}
	\item $C(31, 3) = 4495$.
	\item \begin{align}
		\begin{aligned}
			&\left(\sum_{i = 0}^{8} x^i\right) \cdot \left(\sum_{i = 0}^{6} x^i\right) \cdot \left(\sum_{i = 0}^{12} x^i\right) \cdot \left(\sum_{i = 0}^{9} x^i\right)\\
			= & \frac{x^9 - 1}{x - 1} \cdot \frac{x^7 - 1}{x - 1} \cdot \frac{x^{13} - 1}{x - 1} \cdot \frac{x^{10} - 1}{x - 1}\\
			= & \frac{x^{39} - x^{32} - \cdots}{x^4 - 4x^3 + 6x^2 - 4x^2 + 1}\\
			= & x^{35} + 4 x^{34} + 10 x^{33} + 20 x^{32} + 35 x^{31} + 56 x^{30} + 84 x^{29} + 119 x^{28} + \cdots
		\end{aligned}
	\end{align} There are 119 solutions.
\end{enumerate}

\section{Addresses}
Group addresses $2 i$ and $2 i + 1$ in hole $i$. Put 51 houses in the 50 holes so that 1 hole has 2 houses. Therefore, these 2 houses have consecutive addresses.

\section{Closed form}
\begin{enumerate}
	\item $-\frac{1}{1 - x}$.
	\item $\frac{2x}{1 - 2x}$.
	\item $\left(\frac{1}{1 - x}\right)' - \frac{2}{1 - x} = \frac{-1 + 2x}{(1 - x)^2}$
\end{enumerate}

\section{Sub-List}
Let the list be $\{s_i\}$. Considering $a_i = \sum_{j = 1}^{i} s_j$, if one of the $\{a_i\}$ is divisible by $n$, then we are done. Otherwise, by pigeonhole principle, $\exists k < l, a_k \equiv a_l \pmod{n}$, then $\{a_i \mid k < i \le l\}$ is what we want.

\section{Stuffed Animals}
\begin{align}
	\begin{aligned}
		& \left(\sum_{i = 1}^{3}x^i\right)^6\\
		= & \left(\frac{x^4 - x}{x - 1}\right)^6\\
		= & \frac{x^{24} - 6 x^{21} +\cdots}{x^6 - 6 x^5 + 15 x^4 - 20 x^3 + \cdots}\\
		= & x^{18} + 6 x^{17} + 21 x^{16} + 56 x^{15} + \cdots
	\end{aligned}
\end{align} There are 56 solutions.

\bibliographystyle{ACM-Reference-Format}
%\bibliography{sample-bibliography} 

\end{document}
