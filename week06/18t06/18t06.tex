%!TEX program = xelatex
\documentclass[10pt, compress, handout]{beamer}
\usepackage[titleprogressbar]{../../cls/beamerthemem}

\setbeamertemplate{caption}[numbered]
\setbeamertemplate{theorems}[numbered]
\newcounter{example}
\resetcounteronoverlays{example}
\newtheorem{crl}{Corollary}[theorem]
\newtheorem{eg}[example]{Example}
\newtheorem*{solution*}{Solution}

\usepackage{booktabs}
\usepackage[scale=2]{ccicons}
\usepackage{minted}

\usepackage{cleveref}
\crefname{example}{Example}{Examples}

\usepgfplotslibrary{dateplot}

\usemintedstyle{trac}

\usepackage{algorithm}
\usepackage[noend]{algpseudocode}
\resetcounteronoverlays{algorithm}

\usepackage{version}
%\excludeversion{proof}
%\excludeversion{solution*}

\usepackage{mathtools}
\usepackage{multicol}
\usepackage{qtree}

\usepackage{tikz}

\makeatletter
\def\old@comma{,}
\catcode`\,=13
\def,{%
	\ifmmode%
	\old@comma\discretionary{}{}{}%
	\else%
	\old@comma%
	\fi%
}
\makeatother

\title{CSCI 3190 Tutorial of Week 06}
\subtitle{Counting}
\author{LI Haocheng}
\institute{Department of Computer Science and Engineering}

\begin{document}

\maketitle

\begin{frame}[fragile]
\frametitle{Permutation}
\begin{columns}
	\begin{column}{.6\linewidth}
		\begin{definition}
			A \textbf{permutation} of a set of distinct objects is an ordered arrangement of these objects.
		\end{definition}
		\begin{definition}
			An ordered arrangement of $r$ elements of a set is called an \textbf{$r$-permutation}.
		\end{definition}
		\begin{theorem}
			\label{thm:p}
			If $n$ is a positive integer and $r$ is an integer with $1 \le r \le n$, then there are $P(n, r) = n(n - 1)(n - 2) \cdots (n - r + 1)$
			$r$-permutations of a set with $n$ distinct elements.
		\end{theorem}
	\end{column}
	\begin{column}{.5\linewidth}
		\onslide<2>\begin{proof}
			The first element of the
			permutation can be chosen in $n$ ways. There are $n - 1$ ways to choose the second element of the permutation. Similarly, there are $n - 2$ ways to choose the third element, and so on, until there are exactly $n - (r - 1) = n - r + 1$ ways to choose the $r$th element. Consequently, by the product rule, there are
			$n(n - 1)(n - 2) \cdots (n - r + 1)$
			$r$-permutations of the set.
		\end{proof}
	\end{column}
\end{columns}
\end{frame}

\begin{frame}[fragile]
\frametitle{Onto}
\begin{eg}
	Consider the set $X$ of all functions $f \colon A \rightarrow B$, where $A = \{1, 2, 3, 4, 5, 6, 7\}$, $B = \{a, b, c, d, e\}$. How many functions in $X$ are onto?
\end{eg}
\onslide<2>\begin{solution*}
	For the first situation that 1 of $B$ matches 3 of $A$ and others match 1 of $A$ for each, we select 4 of 5 in $B$ and for each of them select 1 different item in $A$ so that there are ${}_4 C_1 \times {}_7 P_4$ functions in the first situation.
	
	For the second situation that 2 of $B$ match 2 of $A$ for each and others match 1 of $A$ for each, we select 3 of 5 in $B$, for each of them select 1 different item in $A$ and then from the rest 4 items in $A$ select 2 of them to match 1 of 2 rest item in $B$ so that there are ${}_5 C_2 \times {}_7 P_3 \times {}_4 C_2$ functions in the section situation.
	
	Totally, there are ${}_5 C_1 \times {}_7 P_4 + {}_5 C_2 \times {}_7 P_3 \times {}_4 C_2 = 16800$ functions.
\end{solution*}
\end{frame}

\begin{frame}[fragile]
\frametitle{Tokens}
\begin{eg}
	Use generating functions to determine the number of ways to insert tokens worth \$1, \$2,
	and \$5 into a vending machine to pay for an item that costs $r$ dollars in both the cases when
	the order in which the tokens are inserted does not matter.
\end{eg}
\onslide<2>\begin{solution*}
	Because
	we can use any number of \$1 tokens, any number of \$2 tokens, and any number of \$5 tokens,
	the answer is the coefficient of $x^r$ in the generating function \begin{align}
		\begin{aligned}
		& \left(\sum_{i = 0}^{\infty} x^i\right) \left(\sum_{i = 0}^{\infty} x^{2i}\right) \left(\sum_{i = 0}^{\infty} x^{5i}\right)\\
		= & \frac{1}{(1 - x)(1 - x^2)(1 - x^5)}.
		\end{aligned}
	\end{align}
\end{solution*}
\end{frame}

\begin{frame}[fragile]
\frametitle{Recurrence Relation}
\begin{eg}
	Solve the recurrence relation $a_k = 3a_{k - 1}$ for $k = 1, 2, 3, \ldots$ and initial condition $a_0 = 2$.
\end{eg}
\onslide<2>\begin{solution*}
	Let $G(x)$ be the generating function for the sequence $\{a_k\}$, that is, $G(x) = \sum_{k = 0}^{\infty} a_k x^k$.
	Using the recurrence relation, we see that \begin{equation}
		G(x) - 3xG(x) = \sum_{k = 0}^{\infty} a_k x^k - 3 \sum_{k = 1}^{\infty} a_{k - 1} x^{k} = a_0 + \sum_{k = 1}^{\infty} (a_k - 3 a_{k - 1}) x^{k}	= 2.
	\end{equation}
	Solving for $G(x)$ shows that $G(x) = \frac{2}{1 - 3x} = \sum_{k = 0}^\infty 2 \cdot 3^k x^k$.
	Consequently, $a_k = 2 \cdot 3^k$.
\end{solution*}
\end{frame}

\begin{frame}[fragile]
\frametitle{Raccoons}
\begin{eg}
	Let the lifespan of a raccoon be exactly 6 years.
	Suppose there are 4 new-born raccoons at the 0th year and the number of new-born raccoons in each year is 3 times
	that in the previous year. Let $b_r$ be the number of raccoons on $r$-th year where $r \ge 0$.
	Give a closed form generating function for $b_r$.
\end{eg}

\onslide<2>\begin{solution*}
	Let $N(x)$ be the generating function for the number sequence of newborn raccoons.
	Using the recurrence relation, we see that $N(x) - 3xN(x) = 4$.
	Solving for $N(x)$ shows that $N(x) = \frac{4}{1 - 3x}$.
	Let $B(x)$ be the generating function for the number sequence of raccoons, then \begin{equation}
	B(x) = \sum_{i = 0}^{5} x^i N(x) = N(x) \left(\sum_{i = 0}^{5} x^i\right) = N(x) \cdot \frac{(1 - x^6)}{1 - x} = \frac{4(1 - x^6)}{(1 - 3x)(1 - x)}.
	\end{equation}
\end{solution*}
\end{frame}

\plain{Questions?}

\end{document}
