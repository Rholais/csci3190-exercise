% This is "sig-alternate.tex" V2.1 April 2013
% This file should be compiled with V2.5 of "sig-alternate.cls" May 2012
%
% This example file demonstrates the use of the 'sig-alternate.cls'
% V2.5 LaTeX2e document class file. It is for those submitting
% articles to ACM Conference Proceedings WHO DO NOT WISH TO
% STRICTLY ADHERE TO THE SIGS (PUBS-BOARD-ENDORSED) STYLE.
% The 'sig-alternate.cls' file will produce a similar-looking,
% albeit, 'tighter' paper resulting in, invariably, fewer pages.
%
% ----------------------------------------------------------------------------------------------------------------
% This .tex file (and associated .cls V2.5) produces:
%       1) The Permission Statement
%       2) The Conference (location) Info information
%       3) The Copyright Line with ACM data
%       4) NO page numbers
%
% as against the acm_proc_article-sp.cls file which
% DOES NOT produce 1) thru' 3) above.
%
% Using 'sig-alternate.cls' you have control, however, from within
% the source .tex file, over both the CopyrightYear
% (defaulted to 200X) and the ACM Copyright Data
% (defaulted to X-XXXXX-XX-X/XX/XX).
% e.g.
% \CopyrightYear{2007} will cause 2007 to appear in the copyright line.
% \crdata{0-12345-67-8/90/12} will cause 0-12345-67-8/90/12 to appear in the copyright line.
%
% ---------------------------------------------------------------------------------------------------------------
% This .tex source is an example which *does* use
% the .bib file (from which the .bbl file % is produced).
% REMEMBER HOWEVER: After having produced the .bbl file,
% and prior to final submission, you *NEED* to 'insert'
% your .bbl file into your source .tex file so as to provide
% ONE 'self-contained' source file.
%
% ================= IF YOU HAVE QUESTIONS =======================
% Questions regarding the SIGS styles, SIGS policies and
% procedures, Conferences etc. should be sent to
% Adrienne Griscti (griscti@acm.org)
%
% Technical questions _only_ to
% Gerald Murray (murray@hq.acm.org)
% ===============================================================
%
% For tracking purposes - this is V2.0 - May 2012

\documentclass{../../cls/sig-alternate-05-2015}
\usepackage{booktabs}
\usepackage{textcomp}


\begin{document}

% Copyright
%\setcopyright{acmcopyright}
%\setcopyright{acmlicensed}
%\setcopyright{rightsretained}
%\setcopyright{usgov}
%\setcopyright{usgovmixed}
%\setcopyright{cagov}
%\setcopyright{cagovmixed}


% DOI
%\doi{10.475/123_4}

% ISBN
%\isbn{123-4567-24-567/08/06}

%Conference
%\conferenceinfo{PLDI '13}{June 16--19, 2013, Seattle, WA, USA}

%\acmPrice{\$15.00}

%
% --- Author Metadata here ---
%\conferenceinfo{WOODSTOCK}{'97 El Paso, Texas USA}
%\CopyrightYear{2007} % Allows default copyright year (20XX) to be over-ridden - IF NEED BE.
%\crdata{0-12345-67-8/90/01}  % Allows default copyright data (0-89791-88-6/97/05) to be over-ridden - IF NEED BE.
% --- End of Author Metadata ---

%\\TODO:1.tautology prove section 2. set section operation proof 3. function section basic proving onto bijection...
\title{CSCI 3190 \\ Introduction to Discrete Mathematics and Algorithms}
\subtitle{Extended Exercise 5}

\maketitle
\begin{abstract}

\end{abstract}

\keywords{}

\section{Induction and Recursion}
\subsection{Mathematical Induction}
\begin{enumerate}
\item Prove that if $h > -1$, then $1 + nh \le (1 + h)^n$ for all nonnegative integers $n$. This is called \textbf{Bernoulli's inequality}.

\item Prove that for every positive integer $n$,
\begin{equation}
    \sum_{i = 1}^{n} \frac{1}{\sqrt{i}} > 2(\sqrt{n + 1} - 1).
\end{equation}

\item The \textbf{harmonic numbers} $H_j, j = 1, 2, 3, \ldots$, are defined by \begin{equation}
    H_j = \sum_{i = 1}^j \frac{1}{i}.
\end{equation}
Prove that $H_{2^n} \le 1 + n$ whenever $n$ is a nonnegative integer.

\item Prove that $n^2 - 1$ is divisible by 8 whenever $n$ is an odd positive integer.

\item Prove that if $n$ is a positive integer, then 133 divides
$11^{n + 1} + 12^{2n - 1}.$

\item A knight on a chessboard can move one space horizontally
(in either direction) and two spaces vertically (in
either direction) or two spaces horizontally (in either direction)
and one space vertically (in either direction).
Suppose that we have an infinite chessboard,
made up of all squares $(m, n)$ where $m$ and $n$ are nonnegative integers that denote the row number and the column number of the square,
respectively.
Use mathematical induction to show that a knight starting at $(0, 0)$ can visit every square using a finite sequence of moves.

\item Show that \begin{align}
\begin{aligned}
    & [(p_1 \rightarrow p_2) \land (p_2 \rightarrow p_3) \land \cdots (p_{n - 1} \rightarrow p_{n})]\\
    & \qquad \rightarrow [(p_1 \land p_2 \land \cdots \land p_{n - 1}) \rightarrow p_{n}]
\end{aligned}
\end{align} is a tautology whenever $p_1, p_2, \ldots, p_n$ are propositions, where $n \ge 2$.

\item Let $a_1, a_2, \ldots, a_n$ be positive real numbers.
The \textbf{arithmetic mean} of these numbers is defined by \begin{equation}
    A = \frac{1}{n} \sum_{i = 1}^n a_i,
\end{equation} and the \textbf{geometric mean} of these numbers is defined by \begin{equation}
    G = \left(\prod_{i = 1}^{n} a_i\right)^\frac{1}{n}.
\end{equation}
Use mathematical induction to prove that $A \ge G$.

\item Show that if $I_1, I_2, \ldots, I_n$ is a collection of open intervals on the real number line,
$n \ge 2$,
and every pair of these intervals has a nonempty intersection,
that is,
$I_i \cap I_j = \emptyset$ whenever $1 \le i \le n$ and $1 \le j \le n$,
then the intersection of all these sets is nonempty, that is,
$I_1 \cap I_2 \cap \cdots \cap I_n \ne \emptyset$.

\item Show that a three-dimensional $2^n \times 2^n \times 2^n$ checkerboard with one $1 \times 1 \times 1$ cube missing can be completely covered by $2 \times 2 \times 2$ cubes with one $1 \times 1 \times 1$ cube removed.

\end{enumerate}

\subsection{Strong Induction and Well-Ordering}
\begin{enumerate}
\item Which amounts of money can be formed using just two-dollar bills and five-dollar bills? Prove your answer using strong induction.

\item Use strong induction to prove that $\sqrt{2}$ is irrational.

\item A jigsaw puzzle is put together by successively joining
pieces that fit together into blocks. A move is made each
time a piece is added to a block, or when two blocks
are joined. Use strong induction to prove that no matter
how the moves are carried out, exactly $n - 1$ moves are
required to assemble a puzzle with n pieces.

\item Showthat if $a_1, a_2, \ldots, a_n$ are $n$ distinct real numbers, exactly $n - 1$ multiplications are used to compute the product of these $n$ numbers no matter how parentheses are inserted into their product.

\end{enumerate}

\subsection{Recursive Definitions and Structural Induction}
\begin{enumerate}
\item Give a recursive definition of $P_m(n)$,
the product of the integer $m$ and the nonnegative integer $n$.

\item The \textbf{Fibonacci sequence},
$f_0, f_1, f_2, \ldots$,
is defined by the initial conditions $f_0 = 0, f_1 = 1$,
and the recurrence relation \begin{equation}
    f_n = f_{n - 1} + f_{n - 2}
\end{equation} for $n = 2, 3, 4, \cdots$.
Show that $\sum_{i = 0}^{2n - 1} f_i f_{i + 1} = f_{2n}^2$ when $n$ is a positive integer.

\item Give a recursive definition of
\begin{enumerate}
    \item the set of even integers.
    \item the set of positive integers congruent to $2$ modulo $3$.
    \item the set of positive integers not divisible by $5$.
\end{enumerate}


\item A partition of a positive integer $n$ is a way to write $n$ as a sum of positive integers where the order of terms in
the sum does not matter.
For instance, $7 = 3 + 2 + 1 + 1$ is a partition of $7$.
Let $P_m$ equal the number of different partitions of $m$,
and let $P_{m, n}$ be the number of different ways to express $m$ as the sum of positive integers not exceeding $n$. \begin{enumerate}
    \item Show that $P_{m, m} = P_m$.
    \item Show that the following recursive definition for $P_{m, n}$ is correct: \begin{equation}
        P_{m, n} = \begin{cases}
        1 & \text{if } m = 1 \lor n = 1,\\
        P_{m, m} & \text{if } m < n,\\
        1 + P_{m, m - 1} & \text{if } m = n > 1,\\
        P_{m, n - 1} + P_{m - n, n} & \text{if } m > n > 1.
        \end{cases}
    \end{equation}
    \item Find the number of partitions of 5 and of 6 using this recursive definition.
\end{enumerate}

\item Consider an inductive definition of a version of \textbf{Ackermann's function}.
This functionwas named after Wilhelm Ackermann,
a German mathematician who was a student of the great mathematician David Hilbert.
Ackermann's function plays an important role in the theory of recursive functions and in the study of the complexity of certain algorithms involving set unions.
\begin{equation}
    A(m, n) = \begin{cases}
    2n & \text{if } m = 0,\\
    0 & \text{if } m \ge 1 \land n = 0,\\
    2 & \text{if } m \ge 1 \land n = 1,\\
    A(m - 1, A(m, n - 1)) & \text{if } m \ge 1 \land n \ge 2.
    \end{cases}
\end{equation}
Prove that $A(m, n + 1) > A(m, n)$ whenever $m$ and $n$ are
nonnegative integers.

\end{enumerate}

\subsection{Recursive Algorithms}
\begin{enumerate}
\item Devise a recursive algorithm for computing the greatest
common divisor of two nonnegative integers $a$ and $b$ with
$a < b$ using the fact that $gcd(a, b) = gcd(a, b - a)$.

\item Devise a recursive algorithm for computing $n^2$ where $n$ is a nonnegative integer,
using the fact that \begin{equation}
    (n + 1)^2 = n^2 + 2n + 1.
\end{equation}
Then prove that this algorithm is correct.

\item Devise a recursive algorithm to find the $n$th term of the sequence defined by \begin{equation}
    \begin{cases}
    a_0 = 1,\\
    a_1 = 2,\\
    a_n = a_{n-1} \cdot a_{n-2} & \text{for } n = 2, 3, 4, \ldots.
    \end{cases}
\end{equation}

\item Give a recursive algorithm for tiling a $2^n \times 2^n$ checkerboard with one square missing using right triominoes.

\item Prove that the merge sort algorithm is correct.
\end{enumerate}

\section{Advanced Counting Techniques}
\subsection{Applications of Recurrence Relations}
\begin{enumerate}
\item \begin{enumerate}
    \item Find a recurrence relation for the number of bit strings
    of length $n$ that contain a pair of consecutive $0$s.
    \item What are the initial conditions?
    \item How many bit strings of length seven contain two
    consecutive $0$s?
\end{enumerate}

\item \begin{enumerate}
    \item Find a recurrence relation for the number of ways to
    climb $n$ stairs if the person climbing the stairs can take
    one stair or two stairs at a time.
    \item What are the initial conditions?
    \item In how many ways can this person climb a flight of
    eight stairs?
\end{enumerate}
    
\item A string that contains only 0s, 1s, and 2s is called a \textbf{ternary string}. \begin{enumerate}
    \item Find a recurrence relation for the number of ternary
    strings of length $n$ that do not contain two consecutive
    0s or two consecutive 1s.
    \item What are the initial conditions?
    \item How many ternary strings of length six do not contain
    two consecutive 0s or two consecutive 1s?
\end{enumerate}

\item \begin{enumerate}
    \item Find a recurrence relation for the number of ternary
    strings of length $n$ that do not contain consecutive
    symbols that are the same.
    \item What are the initial conditions?
    \item How many ternary strings of length six do not contain
    consecutive symbols that are the same?
\end{enumerate}

\item \begin{enumerate}
    \item Find the recurrence relation satisfied by $R_n$,
    where $R_n$ is the number of regions that a plane is divided into by $n$ lines,
    if no two of the lines are parallel and no three of the lines go through the same point.
    \item Find $R_n$ using iteration.
\end{enumerate}

\item \begin{enumerate}
    \item Find the recurrence relation satisfied by $S_n$,
    where $S_n$ is the number of regions into which three-dimensional space is divided by $n$ planes if every three of the planes meet in one point,
    but no four of the planes go through the same point.
    \item Find $S_n$ using iteration.
\end{enumerate}

\item How many bit sequences of length seven contain an even
number of $0$s?

 \item Let $S(m, n)$ denote the number of onto functions from
a set with $m$ elements to a set with $n$ elements. Show
that $S(m, n)$ satisfies the recurrence relation
\begin{equation}
S(m, n) = n^m - \sum_{k=1}^{n-1} C(n, k)S(m, k)
\end{equation}
whenever $m \geq n$ and $n > 1$, with the initial condition
$S(m, 1) = 1$.

\end{enumerate}

\subsection{Solving Linear Recurrence Relations}


%\item The \textbf{harmonic numbers} $H_j, j = 1, 2, 3, \ldots$, are defined by \begin{equation}
%    H_j = \sum_{i = 1}^j \frac{1}{i}.
%\end{equation}
%Prove that $H_{2^n} \le 1 + n$ whenever $n$ is a nonnegative integer.


\begin{enumerate}
\item
Find the solution to $a_n=7a_{n-2}+6a_{n-3}$ with $a_0=9$, $a_1 = 10$, and $a_2=32$.

\item Solve the recurrence relation \begin{equation}
\begin{cases}
a_0 = 5,\\
a_1 = -9,\\
a_2 = 15,\\
a_n = -3a_{n-1} - 3a_{n-2} - a_{n-3}.
\end{cases}
\end{equation}

\item What is the general form of the solutions of a linear homogeneous recurrence relation if its characteristic equation
has roots $1, 1, 1, 1, -2, -2, -2, 3, 3, -4$?

\item \begin{enumerate}
    \item Determine values of the constants $A$ and $B$ such
    that $a_n = An + B$ is a solution of recurrence relation $a_n = 2a_{n - 1} + n + 5$.
    \item Find the solution of this recurrence relation with
    $a_0 = 4$.
\end{enumerate}

\item \begin{enumerate}
    \item Find all solutions of the recurrence relation
    $a_n = 2a_{n-1} + 3^n$.
    \item Find the solution of the recurrence relation in part (a)
    with initial condition $a_1 = 5$.
\end{enumerate}

\item Find all solutions of the recurrence relation 
$a_n = 5a_{n-1} - 6a_{n-2} + 2^n + 3n$. 
[Hint: Look for a particular
solution of the form $qn2^n + p_1n + p_2$, where $q$, $p_1$ and
$p_2$ are constants.]

\item Suppose that each pair of a genetically engineered species
of rabbits left on an island produces two new pairs of rabbits at the age of $1$ month and six new pairs of rabbits at
the age of $2$ months and every month afterward. None of
the rabbits ever die or leave the island.
\begin{enumerate}
    \item Find a recurrence relation for the number of pairs of
    rabbits on the island n months after one newborn pair
    is left on the island.
    \item By solving the recurrence relation in (a) determine
    the number of pairs of rabbits on the island $n$ months
    after one pair is left on the island.
\end{enumerate}

\item A new employee at an exciting new software company
starts with a salary of $\$50,000$ and is promised that at the
end of each year her salary will be double her salary of
the previous year, with an extra increment of $\$10,000$ for
each year she has been with the company.
\begin{enumerate}
    \item Construct a recurrence relation for her salary for her
    $n$th year of employment.
    \item Solve this recurrence relation to find her salary for her
    $n$th year of employment.
\end{enumerate}



%\item \begin{enumerate}
%    \item Use the formula found in Example $4$ for $f_n$, the $n$th
%    Fibonacci number, to show that $f_n$ is the integer
%    closest to
%    \begin{equation}
%    \frac{1}{\sqrt{5}} \left( \frac{1 + \sqrt{5}}{2} \right)^n
%%    \sum_{i = 1}^{n} \frac{1}{\sqrt{i}} > 2(\sqrt{n + 1} - 1).
%    \end{equation}
%    \item Determine for which $n$ $f_n$ is greater than 
%    $\frac{1}{\sqrt{5}} \left( \frac{1 + \sqrt{5}}{2} \right)^n$ 
%    and for which $n$ $f_n$ is less than
%    $\frac{1}{\sqrt{5}} \left( \frac{1 + \sqrt{5}}{2} \right)^n$.
%\end{enumerate}



\end{enumerate}

\nocite{*}
\bibliographystyle{abbrv}
\bibliography{ref}  % sigproc.bib is the name of the Bibliography in this case
 
\clearpage
%APPENDICES are optional
%\balancecolumns
\appendix
%Appendix A
\section{Answer}
\subsection{Induction and Recursion}
\subsubsection{Mathematical Induction}
\begin{enumerate}
\item Let $P(n)$ be ``$1 + nh \le (1 + h)^n, h > -1$.''

\textit{Basis step:} $P(0)$ is true because \begin{equation}
    1 + 0 \cdot h = 1 \le 1 = (1 + h)^0.
\end{equation}

\textit{Inductive step:} Assume $1 + kh \le (1 + h)^k$.
Then because $(1 + h) > 0$, $(1 + h)^{k + 1} = (1 + h) (1 + h) ^k \ge (1 + h)(1 + k h) = 1 + (k + 1) h + k h^2 \ge 1 + (k + 1) h$.

\item Let $P(n)$ be \begin{equation}
\sum_{i = 1}^{n} \frac{1}{\sqrt{i}} > 2(\sqrt{n + 1} - 1).
\end{equation}

\textit{Basis step:} $P(1)$ is true because $1 > 2 (\sqrt{2} - 1)$.

\textit{Inductive step:} Assume that $P(k)$ is true.
Then \begin{equation}
\sum_{i = 1}^{k} \frac{1}{\sqrt{i}} > 2(\sqrt{k + 1} - 1).
\end{equation}
If we show that $2(\sqrt{k + 1} - 1) + \frac{1}{\sqrt{k + 1}} > 2(\sqrt{k + 2} - 1)$,
it follows that $P(k + 1)$ is true.
This inequality is equivalent to $2(\sqrt{k + 2} - \sqrt{k + 1}) < \frac{1}{\sqrt{k + 1}}$,
which is equivalent to $2(\sqrt{k + 2} - \sqrt{k + 1}) (\sqrt{k + 2} + \sqrt{k + 1}) < \frac{\sqrt{k + 1}}{\sqrt{k + 1}} + \frac{\sqrt{k + 2}}{\sqrt{k + 1}}$.
This is equivalent to $2 < 1 + \frac{\sqrt{k + 2}}{\sqrt{k + 1}}$,
which is clearly true.

\item Let $P(n)$ be ``$H_{2^n} \le 1 + n$.''

\textit{Basis step:} $P(0)$ is true because $H_{2^0} = H_1 = 1 \le 1 + 0$.

\textit{Inductive step:} Assume that $H_{2^k} \le 1 + k$.
Then $H_{2^{k + 1}} = H_{2^k} + \sum_{j = 2^k + 1}^{2^{k + 1}} \frac{1}{j} \le 1 + k + 2^k \left(\frac{1}{2^{k + 1}}\right) < 1 + k + 1 = 1 + (k + 1)$.

\item Let $P(n)$ be the proposition that $(2n - 1)^2 - 1$ is divisible by
8.

\textit{Basis step:} $P(1)$ is true because $8 \mid 0$.

\textit{Inductive step:} Now assume that $P(k)$ is true.
Because $[(2(k + 1) - 1]^2 - 1 = [(2k - 1)^2 - 1] + 8k$,
$P(k + 1)$ is true because both terms on the right-hand side are divisible by 8.
This shows that $P(n)$ is true for all positive integers $n$,
so $m^2 - 1$ is divisible by 8 whenever $m$ is an odd positive integer.

\item \textit{Basis step:} $11^{1 + 1} + 12^{2 \cdot 1 - 1} = 121 + 12 = 133$.

\textit{Inductive step:} Assume the inductive hypothesis,
that $11^{n + 1} + 12^{2n - 1}$ is divisible by 133.
Then $11^{(n + 1) + 1} + 12^{2(n+1) - 1} = 11 \cdot 11^{n + 1} + 144 \cdot 12^{2n - 1} = 11 \cdot 11^{n + 1} + (11 + 133) \cdot 12^{2n - 1} = 11(11^{n + 1} + 12^{2n - 1}) + 133 \cdot 12^{2n - 1}$.
The expression in parentheses is divisible by 133 by the inductive hypothesis,
and obviously the second term is divisible by 133,
so the entire quantity is divisible by 133, as desired.

\item We use the notation $(i, j)$ to mean the square
in row $i$ and column $j$ and use induction on $i + j$ to show that every square can be reached by the knight.

\textit{Basis step:} There are six base cases, for the cases when $i + j \le 2$.
The knight is already at $(0, 0)$ to start,
so the empty sequence of moves reaches that square.
To reach $(1, 0)$,
the knight moves $(0, 0) \rightarrow (2, 1) \rightarrow (0, 2) \rightarrow (1, 0)$.
Similarly, to reach $(0, 1)$,
the knight moves $(0, 0) \rightarrow (1, 2) \rightarrow (2, 0) \rightarrow (0, 1)$.
Note that the knight has reached $(2, 0)$ and $(0, 2)$ in the process.
For the last basis step there is $(0, 0) \rightarrow (1, 2) \rightarrow (2, 0) \rightarrow (0, 1) \rightarrow (2, 2) \rightarrow (0, 3) \rightarrow (1, 1)$.

\textit{Inductive step:} Assume the inductive hypothesis,
that the knight can reach any square $(i, j)$ for which $i + j = k$,
where $k$ is an integer greater than 1.
We must show how the knight can reach each square $(i, j)$ when $i + j = k + 1$.
Because $k + 1 \ge 3$,
at least one of $i$ and $j$ is at least 2.
If $i \ge 2$,
then by the inductive hypothesis,
there is a sequence of moves ending at $(i - 2, j + 1)$,
because $i - 2 + j + 1 = i + j - 1 = k$;
from there it is just one step to $(i, j)$;
similarly, if $j \ge 2$.

\item Let $P(n)$ be \begin{align}
\begin{aligned}
& [(p_1 \rightarrow p_2) \land (p_2 \rightarrow p_3) \land \cdots (p_{n - 1} \rightarrow p_{n})]\\
& \qquad \rightarrow [(p_1 \land p_2 \land \cdots \land p_{n - 1}) \rightarrow p_{n}].
\end{aligned}
\end{align}

\textit{Basis step:} $P(2)$ is true because $(p_1 \rightarrow p_2) \rightarrow (p_1 \rightarrow p_2)$ is a tautology.

\textit{Inductive step:} Assume $P(k)$ is true.
To show \begin{align}
\begin{aligned}
& [(p_1 \rightarrow p_2) \land (p_2 \rightarrow p_3) \land \cdots (p_{k} \rightarrow p_{k + 1})]\\
& \qquad \rightarrow [(p_1 \land p_2 \land \cdots \land p_{k}) \rightarrow p_{k + 1}]
\end{aligned}
\end{align} is a tautology,
assume that the hypothesis of this conditional statement is true.
Because both the hypothesis and $P(k)$ are true,
it follows that $(p_1 \land \cdots \land p_{k - 1}) \rightarrow p_k$ is true.
Because this is true, and because $p_k \rightarrow p_{k + 1}$ is true,
it follows by hypothetical syllogism that $(p_1 \land \cdots \land p_{k - 1}) \rightarrow p_{k + 1}$ is true.
The weaker statement $(p_1 \land \cdots \land p_{k - 1} \land p_k) \rightarrow p_{k + 1}$ follows from this.

\item We will first prove the result when $n$ is a power of 2,
that is, if $n = 2^k, k = 1, 2, \ldots$.
Let $P(k)$ be the statement that $A \ge G$,
where $A$ and $G$ are the arithmetic and geometric means,
respectively, of a set of $n = 2^k$ positive real numbers.

\textit{Basis step:} $k = 1$ and $n = 2^1 = 2$.
Note that $(\sqrt{a_1} - \sqrt{a_2})^2 \ge 0$.
Expanding this shows that $a_1 - 2 \sqrt{a_1 a_2} + a_2 \ge 0$,
that is $\frac{a_1 + a_2}{2} \ge (a_1 a_2)^\frac{1}{2}$.

\textit{Inductive step:} Assume that $P(k)$ is true,
with $n = 2^k$.
We will show that $P(k + 1)$ is true.
We have $2^{k + 1} = 2n$.
Now $\frac{\sum_{i = 1}^{2n} a_i}{2n} = \frac{\frac{\sum_{i = 1}^n a_i}{n} + \frac{\sum_{i = n + 1}^{2n} a_i}{n}}{2}$,
and similarly $\left(\prod_{i = 1}^{2n} a_i\right)^{\frac{1}{2n}} = \left[\left(\prod_{i = 1}^n a_i\right)^\frac{1}{n} \left(\prod_{i = n + 1}^{2n} a_i\right)^\frac{1}{n}\right]^\frac{1}{2}$.
To simplify this notification,
let $A(x, y, \ldots)$ and $G(x, y, \ldots)$ denote the arithmetic mean and geometric mean of $x, y, \ldots$,
respectively.
Also, if $x \le x', y \le y'$, and so on,
then $A(x, y, \ldots) \le A(x', y', \ldots)$ and $G(x, y, \ldots) \le G(x', y', \ldots)$.
Hence, \begin{align}
    \begin{aligned}
    & A(a_1, \ldots, a_{2n})\\
    = & A(A(a_1, \ldots, a_n), A(a_{n + 1}, \ldots, a_{2n}))\\
    \ge & A(G(a_1, \ldots, a_n), G(a_{n + 1}, \ldots, a_{2n}))\\
    \ge & G(G(a_1, \ldots, a_n), G(a_{n + 1}, \ldots, a_{2n}))\\
    \ge & G(a_1, \ldots, a_{2n}).
    \end{aligned}
\end{align}
This finishes the proof for powers of 2.
Now if $n$ is not a power of 2,
let $m$ be the next higher power of 2,
and let $a_{n + 1}, \ldots, a_{m}$ all equal $A(a_1, \ldots, a_n) = \bar{a}$.
Then we have $\left[(a_1 a_2 \cdots a_n) \bar{a}^{m - n}\right]^\frac{1}{m} \le A(a_1, \ldots, a_m)$,
because $m$ is a power of 2.
Because $A(a_1, \ldots, a_m) = \bar{a}$,
it follows that $(a_1 \cdots a_n)^\frac{1}{m} \bar{a}^{1 - \frac{n}{m}} \le \bar{a}^\frac{n}{m}$.
Raising both sides to the $\frac{m}{n}$th power gives $G(a_1, \ldots, a_n) \le A(a_1, \ldots, a_n)$.

\item We prove this by mathematical induction.

\textit{Basis step:} ($n = 2$) is true tautologically.
For $n = 3$, suppose that the intervals are $(a, b), (c, d)$, and $(e, f)$,
where without loss of generality we can assume that $a \le c \le e$.
Because $(a, b) \cap (e, f) = \emptyset$,
we must have $e < b$;
for a similar reason, $e < d$.
It follows that the number halfway between $e$ and the smaller of $b$ and $d$ is common to all three intervals.

\textit{Inductive step:} Assume that whenever we have $k$ intervals that have pairwise nonempty intersections then there is a point common to all the intervals,
and suppose that we are given intervals $I_1, I_2, \ldots, I_{k + 1}$ that have pairwise nonempty intersections.
For each $i$ from $1$ to $k$, let $J_i = I_i \cap I_{k + 1}$.
We claim that the collection $J_1, J_2, \ldots, J_k$ satisfies the inductive hypothesis,
that is, that $J_{i_1} \cap J_{i_2} = \emptyset$ for each choice of subscripts $i_1$ and $i_2$.
This follows from the $n = 3$ case proved above,
using the sets $I_{i_1}, I_{i_2}$, and $I_{k + 1}$.
We can now invoke the inductive hypothesis to conclude that there is a number common to all of the sets $J_i$ for $i = 1, 2, \ldots, k$,
which perforce is in the intersection of all the sets $I_i$ for $i = 1, 2, \ldots, k + 1$.

\item Let $P(n)$ be the statement that every $2^n \times 2^n \times 2^n$ checkerboard with a $1 \times 1 \times 1$ cube removed can be covered by tiles that are $2 \times 2 \times 2$ cubes each with a $1 \times 1 \times 1$ cube removed.

\textit{Basis step:} $P(1)$,
holds because one tile coincides with the solid to be tiled.

\textit{Inductive step:} Assume that $P(k)$ holds.
Now consider a $2^{k + 1} \times 2^{k + 1} \times 2^{k + 1}$ cube with a $ 1 \times 1 \times 1$ cube removed.
Split this object into eight pieces using planes parallel to its faces and running through its center.
The missing $1 \times 1 \times 1$ piece occurs in one of these eight pieces.
Now position one tile with its center at the center of the large object so that the missing $1 \times 1 \times 1$ cube lies in the octant in which the large object is missing a $1 \times 1 \times 1$ cube.
This creates eight $2^k \times 2^k \times 2^k$ cubes,
each missing a $1 \times 1 \times 1$ cube.
By the inductive hypothesis we can fill each of these eight objects with tiles.
Putting these tilings together produces the desired tiling.

\end{enumerate}

\subsubsection{Strong Induction and Well-Ordering}
\begin{enumerate}
\item We can form all amounts except \$1 and \$3. Let $P (n)$
be the statement that we can form $n$ dollars using just 2-dollar
and 5-dollar bills. We want to prove that $P (n)$ is true for all
$n \geq 5$. (It is clear that \$1 and \$3 cannot be formed and that
\$2 and \$4 can be formed.) For the basis step, note that $5 = 5$
and $6 = 2+2+2$. Assume the inductive hypothesis, that $P (j)$
is true for all $j$ with $5 \leq j \leq k$, where $k$ is an arbitrary integer
greater than or equal to $6$. We want to show that $P (k + 1)$ is
true. Because $k-1 \geq $5, we know that $P (k-1)$ is true, that is,
that we can form $k - 1$ dollars. Add another 2-dollar bill, and
we have formed $k + 1$ dollars.
    
    
\item Let $P(n)$ be the statement that there is no positive integer b such that $\sqrt{2} = \frac{n}{b}$.

\textit{Basis step:} $P(1)$ is true because $\sqrt{2} > 1 \ge \frac{1}{b}$ for all positive integers $b$.

\textit{Inductive step:} Assume that $P (j)$ is true for all
$j \le k$,
where $k$ is an arbitrary positive integer;
we prove that $P(k + 1)$ is true by contradiction.
Assume that $\sqrt{2} = \frac{k + 1}{b}$ for some positive integer $b$.
Then $2b^2 = (k + 1)^2$, so $(k + 1)^2$ is even, and hence,
$k + 1$ is even.
So write $k + 1 = 2t$ for some positive integer $t$,
whence $2b^2 = 4t^2$ and $b^2 = 2t^2$.
By the same reasoning as before, $b$ is even,
so $b = 2s$ for some positive integer $s$.
Then $\sqrt{2} = \frac{k + 1}{b} = \frac{2t}{2s} = \frac{t}{s}$.
But $t \le k$,
so this contradicts the inductive hypothesis,
and our proof of the inductive step is complete.

\item Let $P(n)$ be the statement that exactly $n - 1$ moves are required to assemble a puzzle with $n$ pieces.

\textit{Basis step:} $P(1)$ is trivially true.

\textit{Inductive step:} Assume that $P (j)$ is true for all $j \le k$,
and consider a puzzle with $k + 1$ pieces.
The final move must be the joining of two blocks,
of size $j$ and $k + 1 - j$ for some integer $j$ with $1 \le j \le k$.
By the inductive hypothesis,
it required $j - 1$ moves to construct the one block,
and $k + 1 - j - 1 = k - j$ moves to construct the other.
Therefore, $1 + (j - 1) + (k - j) = k$ moves are required in all,
so $P(k + 1)$ is true.

\item Let $P(n)$ be the statement that if $x_1, x_2, \ldots, x_n$ are $n$ distinct real numbers, then $n - 1$ multiplications are used to find the product of these numbers no matter how parentheses are inserted in the product.
We will prove that $P(n)$ is true using strong induction.

\textit{Basis step:} $P(1)$ is true because $1 - 1 = 0$
multiplications are required to find the product of $x_1$,
a product with only one factor.

\textit{Inductive step:} Suppose that $P(k)$ is true for $1 \le k \le n$.
The last multiplication used to find the product of the $n + 1$
distinct real numbers $x_1, x_2, \ldots, x_n, x_{n + 1}$ is a multiplication of the product of the first $k$ of these numbers for some $k$ and the product of the last $n + 1 - k$ of them.
By the inductive hypothesis,
$k - 1$ multiplications are used to find the product of $k$ of the numbers,
no matter how parentheses were inserted in the product of these numbers,
and $n - k$ multiplications are used to find the product of the other $n + 1 - k$ of them,
no matter how parentheses were inserted in the product of these
numbers.
Because one more multiplication is required to find the product of all $n + 1$ numbers,
the total number of multiplications used equals $(k - 1) + (n - k) + 1 = n$.
Hence, $P(n + 1)$ is true.

\end{enumerate}

\subsubsection{Recursive Definitions and Structural Induction}
\begin{enumerate}
\item $P_m(0) = 0$, $P_m(n + 1) = P_m(n) + m$.
    
    
\item \textit{Basis step:} $f_0 f_1 + f_1 f_2 = 0 \cdot 1 + 1 \cdot 1 = 1^2 = f_2^2$.

\textit{Inductive step:} Assume that $\sum_{i = 0}^{2k - 1} f_i f_{i + 1} = f_{2k}^2$.
Then $\sum_{i = 0}^{2k + 1} f_i f_{i + 1} = f_{2k}^2 + f_{2k} f_{2k + 1} + f_{2k + 1} f_{2k + 2} = f_{2k} (f_{2k} + f_{2k + 1}) + f_{2k + 1} f_{2k + 2} = f_{2k} f_{2k + 2} + f_{2k + 1} f_{2k + 2} = (f_{2k} + f_{2k + 1}) f_{2k + 2} = f_{2k + 2}^2$.


\item
\begin{enumerate}
    \item $0 \in S$, and if $x \in S$, then $x + 2 \in S$
    and $x - 2 \in S$. 
    \item $2 \in S$, and if $x \in S$, then $x + 3 \in S$.
    \item $1 \in S$, $2 \in S$, $3 \in S$, $4 \in S$, and if $x \in S$, then $x + 5 \in S$.
\end{enumerate}


\item \begin{enumerate}
    \item $P_{m, m} = P_m$ because a number exceeding $m$ cannot be used in a partition of $m$.
    \item Because there is only one way to partition 1, namely, 1 = 1, it follows that $P_{1,n} = 1$.
    Because there is only one way to partition $m$ into $1$s, $P_{m,1} = 1$. When $n > m$ it follows that $P_{m, n} = P_{m, m}$ because a number exceeding $m$ cannot be used.
    $P_{m, m} = 1 + P_{m, m - 1}$ because one extra partition,
    namely, $m = m$,
    arises when $m$ is allowed in the partition.
    $P_{m, n} = P_{m, n - 1} + P_{m - n, n}$ if $m > n$ because a partition of $m$ into integers not exceeding $n$ either does not use any $n$s and hence,
    is counted in $P_{m, n - 1}$ or else uses an $n$ and a partition of $m - n$, and hence, is counted in $P_{m - n, n}$.
    \item $P_5 = 7, P_6 = 11$.
\end{enumerate}

\item Use a double induction argument to prove the stronger statement:
$A(m, k) > A(m, l)$ when $k > l$.

\textit{Basis step:}
When $m = 0$ the statement is true because $k > l$ implies that $A(0, k) = 2k > 2l = A(0, l)$.

\textit{Inductive step:} Assume that $A(m, x) > A(m, y)$ for all nonnegative integers $x$ and $y$ with $x > y$.
We will show that this implies that $A(m + 1, k) > A(m + 1, l)$ if $k > l$.

\textit{Basis step:} When $l = 0$ and $k > 0$, $A(m + 1, l) = 0$ and either $A(m + 1, k) = 2$ or $A(m + 1, k) = A(m, A(m + 1, k - 1))$.
If $m = 0$, this is $2A(1, k - 1) = 2k$.
If $m > 0$,
this is greater than 0 by the inductive hypothesis.
In all cases, $A(m + 1, k) > 0$,
and in fact, $A(m + 1, k) \ge 2$.
If $l = 1$ and $k > 1$,
then $A(m + 1, l) = 2$ and $A(m + 1, k) = A(m, A(m + 1, k - 1))$,
with $A(m + 1, k - 1) \ge 2$.
Hence, by the inductive hypothesis,
$A(m, A(m + 1, k - 1)) \ge A(m, 2) > A(m, 1) = 2$.

\textit{Inductive step:} Assume that $A(m + 1, r) > A(m + 1, s)$ for all $r > s, s = 0, 1, \ldots, l$.
Then if $k + 1 > l + 1$ it follows that $A(m + 1, k + 1) = A(m, A(m + 1, k)) > A(m, A(m + 1, k)) =
A(m + 1, l + 1)$.

\end{enumerate}

\subsubsection{Recursive Algorithms}
\begin{enumerate}
\item 
\textbf{procedure} $gcd$($a, b$: nonnegative integers)

\{$a < b$ assumed to hold\}

\textbf{if} $a = 0$ \textbf{then return} $b$

\textbf{else if} $a = b - a$ \textbf{then return} $a$

\textbf{else if} $a < b - a$ \textbf{then return} $gcd(a, b - a)$

\textbf{else return} $gcd(b - a, a)$


\item  \textbf{procedure} square($n$: nonnegative integer),

\textbf{if} $n = 0$ \textbf{then retur}n $0$,

\textbf{else return} $square (n - 1) + 2(n - 1) + 1$.

\textbf{Prove:} Let $P (n)$ be the statement that this algorithm correctly computes $n^2$. Because $0^2 = 0$, the algorithm works correctly
(using the \textbf{if} clause) if the input is $0$. Assume that the algorithm works correctly for input $k$. Then for input $k + 1$, it
gives as output (because of the \textbf{else} clause) its output when
the input is $k$, plus $2(k + 1 - 1) + 1$. By the inductive
hypothesis, its output at $k$ is $k^2$, so its output at $k + 1$ is
$k^2 + 2(k + 1 - 1) + 1 = k^2 + 2k + 1 = (k + 1)^2$, as desired.



\item 
\textbf{procedure} $a$($n$: nonnegative integer),

\textbf{if} $n = 0$ \textbf{then return} $1$,

\textbf{else if} $n = 1$ \textbf{then return} $2$,

\textbf{else return} $a(n - 1) \cdot a(n - 2)$.


    
\item If $n = 1$ (basis step),
place the one right triomino so that its armpit corresponds to the hole in the $2 \times 2$ board.

If $n > 1$,
then divide the board into four boards, each of size $2^{n - 1} \times 2^{n - 1}$,
notice which quarter the hole occurs in,
position one right triomino at the center of the board with its armpit in the quarter where the missing square is,
and invoke the algorithm recursively four times -- once on each of the $2^{n - 1} \times 2^{n - 1}$ boards,
each of which has one square missing.

\item If $n = 1$,
then the algorithm does nothing,
which is correct because a list with one element is already sorted.

Assume that the algorithm works correctly for $n = 1$ through $n = k$.
If $n = k + 1$, then the list is split into two lists,
$L_1$ and $L_2$.
By the inductive hypothesis,
mergesort correctly sorts each of these sublists;
furthermore,
merge correctly merges two sorted lists into one because with each comparison the smallest element in $L_1 \cup L_2$ not yet put into $L$ is put there.

\end{enumerate}

\subsection{Advanced Counting Techniques}
\subsubsection{Applications of Recurrence Relations}
\begin{enumerate}
\item \begin{enumerate}
    \item $a_n = a_{n-1} + a_{n-2} + 2^{n-2}$ for $n \geq 2$
    \item $a_0 = 0$, $a_1 = 0$
    \item $94$
\end{enumerate}

\item \begin{enumerate}
    \item $a_n = a_{n-1} + a_{n-2}$ for $n \geq 2$
    \item $a_0 = 1$, $a_1 = 1$
    \item $34$
\end{enumerate}

\item \begin{enumerate}
    \item $a_n  = 2 a_{n - 1} + a_{n - 2}$ for $n \ge 2$.
    \item $a_0 = 1, a_1 = 3$.
    \item 239.
\end{enumerate}

\item \begin{enumerate}
    \item $a_n  = 2 a_{n - 1}$ for $n \ge 2$.
    \item $a_1 = 3$.
    \item 96.
\end{enumerate}

\item \begin{enumerate}
    \item $R_n = n + R_{n - 1}, R_0 = 1$.
    \item $R_n = \frac{n(n + 1)}{2} + 1$.
\end{enumerate}

\item \begin{enumerate}
    \item $S_n = s_{n - 1} + \frac{n^2 - n + 2}{2}, S_0 = 1$.
    \item $S_n = \frac{n^3 + 5n + 6}{6}$.
\end{enumerate}

\item $64$.

\item  Clearly, $S(m, 1) = 1$ for $m \geq 1$. If $m \geq n$, then a function that
is not onto from the set with $m$ elements to the set with $n$
elements can be specified by picking the size of the range,
which is an integer between $1$ and $n - 1$ inclusive, picking
the elements of the range, which can be done in $C(n, k)$ ways,
and picking an onto function onto the range, which can be
done in $S(m, k)$ ways. Hence, there are $\sum_{k=1}^{n-1} C(n, k)S(m, k)$
functions that are not onto. But there are $n^m$ functions
altogether, so $S(m, n) = n^m - \sum_{k=1}^{n-1} C(n, k)S(m, k)$.

\end{enumerate}

\subsubsection{Solving Linear Recurrence Relations}
\begin{enumerate}
\item 
$a_n = 8(-1)^n - 3(-2)^n + 4 \cdot 3^n$

\item $a_n = (n^2 + 3n + 5)(-1)^n$

\item $(a_{1,0} + a_{1,1}n+ a_{1,2}n^2 + a_{1,3}n^3)+
(a_{2,0} + a_{2,1}n+ a_{2,2}n^2)(-2)^n + (a_{3,0} + a_{3,1}n)3^n + a_{4,0}(-4)^n$

\item \begin{enumerate}
    \item $A = -1$, $B = -7$.
    \item $a_n = 11 \cdot 2^n - n - 7$.
\end{enumerate}

\item \begin{enumerate}
    \item $a_n = \alpha 2^n + 3^{n+1}$.
    \item $a_n = -2 \cdot 2^n + 3^{n+1}$.
\end{enumerate}

\item $a_n = \alpha 2^n + \beta 3^n - n \cdot 2^{n+1} + 3n/2 + 21/4$.

\item \begin{enumerate}
    \item $a_n = 3a_{n-1} + 4a_{n-2}$, $a_0 = 2, a_1 = 6$.
    \item $a_n = [4^{n+1} + (-1)^n]/5$.
\end{enumerate}

\item \begin{enumerate}
    \item $a_n = 2a_{n+1} + (n - 1)10,000$.
    \item $a_n = 70,000 \cdot 2^{n-1} - 10,000n - 10,000$.
\end{enumerate}

%\item \begin{enumerate}
%    \item Using the formula for $f_n$, we see that 
%    $\left| f_n - \frac{1}{\sqrt{5}} \left( \frac{1 + \sqrt{5}}{2} \right)^n \right| 
%    = \left|\frac{1}{\sqrt{5}} \left( \frac{1 + \sqrt{5}}{2} \right)^n \right|
%    < \frac{1}{\sqrt{5}} < \frac{1}{2}$.
%    This means that $f_n$ is the integer closest to $\frac{1}{\sqrt{5}} \left( \frac{1 + \sqrt{5}}{2} \right)^n$.
%    \item Less when $n$ is even; greater when $n$ is odd.
%\end{enumerate}

\end{enumerate}

\end{document}
