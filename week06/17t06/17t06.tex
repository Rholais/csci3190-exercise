%!TEX program = xelatex
\documentclass[10pt, compress, handout]{beamer}
\usepackage[titleprogressbar]{../../cls/beamerthemem}

\usepackage{booktabs}
\usepackage[scale=2]{ccicons}
\usepackage{minted}

\usepgfplotslibrary{dateplot}

\usemintedstyle{trac}

\setbeamertemplate{caption}[numbered]
\setbeamertemplate{theorems}[numbered]
\newtheorem{crl}{Corollary}[theorem]
\newtheorem*{solution*}{Solution}

\usepackage{algorithm}
\usepackage[noend]{algpseudocode}

\usepackage{version}
%\excludeversion{proof}
%\excludeversion{solution*}

\usepackage{mathtools}
\usepackage{multicol}
\usepackage{qtree}

\usepackage{tikz}

\makeatletter
\def\old@comma{,}
\catcode`\,=13
\def,{%
    \ifmmode%
    \old@comma\discretionary{}{}{}%
    \else%
    \old@comma%
    \fi%
}
\makeatother

\title{CSCI 3190 Tutorial of Week 06}
\subtitle{Counting}
\author{LI Haocheng}
\institute{Department of Computer Science and Engineering}

\begin{document}

\maketitle

\begin{frame}[fragile]
\frametitle{Combinatorial Proof}
\begin{columns}
	\begin{column}{.6\linewidth}
		\begin{crl}
			Let $0 \le r \le n$. Then $C(n, r) = C(n, n - r)$.
		\end{crl}
		\begin{definition}
			A \textbf{combinatorial proof} of an identity is a proof that uses counting arguments to prove that both sides of the identity count the same objects but in different ways or a proof that is based on showing that there is a bijection between the sets of objects counted by the two sides of the identity. These two types of proofs are called \textbf{double counting proofs} and \textbf{bijective proofs}, respectively.
		\end{definition}
	\end{column}
	\begin{column}{.4\linewidth}
		\begin{proof}
			Suppose that $|S| = n$. The function that maps a subset $A$ of $S$ to $\bar{A}$ is a bijection between subsets of $S$ with $r$ elements and subsets with $n - r$ elements. The identity $C(n, r) = C(n, n - r)$ follows because when there is a bijection between two finite sets, the two sets must have the same number of elements.
		\end{proof}
	\end{column}
\end{columns}
\end{frame}

\begin{frame}[fragile]
	\frametitle{Binomial Coefficients}
	\begin{example}
		Show that if $n$ and $k$ are integers with $1 \le k \le n$, then $\binom{n}{k} \le \frac{n^k}{2^{k - 1}}$.
	\end{example}
	\onslide<2>\begin{solution*}
		\begin{align}
		\begin{aligned}
		&\binom{n}{k}\\
		= & \frac{n(n - 1)(n - 2) \cdots (n - k + 1)}{k(k - 1)(k - 2) \cdots 2}\\
		\le & \frac{n \cdot n \cdot \cdots \cdot n}{2 \cdot 2 \cdot \cdots \cdot 2}\\
		= & \frac{n^k}{2^{k - 1}}.
		\end{aligned}
		\end{align}
	\end{solution*}
\end{frame}

\plain{Questions?}

\end{document}
