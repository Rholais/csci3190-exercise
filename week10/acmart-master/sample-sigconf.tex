\documentclass[sigconf]{acmart}

\usepackage{booktabs} % For formal tables


% Copyright

\setcopyright{none}
%\setcopyright{acmcopyright}
%\setcopyright{acmlicensed}
%\setcopyright{rightsretained}
%\setcopyright{usgov}
%\setcopyright{usgovmixed}
%\setcopyright{cagov}
%\setcopyright{cagovmixed}


% DOI
%\acmDOI{10.475/123_4}

% ISBN
%\acmISBN{123-4567-24-567/08/06}

%Conference
\acmConference[CSCI'3190]{Introduction to Discrete Mathematics and Algorithms}{2017}{The Chinese University of Hong Kong}
\acmYear{2017}
\copyrightyear{2017}


%\acmArticle{4}
%\acmPrice{15.00}

% These commands are optional
%\acmBooktitle{Transactions of the ACM Woodstock conference}
%\editor{Jennifer B. Sartor}
%\editor{Theo D'Hondt}
%\editor{Wolfgang De Meuter}

\newtheorem*{solution*}{Solution}

\makeatletter
\def\old@comma{,}
\catcode`\,=13
\def,{%
	\ifmmode%
	\old@comma\discretionary{}{}{}%
	\else%
	\old@comma%
	\fi%
}
\makeatother

\begin{document}
\title{Quiz 1}
%\titlenote{Produces the permission block, and copyright information}
%\subtitle{}
%\subtitlenote{The full version of the author's guide is available as \texttt{acmart.pdf} document}

% The default list of authors is too long for headers.
%\renewcommand{\shortauthors}{B. Trovato et al.}

%
% The code below should be generated by the tool at
% http://dl.acm.org/ccs.cfm
% Please copy and paste the code instead of the example below. 
%

%\keywords{ACM proceedings, \LaTeX, text tagging}


\maketitle

%\input{samplebody-conf}
\section{Tautologies}
\begin{example}
	Determine if the following is a tautology. Show your proof. \begin{equation}
	(p \to q) \to ((p \lor q) \to q).
	\end{equation}
\end{example}
\begin{proof}
	\begin{align}
		\begin{aligned}
			& ((p \to q) \to ((p \lor q) \to q))\\
			\equiv & \neg (\neg p \lor q) \lor (\neg(p \lor q) \lor q)\\
			\equiv & ((p \land \neg q) \lor (\neg p \land \neg q) \lor q)\\
			\equiv & (((p \lor \neg q) \land (\neg p \lor \neg q) \land \neg q) \lor q )\\
			\equiv & ((\neg q \land \neg q) \lor q)\\
			\equiv & T.
		\end{aligned}
	\end{align}
\end{proof}

\begin{example}
	Determine if the followings are logically equivalent. Show your proof. \begin{equation}
	(p \to q) \land (p \to r) \text{~and~} q \to r.
	\end{equation}
\end{example}
\begin{proof}
	Let $p = 0, q = 1, r = 0$, then $lhs = 1, rhs = 0$.
\end{proof}

\section{Distributive Property}
\begin{example}
	Prove that $(A - B) - (A - C) \equiv A \cap (C - B)$.
\end{example}
\begin{proof}
	\begin{align}
		\begin{aligned}
			& (A - B) - (A - C)\\
			\equiv & (A \cap \overline{B}) \cap \overline{A \cap \overline{C}}\\
			\equiv & A \cap \overline{B} \cap (\overline{A} \cup C)\\
			\equiv & (A \cap \overline{A} \cap \overline{B}) \cup (A \cap \overline{B} \cap C)\\
			\equiv & A \cap (C - B).
		\end{aligned}
	\end{align}
\end{proof}

\section{Relation}
\begin{example}
	Let $A = \{1, 2, 3, 4, 5\}$, $R$ is a relation on $A \times A$ such that $((a, b), (c, d)) \in R$ if and only if $a - d = c - b$. Show that $R$ is an equivalence relation.
\end{example}
\begin{proof}
	Evaluate three properties:
	\begin{itemize}
		\item $a - b = a - b$.
		\item $a - d = c - b \Leftrightarrow c - b = a - d$.
		\item $a - d = c - b, c - f = e - d \Rightarrow a - f = e - b$.
	\end{itemize}
\end{proof}
\begin{example}
	How many equivalence classes are there in $R$?
\end{example}
\begin{solution*}
	9: $[(1, 1)], [(1, 2)], [(1, 3)], [(1, 4)], [(1, 5)], [(2, 5)], [(3, 5)], [(4, 5)], [(5, 5)]$.
\end{solution*}

\section{Generating Function}
\begin{example}
	Given a closed form expression for the generating function of the following sequence:\begin{equation}
	1, -2, 3, -4, 5, -6, \cdots
	\end{equation}
\end{example}
\begin{solution*}
	\begin{align}
	a_r & = -1^{r + 1} & \leftrightarrow & -\frac{1}{1 + x},\\
	b_r & = -1^r (r + 1) & \leftrightarrow & \frac{1}{(1 + x)^2}.
	\end{align}
\end{solution*}

\section{Contradiction}
\begin{example}
	Suppose that every student in a discrete mathematics class of 17 students is either a freshman, a sophomore or a junior. Show that there are at least 4 freshmen, or 10 sophomores or 5 juniors in the class. Explain your answer.
\end{example}
\begin{proof}
	If not, there are at most 3 freshmen and 9 sophomores and 4 juniors in the class who are at most 16 in total. However, there are 17 student in the class, which constructs a contradiction.
\end{proof}

\section{Permutations}
\begin{example}
	Find the number of permutations of \texttt{a, b, c, \dots, y, z} in which non of the patterns \texttt{bank}, \texttt{pickle}, \texttt{kite} occurs. Explain your answer.
\end{example}
\begin{solution*}
	By the inclusion–exclusion principle:
	\begin{enumerate}
		\item number of permutations in which \texttt{bank} occurs: $23!$,
		\item number of permutations in which \texttt{pickle} occurs: $21!$,
		\item number of permutations in which \texttt{kite} occurs: $23!$,
		\item number of permutations in which \texttt{bank} and \texttt{pickle} occurs: $0$,
		\item number of permutations in which \texttt{bank} and \texttt{kite} occurs: $20!$,
		\item number of permutations in which \texttt{pickle} and \texttt{kite} occurs: $0$,
		\item number of permutations in which all occurs: $0$.
		\item number of permutations in which none occurs:\begin{equation}
		26! - 23! \times 2 - 21! + 20!.
		\end{equation}
	\end{enumerate}
\end{solution*}

\bibliographystyle{../../cls/ACM-Reference-Format}
%\bibliography{sample-bibliography} 

\end{document}
