%!TEX program = xelatex
\documentclass[10pt, compress, handout]{beamer}
\usepackage[titleprogressbar]{../../cls/beamerthemem}

\usepackage{booktabs}
\usepackage[scale=2]{ccicons}
\usepackage{minted}

\usepgfplotslibrary{dateplot}

\usemintedstyle{trac}

\setbeamertemplate{caption}[numbered]
\setbeamertemplate{theorems}[numbered]
\newtheorem{crl}{Corollary}[theorem]
\newtheorem*{solution*}{Solution}

\usepackage{algorithm}
\usepackage[noend]{algpseudocode}

\usepackage{version}
\excludeversion{proof}
\excludeversion{solution*}

\usepackage{mathtools}
\usepackage{multicol}
\usepackage{qtree}

\makeatletter
\def\old@comma{,}
\catcode`\,=13
\def,{%
	\ifmmode%
	\old@comma\discretionary{}{}{}%
	\else%
	\old@comma%
	\fi%
}
\makeatother

\title{CSCI 3190 Tutorial of Week 10}
\subtitle{Traversal}
\author{LI Haocheng}
\institute{Department of Computer Science and Engineering}

\begin{document}

\maketitle

\begin{frame}[fragile]
	\frametitle{Complexity}
	\onslide<1->\begin{example}
		Show the following:\begin{enumerate}[(a)]
			\item $3n^2 + 5n + 10 = O(n^2)$
		\end{enumerate}
	\end{example}
	\onslide<2>\begin{proof}
		\begin{enumerate}[(a)]
			\item \begin{description}
				\item[Method 1] $\exists c = 4, \exists N = 6$, such that $\forall n > N, 3n^2 + 5n + 10 < 4n^2$.
				\item[Method 2] Since $3 + \frac{5}{n} + \frac{10}{n^2} = O(1)$ and $n^2 = O(n^2)$, $3n^2 + 5n + 10 = O(n^2)$.
			\end{description}
		\end{enumerate}
	\end{proof}
\end{frame}

\begin{frame}[fragile]
	\frametitle{Complexity}
	\onslide<1->\begin{example}
		Show the following:\begin{enumerate}[(b)]
			\item $100 \log_2 n = O(n)$
		\end{enumerate}
	\end{example}
	\onslide<2>\begin{proof}
		\begin{enumerate}[(b)]
			\item \begin{description}
				\item[Method 1] Since $100 = O(1)$ and $\log_2 n = O(n)$, $100\log_2 n = O(n)$.
				\item[Method 2] \begin{align}
				& \lim\limits_{n \rightarrow \infty} \frac{100\log_2 n}{n}\\
				= & \frac{100}{\ln 2} \lim\limits_{n \rightarrow \infty} \frac{1}{n}\\
				= & 0
				\end{align}
			\end{description}
		\end{enumerate}
	\end{proof}
\end{frame}

\begin{frame}[fragile]
	\frametitle{Complexity}
	\onslide<1->\begin{example}
		Show the following:\begin{enumerate}[(a)]
			\setcounter{enumi}{2}
			\item $n! = O(n^n)$
			\item $2^n = O(n!)$
		\end{enumerate}
	\end{example}
	\onslide<2>\begin{proof}
		\begin{enumerate}[(a)]
			\setcounter{enumi}{2}
			\item $n! = \Pi_{i = 1}^n i \le \Pi_{i = 1}^n n = n^n$.
			\item $\exists c = 2, \exists N = 1$, such that $\forall n > N, 2^n = \Pi_{i = 1}^n 2 = 2 \Pi_{i = 2}^n 2 \le 2 \Pi_{i = 2}^n i = 2(n!)$.
		\end{enumerate}
	\end{proof}
\end{frame}

\plain{Questions?}

\end{document}
