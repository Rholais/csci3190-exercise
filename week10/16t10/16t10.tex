%!TEX program = xelatex
\documentclass[10pt, compress]{beamer}
\usetheme[titleprogressbar]{m}

\usepackage{booktabs}
\usepackage[scale=2]{ccicons}
\usepackage{minted}

\usepgfplotslibrary{dateplot}

\usemintedstyle{trac}

\setbeamertemplate{caption}[numbered]
\setbeamertemplate{theorems}[numbered]
\newtheorem{crl}{Corollary}[theorem]

\usepackage{algorithm}
\usepackage{algpseudocode}

\usepackage{multicol}
\usepackage{qtree}

\makeatletter
\def\old@comma{,}
\catcode`\,=13
\def,{%
	\ifmmode%
	\old@comma\discretionary{}{}{}%
	\else%
	\old@comma%
	\fi%
}
\makeatother

\title{CSCI 3190 Tutorial of Week 10}
\subtitle{Traversal}
\author{LI Haocheng}
\institute{Department of Computer Science and Engineering}

\begin{document}

\maketitle

\begin{frame}[fragile]
	\frametitle{Mathematical Induction}
	\onslide<1->\begin{example}
		For positive integer $n$, prove each of the following by mathematical induction:\begin{enumerate}[(a)]
			\item $3 \mid 2^{2n + 1} + 1$
			\item $9 \mid n^3 + (n + 1)^3 + (n + 2)^3$
		\end{enumerate}
	\end{example}
	\onslide<2>\textbf{Proof.}\begin{enumerate}[(a)]
			\item \begin{enumerate}[(i)]
				\item Let $n = 0$, $3 \mid 2^1 + 1 = 3$.
				\item Suppose $3 \mid 2^{2n - 1} + 1$, $2^{2n + 1} = 3 * 2^{2n - 1} + 2^{2n - 1} + 1 \equiv 0 \pmod{3}$.
			\end{enumerate}
			\item \begin{enumerate}[(i)]
				\item Let $n = 0$, $9 \mid 1^3 + 2^3$.
				\item Suppose $9 \mid (n - 1)^3 + n^3 + (n + 1)^3$, $n^3 + (n + 1)^3 + (n + 2)^3 = (n - 1)^3 + n^3 + (n + 1)^3 + 3((n - 1)^2 + (n - 1)(n + 2) + (n + 2)^2) = (n - 1)^3 + n^3 + (n + 1)^3 + 9(n^2 + n + 1) \equiv 0 \pmod{9}$.
			\end{enumerate}
		\end{enumerate}
\end{frame}

\begin{frame}[fragile]
	\frametitle{Preorder}
	\begin{columns}
		\begin{column}{.6\linewidth}
			\onslide<1->\begin{example}
				Determine the order in which a preorder traversal visits the vertices of the given ordered rooted tree in Figure \ref{f-11-3-e7}.
			\end{example}
			\onslide<2>\textbf{Solution} $a, b, d, e, f, g, c$
		\end{column}
		\onslide<1->\begin{column}{.4\linewidth}
			\begin{figure}
				\centering
				$\Tree [.a [.b d [.e f g ]] c ]$
				\caption{A Rooted Tree $T$}
				\label{f-11-3-e7}
			\end{figure}
		\end{column}
	\end{columns}
\end{frame}

\begin{frame}[fragile]
	\frametitle{Inorder}
	\begin{columns}
		\begin{column}{.6\linewidth}
			\onslide<1->\begin{example}
				Determine the order in which a inorder traversal visits the vertices of the given ordered rooted tree in Figure \ref{f-11-3-e7-1}.
			\end{example}
			\onslide<2>\textbf{Solution} $d, b, f, e, g, a, c$
		\end{column}
		\onslide<1->\begin{column}{.4\linewidth}
			\begin{figure}
				\centering
				$\Tree [.a [.b d [.e f g ]] c ]$
				\caption{A Rooted Tree $T$}
				\label{f-11-3-e7-1}
			\end{figure}
		\end{column}
	\end{columns}
\end{frame}

\begin{frame}[fragile]
	\frametitle{Postorder}
	\begin{columns}
		\begin{column}{.6\linewidth}
			\onslide<1->\begin{example}
				Determine the order in which a postorder traversal visits the vertices of the given ordered rooted tree in Figure \ref{f-11-3-e7-2}.
			\end{example}
			\onslide<2>\textbf{Solution} $d, f, g, e, b, c, a$
		\end{column}
		\onslide<1->\begin{column}{.4\linewidth}
			\begin{figure}
				\centering
				$\Tree [.a [.b d [.e f g ]] c ]$
				\caption{A Rooted Tree $T$}
				\label{f-11-3-e7-2}
			\end{figure}
		\end{column}
	\end{columns}
\end{frame}

\begin{frame}[fragile]
	\frametitle{Expression}
	\begin{columns}
		\begin{column}{.6\linewidth}
			\onslide<1->\begin{example}
				\begin{enumerate}
					\item Represent the expression $((x + 2) \uparrow 3) \cdot (y - (3 + x)) - 5$ using a binary tree.
					\item Write this expression in prefix notation.
					\item Write this expression in postfix notation.
					\item Write this expression in infix notation.
				\end{enumerate}
			\end{example}
			\onslide<2>\textbf{Solution} \begin{enumerate}
				\item $-\ \cdot\ \uparrow\ +\ x\ 2\ 3\ -\ y\ +\ 3\ x\ 5$
				\item $x\ 2\ +\ 3\ \uparrow\ y\ 3\ x\ +\ -\ \cdot\ 5\ -$
				\item $x\ +\ 2\ \uparrow\ 3\ \cdot\ y\ -\ 3\ +\ x\ -\ 5$
			\end{enumerate}
		\end{column}
		\onslide<2>\begin{column}{.4\linewidth}
			\begin{figure}
				\centering
				$\Tree [.- [.\(\cdot\) [.\(\uparrow\) [.+ x 2 ] 3 ] [.- y [.+ 3 x ] ] ] 5 ]$
				\caption{A Rooted Tree $T$}
				\label{f-11-3-e16}
			\end{figure}
		\end{column}
	\end{columns}
\end{frame}

\begin{frame}[fragile]
	\frametitle{Construction}
	\begin{columns}
		\begin{column}{.6\linewidth}
			\onslide<1->\begin{example}
				Construct the ordered rooted tree whose preorder traversal is $a, b, f, c, g, h, i, d, e, j, k, l$, where $a$ has four children, $c$ has three children, $j$ has two children, $b$ and $e$ have one child each, and all other vertices are leaves.
			\end{example}
		\end{column}
		\onslide<2>\begin{column}{.4\linewidth}
			\begin{figure}
				\centering
				$\Tree [.a [.b f ] [.c g h i ] d [.e [.j k l ] ] ]$
				\caption{A Rooted Tree $T$}
				\label{f-11-3-e25}
			\end{figure}
		\end{column}
	\end{columns}
\end{frame}

\begin{frame}[fragile]
	\frametitle{Complexity}
	\onslide<1->\begin{example}
		Show the following:\begin{enumerate}[(a)]
			\item $3n^2 + 5n + 10 = O(n^2)$
		\end{enumerate}
	\end{example}
	\onslide<2>\begin{proof}
		\begin{enumerate}[(a)]
			\item \begin{description}
				\item[Method 1] $\exists c = 4, \exists N = 6$, such that $\forall n > N, 3n^2 + 5n + 10 < 4n^2$.
				\item[Method 2] Since $3 + \frac{5}{n} + \frac{10}{n^2} = O(1)$ and $n^2 = O(n^2)$, $3n^2 + 5n + 10 = O(n^2)$.
			\end{description}
		\end{enumerate}
	\end{proof}
\end{frame}

\begin{frame}[fragile]
	\frametitle{Complexity}
	\onslide<1->\begin{example}
		Show the following:\begin{enumerate}[(b)]
			\item $100 \log_2 n = O(n)$
		\end{enumerate}
	\end{example}
	\onslide<2>\begin{proof}
		\begin{enumerate}[(b)]
			\item \begin{description}
				\item[Method 1] Since $100 = O(1)$ and $\log_2 n = O(n)$, $100\log_2 n = O(n)$.
				\item[Method 2] \begin{align}
				& \lim\limits_{n \rightarrow \infty} \frac{100\log_2 n}{n}\\
				= & \frac{100}{\ln 2} \lim\limits_{n \rightarrow \infty} \frac{1}{n}\\
				= & 0
				\end{align}
			\end{description}
		\end{enumerate}
	\end{proof}
\end{frame}

\begin{frame}[fragile]
	\frametitle{Complexity}
	\onslide<1->\begin{example}
		Show the following:\begin{enumerate}[(a)]
			\setcounter{enumi}{2}
			\item $n! = O(n^n)$
			\item $2^n = O(n!)$
		\end{enumerate}
	\end{example}
	\onslide<2>\begin{proof}
		\begin{enumerate}[(a)]
			\setcounter{enumi}{2}
			\item $n! = \Pi_{i = 1}^n i \le \Pi_{i = 1}^n n = n^n$.
			\item $\exists c = 2, \exists N = 1$, such that $\forall n > N, 2^n = \Pi_{i = 1}^n 2 = 2 \Pi_{i = 2}^n 2 \le 2 \Pi_{i = 2}^n i = 2(n!)$.
		\end{enumerate}
	\end{proof}
\end{frame}

\begin{frame}[allowframebreaks]
	\frametitle{Double Tower of Hanoi}
	\begin{example}
		A Double Tower of Hanoi contains $2n$ disks of $n$ different sizes, two of each size. As usual, we can move
		only one disk at a time, without putting a larger one over a smaller one. Assume that disks of equal size are
		indistinguishable from each other:\begin{enumerate}[(a)]
			\item Write a pseudo-code to solve the above problem recursively.
			\item Set up a recurrence equation to count the number of steps to move $2n$ disks. 
			\item Solve the recurrence equation and give the complexity of your algorithm.
		\end{enumerate}
	\end{example}
	\begin{columns}
		\begin{column}{.5\linewidth}
			\textbf{Solution}\begin{enumerate}[(a)]
				\item See Algorithm \ref{a-3}.		
				\item $H_{2n} = 2H_{2n - 2} + 2$.
				\item \begin{align}
				H_{2n} & = 2(H_{2n - 4} + 2) + 2\\
				& = 2H_{2n - 4} + 4 + 2\\
				& \cdots\\
				& = 2^{n + 1} - 2.\\
				O(H_{2n}) & = 2^n.
				\end{align}
			\end{enumerate}
		\end{column}
		\begin{column}{.5\linewidth}
			\begin{algorithm}[H]
				\caption{Double Tower of Hanoi}
				\label{a-3}
				\begin{algorithmic}
					\Procedure{Hanoi}{$h$, $b$, $e$, $m$}
					\If{$h$ = 0}
					\State \Return
					\ElsIf{$h$ = 2}
					\State \Call{Move}{$b$, $e$}
					\State \Call{Move}{$b$, $e$}
					\Else
					\State \Call{Hanoi}{$h - 2$, $b$, $m$, $e$}
					\State \Call{Hanoi}{2, $b$, $e$}
					\State \Call{Hanoi}{$h - 2$, $m$, $e$, $b$}
					\EndIf
					\EndProcedure
				\end{algorithmic}
			\end{algorithm}
		\end{column}
	\end{columns}
\end{frame}

\plain{Questions?}

\end{document}
