% This is "sig-alternate.tex" V2.1 April 2013
% This file should be compiled with V2.5 of "sig-alternate.cls" May 2012
%
% This example file demonstrates the use of the 'sig-alternate.cls'
% V2.5 LaTeX2e document class file. It is for those submitting
% articles to ACM Conference Proceedings WHO DO NOT WISH TO
% STRICTLY ADHERE TO THE SIGS (PUBS-BOARD-ENDORSED) STYLE.
% The 'sig-alternate.cls' file will produce a similar-looking,
% albeit, 'tighter' paper resulting in, invariably, fewer pages.
%
% ----------------------------------------------------------------------------------------------------------------
% This .tex file (and associated .cls V2.5) produces:
%       1) The Permission Statement
%       2) The Conference (location) Info information
%       3) The Copyright Line with ACM data
%       4) NO page numbers
%
% as against the acm_proc_article-sp.cls file which
% DOES NOT produce 1) thru' 3) above.
%
% Using 'sig-alternate.cls' you have control, however, from within
% the source .tex file, over both the CopyrightYear
% (defaulted to 200X) and the ACM Copyright Data
% (defaulted to X-XXXXX-XX-X/XX/XX).
% e.g.
% \CopyrightYear{2007} will cause 2007 to appear in the copyright line.
% \crdata{0-12345-67-8/90/12} will cause 0-12345-67-8/90/12 to appear in the copyright line.
%
% ---------------------------------------------------------------------------------------------------------------
% This .tex source is an example which *does* use
% the .bib file (from which the .bbl file % is produced).
% REMEMBER HOWEVER: After having produced the .bbl file,
% and prior to final submission, you *NEED* to 'insert'
% your .bbl file into your source .tex file so as to provide
% ONE 'self-contained' source file.
%
% ================= IF YOU HAVE QUESTIONS =======================
% Questions regarding the SIGS styles, SIGS policies and
% procedures, Conferences etc. should be sent to
% Adrienne Griscti (griscti@acm.org)
%
% Technical questions _only_ to
% Gerald Murray (murray@hq.acm.org)
% ===============================================================
%
% For tracking purposes - this is V2.0 - May 2012

\documentclass{../../cls/sig-alternate-05-2015}

\usepackage{algorithm}
\usepackage{algpseudocode}

\usepackage{booktabs}
\usepackage{enumitem}
\usepackage{mathtools}
\usepackage{textcomp}


\begin{document}

% Copyright
%\setcopyright{acmcopyright}
%\setcopyright{acmlicensed}
%\setcopyright{rightsretained}
%\setcopyright{usgov}
%\setcopyright{usgovmixed}
%\setcopyright{cagov}
%\setcopyright{cagovmixed}


% DOI
%\doi{10.475/123_4}

% ISBN
%\isbn{123-4567-24-567/08/06}

%Conference
%\conferenceinfo{PLDI '13}{June 16--19, 2013, Seattle, WA, USA}

%\acmPrice{\$15.00}

%
% --- Author Metadata here ---
%\conferenceinfo{WOODSTOCK}{'97 El Paso, Texas USA}
%\CopyrightYear{2007} % Allows default copyright year (20XX) to be over-ridden - IF NEED BE.
%\crdata{0-12345-67-8/90/01}  % Allows default copyright data (0-89791-88-6/97/05) to be over-ridden - IF NEED BE.
% --- End of Author Metadata ---

%\\TODO:1.tautology prove section 2. set section operation proof 3. function section basic proving onto bijection...
\title{CSCI 3190 \\ Introduction to Discrete Mathematics and Algorithms}
\subtitle{Sample Solution of Assignment 2}

\maketitle
\begin{abstract}

\end{abstract}

\keywords{}

\section{Mathematical Induction}
\textbf{Proof}\begin{enumerate}[label=(\alph*)]
	\item \begin{description}
		\item[Base Step] Let $n = 0$, $3 \mid 2^1 + 1 = 3$.
		\item[Induction Step] Suppose $3 \mid 2^{2n - 1} + 1$, $2^{2n + 1} = 3 * 2^{2n - 1} + 2^{2n - 1} + 1 \equiv 0 \pmod{3}$.
	\end{description}
	\item \begin{description}
		\item[Base Step] Let $n = 0$, $9 \mid 1^3 + 2^3$.
		\item[Induction Step] Suppose $9 \mid (n - 1)^3 + n^3 + (n + 1)^3$, $n^3 + (n + 1)^3 + (n + 2)^3 = (n - 1)^3 + n^3 + (n + 1)^3 + 3((n - 1)^2 + (n - 1)(n + 2) + (n + 2)^2) = (n - 1)^3 + n^3 + (n + 1)^3 + 9(n^2 + n + 1) \equiv 0 \pmod{9}$.
	\end{description}
\end{enumerate}

\section{Complexity}
\textbf{Proof}\begin{enumerate}[label=(\alph*)]
	\item \begin{description}
		\item[Method 1] $\exists c = 4, \exists N = 6$, such that $\forall n > N, 3n^2 + 5n + 10 < 4n^2$.
		\item[Method 2] Since $3 + \frac{5}{n} + \frac{10}{n^2} = O(1)$ and $n^2 = O(n^2)$, $3n^2 + 5n + 10 = O(n^2)$.
	\end{description}
	\item \begin{description}
		\item[Method 1] Since $100 = O(1)$ and $\log_2 n = O(n)$, $100\log_2 n = O(n)$.
		\item[Method 2] \begin{align}
			& \lim\limits_{n \rightarrow \infty} \frac{100\log_2 n}{n}\\
			= & \frac{100}{\ln 2} \lim\limits_{n \rightarrow \infty} \frac{1}{n}\\
			= & 0
		\end{align}
	\end{description}
	\item $n! = \Pi_{i = 1}^n i \le \Pi_{i = 1}^n n = n^n$.
	\item $\exists c = 2, \exists N = 1$, such that $\forall n > N, 2^n = \Pi_{i = 1}^n 2 = 2 \Pi_{i = 2}^n 2 \le 2 \Pi_{i = 2}^n i = 2(n!)$.
\end{enumerate}

\section{Double Tower of Hanoi}
\textbf{Solution}\begin{enumerate}[label=(\alph*)]
	\item See Algorithm \ref{a-3}.
	\begin{algorithm}
		\caption{Double Tower of Hanoi}
		\label{a-3}
		\begin{algorithmic}
			\Procedure{Hanoi}{$height$, $begin$, $end$, $mid$}
			\If{$height$ = 0}
			\State \Return
			\ElsIf{$height$ = 2}
			\State \Call{Move}{$begin$, $end$}
			\State \Call{Move}{$begin$, $end$}
			\Else
			\State \Call{Hanoi}{$height - 2$, $begin$, $mid$, $end$}
			\State \Call{Hanoi}{2, $begin$, $end$}
			\State \Call{Hanoi}{$height - 2$, $mid$, $end$, $begin$}
			\EndIf
			\EndProcedure
		\end{algorithmic}
	\end{algorithm}

	\item $H_{2n} = 2H_{2n - 2} + 2$.
	\item \begin{align}
		H_{2n} & = 2(2H_{2n - 4} + 2) + 2\\
		& = 4H_{2n - 4} + 4 + 2\\
		& \cdots\\
		& = 2^{n + 1} - 2.\\
		O(H_{2n}) & = 2^n.
	\end{align}
\end{enumerate}

\section{Fibonacci}
\textbf{Solution}\begin{enumerate}[label=(\alph*)]
	\item See Algorithm \ref{a-4-1}.
	\begin{algorithm}
		\caption{A Recursive Algorithm for Fibonacci}
		\label{a-4-1}
		\begin{algorithmic}
			\Procedure{Fibonacci}{$n$}
			\If{$n = 0$}
			\State \Return $0$
			\ElsIf{$n = 1$}
			\State \Return $1$
			\Else
			\State \Return \Call{Fibonacci}{$n - 1$} + \Call{Fibonacci}{$n - 2$}
			\EndIf
			\EndProcedure
		\end{algorithmic}
	\end{algorithm}
	The number of additions follows the recursive expression:\begin{equation}
	a_n = \begin{cases}
	0 & \text{If } n = 0,\\
	0 & \text{If } n = 1,\\
	a_{n - 1} + a_{n - 2} + 1 & \text{Otherwise.}
	\end{cases}
	\end{equation} So that $O(a_n) = 2^n$.
	
	\item See Algorithm \ref{a-4-2}.
	\begin{algorithm}
		\caption{A Iterative Algorithm for Fibonacci}
		\label{a-4-2}
		\begin{algorithmic}
			\Procedure{IterativeFibonacci}{$n$}
			\State $x \coloneqq 0$
			\State $y \coloneqq 1$
			\For{$i \coloneqq 0, 1, 2, \cdots, n - 1$} 
			\State $z \coloneqq x + y$
			\State $x \coloneqq y$
			\State $y \coloneqq z$
			\EndFor
			\State \Return $x$
			\EndProcedure
		\end{algorithmic}
	\end{algorithm}

	The number of additions follows the expression $b_n = n - 1$ so that $O(b_n) = n$.
\end{enumerate}

\section{Longest Common Subsequence}
\textbf{Solution} See Algorithm \ref{a-5}.
\begin{algorithm}
	\caption{Print Out ALL the LCSes}
	\label{a-5}
	\begin{algorithmic}
		\Procedure{All}{$T$, $A$, $B$, $j$, $k$}
		\If{$j \cdot k = 0$}
		\State \Return $\{``"\}$
		\ElsIf{$A[j] = B[k]$}
		\State \Return $\{C + A[j] \colon Z \in \Call{All}{T, A, B, j - 1, k - 1}\}$
		\Else
		\State $R \coloneqq \{\}$
		\If{$T[j, k - 1] \ge T[j - 1, k]$}
		\State $R \coloneqq R \cup \Call{All}{T, A, B, j, k - 1}$
		\EndIf
		\If{$T[j - 1, k] \ge T[j, k - 1]$}
		\State $R \coloneqq R \cup \Call{All}{T, A, B, j - 1, k}$
		\EndIf
		\State \Return $R$
		\EndIf
		\EndProcedure
	\end{algorithmic}
\end{algorithm}

\section{LCS between 3 Sequences}
\textbf{Solution} See Algorithm \ref{a-6}.
\begin{algorithm}
	\caption{Compute the LCS among 3 Sequences}
	\label{a-6}
	\begin{algorithmic}
		\Procedure{Three}{$A$, $B$, $C$}
		\State $T \coloneqq \Call{Zeros}{0 \cdots m, 0 \cdots n, 0 \cdots p}$
		\For{$j \coloneqq 1 \cdots m$}
		\State $T[j, 0, 0] \coloneqq 0$
		\EndFor
		\For{$k \coloneqq 1 \cdots n$}
		\State $T[0, k, 0] \coloneqq 0$
		\EndFor
		\For{$l \coloneqq 1 \cdots p$}
		\State $T[0, 0, l] \coloneqq 0$
		\EndFor
		\For{$j \coloneqq 1 \cdots m$}
		\For{$k \coloneqq 1 \cdots n$}
		\For{$l \coloneqq 1 \cdots l$}
		\If{$A[j] = B[k] = C[l]$}
		\State $T[j, k, l] \coloneqq T[j - 1, k - 1, l - 1] + 1$
		\Else
		\State $T[j, k, l] = \Call{Max}{T[j - 1, k, l], T[j, k - 1, l], T[j, k - 1, l]}$
		\EndIf
		\EndFor
		\EndFor
		\EndFor
		\State \Return $T[m, n, l]$
		\EndProcedure
	\end{algorithmic}
\end{algorithm}

\section{Unique Symbols}
Consider the longest common subsequence problem between sequence $A[1\cdots m]$ and sequence $B[1\cdots n]$ where
the symbols in $A$ are all unique. Give a $O(n \log m)$ algorithm that can find the longest common subsequence
between $A[1\cdots m]$ and $B[1\cdots n]$. 

\textbf{Solution} See Algorithm \ref{a-7}.
\begin{algorithm}
	\caption{Compute the LCS with Unique Symbols}
	\label{a-7}
	\begin{algorithmic}
		\Procedure{Unique}{$A$, $B$}
		\State $C \coloneqq \Call{Zeros}{n}$
		\State $D \coloneqq \Call{Zeros}{{}}$
		\For{$j \coloneqq 1 \cdots m$}
		\State $D[A[j]] \coloneqq j$
		\EndFor
		\For{$k \coloneqq 1 \cdots n$}
		\State $C[k] = D[B[k]]$
		\EndFor
		\State $P \coloneqq \Call{Zeros}{n}$
		\State $M \coloneqq \Call{Zeros}{n + 1}$
		\State $l \coloneqq 0$
		\For{$k \coloneqq 1 \cdots n$}
		\State $lo \coloneqq 1$
		\State $hi \coloneqq l$
		\While{$lo \le hi$}
		\State $mid \coloneqq \Call{Ceil}{\frac{lo + hi}{2}}$
		\If{$C[M[mid + 1]] < C[k]$}
		\State $lo \coloneqq mid + 1$
		\Else
		\State $hi \coloneqq mid - 1$
		\EndIf
		\EndWhile
		\State $P[k] = M[lo]$
		\State $M[lo + 1] = k$
		\If{$lo > l$}
		\State $l = lo$
		\EndIf
		\EndFor
		\State $S \coloneqq \Call{Zeros}{l}$
		\State $k = M[l + 1]$
		\For{$i \coloneqq l \cdots 1$}
		\State $S[i]=A[C[k]]$
		\State $k = P[k]$
		\EndFor
		\State \Return $S$
		\EndProcedure
	\end{algorithmic}
\end{algorithm}

\section{Tree}
\textbf{Proof}\begin{enumerate}[label=(\alph*)]
	\item A $n$-vertex tree has $n-1$ edges so that the total degree is $2n - 2$.
	\item \begin{description}
		\item[Base Step] For $n = 2$, this is trivial.
		\item[Induction Step] Suppose $\forall$ positive $d_i$ such that $\Sigma_{i = 1}^{n - 1} d_i = 2n - 4$, $\exists$ a tree $T^\prime$ whose vertices have degrees $d_1, d_2, \cdots, d_{n - 1}$. Then $\forall$ positive $d_i$ such that $\Sigma_{i = 1}^{n} d_i = 2n - 2$, since $d_i$ cannot be all greater than 1 or less than 2, without loss of generality, let $d_{n - 1} > 1, d_n = 1$. Hence we can remove $d_n$ and subtract $d_{n - 1}$ by 1 so that $\Sigma_{i = 1}^{n - 1} d_i = 2n - 4$ and we can find a tree $T^\prime$. After that we can add vertex $d_n$ as a leaf of $d_{n - 1}$ to build the final tree $T$.
	\end{description}
\end{enumerate}

\section{Traversal}
\textbf{Solution}\begin{enumerate}[label=(\alph*)]
	\item \begin{description}
		\item[DFS] B, A, C, F, D, E
		\item[BFS] B, A, C, D, E, F
	\end{description}
	\item \begin{description}
		\item[DFS] B, A, D, E, C
		\item[BFS] B, A, C, D, E
	\end{description}
\end{enumerate}

\end{document}
