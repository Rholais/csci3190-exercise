% This is "sig-alternate.tex" V2.1 April 2013
% This file should be compiled with V2.5 of "sig-alternate.cls" May 2012
%
% This example file demonstrates the use of the 'sig-alternate.cls'
% V2.5 LaTeX2e document class file. It is for those submitting
% articles to ACM Conference Proceedings WHO DO NOT WISH TO
% STRICTLY ADHERE TO THE SIGS (PUBS-BOARD-ENDORSED) STYLE.
% The 'sig-alternate.cls' file will produce a similar-looking,
% albeit, 'tighter' paper resulting in, invariably, fewer pages.
%
% ----------------------------------------------------------------------------------------------------------------
% This .tex file (and associated .cls V2.5) produces:
%       1) The Permission Statement
%       2) The Conference (location) Info information
%       3) The Copyright Line with ACM data
%       4) NO page numbers
%
% as against the acm_proc_article-sp.cls file which
% DOES NOT produce 1) thru' 3) above.
%
% Using 'sig-alternate.cls' you have control, however, from within
% the source .tex file, over both the CopyrightYear
% (defaulted to 200X) and the ACM Copyright Data
% (defaulted to X-XXXXX-XX-X/XX/XX).
% e.g.
% \CopyrightYear{2007} will cause 2007 to appear in the copyright line.
% \crdata{0-12345-67-8/90/12} will cause 0-12345-67-8/90/12 to appear in the copyright line.
%
% ---------------------------------------------------------------------------------------------------------------
% This .tex source is an example which *does* use
% the .bib file (from which the .bbl file % is produced).
% REMEMBER HOWEVER: After having produced the .bbl file,
% and prior to final submission, you *NEED* to 'insert'
% your .bbl file into your source .tex file so as to provide
% ONE 'self-contained' source file.
%
% ================= IF YOU HAVE QUESTIONS =======================
% Questions regarding the SIGS styles, SIGS policies and
% procedures, Conferences etc. should be sent to
% Adrienne Griscti (griscti@acm.org)
%
% Technical questions _only_ to
% Gerald Murray (murray@hq.acm.org)
% ===============================================================
%
% For tracking purposes - this is V2.0 - May 2012

\documentclass{../../cls/sig-alternate-05-2015}

\usepackage{algorithm}
\usepackage{algpseudocode}

\usepackage{booktabs}
\usepackage{enumitem}
\usepackage{mathtools}
\usepackage{textcomp}


\begin{document}

% Copyright
%\setcopyright{acmcopyright}
%\setcopyright{acmlicensed}
%\setcopyright{rightsretained}
%\setcopyright{usgov}
%\setcopyright{usgovmixed}
%\setcopyright{cagov}
%\setcopyright{cagovmixed}


% DOI
%\doi{10.475/123_4}

% ISBN
%\isbn{123-4567-24-567/08/06}

%Conference
%\conferenceinfo{PLDI '13}{June 16--19, 2013, Seattle, WA, USA}

%\acmPrice{\$15.00}

%
% --- Author Metadata here ---
%\conferenceinfo{WOODSTOCK}{'97 El Paso, Texas USA}
%\CopyrightYear{2007} % Allows default copyright year (20XX) to be over-ridden - IF NEED BE.
%\crdata{0-12345-67-8/90/01}  % Allows default copyright data (0-89791-88-6/97/05) to be over-ridden - IF NEED BE.
% --- End of Author Metadata ---

%\\TODO:1.tautology prove section 2. set section operation proof 3. function section basic proving onto bijection...
\title{CSCI 3190 \\ Introduction to Discrete Mathematics and Algorithms}
\subtitle{Sample Solution of Assignment 2}

\maketitle
\begin{abstract}

\end{abstract}

\keywords{}

\section{Mathematical Induction}
\textbf{Proof}\begin{enumerate}[label=(\alph*)]
	\item \begin{description}
		\item[Base Step] Let $n = 0$, $3 \mid 2^1 + 1 = 3$.
		\item[Induction Step] Suppose $3 \mid 2^{2n - 1} + 1$, $2^{2n + 1} = 3 * 2^{2n - 1} + 2^{2n - 1} + 1 \equiv 0 \pmod{3}$.
	\end{description}
	\item \begin{description}
		\item[Base Step] Let $n = 0$, $9 \mid 1^3 + 2^3$.
		\item[Induction Step] Suppose $9 \mid (n - 1)^3 + n^3 + (n + 1)^3$, $n^3 + (n + 1)^3 + (n + 2)^3 = (n - 1)^3 + n^3 + (n + 1)^3 + 3((n - 1)^2 + (n - 1)(n + 1) + (n + 1)^2) \equiv 0 \pmod{3}$.
	\end{description}
\end{enumerate}

\section{Complexity}
\textbf{Proof}\begin{enumerate}[label=(\alph*)]
	\item $\exists c = 4, \exists N = 6$, such that $\forall n > N, 3n^2 + 5n + 10 < 4n^2$.
	\item Since $100 = O(1)$ and $\log_2 n = O(n)$, $100\log_2 n = O(n)$.
	\item $n! = \Pi_{i = 1}^n i \le \Pi_{i = 1}^n n = n^n$.
	\item $\exists c = 2, \exists N = 1$, such that $\forall n > N, 2^n = \Pi_{i = 1}^n 2 = 2 \Pi_{i = 2}^n 2 \le 2 \Pi_{i = 2}^n i = 2(n!)$.
\end{enumerate}

\end{document}
