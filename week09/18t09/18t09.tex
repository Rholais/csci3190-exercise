%!TEX program = xelatex
\documentclass[10pt, compress, handout]{beamer}
\usepackage[titleprogressbar]{../../cls/beamerthemem}

\setbeamertemplate{caption}[numbered]
\setbeamertemplate{theorems}[numbered]
\newcounter{example}
\resetcounteronoverlays{example}
\newtheorem{crl}{Corollary}[theorem]
\newtheorem{eg}[example]{Example}
\newtheorem*{solution*}{Solution}

\usepackage{booktabs}
\usepackage[scale=2]{ccicons}
\usepackage{minted}

\usepackage{cleveref}
\crefname{example}{Example}{Examples}

\usepgfplotslibrary{dateplot}

\usemintedstyle{trac}

\usepackage{algorithm}
\usepackage[noend]{algpseudocode}
\resetcounteronoverlays{algorithm}

\usepackage{version}
%\excludeversion{proof}
%\excludeversion{solution*}

\usepackage{mathtools}
\usepackage{multicol}
\usepackage{qtree}

\usepackage{tikz}

\makeatletter
\def\old@comma{,}
\catcode`\,=13
\def,{%
    \ifmmode%
    \old@comma\discretionary{}{}{}%
    \else%
    \old@comma%
    \fi%
}
\makeatother

\title{CSCI 3190 Tutorial of Week 09}
\subtitle{Quiz 2}
\author{LI Haocheng}
\institute{Department of Computer Science and Engineering}

\begin{document}

\maketitle

\begin{frame}
\frametitle{Mode}
\begin{eg}
    Give a recursive algorithm for finding a mode of a list of integers.
\end{eg}
\onslide<2>\begin{solution*}
    \begin{algorithm}[H]
        \caption{A Recursive Algorithm for Mode}
        \label{a-5}
        \begin{algorithmic}
            \Procedure{mode}{$a_1, \cdots, a_n \in \mathbb{Z}$}
            \If{$n = 1$}
            \Return $a_1$
            \EndIf
            \State $m \eqqcolon $ \Call{mode}{$a_1, \cdots, a_{n - 1}$}
            \If{$m = a_n$}
            \Return $a_n$
            \EndIf
            \State $numM \eqqcolon$ number of $m$ in $a_1, \cdots, a_n$
            \State $numN \eqqcolon$ number of $a_n$ in $a_1, \cdots, a_n$
            \If{$numM \le numN$}
            \Return $a_n$
            \EndIf
            \State\Return $m$
            \EndProcedure
        \end{algorithmic}
    \end{algorithm}
\end{solution*}
\end{frame}

\plain{Questions?}

\end{document}
