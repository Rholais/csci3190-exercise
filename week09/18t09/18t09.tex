%!TEX program = xelatex
\documentclass[10pt, compress, handout]{beamer}
\usepackage[titleprogressbar]{../../cls/beamerthemem}

\setbeamertemplate{caption}[numbered]
\setbeamertemplate{theorems}[numbered]
\newcounter{example}
\resetcounteronoverlays{example}
\newtheorem{crl}{Corollary}[theorem]
\newtheorem{eg}[example]{Example}
\newtheorem*{solution*}{Solution}

\usepackage{booktabs}
\usepackage[scale=2]{ccicons}
\usepackage{minted}

\usepackage{cleveref}
\crefname{example}{Example}{Examples}

\usepgfplotslibrary{dateplot}

\usemintedstyle{trac}

\usepackage{algorithm}
\usepackage[noend]{algpseudocode}
\resetcounteronoverlays{algorithm}

\usepackage{version}
%\excludeversion{proof}
%\excludeversion{solution*}

\usepackage{mathtools}
\usepackage{multicol}
\usepackage{qtree}

\usepackage{tikz}

\makeatletter
\def\old@comma{,}
\catcode`\,=13
\def,{%
    \ifmmode%
    \old@comma\discretionary{}{}{}%
    \else%
    \old@comma%
    \fi%
}
\makeatother

\title{CSCI 3190 Tutorial of Week 09}
\subtitle{Quiz 2}
\author{LI Haocheng}
\institute{Department of Computer Science and Engineering}

\begin{document}

\maketitle

\begin{frame}
\frametitle{Exponential}
\begin{columns}
    \begin{column}{0.6\linewidth}
        \onslide<1->\begin{eg}
            \label{eg:8}
            Devise a recursive algorithm to compute $a^{2^n}$,
            where $a \in \mathbb{R} \land n \in \mathbb{N}^+$.
        \end{eg}
        \onslide<2->\begin{solution*}
            The complexity of~\Cref{a:6-a} is $O(n)$.
            \begin{algorithm}[H]
                \caption{Recursive Algorithm}
                \label{a:6-a}
                \begin{algorithmic}
                    \Procedure{Exp}{$n$, $a$}
                    \If{$n$ = 0}
                    \State \Return $a$
                    \Else
                    \State $p \leftarrow$ \Call{Exp}{$n - 1$, $a$}
                    \State \Return $p \times p$
                    \EndIf
                    \EndProcedure
                \end{algorithmic}
            \end{algorithm}
        \end{solution*}
    \end{column}
    \begin{column}{0.5\linewidth}
        \onslide<1->\begin{eg}
            Repeat~\Cref{eg:8} with an iterative algorithm.
        \end{eg}
        \onslide<3>\begin{solution*}
            The complexity of~\Cref{a:6-b} is $O(n)$.
            \begin{algorithm}[H]
                \caption{Iterative Algorithm}
                \label{a:6-b}
                \begin{algorithmic}
                    \Procedure{Exp}{$n$, $a$}
                    \For{$i = 1, \ldots, n$}
                    \State $a \leftarrow a \times a$
                    \EndFor
                    \State \Return $a$
                    \EndProcedure
                \end{algorithmic}
            \end{algorithm}
        \end{solution*}
    \end{column}
\end{columns}
\end{frame}

\begin{frame}[allowframebreaks]
\frametitle{Fibonacci}
\begin{eg}
Consider the computation of the $n$-th Fibonacci number:\begin{enumerate}
\item Give the pseudo code of a recursive algorithm to compute the nth Fibonacci number. What is the
complexity of your algorithm? Explain your answer.
\item Give the pseudo code of an iterative algorithm to compute the nth Fibonacci number. What is the
complexity of your algorithm? Explain your answer. 
\end{enumerate}
\end{eg}

\newpage

\begin{solution*}
\begin{columns}
\begin{column}{.6\linewidth}
    \begin{algorithm}[H]
        \caption{Recursive Algorithm}
        \label{a-4-1}
        \begin{algorithmic}
            \Procedure{Fib}{$n$}
            \If{$n = 0$}
            \State\Return $0$
            \EndIf
            \If{$n = 1$}
            \State\Return $1$
            \EndIf
            \State\Return \Call{Fib}{$n - 1$} + \Call{Fib}{$n - 2$}
            \EndProcedure
        \end{algorithmic}
    \end{algorithm}
    
    $O(a_n) = \left(\frac{\sqrt{5} + 1}{2}\right)^n$.
\end{column}

\begin{column}{.5\linewidth}
    \begin{algorithm}[H]
        \caption{Iterative Algorithm}
        \label{a-4-2}
        \begin{algorithmic}
            \Procedure{IterFib}{$n$}
            \If{$n = 0$}\ \Return $0$
            \EndIf
            \State $x \coloneqq 0, y \coloneqq 1$
            \For{$i \coloneqq 1, 2, \cdots, n - 1$}
            \State $z \coloneqq x + y, x \coloneqq y, y \coloneqq z$
            \EndFor
            \State \Return $y$
            \EndProcedure
        \end{algorithmic}
    \end{algorithm}
    
    The number of additions follows the expression $b_n = n - 1$ so that $O(b_n) = n$.
\end{column}
\end{columns}
\end{solution*}
\end{frame}

\begin{frame}
\frametitle{Mode}
\begin{eg}
    Give a recursive algorithm for finding a mode of a list of integers.
\end{eg}
\onslide<2>\begin{solution*}
    \begin{algorithm}[H]
        \caption{A Recursive Algorithm for Mode}
        \label{a-5}
        \begin{algorithmic}
            \Procedure{mode}{$a_1, \cdots, a_n \in \mathbb{Z}$}
            \If{$n = 1$}
            \Return $a_1$
            \EndIf
            \State $m \eqqcolon $ \Call{mode}{$a_1, \cdots, a_{n - 1}$}
            \If{$m = a_n$}
            \Return $a_n$
            \EndIf
            \State $numM \eqqcolon$ number of $m$ in $a_1, \cdots, a_n$
            \State $numN \eqqcolon$ number of $a_n$ in $a_1, \cdots, a_n$
            \If{$numM \le numN$}
            \Return $a_n$
            \EndIf
            \State\Return $m$
            \EndProcedure
        \end{algorithmic}
    \end{algorithm}
\end{solution*}
\end{frame}

\plain{Questions?}

\end{document}
