%!TEX program = xelatex
\documentclass[10pt, compress, handout]{beamer}
\usepackage[titleprogressbar]{../../cls/beamerthemem}

\usepackage{booktabs}
\usepackage[scale=2]{ccicons}
\usepackage{minted}

\usepgfplotslibrary{dateplot}

\usemintedstyle{trac}

\setbeamertemplate{caption}[numbered]
\setbeamertemplate{theorems}[numbered]
\newtheorem{crl}{Corollary}[theorem]
\newtheorem*{solution*}{Solution}

\usepackage{algorithm}
\usepackage[noend]{algpseudocode}

\usepackage{version}
%\excludeversion{proof}
%\excludeversion{solution*}

\usepackage{mathtools}
\usepackage{multicol}
\usepackage{qtree}

\usepackage{tikz}

\makeatletter
\def\old@comma{,}
\catcode`\,=13
\def,{%
	\ifmmode%
	\old@comma\discretionary{}{}{}%
	\else%
	\old@comma%
	\fi%
}
\makeatother

\title{CSCI 3190 Tutorial of Week 09}
\subtitle{Mathematical Induction}
\author{LI Haocheng}
\institute{Department of Computer Science and Engineering}

\begin{document}

\maketitle

\begin{frame}[allowframebreaks]
\frametitle{Equivalence Relation}
\begin{theorem}
	Let $R$ be an equivalence relation on a set $A$. These statements for elements $a$ and $b$ of $A$ are equivalent:\begin{enumerate}
		\item $aRb$
		\item $[a]=[b]$
		\item $[a]\cap[b] \ne \emptyset$
	\end{enumerate}
\end{theorem}
\textbf{Proof.} We first show that 1 implies 2. Assume that $aRb$.
We will prove that $[a]=[b]$ by showing $[a]\subseteq[b]$ and $[b]\subseteq[a]$. Suppose $c \in [a]$. Then $aRc$.
Because $aRb$ and $R$ is symmetric, we know that $bRa$. 

Furthermore, because $R$ is transitive and $bRa$ and $aRc$, it follows that $bRc$. Hence, $c \in [b]$.
This shows that $[a]\subseteq[b]$. The proof that $[b]\subseteq[a]$ is similar.

Second, we will show that 2 implies 3. Assume that $[a]=[b]$.
It follows that $[a]\cap[b] \ne \emptyset$ because $[a]$ is nonempty (because $a \in [a]$ because $R$ is reflexive).

Next, we will show that 3 implies 1. Suppose that $[a]\cap[b] \ne \emptyset$. Then there is an
element $c$ with $c \in [a]$ and $c \in [b]$. In other words, $aRc$ and $bRc$. By the symmetric property, $cRb$.
Then by transitivity, because $aRc$ and $cRb$, we have $aRb$.

Because 1 implies 2, 2 implies 3, and 3 implies 1, the three statements, 1, 2, and 3, are equivalent.
\end{frame}

\begin{frame}[fragile]
\frametitle{Equivalence Class}
\begin{example}
	Find smallest equivalence relation on set $\{a, b, c, d, e\}$ containing relation $\{(a, b), (b, c), (d, c)\}$.
\end{example}
\begin{solution*}
	\begin{enumerate}
		\item<2-> Find Equivalence Classes: $[a] = \{a, b, c, d\}, [e] = \{e\}$.
		\item<3> Find Equivalence Relation: $\{(a, a), (a, b), (a, c), (a, d), (b, a), (b, b), (b, c), (b, d), (c, a),$
		$(c, b), (c, c), (c, d), (d, a), (d, b), (d, c), (d, d), (e, e)\}$
	\end{enumerate}
\end{solution*}
\end{frame}

\begin{frame}[fragile]
\frametitle{Raccoons}
\begin{example}
	Let the lifespan of a raccoon be exactly 6 years.
	Suppose there are 4 new-born raccoons at the 0th year and the number of new-born raccoons in each year is 3 times
	that in the previous year. Let $b_r$ be the number of raccoons on $r$-th year where $r \ge 0$.
	Give a closed form generating function for $b_r$.
\end{example}

\onslide<2>\begin{solution*}
	\begin{equation}
		n_r = \begin{cases}
		4 \times 3^r, & r \ge 0\\
		0, & \text{otherwise}.
		\end{cases} \leftrightarrow \frac{4}{1 - 3x},
	\end{equation} so that number of racoons is \begin{equation}
		b_r = \Sigma_{i = r - 5}^{r} n_i \leftrightarrow \frac{4(1 - x^6)}{(1 - 3x)(1 - x)}.
	\end{equation}
\end{solution*}
\end{frame}

\begin{frame}[fragile]
\frametitle{Mathematical Induction}
\begin{example}
	For $n \in \mathbb{N}^+$, prove each of the following by mathematical induction:\begin{enumerate}
		\item $3 | 2^{2n + 1} + 1$;
		\item $9 | n^3 + (n + 1)^3 + (n + 2)^3$.
	\end{enumerate}
\end{example}
\begin{proof}
	\begin{enumerate}
		\item<2-> \begin{description}
			\item[Base] Let $n = 0$, $3 \mid 2^1 + 1 = 3$.
			\item[Induction] Suppose $3 \mid 2^{2n - 1} + 1$, $2^{2n + 1} = 3 * 2^{2n - 1} + 2^{2n - 1} + 1$
			$\equiv 0 \pmod{3}$.
		\end{description}
		\item<3> \begin{description}
			\item[Base] Let $n = 0$, $9 \mid 1^3 + 2^3$.
			\item[Induction] Suppose $9 \mid (n - 1)^3 + n^3 + (n + 1)^3$, $n^3 + (n + 1)^3 + (n + 2)^3$
			$= (n - 1)^3 + n^3 + (n + 1)^3 + 3((n - 1)^2 + (n - 1)(n + 2) + (n + 2)^2)$
			$= (n - 1)^3 + n^3 + (n + 1)^3 + 9(n^2 + n + 1) \equiv 0 \pmod{9}$.
		\end{description}
	\end{enumerate}
\end{proof}
\end{frame}

\plain{Questions?}

\end{document}
