%!TEX program = xelatex
\documentclass[10pt, compress, handout]{beamer}
\usepackage[titleprogressbar]{../../cls/beamerthemem}

\usepackage{booktabs}
\usepackage[scale=2]{ccicons}
\usepackage{minted}

\usepgfplotslibrary{dateplot}

\usemintedstyle{trac}

\setbeamertemplate{caption}[numbered]
\setbeamertemplate{theorems}[numbered]
\newtheorem{crl}{Corollary}[theorem]
\newtheorem*{solution*}{Solution}

\usepackage{algorithm}
\usepackage[noend]{algpseudocode}

\usepackage{version}
%\excludeversion{proof}
%\excludeversion{solution*}

\usepackage{mathtools}
\usepackage{multicol}
\usepackage{qtree}

\usepackage{tikz}

\makeatletter
\def\old@comma{,}
\catcode`\,=13
\def,{%
	\ifmmode%
	\old@comma\discretionary{}{}{}%
	\else%
	\old@comma%
	\fi%
}
\makeatother

\title{CSCI 3190 Tutorial of Week 09}
\subtitle{Mathematical Induction}
\author{LI Haocheng}
\institute{Department of Computer Science and Engineering}

\begin{document}

\maketitle

\begin{frame}[allowframebreaks]
\frametitle{Equivalence Relation}
\begin{theorem}
	Let $R$ be an equivalence relation on a set $A$. These statements for elements $a$ and $b$ of $A$ are equivalent:\begin{enumerate}
		\item $aRb$
		\item $[a]=[b]$
		\item $[a]\cap[b] \ne \emptyset$
	\end{enumerate}
\end{theorem}
\textbf{Proof.} We first show that 1 implies 2. Assume that $aRb$.
We will prove that $[a]=[b]$ by showing $[a]\subseteq[b]$ and $[b]\subseteq[a]$. Suppose $c \in [a]$. Then $aRc$.
Because $aRb$ and $R$ is symmetric, we know that $bRa$. 

Furthermore, because $R$ is transitive and $bRa$ and $aRc$, it follows that $bRc$. Hence, $c \in [b]$.
This shows that $[a]\subseteq[b]$. The proof that $[b]\subseteq[a]$ is similar.

Second, we will show that 2 implies 3. Assume that $[a]=[b]$.
It follows that $[a]\cap[b] \ne \emptyset$ because $[a]$ is nonempty (because $a \in [a]$ because $R$ is reflexive).

Next, we will show that 3 implies 1. Suppose that $[a]\cap[b] \ne \emptyset$. Then there is an
element $c$ with $c \in [a]$ and $c \in [b]$. In other words, $aRc$ and $bRc$. By the symmetric property, $cRb$.
Then by transitivity, because $aRc$ and $cRb$, we have $aRb$.

Because 1 implies 2, 2 implies 3, and 3 implies 1, the three statements, 1, 2, and 3, are equivalent.
\end{frame}

\begin{frame}[fragile]
\frametitle{Equivalence Class}
\begin{example}
	Find smallest equivalence relation on set $\{a, b, c, d, e\}$ containing relation $\{(a, b), (b, c), (d, c)\}$.
\end{example}
\begin{solution*}
	\begin{enumerate}
		\item<2-> Find Equivalence Classes: $[a] = \{a, b, c, d\}, [e] = \{e\}$.
		\item<3> Find Equivalence Relation: $\{(a, a), (a, b), (a, c), (a, d), (b, a), (b, b), (b, c), (b, d), (c, a),$
		$(c, b), (c, c), (c, d), (d, a), (d, b), (d, c), (d, d), (e, e)\}$
	\end{enumerate}
\end{solution*}
\end{frame}

\plain{Questions?}

\end{document}
