%!TEX program = xelatex
\documentclass[10pt, compress]{beamer}
\usepackage[titleprogressbar]{../../cls/beamerthemem}

\usepackage{booktabs}
\usepackage[scale=2]{ccicons}
\usepackage{minted}

\usepgfplotslibrary{dateplot}

\usemintedstyle{trac}

\setbeamertemplate{caption}[numbered]
\setbeamertemplate{theorems}[numbered]
\newtheorem{crl}{Corollary}[theorem]
\newtheorem*{solution*}{Solution}

\usepackage{algorithm}
\usepackage[noend]{algpseudocode}

\usepackage{version}
%\excludeversion{proof}
%\excludeversion{solution*}

\usepackage{mathtools}
\usepackage{multicol}
\usepackage{qtree}

\usepackage{tikz}

\makeatletter
\def\old@comma{,}
\catcode`\,=13
\def,{%
	\ifmmode%
	\old@comma\discretionary{}{}{}%
	\else%
	\old@comma%
	\fi%
}
\makeatother

\title{CSCI 3190 Tutorial of Week 09}
\subtitle{Mathematical Induction}
\author{LI Haocheng}
\institute{Department of Computer Science and Engineering}

\begin{document}

\maketitle

\begin{frame}[allowframebreaks]
\frametitle{Equivalence Relation}
\begin{theorem}
	Let $R$ be an equivalence relation on a set $A$. These statements for elements $a$ and $b$ of $A$ are equivalent:\begin{enumerate}
		\item $aRb$
		\item $[a]=[b]$
		\item $[a]\cap[b] \ne \emptyset$
	\end{enumerate}
\end{theorem}
\textbf{Proof.} We first show that 1 implies 2. Assume that $aRb$. We will prove that $[a]=[b]$ by showing $[a]\subseteq[b]$ and $[b]\subseteq[a]$. Suppose $c \in [a]$. Then $aRc$. Because $aRb$ and $R$ is symmetric, we know that $bRa$. 

Furthermore, because $R$ is transitive and $bRa$ and $aRc$, it follows that $bRc$. Hence, $c \in [b]$. This shows that $[a]\subseteq[b]$. The proof that $[b]\subseteq[a]$ is similar.

Second, we will show that 2 implies 3. Assume that $[a]=[b]$. It follows that $[a]\cap[b] \ne \emptyset$ because $[a]$ is nonempty (because $a \in [a]$ because $R$ is reflexive).

Next, we will show that 3 implies 1. Suppose that $[a]\cap[b] \ne \emptyset$. Then there is an
element $c$ with $c \in [a]$ and $c \in [b]$. In other words, $aRc$ and $bRc$. By the symmetric property, $cRb$. Then by transitivity, because $aRc$ and $cRb$, we have $aRb$.

Because 1 implies 2, 2 implies 3, and 3 implies 1, the three statements, 1, 2, and 3, are equivalent.
\end{frame}

\begin{frame}[fragile]
\frametitle{Raccoons}
\begin{example}
	Let the lifespan of a raccoon be exactly 6 years. Suppose there are 4 new-born raccoons at the 0th year and the number of new-born raccoons in each year is 3 times that in the previous year. Let $b_r$ be the number of raccoons on $r$-th year where $r \ge 0$. Give a closed form generating function for $b_r$.
\end{example}

\onslide<2>\begin{solution*}
	\begin{equation}
		n_r = \begin{cases}
		4 \times 3^r, & r \ge 0\\
		0, & \text{otherwise}.
		\end{cases} \leftrightarrow \frac{4}{1 - 3x},
	\end{equation} so that number of racoons is \begin{equation}
		b_r = \Sigma_{i = r - 5}^{r} n_i \leftrightarrow \frac{4(1 - x^6)}{(1 - 3x)(1 - x)}.
	\end{equation}
\end{solution*}
\end{frame}

\begin{frame}[fragile]
\frametitle{Mathematical Induction}
\begin{example}
	For $n \in \mathbb{N}^+$, prove each of the following by mathematical induction:\begin{enumerate}
		\item $3 | 2^{2n + 1}$;
		\item $9 | n^3 + (n + 1)^3 + (n + 2)^3$.
	\end{enumerate}
\end{example}
\begin{proof}
	\begin{enumerate}
		\item<2-> \begin{description}
			\item[Base] Let $n = 0$, $3 \mid 2^1 + 1 = 3$.
			\item[Induction] Suppose $3 \mid 2^{2n - 1} + 1$, $2^{2n + 1} = 3 * 2^{2n - 1} + 2^{2n - 1} + 1 \equiv 0 \pmod{3}$.
		\end{description}
		\item<3> \begin{description}
			\item[Base] Let $n = 0$, $9 \mid 1^3 + 2^3$.
			\item[Induction] Suppose $9 \mid (n - 1)^3 + n^3 + (n + 1)^3$, $n^3 + (n + 1)^3 + (n + 2)^3 = (n - 1)^3 + n^3 + (n + 1)^3 + 3((n - 1)^2 + (n - 1)(n + 2) + (n + 2)^2) = (n - 1)^3 + n^3 + (n + 1)^3 + 9(n^2 + n + 1) \equiv 0 \pmod{9}$.
		\end{description}
	\end{enumerate}
\end{proof}
\end{frame}

\begin{frame}
\frametitle{Right Trominoes}
\begin{columns}
	\begin{column}{.3\linewidth}
		\begin{example}
			Let $n \in \mathbb{N}^+$. Show that every $2^n \times 2^n$ checkerboard with one square removed can be tiled using right trominoes, where these pieces cover three squares at a time.
		\end{example}
		\begin{figure}
			\includegraphics[width=.4\linewidth]{tromino}
			\caption{Tromino}
		\end{figure}
	\end{column}
	\begin{column}{.8\linewidth}
		\begin{proof}
			\begin{description}
				\item[Base]<2-> $P(1)$ is true, obviously.
				\item[Induction]<3> Split the $2^{k + 1} \times 2^{k + 1}$ checkerboard with one square removed into four checkerboards of size $2^k \times 2^k$. Now temporarily remove the
				square from each of the three full checkerboards at the center of the original. By the inductive hypothesis,
				each of these four checkerboards with a square removed can be tiled by right trominoes. Furthermore, the three squares that were temporarily removed can be covered by one right
				tromino.
			\end{description}
		\end{proof}
	\end{column}
\end{columns}
\end{frame}

\begin{frame}[fragile]
\frametitle{Recursive Algorithms}
\begin{columns}
	\begin{column}{.4\linewidth}
		\begin{example}
			\setcounter{algorithm}{0}
			\begin{algorithm}[H]
				\caption{Recursive Algorithm to Find $a^{2^n}$}
				\label{alg:a2n}
				\begin{algorithmic}[1]
					\Require $a \in \mathbb{R}, n \in \mathbb{N}^+$.
					\Ensure $a^{2^n}$.
					\Procedure{pow2}{$a$, $n$}
					\onslide<2->\If {$n = 1$}
					\State\Return $a$
					\EndIf
					\State $p \gets$ \Call{pow2}{$a$, $n - 1$}
					\State\Return $p \times p$
					\EndProcedure
				\end{algorithmic}
			\end{algorithm}
		\end{example}
	\end{column}
	\begin{column}{.6\linewidth}
		\onslide<1->\begin{example}
			\setcounter{algorithm}{1}
			\begin{algorithm}[H]
				\caption{Recursive Algorithm to Find $a^{n}$}
				\label{alg:an}
				\begin{algorithmic}[1]
					\Require $a \in \mathbb{R}, n \in \mathbb{N}$.
					\Ensure $a^{n}$.
					\Procedure{pow}{$a$, $n$}
					\onslide<3>\If {$n = 0$}
					\Return $1$
					\EndIf
					\If {$n = 1$}
					\Return $a$
					\EndIf
					\State $p \gets$ \Call{pow2}{$a$, $\lfloor\frac{n}{2}\rfloor$}
					\If {$n \equiv 0 \pmod{2}$}
					\Return $p \times p$
					\EndIf
					\State\Return $p \times p \times a$
					\EndProcedure
				\end{algorithmic}
			\end{algorithm}
		\end{example}
	\end{column}
\end{columns}
\end{frame}

\begin{frame}[allowframebreaks]
	\frametitle{Tree}
	\begin{theorem}
		An undirected graph is a tree if and only if there is a unique path without cycle between any two of its vertices.
	\end{theorem}
	\textbf{Proof.} First assume that $T$ is a tree which is connected graph without circuit.
	Because $T$ is connected, there is a path without cycle between two vertices $x$ and $y$.
	Moreover, this path must be unique, for if there were a second such path,
	they would form a circuit.
	
	\newpage
	
	This implies that there is a circuit in $T$. Hence, there is a unique path without cycle between any two vertices of a tree. 
	
	Now assume that there is a unique path without cycle between any two vertices of a graph $T$.
	Then $T$ is connected, because there is a path between any two of its vertices.
	Furthermore, $T$ can have no circuits.
	To see that this is true, suppose $T$ had a circuit that contained the vertices $x$ and $y$.
	Then there would be two simple paths between $x$ and $y$,
	Hence, a graph with a unique simple path between any two vertices is a tree.
\end{frame}

\plain{Questions?}

\end{document}
