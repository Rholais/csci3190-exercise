%!TEX program = xelatex
\documentclass[10pt, compress]{beamer}
\usetheme[titleprogressbar]{m}

\usepackage{booktabs}
\usepackage[scale=2]{ccicons}
\usepackage{minted}

\usepgfplotslibrary{dateplot}

\usemintedstyle{trac}

\setbeamertemplate{caption}[numbered]
\setbeamertemplate{theorems}[numbered]
\newtheorem{crl}{Corollary}[theorem]

\usepackage{multicol}
\usepackage{qtree}

\makeatletter
\def\old@comma{,}
\catcode`\,=13
\def,{%
	\ifmmode%
	\old@comma\discretionary{}{}{}%
	\else%
	\old@comma%
	\fi%
}
\makeatother

\title{CSCI 3190 Tutorial of Week 9}
\subtitle{Trees}
\author{LI Haocheng}
\institute{Department of Computer Science and Engineering}

\begin{document}

\maketitle

\begin{frame}[fragile]
	\frametitle{Subset}
	\onslide<1->\begin{example}
		Let $S$ be a set of seven positive integers the maximum of which is at most 24. Prove that the sums of the elements in all the nonempty subsets of $S$ cannot be distinct.
	\end{example}
	
	\onslide<2>\begin{proof}
		Consider subset of size at most 5. Their total sum is at most $20 + 21 + \cdots + 24 = 110$. There are $128 - 1 - 1 - 7 = 119$ non-empty subsets with at most 5 elements so at least 2 of them have equal sum.
	\end{proof}
\end{frame}

\begin{frame}[fragile]
	\frametitle{Relation}
	\onslide<1->\begin{example}
		Let $n \in \mathbb{Z}^n$, let $U = \{1, 2, \cdots, n\}$. Define the relation $R$ on the power set of $U$ by $(A, B) \in R$ if and only if $A \not\subset B$ and $B \not\subset A$. Is $R$ an equivalence relation? What is $|R|$? (Note that the power set of $U$ is set of all subset of $U$.)
	\end{example}
	\onslide<2>\textbf{Solution} \begin{enumerate}
		\item $R$ is not an equivalence relation. Let $n = 3, A = \{1\}, B = \{2\}$, $C = \{1, 3\}$ so that $(A, B) \in R, (B, C) \in R, (A, C) \notin R$.
		\item $|R| = 2^{2n} - 2(3^n - 2^n)$.
	\end{enumerate}
\end{frame}

\begin{frame}[fragile]
	\frametitle{Equivalence Class}
	\begin{definition}
		Let $R$ be an equivalence relation on a set $A$. The set of all elements that are related to an element $a$ of $A$ is called the equivalence class of $a$. The equivalence class of a with respect
		to $R$ is denoted by $[a]_R$. When only one relation is under consideration, we can delete the subscript $R$ and write $[a]$ for this equivalence class.
	\end{definition}
\end{frame}

\plain{Questions?}

\end{document}
