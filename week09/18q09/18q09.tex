% This is "sig-alternate.tex" V2.1 April 2013
% This file should be compiled with V2.5 of "sig-alternate.cls" May 2012
%
% This example file demonstrates the use of the 'sig-alternate.cls'
% V2.5 LaTeX2e document class file. It is for those submitting
% articles to ACM Conference Proceedings WHO DO NOT WISH TO
% STRICTLY ADHERE TO THE SIGS (PUBS-BOARD-ENDORSED) STYLE.
% The 'sig-alternate.cls' file will produce a similar-looking,
% albeit, 'tighter' paper resulting in, invariably, fewer pages.
%
% ----------------------------------------------------------------------------------------------------------------
% This .tex file (and associated .cls V2.5) produces:
%       1) The Permission Statement
%       2) The Conference (location) Info information
%       3) The Copyright Line with ACM data
%       4) NO page numbers
%
% as against the acm_proc_article-sp.cls file which
% DOES NOT produce 1) thru' 3) above.
%
% Using 'sig-alternate.cls' you have control, however, from within
% the source .tex file, over both the CopyrightYear
% (defaulted to 200X) and the ACM Copyright Data
% (defaulted to X-XXXXX-XX-X/XX/XX).
% e.g.
% \CopyrightYear{2007} will cause 2007 to appear in the copyright line.
% \crdata{0-12345-67-8/90/12} will cause 0-12345-67-8/90/12 to appear in the copyright line.
%
% ---------------------------------------------------------------------------------------------------------------
% This .tex source is an example which *does* use
% the .bib file (from which the .bbl file % is produced).
% REMEMBER HOWEVER: After having produced the .bbl file,
% and prior to final submission, you *NEED* to 'insert'
% your .bbl file into your source .tex file so as to provide
% ONE 'self-contained' source file.
%
% ================= IF YOU HAVE QUESTIONS =======================
% Questions regarding the SIGS styles, SIGS policies and
% procedures, Conferences etc. should be sent to
% Adrienne Griscti (griscti@acm.org)
%
% Technical questions _only_ to
% Gerald Murray (murray@hq.acm.org)
% ===============================================================
%
% For tracking purposes - this is V2.0 - May 2012

\documentclass{../../cls/sig-alternate-05-2015}

\usepackage{algorithm}
\usepackage{algpseudocode}

\usepackage{booktabs}
\usepackage{cleveref}
\usepackage{color}
\usepackage{enumitem}
\usepackage{mathtools}
\usepackage{soul}
\usepackage{textcomp}

\begin{document}

% Copyright
%\setcopyright{acmcopyright}
%\setcopyright{acmlicensed}
%\setcopyright{rightsretained}
%\setcopyright{usgov}
%\setcopyright{usgovmixed}
%\setcopyright{cagov}
%\setcopyright{cagovmixed}


% DOI
%\doi{10.475/123_4}

% ISBN
%\isbn{123-4567-24-567/08/06}

%Conference
%\conferenceinfo{PLDI '13}{June 16--19, 2013, Seattle, WA, USA}

%\acmPrice{\$15.00}

%
% --- Author Metadata here ---
%\conferenceinfo{WOODSTOCK}{'97 El Paso, Texas USA}
%\CopyrightYear{2007} % Allows default copyright year (20XX) to be over-ridden - IF NEED BE.
%\crdata{0-12345-67-8/90/01}  % Allows default copyright data (0-89791-88-6/97/05) to be over-ridden - IF NEED BE.
% --- End of Author Metadata ---

\makeatletter
\def\old@comma{,}
\catcode`\,=13
\def,{%
    \ifmmode%
    \old@comma\discretionary{}{}{}%
    \else%
    \old@comma%
    \fi%
}
\makeatother

\title{CSCI 3190 \\ Introduction to Discrete Mathematics and Algorithms}
\subtitle{Sample Solution of Quiz 2}

\maketitle
\begin{abstract}

\end{abstract}

\keywords{}

\section{Complexity}
Show that $n^2 + 100 \sqrt{n} + 7 = O(n^2)$.

\textbf{Proof} Let $C = 2, N = 25$, then $\forall n \ge N, n^2 + 100 \sqrt{n} + 7 < Cn^2$.

\section{Mathematical Induction}
Prove by mathematical induction that \begin{equation}
\sum_{i = 1}^{n} \frac{1}{i^2} < 2 - \frac{1}{n}.
\end{equation}

\textbf{Proof} \begin{description}
    \item[Basis step] When $n = 2$,
    it's obvious that $1 + \frac{1}{4} < 2 - \frac{1}{2}$.
    \item[Inductive step] Let statement $P(n)$ be $\sum_{i = 1}^{n} \frac{1}{i^2} < 2 - \frac{1}{n}$. When $P(k)$ is true, \begin{align}
    \begin{aligned}
    \sum_{i = 1}^{k + 1} \frac{1}{i^2} = & \sum_{i = 1}^{k} \frac{1}{i^2} + \frac{1}{(k + 1)^2}\\
    < & 2 - \frac{1}{k} + \frac{1}{k(k + 1)}\\
    = & 2 - \frac{1}{k + 1}.
    \end{aligned}
    \end{align}
    Then $P(k + 1)$ is also true.
\end{description}

\section{Recurrence Equation}
Consider~\Cref{a:3} to find the minimum of
$n$ numbers.Let $T(n)$ be the running time to solve FindMin() where $n$ is the number of elements in $A[]$
\begin{algorithm}[H]
    \caption{Recursive Algorithm}
    \label{a:3}
    \begin{algorithmic}
        \Procedure{FindMin}{$A[]$}
        \If{\Call{len}{$A[]$} = 1}
        \State \Return $A[0]$
        \Else
        \State $p \leftarrow$ \Call{FindMin}{1H of $A[]$}
        \State $q \leftarrow$ \Call{FindMin}{2H of $A[]$}
        \State \Return \Call{min}{$p$, $q$}
        \EndIf
        \EndProcedure
    \end{algorithmic}
\end{algorithm}
\begin{enumerate}
    \item Write a recurrence equation for $T(n)$.
    \item Solve the recurrence and find the complexity of FindMin().
\end{enumerate}

\textbf{Proof} \begin{enumerate}
    \item \begin{equation}
    \begin{cases}
    T(1) = a,\\
    T(n) = 2T(\frac{n}{2}) + b.
    \end{cases}
    \end{equation}
    \item $T(n) = 2T(\frac{n}{2}) + b = 4T(\frac{n}{4}) + 3b = \cdots = nT(1) + (n - 1)b = (a + b)n - b = O(n)$.
\end{enumerate}

\section{Bakery}
Jim wants to order $k$ pies from the bakery.
Unfortunately,
the bakery is only left with 3 apple pies,
2 banana pies and 4 cheese pies.
Let $d_k$ be the number of ways
to make up an order of $k$ pies.
\begin{enumerate}
    \item Give a generating function for $d_k$.
    \item What are $d_k$ when $k = 0, 1, \ldots, 9$?
\end{enumerate}
\textbf{Solution} \begin{enumerate}
    \item \begin{align}
    \begin{aligned}
    & (1 + x + x^2)(1 + x + x^2 + x^3)\\
    & \qquad \cdot (1 + x + x^2 + x^3 + x^4)\\
    = & (1 + 2x + 3x^2 + 3x^3 + 2x^4 + x^5)\\
    & \qquad \cdot (1 + x + x^2 + x^3 + x^4)\\
    = & 1 + 3x + 6x^2 + 9x^3 + 11x^4 + 11x^5\\
    & \qquad + 9x^6 + 6x^7 + 3x^8 + x^9.
    \end{aligned}
    \end{align}
    \item $d_0 = 1, d_1 = 3, d_2 = 6, d_3 = 9, d_4 = 11, d_5 = 11, d_6 = 9, d_7 = 6, d_8 = 3, d_9 = 1$.
\end{enumerate}

\end{document}
