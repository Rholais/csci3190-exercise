% This is "sig-alternate.tex" V2.1 April 2013
% This file should be compiled with V2.5 of "sig-alternate.cls" May 2012
%
% This example file demonstrates the use of the 'sig-alternate.cls'
% V2.5 LaTeX2e document class file. It is for those submitting
% articles to ACM Conference Proceedings WHO DO NOT WISH TO
% STRICTLY ADHERE TO THE SIGS (PUBS-BOARD-ENDORSED) STYLE.
% The 'sig-alternate.cls' file will produce a similar-looking,
% albeit, 'tighter' paper resulting in, invariably, fewer pages.
%
% ----------------------------------------------------------------------------------------------------------------
% This .tex file (and associated .cls V2.5) produces:
%       1) The Permission Statement
%       2) The Conference (location) Info information
%       3) The Copyright Line with ACM data
%       4) NO page numbers
%
% as against the acm_proc_article-sp.cls file which
% DOES NOT produce 1) thru' 3) above.
%
% Using 'sig-alternate.cls' you have control, however, from within
% the source .tex file, over both the CopyrightYear
% (defaulted to 200X) and the ACM Copyright Data
% (defaulted to X-XXXXX-XX-X/XX/XX).
% e.g.
% \CopyrightYear{2007} will cause 2007 to appear in the copyright line.
% \crdata{0-12345-67-8/90/12} will cause 0-12345-67-8/90/12 to appear in the copyright line.
%
% ---------------------------------------------------------------------------------------------------------------
% This .tex source is an example which *does* use
% the .bib file (from which the .bbl file % is produced).
% REMEMBER HOWEVER: After having produced the .bbl file,
% and prior to final submission, you *NEED* to 'insert'
% your .bbl file into your source .tex file so as to provide
% ONE 'self-contained' source file.
%
% ================= IF YOU HAVE QUESTIONS =======================
% Questions regarding the SIGS styles, SIGS policies and
% procedures, Conferences etc. should be sent to
% Adrienne Griscti (griscti@acm.org)
%
% Technical questions _only_ to
% Gerald Murray (murray@hq.acm.org)
% ===============================================================
%
% For tracking purposes - this is V2.0 - May 2012

\documentclass{sig-alternate-05-2015}

\usepackage{booktabs}

\begin{document}

% Copyright
%\setcopyright{acmcopyright}
%\setcopyright{acmlicensed}
%\setcopyright{rightsretained}
%\setcopyright{usgov}
%\setcopyright{usgovmixed}
%\setcopyright{cagov}
%\setcopyright{cagovmixed}


% DOI
%\doi{10.475/123_4}

% ISBN
%\isbn{123-4567-24-567/08/06}

%Conference
%\conferenceinfo{PLDI '13}{June 16--19, 2013, Seattle, WA, USA}

%\acmPrice{\$15.00}

%
% --- Author Metadata here ---
%\conferenceinfo{WOODSTOCK}{'97 El Paso, Texas USA}
%\CopyrightYear{2007} % Allows default copyright year (20XX) to be over-ridden - IF NEED BE.
%\crdata{0-12345-67-8/90/01}  % Allows default copyright data (0-89791-88-6/97/05) to be over-ridden - IF NEED BE.
% --- End of Author Metadata ---

\title{CSCI 3190 \\ Introduction to Discrete Mathematics and Algorithms}
\subtitle{Extended Exercise 1}

\maketitle
\begin{abstract}

\end{abstract}

\keywords{}

\section{Propositional Logic}
\subsection{Propositional Logic}

\begin{enumerate}
\item Explain, without using a truth table, why $(p \vee q \vee r) \wedge
(\neg p \vee \neg q \vee \neg r)$ is true when at least one of $p$, $q$, and $r$
is true and at least one is false, but is false when all three
variables have the same truth value.
\end{enumerate}

\subsection{Application}

\begin{enumerate}
\item Each inhabitant of a remote village always tells the truth
or always lies.A villager will give only a \textquotedblleft Yes\textquotedblright or a \textquotedblleft No\textquotedblright
response to a question a tourist asks. Suppose you are a
tourist visiting this area and come to a fork in the road.
One branch leads to the ruins you want to visit; the other
branch leads deep into the jungle. A villager is standing
at the fork in the road. What one question can you ask the
villager to determine which branch to take?
\end{enumerate}

\subsection{Propositional Equivalences}

\begin{enumerate}
\item Show that $(p \wedge q) \rightarrow p$ is a tautology
by using truth tables.

\item Determine whether $(\neg q \wedge (p \rightarrow q)) \rightarrow \neg p$ is a tautology.

\item Show that $(p \vee q) \wedge (\neg p \vee r) \rightarrow (q \vee r)$ is a tautology.

\end{enumerate}

\subsection{Predicates and Quantifiers}

\begin{enumerate}
\item Let $P(x)$ denote the statement \textquotedblleft $x \le 4$.\textquotedblright What are these
truth values?
\begin{enumerate}
	\item $P(0)$
	\item $P(4)$
	\item $P(6)$
\end{enumerate}

\item Let $P(x)$ be the statement \textquotedblleft $x$ spends more than five hours
every weekday in class,\textquotedblright where the domain for $x$ consists
of all students. Express each of these quantifications in
English.
\begin{enumerate}
	\item $\exists x P(x)$
	\item $\forall x P(x)$
	\item $\exists x \neg P(x)$
	\item $\forall x \neg P(x)$
\end{enumerate}

\item Translate these statements into English, where $C(x)$ is \textquotedblleft $x$
is a comedian\textquotedblright and $F(x)$ is \textquotedblleft $x$ is funny\textquotedblright and the domain
consists of all people.
\begin{enumerate}
	\item $\forall x (C(x) \rightarrow F(x))$
	\item $\forall x (C(x) \wedge F(x))$
	\item $\exists x (C(x) \rightarrow F(x))$
	\item $\exists x (C(x) \wedge F(x))$
\end{enumerate}

\item Let $P(x)$ be the statement \textquotedblleft $x$ can speak Cantonese\textquotedblright and let
$Q(x)$ be the statement \textquotedblleft $x$ knows the computer language
C++.\textquotedblright Express each of these sentences in terms of $P(x)$,
$Q(x)$, quantifiers, and logical connectives. The domain
for quantifiers consists of all students at CU.
\begin{enumerate}
	\item There is a student at CU who can speak Cantonese
	and who knows C++.
	\item There is a student at CU who can speak Cantonese
	but who doesn’t know C++.
	\item Every student at CU either can speak Cantonese
	or knows C++.
	\item No student at CU can speak Cantonese or knows
	C++.
\end{enumerate}

\item Let $P(x)$ be the statement \textquotedblleft $x = x^2$.\textquotedblright If the domain consists
of the integers, what are these truth values?
\begin{enumerate}
	\item $P(0)$
	\item $P(1)$
	\item $P(2)$
	\item $P(-1)$
	\item $\exists x P(x)$
	\item $\forall x P(x)$
\end{enumerate}

\item Suppose that the domain of the propositional function
$P(x)$ consists of the integers 0 and 1. Write out
each of these propositions using disjunctions, conjunctions,
and negations.
\begin{enumerate}
	\item $\exists x P(x)$
	\item $\forall x P(x)$
	\item $\exists x \neg P(x)$
	\item $\forall x \neg P(x)$
	\item $\neg \exists x P(x)$
	\item $\neg \forall x P(x)$
\end{enumerate}

\item For each of these statements find a domain for which the
statement is true and a domain for which the statement is
false.
\begin{enumerate}
	\item Everyone is studying discrete mathematics.
	\item Everyone is older than 21 years.
	\item Every two people have the same mother.
	\item No two different people have the same grandmother.
\end{enumerate}

\item Find a counterexample, if possible, to these universally
quantified statements, where the domain for all variables
consists of all integers.
\begin{enumerate}
	\item $\forall x (x^2 \ge x)$
	\item $\forall x (x > 0 \vee x < 0)$
	\item $\forall (x = 1)$
\end{enumerate}

\item Determine whether $\forall x (P(x) \rightarrow Q(x))$ and $\forall x P(x) \rightarrow
\forall x Q(x)$ are logically equivalent. Justify your answer.

\item Show that $\exists x (P(x) \vee Q(x))$ and $\exists x P(x) \vee \exists x Q(x)$ are
logically equivalent.

\end{enumerate}

\subsection{Nested Quantifiers}

\begin{enumerate}
\item Translate these statements into English, where the domain
for each variable consists of all real numbers.
\begin{enumerate}
	\item $\forall x \exists y (x < y)$
	\item $\forall x \forall y (((x \ge 0) \wedge (y \ge 0)) \rightarrow (x y \ge 0))$
	\item $\forall x \forall y \exists z (x y = z)$
\end{enumerate}
	
\item Show that $\forall x P(x) \wedge \exists x Q(x)$ is logically equivalent
to $\forall x \exists y (P(x) \wedge Q(y))$, where all quantifiers have
the same nonempty domain.

\item Show that $\forall xP(x) \vee \exists x Q(x)$ is equivalent to $\forall x \exists y
(P(x) \vee Q(y))$, where all quantifiers have the same
nonempty domain.
\end{enumerate}

\newpage
%APPENDICES are optional
%\balancecolumns
\appendix
%Appendix A
\section{Answer}
\subsection{Propositional Logic}
\subsubsection{Propositional Logic}
\begin{enumerate}

\item The first clause is true if and only if at least one of $p$, $q$, and
$r$ is true. The second clause is true if and only if at least one of
the three variables is false. Therefore the entire statement is
true if and only if there is at least one T and one F among the
truth values of the variables, in other words, that they don’t all
have the same truth value.
\end{enumerate}

\subsubsection{Application}

\begin{enumerate}
\item \textquotedblleft If I were to ask you whether the right branch
leads to the ruins, would you answer yes? \textquotedblright
\end{enumerate}

\subsubsection{Propositional Equivalences}
\begin{enumerate}
\item {\ }
\begin{table}[htb]
	\centering
	\caption{Truth Table}
	\begin{tabular}{cccc} \toprule
		$p$ & $q$ & $p \wedge q$ & $(p \wedge q) \rightarrow p$\\ \midrule
		T & T & T & T\\ 
		T & F & F & T\\
		F & T & F & T\\
		F & F & F & T\\
		\hline\end{tabular}
\end{table}


\item It is a tautology.

\item If both $q$ and $r$ are false, $(p \vee q)$ and $(\neg p \vee r)$ is opposite, then $(\neg q \wedge \neg r) \rightarrow \neg (p \vee q) \vee \neg (\neg p \vee r)$ is tautology, so $\neg(q \vee r) \rightarrow \neg ((p \vee q) \wedge (\neg p \vee r))$ is tautology, and $(p \vee q) \wedge (\neg p \vee r) \rightarrow (q \vee r)$ is tautology.

\end{enumerate}

\subsubsection{Predicates and Quantifiers}

\begin{enumerate}
\item 
\begin{enumerate}
	\item T
	\item T
	\item F
\end{enumerate}

\item 
\begin{enumerate}
	\item There
	is a student who spends more than 5 hours every weekday
	in class.
	\item Every student spends more than 5 hours every
	weekday in class.
	\item There is a student who does not
	spend more than 5 hours every weekday in class.
	\item No
	student spends more than 5 hours every weekday in class.
\end{enumerate}

\item 
\begin{enumerate}
	\item Every comedian is funny.
	\item Every person is a funny
	comedian.
	\item There exists a person such that if she or he is
	a comedian, then she or he is funny.
	\item Some comedians
	are funny.
\end{enumerate}

\item 
\begin{enumerate}
	\item $\exists x (P(x) \wedge Q(x))$
	\item $\exists x (P(x) \wedge \neg Q(x))$
	\item $\forall x (P(x) \vee Q(x))$
	\item $\forall x \neg (P(x) \vee Q(x)$
\end{enumerate}

\item 
\begin{enumerate}
	\item T
	\item T
	\item F
	\item F
	\item T
	\item F
\end{enumerate}

\item 
\begin{enumerate}
	\item $P(0) \vee P(1)$
	\item $P(0) \wedge P(1)$
	\item $\neg P(0) \vee \neg P(1)$
	\item $\neg P(0) \wedge \neg P(1)$
	\item $\neg(P(0) \vee P(1))$
	\item $\neg(P(0) \wedge P(1))$
\end{enumerate}

\item Many answers are possible.
\begin{enumerate}
	\item All students in CSCI3190; all students in	the world
	\item All members of the Legislative Council; all students in CSCI3190
	\item Tan Sri Dr Runme Shaw and Sir Run Run Shaw; all members of the Legislative Council
	\item Sir Donald Tsang Yam-kuen and Leung Chun-ying; all students in the world
\end{enumerate}

\item 
\begin{enumerate}
	\item There is no counterexample.
	\item $x = 0$
	\item $x = 2$
\end{enumerate}

\item They are not equivalent.
Let $P(x)$ be any propositional function that is sometimes
true and sometimes false, and let $Q(x)$ be any propositional
function that is always false. Then $\forall x (P(x) \rightarrow Q(x))$ is false
but $\forall x P(x) \rightarrow \forall x Q(x)$ is true.

\item Both statements are
true precisely when at least one of $P(x)$ and $Q(x)$ is true for
at least one value of $x$ in the domain.

\end{enumerate}

\subsubsection{Nested Quantifiers}

\begin{enumerate}
\item 
\begin{enumerate}
	\item For every real number $x$ there exists a real number $y$
	such that $x$ is less than $y$.
	\item For every real number $x$ and real
	number $y$, if $x$ and $y$ are both nonnegative, then their product
	is nonnegative.
	\item or every real number $x$ and real number
	$y$, there exists a real number $z$ such that $xy = z$.
\end{enumerate}
	
\item Suppose that $\forall x P(x) \wedge \exists x Q(x)$ is true. Then $P(x)$ is true for all $x$ and there is an element $y$ for which $Q(y)$ is true.
Because $P(x) \wedge Q(y)$ is true for all $x$ and there is a $y$ for which
$Q(y)$ is true, $\forall x \exists y (P(x) \wedge Q(y))$ is true. Conversely, suppose
that the second proposition is true. Let x be an element
in the domain. There is a $y$ such that $Q(y)$ is true, so $\exists x Q(x)$
is true. Because $\forall x P(x)$ is also true, it follows that the first
proposition is true.
	
\item Suppose that $\forall x P(x) \vee \exists x Q(x)$ is
true. Then either $P(x)$ is true for all $x$, or there exists a $y$ for
which $Q(y)$ is true. In the former case, $P(x) \vee Q(y)$ is true
for all x, so $\forall x \exists y (P(x) \vee Q(y))$ is true. In the latter case,
$Q(y)$ is true for a particular $y$, so $P(x) \vee Q(y)$ is true for all
$x$ and consequently $\forall x \exists y(P(x) \vee Q(y))$ is true. Conversely,
suppose that the second proposition is true. If $P(x)$ is true for
all $x$, then the first proposition is true. If not, $P(x)$ is false for
some $x$, and for this $x$ there must be a $y$ such that $P(x) \vee Q(y)$
is true. Hence, $Q(y)$ must be true, so $\exists y Q(y)$ is true. It follows
that the first proposition must hold.
\end{enumerate}


\end{document}
