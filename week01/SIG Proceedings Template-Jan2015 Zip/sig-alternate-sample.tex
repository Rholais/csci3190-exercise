% This is "sig-alternate.tex" V2.1 April 2013
% This file should be compiled with V2.5 of "sig-alternate.cls" May 2012
%
% This example file demonstrates the use of the 'sig-alternate.cls'
% V2.5 LaTeX2e document class file. It is for those submitting
% articles to ACM Conference Proceedings WHO DO NOT WISH TO
% STRICTLY ADHERE TO THE SIGS (PUBS-BOARD-ENDORSED) STYLE.
% The 'sig-alternate.cls' file will produce a similar-looking,
% albeit, 'tighter' paper resulting in, invariably, fewer pages.
%
% ----------------------------------------------------------------------------------------------------------------
% This .tex file (and associated .cls V2.5) produces:
%       1) The Permission Statement
%       2) The Conference (location) Info information
%       3) The Copyright Line with ACM data
%       4) NO page numbers
%
% as against the acm_proc_article-sp.cls file which
% DOES NOT produce 1) thru' 3) above.
%
% Using 'sig-alternate.cls' you have control, however, from within
% the source .tex file, over both the CopyrightYear
% (defaulted to 200X) and the ACM Copyright Data
% (defaulted to X-XXXXX-XX-X/XX/XX).
% e.g.
% \CopyrightYear{2007} will cause 2007 to appear in the copyright line.
% \crdata{0-12345-67-8/90/12} will cause 0-12345-67-8/90/12 to appear in the copyright line.
%
% ---------------------------------------------------------------------------------------------------------------
% This .tex source is an example which *does* use
% the .bib file (from which the .bbl file % is produced).
% REMEMBER HOWEVER: After having produced the .bbl file,
% and prior to final submission, you *NEED* to 'insert'
% your .bbl file into your source .tex file so as to provide
% ONE 'self-contained' source file.
%
% ================= IF YOU HAVE QUESTIONS =======================
% Questions regarding the SIGS styles, SIGS policies and
% procedures, Conferences etc. should be sent to
% Adrienne Griscti (griscti@acm.org)
%
% Technical questions _only_ to
% Gerald Murray (murray@hq.acm.org)
% ===============================================================
%
% For tracking purposes - this is V2.0 - May 2012

\documentclass{sig-alternate-05-2015}
\usepackage{booktabs}


\begin{document}

% Copyright
%\setcopyright{acmcopyright}
%\setcopyright{acmlicensed}
%\setcopyright{rightsretained}
%\setcopyright{usgov}
%\setcopyright{usgovmixed}
%\setcopyright{cagov}
%\setcopyright{cagovmixed}


% DOI
%\doi{10.475/123_4}

% ISBN
%\isbn{123-4567-24-567/08/06}

%Conference
%\conferenceinfo{PLDI '13}{June 16--19, 2013, Seattle, WA, USA}

%\acmPrice{\$15.00}

%
% --- Author Metadata here ---
%\conferenceinfo{WOODSTOCK}{'97 El Paso, Texas USA}
%\CopyrightYear{2007} % Allows default copyright year (20XX) to be over-ridden - IF NEED BE.
%\crdata{0-12345-67-8/90/01}  % Allows default copyright data (0-89791-88-6/97/05) to be over-ridden - IF NEED BE.
% --- End of Author Metadata ---

\title{CSCI 3190 \\ Introduction to Discrete Mathematics and Algorithms}
\subtitle{Extended Exercise 1}

\maketitle
\begin{abstract}

\end{abstract}

\keywords{}

\section{Propositional Logic}
\begin{enumerate}
\item Which of these sentences are propositions? What are the
truth values of those that are propositions?
\begin{enumerate}
	\item $2 + 3 = 5$.
	\item $5 + 7 = 10$.
	\item $x + 2 = 11$.
	\item Answer this question.
\end{enumerate}

\item What is the negation of each of these propositions?
\begin{enumerate}
	\item Sharon has an MP3 player.
	\item There is no pollution in New Jersey.
	\item $2 + 1 = 3$
	\item The summer in HK is hot and sunny.
\end{enumerate}

\item Suppose that during the most recent fiscal year, the annual
revenue of Acme Computer was 138 billion dollars
and its net profit was 8 billion dollars, the annual revenue
of Nadir Software was 87 billion dollars and its net profit
was 5 billion dollars, and the annual revenue of Quixote
Media was 111 billion dollars and its net profit was
13 billion dollars. Determine the truth value of each of
these propositions for the most recent fiscal year.
\begin{enumerate}
	\item Quixote Media had the largest annual revenue.
	\item Nadir Software had the lowest net profit and Acme
	Computer had the largest annual revenue.
	\item Acme Computer had the largest net profit or Quixote
	Media had the largest net profit.
	\item If Quixote Media had the smallest net profit, then
	Acme Computer had the largest annual revenue.
	\item Nadir Software had the smallest net profit if and only
	if Acme Computer had the largest annual revenue.
\end{enumerate}

\item Let p and q be the propositions \textquotedblleft Swimming at HK shore is allowed\textquotedblright\ and \textquotedblleft Sharks have been spotted
near the shore,\textquotedblright\ respectively. Express each of these compound
propositions as an English sentence.
\begin{enumerate}
	\item $\neg q$
	\item $p \wedge q$
	\item $\neg p \vee q$
	\item $p \rightarrow \neg q$
	\item $\neg q \rightarrow p$
	\item $\neg p \rightarrow \neg q$
	\item $p \leftrightarrow \neg q$
	\item $\neg p \wedge (p \vee \neg q)$
\end{enumerate}

\item Let $p$, $q$, and $r$ be the propositions.

$p$ : Grizzly bears have been seen in the area.

$q$ : Hiking is safe on the trail.

$r$ : Berries are ripe along the trail.

Write these propositions using $p$, $q$, and $r$ and logical
connectives (including negations).
\begin{enumerate}
\item Berries are ripe along the trail, but grizzly bears have
not been seen in the area.
\item Grizzly bears have not been seen in the area and hiking
on the trail is safe, but berries are ripe along the
trail.
\item If berries are ripe along the trail, hiking is safe if and
only if grizzly bears have not been seen in the area.
\item It is not safe to hike on the trail, but grizzly bears have
not been seen in the area and the berries along the trail
are ripe.
\item For hiking on the trail to be safe, it is necessary but not
sufficient that berries not be ripe along the trail and
for grizzly bears not to have been seen in the area.
\item Hiking is not safe on the trail whenever grizzly bears
have been seen in the area and berries are ripe along
the trail.
\end{enumerate}

\item Let $p$, $q$, and $r$ be the propositions.

$p$ :You have the flu.

$q$ :You miss the final examination.

$r$ :You pass the course.

Express each of these propositions as an English sentence.

\begin{enumerate}
	\item $p\rightarrow q$
	\item $\neg q \leftrightarrow r$
	\item $q \rightarrow \neg r$
	\item $p \vee q \vee r$
	\item $(p \rightarrow \neg r ) \vee (q \rightarrow \neg r)$
	\item $(p\wedge q)\vee (\neg q \wedge r)$
\end{enumerate}

\item Explain, without using a truth table, why $(p \vee q \vee r) \wedge
(\neg p \vee \neg q \vee \neg r)$ is true when at least one of $p$, $q$, and $r$
is true and at least one is false, but is false when all three
variables have the same truth value.

\item The $n$th statement in a list of 100 statements is \textquotedblleft Exactly
$n$ of the statements in this list are false.\textquotedblright
\begin{enumerate}
	\item What conclusions can you draw from these statements?
	\item Answer part (a) if the nth statement is \textquotedblleft At least n of
	the statements in this list are false.\textquotedblright
	\item Answer part (b) assuming that the list contains 99
	statements.
\end{enumerate}

\end{enumerate}

\section{Application}

\begin{enumerate}
\item Each inhabitant of a remote village always tells the truth
or always lies. A villager will give only a \textquotedblleft Yes\textquotedblright or a \textquotedblleft No\textquotedblright
response to a question a tourist asks. Suppose you are a
tourist visiting this area and come to a fork in the road.
One branch leads to the ruins you want to visit; the other
branch leads deep into the jungle. A villager is standing
at the fork in the road. What one question can you ask the
villager to determine which branch to take?

\item To use the wireless network in the airport you must pay
the daily fee unless you are a subscriber to the service.
Express your answer in terms of $w$: \textquotedblleft You can use the wireless
network in the airport,\textquotedblright\ $d$: \textquotedblleft You pay the daily fee,\textquotedblright\ 
and $s$: \textquotedblleft You are a subscriber to the service.\textquotedblright
\newline\newline
Exercises $3 - 7$ relate to inhabitants of the island of knights
and knaves created by Smullyan, where knights always tell
the truth and knaves always lie. You encounter two people,
A and B. Determine, if possible, what A and B are if they
address you in the ways described. If you cannot determine
what these two people are, can you draw any conclusions?

\item A says \textquotedblleft At least one of us is a knave\textquotedblright\ and B says nothing.
\item A says \textquotedblleft The two of us are both knights\textquotedblright\ and B says \textquotedblleft A
is a knave.\textquotedblright
\item A says \textquotedblleft I am a knave or B is a knight\textquotedblright\ and B says nothing.
\item Both A and B say \textquotedblleft I am a knight.\textquotedblright 
\item A says \textquotedblleft We are both knaves\textquotedblright\ and B says nothing.

\end{enumerate}

\section{Propositional Equivalences}

\begin{enumerate}
\item Show that $(p \wedge q) \rightarrow p$ is a tautology
by using truth tables.

\item Determine whether $(\neg q \wedge (p \rightarrow q)) \rightarrow \neg p$ is a tautology.

\item Show that $(p \vee q) \wedge (\neg p \vee r) \rightarrow (q \vee r)$ is a tautology.

\item Show that $(\neg(p \vee F) \wedge (\neg q \wedge T)) \vee p$ and $q\rightarrow p$ are equivalent.

\item Show that $((\neg p) \rightarrow (r\vee q)) \equiv ((\neg r)\rightarrow ((\neg p)\rightarrow q))$.

\item Show that $(p \rightarrow (r \rightarrow q))\equiv (p \rightarrow q)\vee \neg r$.

\item Show that $(p \rightarrow q) \vee (p \rightarrow r) \equiv p \rightarrow (q\vee r)$.

\item Show that $(s \rightarrow r) \wedge (q \rightarrow r) \equiv s\wedge q \rightarrow  r$.

\item Show that $\neg p \rightarrow (q \rightarrow \neg r)\equiv r \rightarrow (q \rightarrow p)$.

\item Show that $(\neg p \vee q) \wedge (r \vee \neg q) \wedge (p \vee q) \equiv q\wedge r$.

\item Show that $(\neg p \wedge (p \vee q))\rightarrow q$ is a tautology.

\item Show that $((p\rightarrow q)\wedge (q\rightarrow r)) \rightarrow (p\rightarrow r)$ is a  tautology.

\item Show that $(p\wedge (p \rightarrow q))\rightarrow q$ is a tautology.

\item Show that $((p\vee q)\wedge (p \rightarrow r) \wedge (q\rightarrow r))\rightarrow r$ is a  tautology.
\end{enumerate}

\section{Predicates and Quantifiers}

\begin{enumerate}
\item Let $P(x)$ denote the statement \textquotedblleft $x \le 4$.\textquotedblright What are these
truth values?
\begin{enumerate}
	\item $P(0)$
	\item $P(4)$
	\item $P(6)$
\end{enumerate}

\item Let $P(x)$ be the statement \textquotedblleft $x$ spends more than five hours
every weekday in class,\textquotedblright where the domain for $x$ consists
of all students. Express each of these quantifications in
English.
\begin{enumerate}
	\item $\exists x P(x)$
	\item $\forall x P(x)$
	\item $\exists x \neg P(x)$
	\item $\forall x \neg P(x)$
\end{enumerate}

\item Translate these statements into English, where $C(x)$ is \textquotedblleft $x$
is a comedian\textquotedblright and $F(x)$ is \textquotedblleft $x$ is funny\textquotedblright and the domain
consists of all people.
\begin{enumerate}
	\item $\forall x (C(x) \rightarrow F(x))$
	\item $\forall x (C(x) \wedge F(x))$
	\item $\exists x (C(x) \rightarrow F(x))$
	\item $\exists x (C(x) \wedge F(x))$
\end{enumerate}

\item Let $P(x)$ be the statement \textquotedblleft $x$ can speak Cantonese\textquotedblright and let
$Q(x)$ be the statement \textquotedblleft $x$ knows the computer language
C++.\textquotedblright Express each of these sentences in terms of $P(x)$,
$Q(x)$, quantifiers, and logical connectives. The domain
for quantifiers consists of all students at CU.
\begin{enumerate}
	\item There is a student at CU who can speak Cantonese
	and who knows C++.
	\item There is a student at CU who can speak Cantonese
	but who doesn\textquoteright t know C++.
	\item Every student at CU either can speak Cantonese
	or knows C++.
	\item No student at CU can speak Cantonese or knows
	C++.
\end{enumerate}

\item Let $P(x)$ be the statement \textquotedblleft $x = x^2$.\textquotedblright If the domain consists
of the integers, what are these truth values?
\begin{enumerate}
	\item $P(0)$
	\item $P(1)$
	\item $P(2)$
	\item $P(-1)$
	\item $\exists x P(x)$
	\item $\forall x P(x)$
\end{enumerate}

\item Suppose that the domain of the propositional function
$P(x)$ consists of the integers 0 and 1. Write out
each of these propositions using disjunctions, conjunctions,
and negations.
\begin{enumerate}
	\item $\exists x P(x)$
	\item $\forall x P(x)$
	\item $\exists x \neg P(x)$
	\item $\forall x \neg P(x)$
	\item $\neg \exists x P(x)$
	\item $\neg \forall x P(x)$
\end{enumerate}

\item For each of these statements find a domain for which the
statement is true and a domain for which the statement is
false.
\begin{enumerate}
	\item Everyone is studying discrete mathematics.
	\item Everyone is older than 21 years.
	\item Every two people have the same mother.
	\item No two different people have the same grandmother.
\end{enumerate}

\item Find a counterexample, if possible, to these universally
quantified statements, where the domain for all variables
consists of all integers.
\begin{enumerate}
	\item $\forall x (x^2 \ge x)$
	\item $\forall x (x > 0 \vee x < 0)$
	\item $\forall x (x = 1)$
\end{enumerate}

\item Determine whether $\forall x (P(x) \rightarrow Q(x))$ and $\forall x P(x) \rightarrow
\forall x Q(x)$ are logically equivalent. Justify your answer.

\item Show that $\exists x (P(x) \vee Q(x))$ and $\exists x P(x) \vee \exists x Q(x)$ are
logically equivalent.

\end{enumerate}

\section{Nested Quantifiers}

\begin{enumerate}
\item Translate these statements into English, where the domain
for each variable consists of all real numbers.
\begin{enumerate}
	\item $\forall x \exists y (x < y)$
	\item $\forall x \forall y (((x \ge 0) \wedge (y \ge 0)) \rightarrow (x y \ge 0))$
	\item $\forall x \forall y \exists z (x y = z)$
\end{enumerate}

\item Let $Q(x, y)$ be the statement \textquotedblleft $x$ has sent an e-mail message
to $y$,\textquotedblright where the domain for both $x$ and $y$ consists of
all students in your class. Express each of these quantifications in English.
\begin{enumerate}
	\item $\exists x \exists y Q(x, y)$
	\item $\exists x \forall y Q(x, y)$
	\item $\forall x \exists y Q(x, y)$
	\item $\forall x \forall y Q(x, y)$
\end{enumerate}

\item Let $S(x)$ be the predicate \textquotedblleft $x$ is a student,\textquotedblright $F(x)$ the predicate
\textquotedblleft $x$ is a faculty member,\textquotedblright and $A(x, y)$ the predicate
\textquotedblleft $x$ has asked $y$ a question,\textquotedblright where the domain consists of
all people associated with CU. Use quantifiers to
express each of these statements.
\begin{enumerate}
	\item Lois has asked Professor Michaels a question.
	\item Every student has asked Professor Gross a question
	\item Every faculty member has either asked Professor
	Miller a question or been asked a question by Professor
	Miller
	\item Some student has not asked any faculty member a
	question.
	\item There is a faculty member who has never been asked
	a question by a student.
	\item Some student has asked every faculty member a question.
	\item There is a faculty member who has asked every other
	faculty member a question.
	\item Some student has never been asked a question by a
	faculty member.
\end{enumerate}
	
\item Show that $\forall x P(x) \wedge \exists x Q(x)$ is logically equivalent
to $\forall x \exists y (P(x) \wedge Q(y))$, where all quantifiers have
the same nonempty domain.

\item Show that $\forall xP(x) \vee \exists x Q(x)$ is equivalent to $\forall x \exists y
(P(x) \vee Q(y))$, where all quantifiers have the same
nonempty domain.
\end{enumerate}

\nocite{*}
\bibliographystyle{abbrv}
\bibliography{ref}  % sigproc.bib is the name of the Bibliography in this case
 
\newpage
%APPENDICES are optional
%\balancecolumns
\appendix
%Appendix A
\section{Answer}
\subsection{Propositional Logic}
\begin{enumerate}
\item 
\begin{enumerate}
	\item Yes, T
	\item Yes, F
	\item No
	\item No
\end{enumerate}

\item 
\begin{enumerate}
	\item Mei does not have an MP3 player.
	\item There is pollution
	in New Jersey.
	\item $2 + 1 \ne 3$
	\item The summer in HK is not
	hot or it is not sunny.
\end{enumerate}

\item 
\begin{enumerate}
	\item F
	\item T
	\item T
	\item T
	\item T
\end{enumerate}

\item 
\begin{enumerate}
	\item Sharks have
	not been spotted near the shore
	\item Swimming at HK shore is allowed, and sharks have been spotted near the
	shore.
	\item Swimming at HK shore is not allowed,
	or sharks have been spotted near the shore.
	\item If swimming
	at HK shore is allowed, then sharks have not been
	spotted near the shore.
	\item If sharks have not been spotted near
	the shore, then swimming at HK shore is allowed.
	\item If swimming at HK shore is not allowed, then
	sharks have not been spotted near the shore.
	\item Swimming at HK shore is allowed if and only if sharks have
	not been spotted near the shore.
	\item Swimming at HK shore is not allowed, and either swimming at HK shore is allowed or sharks have not been spotted near
	the shore.
\end{enumerate}

\item
\begin{enumerate}
\item $r \wedge \neg  p$
\item $\neg p \wedge q\wedge r$
\item $r \rightarrow (q \leftrightarrow \neg p) $
\item $\neg q \wedge \neg p \wedge r$
\item $(q \rightarrow (\neg r\wedge \neg p))\wedge \neg((\neg r \wedge \neg p) \rightarrow q)$
\item $(p \wedge r) \rightarrow \neg q$
\end{enumerate}
\item
\begin{enumerate}
	\item If you have the flu, then you will miss the final exam.

	\item Not missing the final exam is necessary and sufficient for passing the course. OR: You
	will pass the course if and only if you don't miss the final exam
	\item If you miss the final, you will not pass the course.
	\item You have the flu, or you miss the final, or you pass the course.
	\item If you have the flu, then you will not pass the course, or, if you miss the final, you will
	not pass the course.
	\item You have the flu and miss the final exam, or you don't miss the final exam and pass
	the course.
\end{enumerate}

\item The first clause is true if and only if at least one of $p$, $q$, and
$r$ is true. The second clause is true if and only if at least one of
the three variables is false. Therefore the entire statement is
true if and only if there is at least one T and one F among the
truth values of the variables, in other words, that they don\textquoteright t all
have the same truth value.

\item 
\begin{enumerate}
	\item The
	99th statement is true and the rest are false.
	\item Statements
	1 through 50 are all true and statements 51 through 100 are
	all false.
	\item This cannot happen; it is a paradox, showing that
	these cannot be statements.
\end{enumerate}

\end{enumerate}

\subsection{Application}

\begin{enumerate}
\item \textquotedblleft If I were to ask you whether the right branch
leads to the ruins, would you answer yes? \textquotedblright
\item $w \rightarrow (d \vee s)$
\item A is a knight and B is a knave
\item A is a knave and B is a knight.
\item A is a knight and B is a knight.
\item We don\textquoteright t know anything about A and B. If A (or B) is a knight, then A (or B) would
truthfully say, \textquotedblleft I am a knight\textquotedblright. Likewise, if A (or B) is a knave, they would untruthfully
say, \textquotedblleft I am a knight\textquotedblright.
\item A is a knave and B is a knight.
\end{enumerate}

\subsection{Propositional Equivalences}
\begin{enumerate}
\item {\ }
\begin{table}[htb]
	\centering
	\caption{Truth Table}
	\begin{tabular}{cccc} \toprule
		$p$ & $q$ & $p \wedge q$ & $(p \wedge q) \rightarrow p$\\ \midrule
		T & T & T & T\\ 
		T & F & F & T\\
		F & T & F & T\\
		F & F & F & T\\
		\hline\end{tabular}
\end{table}


\item It is a tautology.

$(\neg q \wedge (p \rightarrow q)) \rightarrow \neg p$

$\equiv \neg (\neg q \wedge (p \rightarrow q)) \vee \neg p$

$\equiv q \vee \neg (\neg p \vee q)) \vee \neg p$

$\equiv q \vee ( p \wedge \neg q) \vee \neg p$

$\equiv ((q \vee p) \wedge (q \vee \neg q) ) \vee \neg p$

$\equiv ((q \vee p) \wedge T ) \vee \neg p$

$\equiv q \vee p \vee \neg p$
$\equiv T$

\item If both $q$ and $r$ are false, $(p \vee q)$ and $(\neg p \vee r)$ is opposite, then $(\neg q \wedge \neg r) \rightarrow \neg (p \vee q) \vee \neg (\neg p \vee r)$ is tautology, so $\neg(q \vee r) \rightarrow \neg ((p \vee q) \wedge (\neg p \vee r))$ is tautology, and $(p \vee q) \wedge (\neg p \vee r) \rightarrow (q \vee r)$ is tautology.

Or:

$(p\vee q) \wedge (\neg p \vee r)\rightarrow (q \vee r)$

$\equiv \neg (p\vee q) \vee \neg (\neg p \vee r)\vee  (q \vee r)$

$\equiv (\neg p\wedge \neg q) \vee (p \wedge \neg r)\vee  (q \vee r)$

$\equiv (\neg p\wedge \neg q) \vee q \vee (p \wedge \neg r) \vee r$

$\equiv ((\neg p\vee q)\wedge (\neg q\vee q) ) \vee (p \wedge \neg r) \vee r$

$\equiv ((\neg p\vee q)\wedge T) \vee ((p \vee r) \wedge (\neg r \vee r)) $

$\equiv (\neg p\vee q) \vee ((p \vee r) \wedge T) $

$\equiv \neg p\vee q \vee p \vee r $

$\equiv (\neg p \vee p)\vee q \vee r $
$\equiv T $
\item  $(\neg(p \vee F) \wedge (\neg q \wedge T)) \vee p$

$\equiv (\neg p \wedge \neg q)\vee p$

$\equiv (\neg p\vee p) \wedge (\neg q \vee p)$

 $\equiv T \wedge (\neg q \vee p)$

 $\equiv \neg q \vee p$
  
  $\equiv  q \rightarrow p$
 
 \item $ (\neg p) \rightarrow (r\vee q)$
 
 $\equiv  \neg(\neg p) \vee (r\vee q)$

$\equiv  p \vee (r\vee q)$

$\equiv  r \vee (p\vee q)$

$\equiv  (\neg(\neg r)) \vee (p\vee q)$

$\equiv  (\neg r)\rightarrow (p\vee q)$

$\equiv  (\neg r)\rightarrow ((\neg(\neg p))\vee q)$

$\equiv  (\neg r)\rightarrow ((\neg p)\rightarrow q)$

\item $p \rightarrow (r \rightarrow q)$

$\equiv \neg p \vee  (r \rightarrow q)$

$\equiv \neg p \vee  (\neg r \vee q)$

$\equiv (\neg p \vee   q) \vee \neg r$

$\equiv ( p \rightarrow q) \vee \neg r$

\item  $(p \rightarrow q) \vee (p \rightarrow r)$

$\equiv (\neg p \vee q) \vee (\neg p \vee r)$

$\equiv (\neg p \vee \neg p) \vee ( q \vee r)$

$\equiv  p \rightarrow ( q \vee r)$

\item LHS $\equiv (\neg s \vee r) \wedge (\neg q \vee r)$

$\equiv(\neg s \wedge \neg q) \vee r $

$\equiv\neg ( s \wedge q) \vee r $

$(s \wedge q)\rightarrow r \equiv$ RHS 

\item  $\neg p \rightarrow (q \rightarrow \neg r)$

$\equiv \neg \neg p \vee (\neg q \vee \neg r)$

$\equiv  p \vee (\neg q \vee \neg r)$

$\equiv  \neg r \vee (\neg q \vee p)$

$\equiv r \rightarrow (q \rightarrow p)$

\item $(\neg p \vee q) \wedge (r \vee \neg q) \wedge (p \vee q)$

$ \equiv (q \vee \neg p)\wedge (q \vee p) \wedge (r \vee \neg q) $

$\equiv(q\vee (p \wedge \neg p)) \wedge (r \vee \neg q)$

$\equiv(q\vee F)\wedge (r \vee \neg q)$

$\equiv q \wedge (r \vee \neg q)$

$\equiv (q \wedge r )\vee( q \wedge \neg q)$

$\equiv (q\wedge r)\vee F$
$\equiv q\wedge r$

\item $(\neg p \wedge (p \vee q))\rightarrow q$

$\equiv \neg (\neg p \wedge (p \vee q))\vee q$

$\equiv  ( p \vee (\neg p \wedge \neg q))\vee q$

$\equiv  ( (p \vee \neg p) \wedge (p \vee \neg q))\vee q$

$\equiv   p \vee \neg q \vee q$ $\equiv  T$
\item $((p\rightarrow q)\wedge (q\rightarrow r)) \rightarrow (p\rightarrow r)$

$\equiv \neg ((\neg p\vee q)\wedge (\neg q\vee r)) \vee(\neg p\vee r)$

$\equiv (( p\wedge \neg q)\vee (q\wedge \neg r)) \vee(\neg p\vee r)$

$\equiv (( p\vee \neg p)\wedge (\neg q \vee \neg p))\vee (q\wedge \neg r)) \vee r$

$\equiv  (\neg q \vee \neg p)\vee ((q\vee r)\wedge (\neg r\vee r))$

$\equiv  \neg q \vee \neg p\vee q\vee r$ $\equiv T$
\item $(p\wedge (p \rightarrow q))\rightarrow q$

$\equiv \neg (p\wedge (\neg p \vee q))\vee q$

$\equiv \neg p\vee ( p \wedge \neg q)\vee q$

$\equiv ((\neg p\vee p) \wedge (\neg p \vee \neg q)\vee q$

$\equiv \neg p \vee \neg q\vee q$ $\equiv T$

\item $((p\vee q)\wedge (p \rightarrow r) \wedge (q\rightarrow r))\rightarrow r$

$\equiv \neg((p\vee q)\wedge (\neg p \vee r) \wedge (\neg q\vee r))\vee r$

$\equiv (\neg p\wedge \neg q)\vee (p \wedge \neg r) \vee (q\wedge \neg r)\vee r$

$\equiv (\neg p\wedge \neg q)\vee (p \wedge \neg r) \vee ((q\vee r) \wedge (\neg r\vee r))$

$\equiv (\neg p\wedge \neg q)\vee (p \wedge \neg r) \vee q\vee r$

$\equiv ((\neg p\vee q )\wedge (\neg q\vee q)) \vee ((p \vee r)\wedge (\neg r\vee r))$

$\equiv \neg p\vee q  \vee p \vee r$ $\equiv T$
\end{enumerate}

\subsection{Predicates and Quantifiers}

\begin{enumerate}
\item 
\begin{enumerate}
	\item T
	\item T
	\item F
\end{enumerate}

\item 
\begin{enumerate}
	\item There
	is a student who spends more than 5 hours every weekday
	in class.
	\item Every student spends more than 5 hours every
	weekday in class.
	\item There is a student who does not
	spend more than 5 hours every weekday in class.
	\item No
	student spends more than 5 hours every weekday in class.
\end{enumerate}

\item 
\begin{enumerate}
	\item Every comedian is funny.
	\item Every person is a funny
	comedian.
	\item There exists a person such that if she or he is
	a comedian, then she or he is funny.
	\item Some comedians
	are funny.
\end{enumerate}

\item 
\begin{enumerate}
	\item $\exists x (P(x) \wedge Q(x))$
	\item $\exists x (P(x) \wedge \neg Q(x))$
	\item $\forall x (P(x) \vee Q(x))$
	\item $\forall x \neg (P(x) \vee Q(x)$
\end{enumerate}

\item 
\begin{enumerate}
	\item T
	\item T
	\item F
	\item F
	\item T
	\item F
\end{enumerate}

\item 
\begin{enumerate}
	\item $P(0) \vee P(1)$
	\item $P(0) \wedge P(1)$
	\item $\neg P(0) \vee \neg P(1)$
	\item $\neg P(0) \wedge \neg P(1)$
	\item $\neg(P(0) \vee P(1))$
	\item $\neg(P(0) \wedge P(1))$
\end{enumerate}

\item Many answers are possible.
\begin{enumerate}
	\item All students in CSCI3190; all students in	the world
	\item All members of the Legislative Council; all students in CSCI3190
	\item Tan Sri Dr Runme Shaw and Sir Run Run Shaw; all members of the Legislative Council
	\item Sir Donald Tsang Yam-kuen and Leung Chun-ying; all students in the world
\end{enumerate}

\item 
\begin{enumerate}
	\item There is no counterexample.
	\item $x = 0$
	\item $x = 2$
\end{enumerate}

\item They are not equivalent.
Let $P(x)$ be any propositional function that is sometimes
true and sometimes false, and let $Q(x)$ be any propositional
function that is always false. Then $\forall x (P(x) \rightarrow Q(x))$ is false
but $\forall x P(x) \rightarrow \forall x Q(x)$ is true.

\item Both statements are
true precisely when at least one of $P(x)$ and $Q(x)$ is true for
at least one value of $x$ in the domain.

\end{enumerate}

\subsection{Nested Quantifiers}

\begin{enumerate}
\item 
\begin{enumerate}
	\item For every real number $x$ there exists a real number $y$
	such that $x$ is less than $y$.
	\item For every real number $x$ and real
	number $y$, if $x$ and $y$ are both nonnegative, then their product
	is nonnegative.
	\item or every real number $x$ and real number
	$y$, there exists a real number $z$ such that $xy = z$.
\end{enumerate}

\item 
\begin{enumerate}
	\item There is some student in your class who has sent a message to
	some student in your class.
	\item There is some student in your
	class who has sent a message to every student in your class.
	\item Every student in your class has sent a message to at least
	one student in your class.
	\item Every student in the class
	has sent a message to every student in the class.
\end{enumerate}

\item 
\begin{enumerate}
	\item $A(Lois, Prof. Michaels)$
	\item $\forall x (S(x) \rightarrow A(x, Prof. Gross))$
	\item $\forall x (F(x) \rightarrow (A(x, Prof. Miller) \vee A(Prof. Miller, x)))$
	\item $\exists x (S(x) \wedge \forall y (F(y) \rightarrow \neg A(x, y)))$
	\item $\exists x (F(x) \wedge \forall y (S(y) \rightarrow \neg A(y, x)))$
	\item $\forall y (F(y) \rightarrow \exists x (S(x) \vee A(x, y)))$
	\item $\exists x (F(x) \wedge \forall y ((f(y) \wedge (y \ne x)) \rightarrow A(x, y)))$
	\item $\exists x (S(x) \wedge \forall y (F(y) \rightarrow \neg A(y, x)))$
\end{enumerate}
	
\item Suppose that $\forall x P(x) \wedge \exists x Q(x)$ is true. Then $P(x)$ is true for all $x$ and there is an element $y$ for which $Q(y)$ is true.
Because $P(x) \wedge Q(y)$ is true for all $x$ and there is a $y$ for which
$Q(y)$ is true, $\forall x \exists y (P(x) \wedge Q(y))$ is true. Conversely, suppose
that the second proposition is true. Let x be an element
in the domain. There is a $y$ such that $Q(y)$ is true, so $\exists x Q(x)$
is true. Because $\forall x P(x)$ is also true, it follows that the first
proposition is true.
	
\item Suppose that $\forall x P(x) \vee \exists x Q(x)$ is
true. Then either $P(x)$ is true for all $x$, or there exists a $y$ for
which $Q(y)$ is true. In the former case, $P(x) \vee Q(y)$ is true
for all x, so $\forall x \exists y (P(x) \vee Q(y))$ is true. In the latter case,
$Q(y)$ is true for a particular $y$, so $P(x) \vee Q(y)$ is true for all
$x$ and consequently $\forall x \exists y(P(x) \vee Q(y))$ is true. Conversely,
suppose that the second proposition is true. If $P(x)$ is true for
all $x$, then the first proposition is true. If not, $P(x)$ is false for
some $x$, and for this $x$ there must be a $y$ such that $P(x) \vee Q(y)$
is true. Hence, $Q(y)$ must be true, so $\exists y Q(y)$ is true. It follows
that the first proposition must hold.
\end{enumerate}

\end{document}