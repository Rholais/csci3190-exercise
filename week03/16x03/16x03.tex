% This is "sig-alternate.tex" V2.1 April 2013
% This file should be compiled with V2.5 of "sig-alternate.cls" May 2012
%
% This example file demonstrates the use of the 'sig-alternate.cls'
% V2.5 LaTeX2e document class file. It is for those submitting
% articles to ACM Conference Proceedings WHO DO NOT WISH TO
% STRICTLY ADHERE TO THE SIGS (PUBS-BOARD-ENDORSED) STYLE.
% The 'sig-alternate.cls' file will produce a similar-looking,
% albeit, 'tighter' paper resulting in, invariably, fewer pages.
%
% ----------------------------------------------------------------------------------------------------------------
% This .tex file (and associated .cls V2.5) produces:
%       1) The Permission Statement
%       2) The Conference (location) Info information
%       3) The Copyright Line with ACM data
%       4) NO page numbers
%
% as against the acm_proc_article-sp.cls file which
% DOES NOT produce 1) thru' 3) above.
%
% Using 'sig-alternate.cls' you have control, however, from within
% the source .tex file, over both the CopyrightYear
% (defaulted to 200X) and the ACM Copyright Data
% (defaulted to X-XXXXX-XX-X/XX/XX).
% e.g.
% \CopyrightYear{2007} will cause 2007 to appear in the copyright line.
% \crdata{0-12345-67-8/90/12} will cause 0-12345-67-8/90/12 to appear in the copyright line.
%
% ---------------------------------------------------------------------------------------------------------------
% This .tex source is an example which *does* use
% the .bib file (from which the .bbl file % is produced).
% REMEMBER HOWEVER: After having produced the .bbl file,
% and prior to final submission, you *NEED* to 'insert'
% your .bbl file into your source .tex file so as to provide
% ONE 'self-contained' source file.
%
% ================= IF YOU HAVE QUESTIONS =======================
% Questions regarding the SIGS styles, SIGS policies and
% procedures, Conferences etc. should be sent to
% Adrienne Griscti (griscti@acm.org)
%
% Technical questions _only_ to
% Gerald Murray (murray@hq.acm.org)
% ===============================================================
%
% For tracking purposes - this is V2.0 - May 2012

\documentclass{sig-alternate-05-2015}
\usepackage{booktabs}


\begin{document}

% Copyright
%\setcopyright{acmcopyright}
%\setcopyright{acmlicensed}
%\setcopyright{rightsretained}
%\setcopyright{usgov}
%\setcopyright{usgovmixed}
%\setcopyright{cagov}
%\setcopyright{cagovmixed}


% DOI
%\doi{10.475/123_4}

% ISBN
%\isbn{123-4567-24-567/08/06}

%Conference
%\conferenceinfo{PLDI '13}{June 16--19, 2013, Seattle, WA, USA}

%\acmPrice{\$15.00}

%
% --- Author Metadata here ---
%\conferenceinfo{WOODSTOCK}{'97 El Paso, Texas USA}
%\CopyrightYear{2007} % Allows default copyright year (20XX) to be over-ridden - IF NEED BE.
%\crdata{0-12345-67-8/90/01}  % Allows default copyright data (0-89791-88-6/97/05) to be over-ridden - IF NEED BE.
% --- End of Author Metadata ---

%\\TODO:1.tautology prove section 2. set section operation proof 3. function section basic proving onto bijection...
\title{CSCI 3190 \\ Introduction to Discrete Mathematics and Algorithms}
\subtitle{Extended Exercise 3}

\maketitle
\begin{abstract}

\end{abstract}

\keywords{}

\section{Relations}
\subsection{Relations and Their Properties}
Let $R$ be a relation from a set $A$ to a set $B$. The \textbf{inverse relation}
from $B$ to $A$, denoted by $R^{−1}$, is the set of ordered pairs
$\{(b, a) \mid (a, b) \in R\}$. The \textbf{complementary relation} $\overline{R}$ is the
set of ordered pairs $\{(a, b) \mid (a, b) \notin R\}$.

\begin{enumerate}
\item Let $R$ be the relation $R = \{(a, b) \mid a < b\}$ on the set of
integers. Find
\begin{enumerate}
	\item $R^{-1}$.
	\item $\overline{R}$.
\end{enumerate}
\end{enumerate}

\subsection{$n$-ary Relations and Their Applications}
\subsection{Representing Relations}
\subsection{Closures of Relations}
\subsection{Equivalence Relations}
\begin{enumerate}
\item Each bead on a bracelet with three beads is either red,
white, or blue, as illustrated in the figure shown.
Define the relation $R$ between bracelets as: $(B_1,B_2)$,
where $B_1$ and $B_2$ are bracelets, belongs to $R$ if and only
if $B_2$ can be obtained from $B_1$ by rotating it or rotating it
and then reflecting it.
\begin{enumerate}
	\item Show that $R$ is an equivalence relation.
	\item What are the equivalence classes of $R$?
\end{enumerate}
\end{enumerate}

\nocite{*}
\bibliographystyle{abbrv}
\bibliography{ref}  % sigproc.bib is the name of the Bibliography in this case
 
\newpage
%APPENDICES are optional
%\balancecolumns
\appendix
%Appendix A
\section{Answer}
\subsection{Relations}
\subsubsection{Relations and Their Properties}
\begin{enumerate}
\item 
\begin{enumerate}
	\item $R = \{(a, b) \mid a > b\}$.
	\item $R = \{(a, b) \mid a \ge b\}$.
\end{enumerate}
\end{enumerate}

\subsubsection{$n$-ary Relations and Their Applications}
\subsubsection{Representing Relations}
\subsubsection{Closures of Relations}
\subsubsection{Equivalence Relations}
\begin{enumerate}
\item 
\begin{enumerate}
	\item $R$ is
	reflexive because any coloring can be obtained from itself via
	a 360-degree rotation. To see that $R$ is symmetric and transitive,
	use the fact that each rotation is the composition of two reflections and conversely the composition of two reflections
	is a rotation. Hence, $(B_1, B_2)$ belongs to $R$ if and only
	if $B_2$ can be obtained from $B_1$ by a composition of reflections.
	So if $(B_1, B_2)$ belongs to $R$, so does $(B_2, B_1)$ because
	the inverse of the composition of reflections is also a composition
	of reflections (in the opposite order). Hence, $R$ is
	symmetric. To see that $R$ is transitive, suppose $(B_1, B_2)$ and
	$(B_2, B_3)$ belong to $R$. Taking the composition of the reflections
	in each case yields a composition of reflections, showing
	that $(B_1,B_3)$ belongs to $R$. 
	\item We express colorings with sequences
	of length four, with $r$, $w$ and $b$ denoting red, white and blue,
	respectively. We list letters denoting the colors of the upper
	ball, right ball, and left ball, in that order. The equivalence classes are: $\{rrr\}$, $\{www\}$, $\{bbb\}$, $\{rrw, rwr, wrr\}$, $\{rrb, rbr, brr\}$, $\{wwr, wrw, rww\}$, $\{wwb, wbw, bww\}$, $\{bbr, brb, rbb\}$, $\{bbw, bwb, wbb\}$.
\end{enumerate}
\end{enumerate}

\end{document}
