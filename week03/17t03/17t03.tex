%!TEX program = xelatex
\documentclass[10pt, compress, handout]{beamer}
\usepackage[titleprogressbar]{../../cls/beamerthemem}

\usepackage{booktabs}
\usepackage[scale=2]{ccicons}
\usepackage{minted}

\usepgfplotslibrary{dateplot}

\usemintedstyle{trac}

\setbeamertemplate{caption}[numbered]
\setbeamertemplate{theorems}[numbered]
\newtheorem{crl}{Corollary}[theorem]
\newtheorem*{solution*}{Solution}

\usepackage{algorithm}
\usepackage[noend]{algpseudocode}

\usepackage{version}
%\excludeversion{proof}
%\excludeversion{solution*}

\usepackage{mathtools}
\usepackage{multicol}
\usepackage{qtree}

\usepackage{tikz}

\makeatletter
\def\old@comma{,}
\catcode`\,=13
\def,{%
	\ifmmode%
	\old@comma\discretionary{}{}{}%
	\else%
	\old@comma%
	\fi%
}
\makeatother

\title{CSCI 3190 Tutorial of Week 03}
\subtitle{Relations}
\author{LI Haocheng}
\institute{Department of Computer Science and Engineering}

\begin{document}

\maketitle

\begin{frame}[fragile]
	\frametitle{Division}
	\begin{columns}
		\onslide<1->\begin{column}{.5\linewidth}
			\begin{example}
				Let $R$ be the relation $R = \{(a, b) \mid a \text{ divides } b\}$ on the set of integers. Find \begin{enumerate}
					\onslide<1->\item $R^{-1}$ \onslide<2->$= \{(a, b) \mid b \text{ divides } a\}$
					\onslide<1->\item $\bar{R}$ \onslide<2->$= \{(a, b) \mid a \text{ doesn't divide } b\}$
				\end{enumerate}
			\end{example}
		\end{column}
		\onslide<1->\begin{column}{.5\linewidth}
			\begin{example}
				Let $R$ be the relation on the set of all districts in Hong Kong consisting of pairs $(a, b)$ where district $a$ borders district $b$. Find \begin{enumerate}
					\onslide<1->\item $R^{-1}$ \onslide<3>$= \{(a, b) \mid a \text{ borders } b\}$
					\onslide<1->\item $\bar{R}$ \onslide<3>$= \{(a, b) \mid a \text{ doesn't border } b\}$
				\end{enumerate}
			\end{example}
		\end{column}
	\end{columns}
\end{frame}

\begin{frame}[fragile]
	\frametitle{One-to-One Correspondence}
	\begin{example}
		Suppose that the function $f$ from $A$ to $B$ is a one-to-one correspondence. Let $R$ be the relation that equals the
		graph of $f$. That is, $R = \{(a, f(a)) \mid a \in A\}$. What is the inverse relation $R^{-1}$?
	\end{example}
	\onslide<2>\begin{solution*}
		The graph of $f^{-1}$.
	\end{solution*}
\end{frame}

\begin{frame}[fragile]
	\frametitle{Closures}
	\begin{columns}
		\begin{column}{.4\linewidth}
			\begin{definition}
				A \textbf{reflexive closure} of $R$\begin{enumerate}
					\item contains $R$,
					\item is reflexive,
					\item is contained within every reflexive relation that contains $R$.
				\end{enumerate}
			\end{definition}
			\begin{definition}
				A \textbf{diagonal relation} $\Delta$ on $A$ is $\{(a, a) \mid a \in A \}$.
			\end{definition}
		\end{column}
		\begin{column}{.6\linewidth}
			\begin{theorem}
				Given a relation $R$ on a set $A$, the reflexive closure of $R$ equals $R \cup \Delta$.
			\end{theorem}
			\onslide<2>\begin{proof}
				\begin{enumerate}
					\item $R \cup \Delta \supseteq R$.
					\item $R \cup \Delta \supseteq \Delta$.
					\item If there exists a reflexive set $S \supset \Delta$, $r \in R \cup \Delta$, $r \notin S$. Therefore $r \notin \Delta$, $r \in R$ so that $S \nsupseteq R$ which forms a contradictory.
				\end{enumerate}
			\end{proof}
		\end{column}
	\end{columns}
\end{frame}

\begin{frame}[fragile]
	\frametitle{Reflexive Closure}
	\onslide<1->\begin{example}
		What is the reflexive closure of the relation $R = \{(a, b) \mid a < b\}$ on the set of integers?
	\end{example}
	\onslide<2>\begin{solution*}
		\begin{align}
			\begin{aligned}
			& R \cup \Delta \\
			= & \{(a, b) \mid a < b \} \cup \{(a, a) \mid a \in \mathbb{Z}\} \\
			= & \{(a, b) \mid a \le b\}.
			\end{aligned}
		\end{align}
	\end{solution*}
\end{frame}

\begin{frame}[fragile]
	\frametitle{Symmetric Closure}
	\begin{columns}
		\begin{column}{.4\linewidth}
			\begin{definition}
				A \textbf{symmetric closure} of $R$\begin{enumerate}
					\item contains $R$,
					\item is symmetric,
					\item is contained within every reflexive relation that contains $R$.
				\end{enumerate}
			\end{definition}
		\end{column}
		\begin{column}{.6\linewidth}
			\begin{theorem}
				Given a relation $R$ on a set $A$, the reflexive closure of $R$ equals $R \cup R^{-1}$.
			\end{theorem}
			\onslide<2>\begin{proof}
				\begin{enumerate}
					\item $R \cup R^{-1} \supseteq R$.
					\item $(R \cup R^{-1})^{-1} = R^{-1} \cup {R^{-1}}^{-1} = R \cup R^{-1}$.
					\item If there exists a set $S \supseteq R$, $r \in R \cup R^{-1}$, $r \notin S$. Therefore $r \notin R$, $r \in R^{-1}$ so that $S \ne S^{-1}$ which forms a contradictory.
				\end{enumerate}
			\end{proof}
		\end{column}
	\end{columns}
\end{frame}

\begin{frame}[fragile]
	\frametitle{Example of Symmetric Closure}
	\begin{example}
		What is the symmetric closure of the relation $R =\{(a, b) \mid a > b\}$ on the set of positive integers?
	\end{example}
	\onslide<2>\begin{solution*}
		\begin{align}
		\begin{aligned}
		& R \cup R^{-1} \\
		= & \{(a, b) \mid a > b \} \cup \{(a, a) \mid a < b\} \\
		= & \{(a, b) \mid a \ne b\}.
		\end{aligned}
		\end{align}
	\end{solution*}
\end{frame}

\begin{frame}[fragile]
	\frametitle{Equivalence Relation}
	\begin{definition}
		A relation on a set $A$ is called an \textbf{equivalence relation} if it is reflexive, symmetric, and transitive.
	\end{definition}
	\begin{example}
		Let $R$ be the relation on the set of real numbers such that $(a, b) \in R$ if and only if $a - b$ is an integer. Is $R$ an equivalence relation?
	\end{example}
	\onslide<2>\begin{solution*}
		Because $a - a = 0$ is an integer for all real numbers $a$, $\forall a \in \mathbb{R}, (a, a) \in R$. Hence, $R$ is reflexive. Now suppose that $(a, b) \in R$. Then $a - b$ is an integer, so $b - a$ is also an integer. Hence, $(b, a) \in R$. It follows that $R$ is symmetric. If $(a, b), (b, c) \in R$, then $a - b$ and $b - c$ are integers. Therefore, $a - c = (a - b) + (b - c)$ is also an integer. Hence, $(a, c) \in R$. Thus, $R$ is transitive. Consequently, $R$ is an equivalence relation.
	\end{solution*}
\end{frame}

\begin{frame}[fragile]
	\frametitle{Example of Equivalence Relation}
	\begin{example}
		Show the smallest equivalence relation contains $\{(a, b), (c, b), (d, c)\}$ on set $\{a, b, c, d, e\}$.
	\end{example}
	\onslide<2>\begin{solution*}
		\begin{enumerate}
			\item Find Reflexsive Closure: $\{(a, a), (a, b), (b, b), (c, b), (c, c), (d, c), (d, d), (e, e)\}$.
			\item Find Symmetric Closure: $\{(a, a), (a, b), (b, a), (b, b), (b, c), (c, b), (c, c), (c, d), (d, c), (d, d), (e, e)\}$.
			\item Find Transitive Closure: $\{(a, a), (a, b), (a, c), (a, d), (b, a), (b, b), (b, c), (b, d), (c, a), (c, b), (c, c), (c, d), (d, a), (d, b), (d, c), (d, d), (e, e)\}$
		\end{enumerate}
	\end{solution*}
\end{frame}

\plain{Questions?}

\end{document}
